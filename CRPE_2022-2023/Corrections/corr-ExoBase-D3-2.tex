\ \\ [-5mm]
   \begin{enumerate}
      \item On obtient le tableau suivant : \\ [2mm]
      {\hautab{1.3}
      \begin{CLtableau}{0.95\linewidth}{6}{c}
         \hline
         & 15-25 ans & 26-44 ans & 45-60 ans & +60 ans & Total \\
         \hline
         Pas du tout &\textcolor{blue}{22} & 82 & 415 & 147 & 666 \\
         \hline
         Une fois & 682 &\textcolor{blue}{3\,794} & 1\,243 & 589 &\textcolor{blue}{6\,308} \\
         \hline
         Deux fois &\textcolor{blue}{413} & 634 & 552 & 138 & 1\,737 \\
         \hline
         Trois fois & 174 & 95 &\textcolor{blue}{384} &\textcolor{blue}{1\,254} & 1\,907 \\
         \hline
         Quatre fois ou plus & 251 & 418 & 923 & 317 &\textcolor{blue}{1\,909} \\
         \hline
         Total & 1\,542 &\textcolor{blue}{5\,023} & 3\,517 & 2\,445 &  \textcolor{blue}{12\,527} \\
         \hline
      \end{CLtableau}}
      \medskip
      \item Nous sommes dans une situation d'équiprobablité puisque l'on effectue un tirage \og au hasard \fg. \\
         On note $\Omega$ l'ensemble des personnes interrogées. \\
         \begin{enumerate}
            \item Soit $A$ l'événement  \og La personne est allée deux fois au restaurant en janvier 2017 \fg{}. \\ [1mm]
               $\mathcal{P}(A) =\dfrac{\textrm{Card}(A)}{\textrm{Card}(\Omega)} =\dfrac{1\,737}{12\,527} \approx 0,14.$ \\ [1mm]
               {\blue La probabilité que le numéro corresponde à une personne qui est allée exactement deux fois au restaurant pendant le mois de janvier 2017 est d'environ 0,14}. \\
            \item Soit $B$ l'événement  \og La personne a moins de 45 ans \fg{}. \\ [1mm]
               $\mathcal{P}(B) =\dfrac{\textrm{Card}(B)}{\textrm{Card}(\Omega)} =\dfrac{1\,542+5\,023}{12\,527} = \dfrac{6\,565}{12\,527} \approx 0,52.$ \\ [1mm]
               {\blue La probabilité que le numéro corresponde à une personne qui a moins de 45 ans est d'environ 0,52}. \\
            \item Soit $C$ l'événement  \og La personne a plus de 60 ans et est allée au moins trois fois au restaurant pendant le mois de janvier 2017\fg{}. \\ [1mm]
               $\mathcal{P}(C) =\dfrac{\textrm{Card}(C)}{\textrm{Card}(\Omega)} =\dfrac{1\,254+317}{12\,527} =\dfrac{1\,571}{12\,527} \approx 0,13.$ \\ [1mm]
               {\blue La probabilité que le numéro corresponde à une personne de plus 60 ans qui est allée au moins trois fois au restaurant en janvier 2017 est d'environ 0,13}.
         \end{enumerate}
   \end{enumerate}
