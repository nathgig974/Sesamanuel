\ \\ [-5mm]
\begin{enumerate}
   \item Le caractère étudié est \bm{la durée de trajets en minutes} et la population est \bm{une cohorte de 36 personnes d'une même entreprise.}
   \smallskip
   \item
   \begin{enumerate}
      \item
      \begin{LCtableau}{0.9\linewidth}{13}{c}
         \hline
         Durée & 5 & 10 & 15 & 20 & 25 & 30 & 35 & 40 & 45 & 50 & 55 & 60 \\
         \hline
         Effectif & 1 & 3 & 2 & 2 & 4 & 7 & 5 & 4 & 3 & 2 & 0 & 3 \\
         \hline
         Fréquence en $\%$ & 2,8 & 8,3 & 5,6 & 5,6 & 11,1 & 19,4 & 13,9 & 11,1 & 8,3 & 5,6 & 0 & 8,3 \\
         \hline
      \end{LCtableau}
      \item $60-5 =55$, donc, l'étendue vaut \bm{$e =55$ minutes} et le mode est la durée correspondant au plus grand effectif, c'est-à-dire \bm{30 minutes.}
      \item La calculatrice donne : \bm{Moyenne = 32,4 ; Médiane = 30 ; $Q_1$ = 25 et $Q_3$ = 40.} \\ [1mm]
   \end{enumerate}
   \item
   \begin{enumerate}
      \item
      \begin{lctableau}{0.9\linewidth}{5}
         \hline
         Durée en min & $]\,0\,;\,15\,]$ & $]\,15\,;\,30\,]$ & $]\,30\,;\,45\,]$ & $]\,45\,;\,60\,]$ \\
         \hline
         Effectif & 6 & 13 & 12 & 5 \\
         \hline
      \end{lctableau}
      \item On trouve $\overline{m} =\dfrac{6\times7,5+13\times22,5+12\times37,5+5\times52,5}{36} \approx$ \bm{29,2 minutes.} \\
      Ce résultats est assez différent de celui trouvé dans la  question 2)c) car la moyenne est faite à partir de la répartition par classe, on perd donc en précision, puisqu'alors on ne s'occupe plus de la valeur exacte, mais de l'appartenance à un intervalle plus grand.
   \end{enumerate}
\end{enumerate}
