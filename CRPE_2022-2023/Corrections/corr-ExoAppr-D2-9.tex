\ \\ [-5mm]
\begin{enumerate}
   \item Une vitesse de 90 km/h correspond à : $\dfrac{90\,000\text{ m}}{3\,600\text{ s}} =25$ m/s. \\ [1mm]
   Sachant que la constante vaut $k =0,14$, en utilisant la formule donnée on obtient en mètre : \\
   $d_A =25\times0,75+0,14\times25^2 =106,25$. \\
   \bm{Un véhicule circulant sur route mouillée à 90 km/h mettra 106,25 mètres pour s'arrêter.} \\
   \item Pour un conducteur vigilant sur route sèche, on a $t_R =0,75$ et $k =0,14$, d'où : \\
   $d_A =0,75\,v+0,073\,v^2$. \\
       L'expression de la distance en fonction de la vitesse n'est pas une fonction linéaire à cause de la présence du \og $v^2$ \fg{} (il s'agit d'une fonction du second degré représentée par une parabole) donc, \\
   \bm{la distance d'arrêt sur route sèche n'est pas proportionnelle à la vitesse.}
   \item
   \begin{enumerate}
       \item Un véhicule roulant à 110 km/h s'arrête en \bm{101 m.}
       \item La distance de freinage à 80 km/h est de \bm{41 m.}
       \item Un véhicule qui roule à 130 km/h met \bm{6,76 s} pour s'arrêter.
       \item Un distance de réaction de 25 m correspond à une vitesse de \bm{120 km/h.}
       \item Un conducteur roulant à 27,8 m/s mettra 85,4 m pour s'arrêter, donc \bm{il évitera l'obstacle.}
    \end{enumerate}
\end{enumerate}
