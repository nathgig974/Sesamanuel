\ \\ [-5mm]
\begin{enumerate}
   \item Volume de la canette classique de rayon 3,3 cm et de hauteur 9,8 cm : \\
   $\mathcal{V}_c =\pi\times(3,3\text{ cm})^2\times9,8\text{ cm} \approx 335\text{ cm}^3 \approx0,335\text{ dm}^3 \approx0,335\text{ L} \approx33,5\text{ cL}$. \\
   Or, 33,5 cL > 33 cL donc, \bm{le volume de cette canette est supérieur à 33 cL.}
   \item Une contenance de 33 cL = 0,33 L correspond à un volume de 0,33 dm$^3$ = 330 cm$^3$. \\
   Volume de la canette slim de rayon 2,8 cm et de hauteur $h$ en cm : \\
   $\mathcal{V}_s =\pi\times(2,8\text{ cm})^2\times h\text{ cm} \geq 330\text{ cm}^3 \iff h \geq \dfrac{330\text{ cm}^3}{\pi\times7,84\text{ cm}^2} \iff h \geq 13,39\text{ cm}$. \\ [1mm]
   \bm{Pour avoir un volume au moins égal à 33 cL, la hauteur doit être au moins égale à 13,4 cm.}
   \item Dans toutes cette partie, les mesures de longueur sont exprimées en centimètre et les mesures d'aire en centimètre carré. \\
   \begin{enumerate}
      \item $\mathcal{V} =\pi\times r^2\times h =330 \iff$ \bm{$h =\dfrac{330}{\pi r^2}$.}
      \item La longueur $L$ du rectangle correspond au périmètre du disque de rayon $r$ donc : \bm{$L =2\pi r$.}
      \item $\mathcal{A} =L\times h =2\pi r\times\dfrac{330}{\pi r^2} =$ \bm{$\dfrac{660}{r}$.}
      \item L'aire totale du patron est la somme de l'aire du rectangle et des deux disques : \bm{$\mathcal{A}_T =\dfrac{660}{r}+2\pi r^2$.}
   \end{enumerate}
\end{enumerate}
