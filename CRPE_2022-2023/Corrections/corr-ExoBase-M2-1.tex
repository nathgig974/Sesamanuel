\ \\ [-5mm]
   \begin{enumerate}
      \item On procède par étapes.
         \begin{itemize}
            \item Volume d'eau dans le vase à vide : le vase plein d'eau pèse \ug{2300} et le vase vide \ug{500} donc, la masse d'eau est de \ug{1800}. \\
            Or, la masse volumique de l'eau est de 1 g/\ucmc{}, donc \ug{1800} ont un volume de \ucmc{1800}.
            \item Volume d'eau déplacé : une fois la statue dans le vase plein d'eau, elle déplace autant d'eau que son volume, d'après le principe d'Archimède. Son poids est alors de \ug{2600}. \\
            Si on lui enlève le poids du vase et de la statue ($\ug{500}+ \ug{340} =\ug{840}$), on obtient $\ug{2600}-\ug{840} =\ug{1760}$. \\
            Par rapport au volume d'eau à vide, il manque donc \ug{40} ($1\,800-1\,760 = 40$).
            \item Volume de la statue : \\
               \ug{40} d'eau correspondent à \ucmc{40}, donc : {\blue le volume de la statue est de \ucmc{40}}.
         \end{itemize}
         \medskip
         \item $\mu=\dfrac{\text{masse en g}}{\text{volume en \ucmc{}}} =\dfrac{\ug{340}}{\ucmc{40}} ={\blue\ug{8,5}\slash \ucmc{}}$. \\ [1mm]
            ou encore : $\mu=\dfrac{\text{masse en kg}}{\text{volume en L}} =\dfrac{\ug{0,34}}{\udmc{0,04}} ={\blue \ukg{8,5}\slash\ul{}}$. \\
         \medskip
         \item Le poids de ce nouveau liquide est de $\ug{1940}-\ug{500} =\ug{1440}$. \\
            Il occupe un volume de \ucmc{1800}, donc, sa masse volumique est de $\mu=\dfrac{\text{masse en g}}{\text{volume en \ucmc{}}} =\dfrac{\ug{1440}}{\ucmc{1800}}$. \\
             Soit : {\blue la masse volumique du nouveau liquide est de \ug{0,8}\slash\ucmc{}.}
   \end{enumerate}
