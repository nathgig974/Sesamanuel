\ \\ [-5mm]
   \begin{enumerate}
      \item Soit $N=\overline{cdu}$. Si l'un des chiffres vaut 0, alors le produit vaudra 0, ce qui est  impossible. \\
         Si l'un des chiffres est 1, alors le produit vaudra au maximum $9\times9 =81$, alors qu'il devrait être de 120. \\
         Si l'un des chiffres est 2, le produit des deux autres chiffres doit valoir $60$. Or, 60 est le produit de 2 par 30, ou 3 par 20, ou 4 par 15, ou 5 par 12 ou 6 par 10. Toutes ces décompositions sont impossibles puisque l'un des deux nombres est supérieur à 9. Conclusion : {\blue N ne peut contenir ni 0, ni 1, ni 2.}
      \item 120 n'est pas divisible par 7, pas plus que par 9 donc, {\blue N ne peut contenir ni 7, ni 9.}
      \item On peut  chercher une décomposition de 120 en produit de trois facteurs, tous différents de 0, 1, 2, 7 et 9 : la décomposition en produit de facteurs premiers de 120 est $120 =2^3\times3\times5$. \\
         Donc, par exemple, $120 =8\times3\times5$, et $8+3+5 =16$. {\blue Le nombre 835 est une solution du problème.} \\
         Pour trouver d'autres solutions, il suffit de faire des permutations des trois chiffres 8, 3 et 5 ce qui donne : \\
         {\blue 853, 538, 583, 358, 385.}
      \item On détermine les autres décompositions de 120 en produit de trois facteurs, tous inférieurs à 10 et différents de 0, 1, 2, 7, 9 : seul $120 =4\times6\times5$ convient mais $4+6+5 =15 \neq 18$. \\
         {\blue Il y a six solutions au problème : 835, 853, 538, 583, 358 et 385.}
   \end{enumerate}
