\ \\ [-5mm]
\begin{enumerate}
      \item À faire sur tablette ou ordinateur.
      \item On peut conjecturer que \bm{$MD =ME$}.
      \item On applique le Théorème de Thalès dans les triangles $AMB$ et $AMC$.
      \begin{itemize}
         \item Dans le triangle $AMB$, les points $M, R, A$ et $M, D, B$ sont alignés dans cet ordre. \\
         Les droites $(AB)$ et $(RD)$ sont parallèles donc, d'après le théorème de Thalès : $\dfrac{MR}{MA} =\dfrac{MD}{MB}$.
         \item Dans le triangle $AMC$, les points $M, R, A$ et $M, E, C$ sont alignés dans cet ordre. \\
         Les droites $(AC)$ et $(RE)$ sont parallèles donc, d'après le théorème de Thalès : $\dfrac{MR}{MA} =\dfrac{ME}{MC}$.
   \end{itemize}
   On a alors $\dfrac{MD}{MB} =\dfrac{ME}{MC}$. \\
   Or, $MB =MC$ puisque $M$ est le milieu de $[BC]$ d'où $\dfrac{MD}{MB} =\dfrac{ME}{MB} \iff$ \bm{MD = ME}.
   \end{enumerate}
