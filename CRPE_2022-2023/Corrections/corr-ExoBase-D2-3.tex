\ \\ [-5mm]
   \begin{enumerate}
      \item L'image de 7 est d'environ 7,5. {\blue Pour 7 litres d'eau liquide, on obtient environ 7,5 litres de glace.}
      \item On lit l'antécédent de 9, qui est environ 8,3 (avec la précision permise par le graphique). \\
         {\blue Il faut environ 8,3 litres d'eau liquide pour obtenir 9 litres de glace.}
      \item On observe sur le graphique que le volume de glace en fonction du volume d'eau liquide est représenté par un segment de droite passant par l'origine. {\blue Le volume de glace est proportionnel au volume d'eau liquide.}
      \item Le volume de l'eau passant de l'état liquide à l'état solide augmente de 0,8 litres pour 10 litres. Donc, il augmenterait de 8 litres pour 100 litres. {\blue L'augmentation due à la solidification est de 8\%.}
      \item Pour un jour de nettoyage, la ville fournit un volume de $20 \text{ m}^3 = 20\,000 \text{ dm}^3$, soit une capacité de $20\,000$ L. \\
         Donc, pour 30 jours de nettoyage, la capacité est de $30\times 20\,000$ L $=600\,000$ L. \\
         La transformation à l'état solide augmente de 8\,\% la capacité initiale, ce qui correspond à un coefficient multiplicateur de $1+\dfrac{8}{100} =1,08$. On obtient alors $1,08\times600\,000 \text{ L} = 648\,000 \text{ L}$. \\ [1mm]
         {\blue Le volume d'eau fourni par la ville de Lyon pour $30$ jours correspond à $648\,000$ L de glace.}
   \end{enumerate}
