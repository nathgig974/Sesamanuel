\ \\ [-5mm]
   \begin{enumerate}
      \item $4+3+1 =8$ employés gagnent plus de \ueuro{2000} à Toulouse, sur un total de $2+4+3+2+4+3+1 =19$. Ce qui correspond à un pourcentage de $\dfrac{8}{19}\times100 \approx42,11\,\%>40\,\%$. \\ [1mm]
         {\blue L'affirmation de la cheffe d'entreprise est vraie}.
      \item
         \begin{enumerate}
            \item Le salaire minimal est de \ueuro{1410} et l'étendue de \ueuro{1890} donc, {\blue le salaire maximal vaut} $\ueuro{1410}+\ueuro{1890} ={\blue \ueuro{3300}}$. \\
            \item Soit $s$ le salaire recherché, en \ueuro{}. On a : \\ [2mm]
               $\dfrac{2\times\ueuro{1410}+4\times\ueuro{1590}+3\times\ueuro{1760}+2\times\ueuro{1920}+4\times\ueuro{2100}+3\times s+1\times\ueuro{3300}}{19} =\ueuro{1935}$ \\ [1mm]
               $\iff \dfrac{\ueuro{30000}+3s}{19} =\ueuro{1935}$ \\ [2mm]
               $\iff \ueuro{30000}+3s =19\times\ueuro{1935} =\ueuro{36765}$ \\ [1mm]
               $\iff 3s =\ueuro{36765}-\ueuro{30000} =\ueuro{6765}$ \\ [2mm]
               $\iff s =\dfrac{\ueuro{6765}}{3} =\ueuro{2255}$. Donc, {\blue L'avant-dernière barre correspond à un salaire de \ueuro{2255}}. \medskip
         \end{enumerate}
      \setcounter{enumi}{2}
      \item Il y a 19 employés à Toulouse, donc le salaire médian est le 10\up{e} salaire lorsqu'on les ordonne su plus petit au plus grand, par exemple. Par lecture sur le diagramme en bâtons, le 10\up{e} salaire vaut \ueuro{1920}. \\
         {\blue Le salaire médian à Toulouse vaut \ueuro{1920}}. \medskip
      \item $m =\dfrac{19\times\ueuro{1935}+12\times\ueuro{1520}}{19+12} =\dfrac{\ueuro{55005}}{31} \approx\ueuro{1774,35}$. \\ [2mm]
         {\blue Le salaire moyen de l'entreprise vaut environ \ueuro{1774}}.
      \item
         \begin{enumerate}
            \item Une augmentation de 10\,\% correspond à un coefficient multiplicateur de $1+\dfrac{10}{100} =1,1$. \\ [1mm]
               Or, $\ueuro{1410}\times1,1 =\ueuro{1551}$. \\
               {\blue Le montant du salaire minimum à Montauban en 2021 sera de \ueuro{1551}}.
            \item On cherche le coefficient multiplicateur $c_m$ qui permet de passe de \ueuro{1520} à \ueuro{1935} : \\ [1mm]
                $\ueuro{1520}\times c_m =\ueuro{1935} \iff c_m =\dfrac{\ueuro{1935}}{\ueuro{1520}} \approx1,2730$. \\ [2mm]
                Or, $1,2730 =1+0,2730 =1+\dfrac{27,30}{100}$ donc, {\blue il faudrait augmenter les salaires de Montauban d'environ 27,3\,\%}.
         \end{enumerate}
   \end{enumerate}
