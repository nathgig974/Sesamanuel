   \ \\ [-5mm]
   \begin{enumerate}
      \item On peut pour cela utiliser un logiciel comme CarMetal ou GeoGebra, on construit pas à pas chacun des objets qui constituent les figures demandées en respectant les contrainte de mesure et d'orientation. \\
         L'aspect \og dynamique \fg{} permet de vérifier sa construction : en effet, si l'un des objets est mal construit et que l'on déplace l'un des points de la figure, celle-ci va se transformer.
      \item Pour les figures 1 et 2, il y a une seule configuration possible, à isométries près. \\
         Pour la figures 3, il y a une infinité de configurations qui dépendant de l'endroit où l'on place le troisième point. \\
         \begin{pspicture*}(-0.5,-0.5)(3.5,4)
            \pspolygon[fillstyle=solid,fillcolor=lightgray!50](0,0)(3,0)(3,3)(0,3)
            \psset{linecolor=gray}
            \pscircle(3,0){3}
            \psline(0,0)(0,4)
            \psline(3,0)(3,4)
            \psline(-1,3)(3,3)
            \rput(-0.3,-0.3){A}
            \rput(3.3,-0.3){B}
            \rput(3.3,3.3){C}
            \rput(-0.3,3.3){D}
         \end{pspicture*}
         \begin{pspicture*}(-3.7,-2.5)(3.5,2.5)
            \pspolygon[fillstyle=solid,fillcolor=lightgray!50](-3,0)(0,-2)(3,0)(0,2)
            \psset{linecolor=gray}
            \psline(-3,0)(3,0)
            \psline(0,-3)(0,3)
            \pscircle(0,0){2}
            \rput(-3.3,0){A}
            \rput(3.3,0){C}
            \rput(0.3,-2.3){B}
            \rput(0.3,2.3){D}
         \end{pspicture*}
         \begin{pspicture*}(-0.4,-0.5)(4.4,2.5)
            \pspolygon[fillstyle=solid,fillcolor=lightgray!50](0,0)(3,0)(3.9,1.8)(0.9,1.8)
            \psset{linecolor=gray}
            \pscircle(0,0){2}
            \pscircle(3,0){2}
            \pscircle(0.9,1.8){3}
            \rput(-0.3,0){A}
            \rput(3.3,0){B}
            \rput(1,2.2){D}
            \rput(4.2,2.1){C}
         \end{pspicture*}
   \end{enumerate}
