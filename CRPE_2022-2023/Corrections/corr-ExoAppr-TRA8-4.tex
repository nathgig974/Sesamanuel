\ \\ [-5mm]
\begin{enumerate}
   \item On peut utiliser les procédures suivantes :
   \begin{itemize}
      \item On a 30 dragées, soit 4 fois moins de dragées puisque 120 dragées $\div$ 4 = 30 dragées donc, la masse de 30 dragées est 4 fois plus petite et vaut 360 g $\div$ 4 = 90 g. \\
      Utilisation d'un coefficient scalaire (coefficient de linéarité multiplicative) égal à 1$\div$4.
      \item $120\times3 =360$ donc, 30$\times$3 =90. La masse de 30 dragées est de 90 g. \\
      Utilisation du coefficient de proportionnalité qui vaut 30.
      \smallskip
      \item 120 dragées pèsent 360 g donc, une dragée pèse $\dfrac{360\text{ g}}{120} =3$ g. \\
      D'où 30 dragées pèsent $30\times3$ g = 90 g. \\
      Utilisation du passage par l'unité.
   \end{itemize}
   \item Pour \og espérer \fg{} une procédure par retour à l'unité, il faut notamment que les autres procédures soient moins évidentes. Or, les nombres en jeu sont propices à l'utilisation d'un coefficient scalaire (un quart) ou de proportionnalité (30) qui sont simples à obtenir mentalement. \\
   On pourrait donc, par exemple, modifier l'énoncé de la façon suivante :
   \medskip
   \fbox{\begin{minipage}{12.5cm} Une boîte contient des dragées toutes identiques. 120 dragées pèsent 270 g. \\
   Combien pèsent 32 dragées ? \end{minipage}} \\
   \medskip
   Le coefficient scalaire devient 3,75 et le coefficient de proportionnalité 2,25 qui ne sont pas des coefficients \og simples \fg.
\end{enumerate}
