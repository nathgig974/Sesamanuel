\ \\ [-5mm]
   \begin{enumerate}
      \item
         \begin{enumerate}
            \item Pour l'abscisse $r =\ucm{1,5}$, on lit une aire environ égale à {\blue \ucmq{450}}.
            \item Pour une ordonnée $\mathcal{A} =\ucmq{300}$, on lit deux valeurs pour le rayon : {\blue $r_1 \approx\ucm{2,5}$ et $r_2 \approx \ucm{5,25}$}.
            \item Pour une canette classique de rayon \ucm{3,3}, on lit une aire d'environ \ucmq{270} et pour une canette slim de rayon \ucm{2,8}, on lit une aire d'environ \ucmq{285}. Donc, {\blue c'est la canette classique qui demande le moins de surface de métal}.
            \item La surface minimale est atteinte pour un rayon d'environ {\blue \ucm{3,75}}.
         \end{enumerate}
      \setcounter{enumi}{1}
      \item
         \begin{enumerate}
            \item Dans \texttt{B2}, on peut écrire : {\blue \texttt{=2*PI()*B1$\wedge$2+660/B1}}
            \item Les valeurs minimales pour l'aire dans le tableau sont 264,40 et 264,41, elles correspondent à \\
   {\blue un rayon compris entre 3,7 cm et 3,8 cm.}
         \end{enumerate}
   \end{enumerate}
