{\bf A.}
\begin{enumerate}
   \item On pourrait compléter, par exemple, par \og La coopérative avait acheté 25 tickets d'entrée, tous aux même prix, elle en achète 20 autres au même tarif \fg.
   \item \og Sachant qu'il y a une réduction de 2 \euro{} par ticket à partir du vingtième élève. \fg{} \\
\end{enumerate}

{\bf B.}
\begin{enumerate}
   \item L'exemple 1 illustre l'utilisation du {\bf coefficient de proportionnalité} (toutes les valeurs d'une même grandeur sont obtenues en multipliant les valeurs de l'autre grandeur par le même nombre).
   \item L'exemple 2 illustre l'utilisation de la {\bf linéarité multiplicative}, ou homogénéité (quand l'une des grandeurs est multipliée ou divisée par un nombre, l'autre grandeur est multipliée ou divisée par le même nombre).
   \item Dans l'exemple 1, le rapport est le coefficient de proportionnalité. Il est le même pour toutes les données. Dans l'exemple 2, il s'agit d'un coefficient de linéarité, ou coefficient scalaire entre deux grandeurs identiques.
   \item Il s'agit ici de la propriété multiplicative de la linéarité, qui n'est pas vérifiée. \\
   On aurait également pu remarquer que 6 stylos, c'est 3 stylos + 3 stylos. Or le prix n'est pas égal à 5 \euro{} + 5 \euro{}, la situation n'est donc pas proportionnelle en vertu de la propriété additive de la linéarité qui n'est pas vérifiée. \\
\end{enumerate}

{\bf C.} {\bf Auriane} effectue un {\bf passage par l'unité} : elle calcule le nombre d'\oe ufs nécessaires pour une personne, puis pour 20 personnes. Son raisonnement, sa rédaction et son résultat sont corrects. \\
   {\bf émeric} utilise utilise la propriété {\bf additive} de la linéarité : pour 20 personnes (8+12), il faut 15 \oe ufs (6+9). Son raisonnement et son résultat sont corrects. \\
   {\bf Nicolas} se trompe dans son raisonnement : il ajoute une même grandeur (12 personnes) à la fois aux 8 personnes (ce qui est cohérent), mais aussi au nombre d'\oe ufs ! Son résultat est donc faux. Il est possible qu'il ait souhaité utilisé la propriété additive de la linéarité, qu'il ne maîtrise pas. \\
   {\bf Kévin} effectue le {\bf passage par l'unité}. Son raisonnement, sa rédaction et son résultat son corrects. En revanche, il n'effectue pas les opérations dans l'ordre de son raisonnement écrit : il commence par multiplier par 20 (le nombre de personnes) au lieu de diviser par 6 (le nombre d'\oe ufs). Cependant, étant donné la commutativité de la multiplication et de la division, le résultat demeure juste. \\

{\bf D.}
\begin{enumerate}
   \item La notion de pourcentage relève de la proportionnalité : il s'agit du nombre qui aurait été proportionnellement obtenu si l'effectif avait été de 100. Il sera donné plutôt en 6\up{e} puisque, seuls les pourcentages simples (25\,\%, 50\,\%\dots) en lien avec les fractions doivent être acquis à l'école.
   \item
   \begin{enumerate}
      \item Paul effectue la moyenne des deux pourcentages. Ce qui est faux dans la plupart des cas. C'est juste uniquement lorsque l'effectif de départ est strictement le même dans les deux situations.
      \item Pour que 50\,\% soit la bonne réponse, il faudrait que les deux médiathèques disposent du même nombre de livres, soit 4\,000 livres à la bibliothèque jean Jaurès.
   \end{enumerate}
\end{enumerate}
