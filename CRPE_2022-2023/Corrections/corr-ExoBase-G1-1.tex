\ \\ [-5mm]
\begin{enumerate}
   \item On peut classer les solides suivant leur caractère polyédrique ou non :
   \begin{itemize}
      \item les polyèdres : A, B, C ;
      \item les non polyèdres : D, E, F, G.
   \end{itemize}
   \item Dans les programme du cycle 2, les solides présents sont la boule, le cylindre, le cône, le cube, le pavé droit et la pyramide. Il s'agit de les reconnaître, les trier et les nommer. \\
   Cependant, il est important de présenter également d'autres types de solides afin de montrer qu'il en existe justement d'autres, et de pouvoir identifier les solides les uns par rapport aux autres ainsi que leurs caractéristiques principales.
   \item Les élèves pourraient proposer un classement du type \og sucré, salé \fg{}, ou \og ça se mange - ça ne se mange pas \fg{} ou encore proposer un classement par couleur, par taille\dots. \\
   Si on ne précise pas qu'il faut s'intéresser uniquement aux formes des emballages, on peut donc obtenir une multitude de classements n'ayant aucun rapport avec ce que l'enseignant souhaite obtenir !
     \item Tout d'abord, le passage du plan à l'espace n'est pas forcément clair pour les élèves : ils sont beaucoup plus habitués à travailler dans le plan. \\
     Ensuite, ils ont eu l'habitude, notamment en maternelle, de reconnaître, classer et nommer des formes simples comme le carré, le triangle, le disque.
   \item Le rouleau de papier d'aluminium est un objet non fermé. Contrairement aux autres qui, même si on peut les ouvrir (comme les différentes boites), peuvent tous être complètement fermés.
   \item Pour le \textbf{groupe 1}, il semble que les élèves ont \og repéré \fg{} les polyèdres. Les boules forment une catégorie, les autres emballages une troisième catégorie. \\
   On remarque que les cylindres et les cônes n'apparaissent pas dans le classement, sûrement en raison de leur forme mélangeant des faces planes comme pour les polyèdres A, B, C et des surfaces non planes comme pour la sphère.  \\
   Pour le \textbf{groupe 2}, les élèves ont classé ensemble les prismes et les cylindres, c'est à dire les solides ayant des faces opposées parallèles ; puis les solides \og pointus \fg{} (pyramides et cônes) et enfin les boules et autres emballages, peut-être comme la catégorie des formes \og arrondies \fg{}, les cylindres et les cônes ayant déjà été classés.
   \item
   \begin{enumerate}
      \item Le but de l'enseignant est de les faire classer les objets selon si ce sont des polyèdres (ceux qui ne roulent pas) ou non (ceux qui roulent), et à cette occasion d'introduire le terme de polyèdre.
      \item Ce critère n'est cependant pas pertinent didactiquement : par exemple un cube ayant une face bombée vers l'intérieur ne roule pas. Pourtant, ce n'est pas un polyèdre. \\
      Inversement, un polyèdre régulier avec de multiples faces donne l'impression qu'il roule lorsqu'on le lance.
    \end{enumerate}
\end{enumerate}
