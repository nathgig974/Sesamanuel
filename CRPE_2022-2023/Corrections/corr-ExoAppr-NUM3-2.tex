\ \\ [-5mm]
\begin{enumerate}
   \item Une \textbf{division décimale} permet de répondre à la question, et plus particulièrement la division de deux nombres entiers avec quotient décimal : $\opdiv[decimalsepsymbol={,}]{9}{5}$. \quad La réponse attendue est donc 1,8 L. \\
   \item \textbf{Julia} schématise la situation par 9 bâtons pour les 9 litres d'huile et 5 bâtons pour les 5 bidons. Ensuite, elle associe 5 litres à 5 bidons en barrant 5 bâtons. Il lui reste 4 litres qu'elle transforme en 8 demi-litres. Parmi ces 8 demi-litres, elle en attribue 5 aux 5 bidons ce qui lui donne un litre plus un demi-litre, soit 1,5 litre par bidon. Il lui reste 3 demi-litres qu'elle laisse ainsi. \\
      Son raisonnement n'est pas faux, son résultat est cohérent mais il ne correspond pas tout à fait à la question. \\
   \textbf{Karima} commence par procéder par essais-erreurs : elle fait une addition itérée de 2,5 (5 fois), elle calcule la somme des dixièmes et il semble qu'elle calcule également la somme des unités mais qu'elle ne l'écrive pas car la valeur n'est pas celle qu'elle doit trouver. Elle se rend compte néanmoins que le résultat est trop grand, et c'est la raison pour laquelle elle teste la même procédure avec 2,2. \\
      Après deux tentatives, elle cherche une autre procédure plus efficace, elle pense à une division en commençant par une division par 9, qu'elle barre rapidement pour effectuer la division de 9 par 5, ce qui est une procédure experte. Cependant, elle fait une division euclidienne qui ne donne pas un résultat, mais deux (quotient et reste) qui ne lui permettent théoriquement pas de conclure. Elle répond en donnant un résultat composé du quotient comme chiffre des unités, et du reste comme chiffre des dixième, ce qui démontre une mauvaise compréhension des termes obtenus dans une division euclidienne et de la maîtrise des nombres décimaux. Enfin, elle tente une vérification avec le nombre 1,4 trouvé, mais elle n'effectue pas le calcul. \\
      Son résultat est erroné.
\end{enumerate}
\Coupe
   \textbf{Louis} procède par essais-erreurs également mais avec une procédure plus rapide : il cherche la valeur en effectuant la multiplication. Il commence par multiplier par 5 par 1,5, son résultat est trop petit. Il tente donc 1,75, c'est toujours trop petit. Puis il essaie avec 2, le résultat est trop grand, enfin il tente avec 1,8 ce qui lui donne le bon résultat. \\
      Sa réponse est correcte. \\
\begin{enumerate}
\setcounter{enumi}{2}
   \item Si l'objectif est d'utiliser la technique experte (la division décimale), le quotient ne peut être qu'un nombre entier, donc le nombre de bidons est un nombre entier. \\
   En revanche, on peut \textbf{modifier le nombre de litres} de manière à rendre la procédure par essais-erreurs plus longue, par exemple en choisissant 9,1 litres.
\end{enumerate}
