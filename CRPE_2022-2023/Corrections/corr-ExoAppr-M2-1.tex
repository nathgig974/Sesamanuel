\ \\ [-5mm]
\begin{enumerate}
   \item On procède par étapes.
   \begin{itemize}
      \item Volume d'eau dans le vase à vide : le vase plein d'eau pèse 2 300 g et le vase vide 500 g donc, la masse d'eau est de 1 800 g. \\
      Or, la masse volumique de l'eau est de 1 g\slash cm$^3$, donc 1 800 g ont un volume de 1 800 cm$^3$.
      \item Volume d'eau déplacé : une fois la statue dans le vase plein d'eau, elle déplace autant d'eau que son volume, d'après le principe d'Archimède. Son poids est alors de 2 600 g. \\
      Si on lui enlève le poids du vase et de la statue (500 g + 340 g = 840 g), on obtient 2 600 g $-$ 840 g =1 760 g. \\
   Par rapport au volume d'eau à vide, il manque donc 40 g (1 800 $-$ 1760 = 40).
   \item Volume de la statue : \\
   40 g d'eau correspondent à \ucmc{40}, donc : \bm{le volume de la statue est de 40 cm$^3$.}
   \end{itemize}
   \medskip
   \item $\mu=\dfrac{\text{masse en g}}{\text{volume en cm$^3$}} =\dfrac{\ug{340}}{\ucmc{40}} =\,$\bm{8,5 g\slash cm$^3$.} \\ [1mm]
   ou encore : $\mu=\dfrac{\text{masse en kg}}{\text{volume en L}} =\dfrac{\ug{0,34}}{\udmc{0,04}} =\,$\bm{8,5 kg\slash L.} \\
   \medskip
   \item Le poids de ce nouveau liquide est de 1 940 g $-$ 500 g = 1 440 g. \\
   Il occupe un volume de 1 800 cm$^3$, donc, sa masse volumique est de $\mu=\dfrac{\text{masse en g}}{\text{volume en cm$^3$}} =\dfrac{\ug{1440}}{\ucmc{1800}}$ \\
    Soit : \bm{la masse volumique du nouveau liquide est de 0,8 g\slash cm$^3$.}
\end{enumerate}
