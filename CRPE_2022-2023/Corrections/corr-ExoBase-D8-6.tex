\ \\ [-5mm]
   \begin{enumerate}
      \item
         \begin{enumerate}
            \item Soit $n$ la note de Luc au devoir 6, elle correspond à la moyenne des cinq premiers, sa moyenne totale ne varie donc pas après ce sixième devoir. \\ [1mm]
               $n =\dfrac{12+5+18+11+19}{5} =\dfrac{65}{5} =13$. {\blue La moyenne de Luc est de 13}. \smallskip
            \item Soit $n'$ la note de Luc au devoir 6 pour obtenir 15 de moyenne, on a : \\ [1mm]
               $\dfrac{12+5+18+11+19+n'}{6} =15 \iff \dfrac{65+n'}{6}=15$ \\ [2mm]
               \hspace*{4.6cm} $\iff 65+n' =15\times6 =90$ \\ [1mm]
               \hspace*{4.6cm} $\iff n' =90-65 =25$. \\
               {\blue Luc ne peut pas avoir 15 de moyenne après le sixième devoir.}
         \end{enumerate}
      \setcounter{enumi}{1}
      \item
         \begin{enumerate}
            \item Une augmentation de 25\,\% correspond à un coefficient multiplicateur de $1+\dfrac{25}{100} =1,25$. \\
               Donc, on a {\blue $y=1,25x$} \medskip
            \item $\dfrac{20+15+4+9+x+y}{6} =12,5 \iff \dfrac{48+x+1,25x}{6}=12,5$ \\ [2mm]
               \hspace*{4.95cm} $\iff \dfrac{48+2,25x}{6} =12,5$ \\ [2mm]
               \hspace*{4.95cm} $\iff 48+2,25x =12,5\times6 =75$ \\ [1mm]
               \hspace*{4.95cm} $\iff 2,25x =75-48 =27$ \\ [2mm]
               \hspace*{4.95cm} $ \iff x =\dfrac{27}{2,25} =12$ \\ [1mm]
               Dans ce cas, $y=1,25\times12 =15$. \\
               {\blue Julie a obtenu les notes de 12 au devoir 5 et 15 au devoir 6.}
         \end{enumerate}
   \end{enumerate}
