\ \\ [-5mm]
\begin{enumerate}
   \item Exemple de classement : \\
   {\renewcommand{\arraystretch}{1}
   \begin{Ltableau}{0.96\linewidth}{4}{p{3cm}|C{0.5}|p{5.5cm}|p{4.8cm}}
      \hline
      & Élève & Description de l'erreur & Hypothèses sur son origine \\
      \hline
      Production correcte. & C & & \\
      \hline
      Algorithme correct, & E & $2+3+5+7 =18$ au lieu de 17. & Répertoire additif mal maîtrisé \\
      \cline{2-3}
      erreur de calcul & G & $2+3+5+7 =16$ au lieu de 17. & ou erreur de comptage. \\
      \hline
      Erreurs de retenues (technique opératoire mal comprise). & A & Retenue de 1 au lieu de 2 sur le chiffre des dizaines. & Non compréhension de la retenue et habitude de rencontrer uniquement des retenues de 1. \\
      \cline{2-4}
      & B & Inversion entre le chiffre à poser et la retenue. & Non compréhension de la retenue, numération mal maîtrisée. \\
      \cline{2-4}
      & D & Pose systématique de la retenue au dessus du chiffre le plus à gauche. & Aucun sens donné à la retenue. \\
      \cline{2-4}
      & F & La retenue n'est pas utilisée. & Rôle non compris de la retenue. \\
      \cline{2-4}
      & H & Pose des sommes par colonne. & Méconnaissance du principe de la retenue. \\
      \hline
   \end{Ltableau}}
   \item L'élève doit avoir acquis les connaissances et compétences suivantes : connaître et utiliser la technique opératoire de l'addition ; connaître les tables d'addition ; connaître la structure des nombres entiers et de son lien avec la retenue.
\end{enumerate}
