\begin{itemize}
    \item {\bf Linéarité mixte, multiplicative puis additive :} \\
    \begin{tabular}{p{7cm}c|cp{7.5cm}}
       Avec utilisation des décimaux
       & & &
       Sans utilisation des décimaux \\
       \hline
       Pour 4 \euro, on a 2 kg de pommes, donc pour 1 \euro, on a 0,5 kg de pommes et pour 5 \euro{} = 4 \euro + 1 \euro, on a 2 kg + 0,5 kg  = 2,5 kg de pommes.
       & & &
       Pour 4 \euro, on a 2\,000 g de pommes, donc pour 1 \euro, on a 500 g de pommes et pour 5 \euro{} = 4 \euro + 1 \euro, on a 2\,000 g + 500 g  = 2\,500 g de pommes.
   \end{tabular}
   \smallskip
   \item {\bf Passage par l'unité :} \\
   \begin{tabular}{p{7cm}c|cp{7cm}}
       Avec utilisation des décimaux
       & & &
       Sans utilisation des décimaux \\
       \hline
       2 kg de pommes coûtent 4 \euro, donc pour 1 \euro, on a 0,5 kg (2 kg$\div$4) et pour 5 \euro, on a 2,5~kg ($5\times0,5\text{ kg} =2,5$ kg).
       & & &
       2\,000 g de pommes coûtent 4 \euro, donc, pour 1 \euro, on a 500 g (2\,000 g$\div$4) et pour 5 \euro, on a 2\,500 g ($5\times500\text{ g} =2\,500$ g).
   \end{tabular}
   \medskip
   \item {\bf Coefficient de proportionnalité :} \\
   pour passer de 2 à 4, on multiplie par 2, il s'agit du coefficient de proportionnalité. Donc, pour obtenir un prix de 5 \euro, il faut 2,5 kg de pommes ($2,5\times2 =5$).
\end{itemize}
