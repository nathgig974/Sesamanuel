\ \\ [-5mm]
\begin{enumerate}
   \item 117 est divisible au moins par 3 puisque la somme de ses chiffres est égale à 9 donc, 117 n'est pas un nombre premier. \\
   \bm{L'affirmation est fausse}.
   \item
   $(n+2)^2 - (n-2)^2 =(n^2+4n+4)-(n^2-4n+4)$ \\
   \hspace*{3.15cm} $=\cancel{n^2}+4n+\cancel{4}-\cancel{n^2}+4n-\cancel{4}$ \\
   \hspace*{3.15cm} $=8n$. \\
   \begin{enumerate}
      \item D'après le résultat obtenu après développement et réduction, \\
      \bm{l'affirmation est vraie}.
      \item Pour $n =1, (n+2)^2 - (n-2)^2 =8$ qui n'est pas un multiple de 32 donc, \\
      \bm{l'affirmation est fausse}.
   \end{enumerate}
   \item Le nombre recherché est multiple de 2 et de 3, mais pas de $2^2$ et $3^2$. Il suffit donc de choisir un nombre dont la décomposition en produit de facteurs premiers commence par $2\times3\times\dots$. Par exemple : $2\times3\times5 =30$. \\
   \bm{L'affirmation est vraie}.
   \item Une série de perles \og jaune-rouges-blanches \fg{} comporte 6 perles. Or, $147 =6\times24+3$ donc, la 147\up{ème} perle sera de la même couleur que la 3\up{ème} perle de la série, soit rouge. \\
   \bm{L'affirmation est vraie}.
   \item $126 =2^1\times3^2\times7^1$ donc, 126 possède $(1+1)\times(2+1)\times(1+1)$ diviseurs $=2\times3\times2$ diviseurs = 12 diviseurs (ces diviseurs sont : 1 - 2 - 3 - 6 - 7 - 9 - 14 - 18 -21 - 42 - 63 - 126). \\
   \bm{L'affirmation est fausse}.
\end{enumerate}
