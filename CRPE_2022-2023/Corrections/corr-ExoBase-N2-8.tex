   On remarque que $85 =5\times17$. L'objectif est donc, suivant le cas, de pouvoir simplifier par 85, par 17 ou par 5. \\
   On peut choisir, par exemple :
   \begin{center}
      {\hautab{2.5}
      \begin{tabular}{|*3{C{4}|}}
         \hline
         un entier naturel & un décimal non naturel & un rationnel non décimal \\
         \hline
         $\dfrac{\blue 85}{85} =1$ & $\dfrac{\blue17}{85} \;   =\dfrac{\cancel{17}\times1}{\cancel{17}\times5} =\dfrac15$ & $\dfrac{\blue 5}{85} \; =\dfrac{\cancel{5}\times1}{\cancel{5}\times17} =\dfrac{1}{17}$ \\
         \hline
         $\dfrac{85}{\blue 1} =85$ & $\dfrac{85}{\blue 850}\;  =\dfrac{\cancel{85}\times1}{\cancel{85}\times10} =\dfrac{1}{10}$ & $\dfrac{85}{\blue 3} \; =\dfrac{5\times17}{3}$ \\
         \hline
      \end{tabular}}
   \end{center}
