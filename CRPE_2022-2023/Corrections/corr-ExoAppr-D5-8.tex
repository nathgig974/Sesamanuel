\ \\ [-5mm]
\begin{enumerate}
   \item Les dés sont équilibrés, nous sommes donc dans une situation d'équiprobabilité. \\
   \begin{enumerate}
      \item Avec le dé à six faces, la probabilité d'obtenir un 3 est $\mathcal{P}_1 =\dfrac16$ alors que pour le dé tétraédrique, elle est de $\mathcal{P}_2=\dfrac14$. Or, $\dfrac14>\dfrac16$ donc, \bm{il est plus probable d'obtenir un 3 avec le dé tétraédrique}.
      \item Avec le dé à six faces, la probabilité d'obtenir un multiple de 3 (donc 3 ou 6) est $\mathcal{P}_3 =\dfrac26 =\dfrac13$ alors que pour le dé tétraédrique, elle est de $\mathcal{P}_4 =\dfrac14$. Or, $\dfrac13>\dfrac14$ donc, \bm{il est plus probable d'obtenir un multiple de 3 avec le dé à six faces}.
      \item D'après l'arbre ci-dessous, on a 10 issues menant à un résultats favorable sur un total de 24 issues ($6\times4$). Donc, \bm{la probabilité d'obtenir avec le dé à 4 faces un nombre supérieur ou égal au nombre obtenu avec le dé à 6 faces est de $\dfrac{10}{24} =\dfrac{5}{12}$}.
   \end{enumerate}
   \smallskip
   \item
   \begin{enumerate}
      \item On a 12 issues menant à une somme paire sur un total de 24 issues possibles. \\
      Donc, \bm{la probabilité d'obtenir une somme paire est de $\dfrac{12}{24} =\dfrac12$}.
      \smallskip
      \item On a 3 issues menant à une somme inférieure ou égale à 3 sur une total de 24 issues possibles. \\
      Soit $24-3 =21$ issues menant à une somme supérieure strictement à 3, donc, \bm{la probabilité d'obtenir une somme strictement supérieures à 3 est de $\dfrac{21}{24} =\dfrac78$}. \\
   \end{enumerate}
\end{enumerate}
\bigskip
Arbre (non pondéré) modélisant la situation. \\
\bigskip
\hspace*{2.5cm}
     \pstree[treemode=R,nodesep=4pt,levelsep=5cm,treesep=0.3cm]{\Tp}{%
   \pstree{\TR{1}}{%
         \TR{1 \qquad S = 2} \TR{2 \qquad S = 3} \TR{3 \qquad S = 4}\TR{4 \qquad S = 5}}
   \pstree{\TR{2}}{%
         \TR{1 \qquad S = 3} \TR{2 \qquad S = 4} \TR{3 \qquad S = 5}\TR{4 \qquad S = 6}}
   \pstree{\TR{3}}{%
         \TR{1 \qquad S = 4} \TR{2 \qquad S = 5} \TR{3 \qquad S = 6}\TR{4 \qquad S = 7}}
   \pstree{\TR{4}}{%
         \TR{1 \qquad S = 5} \TR{2 \qquad S = 6} \TR{3 \qquad S = 7}\TR{4 \qquad S = 8}}
   \pstree{\TR{5}}{%
         \TR{1 \qquad S = 6} \TR{2 \qquad S = 7} \TR{3 \qquad S = 8}\TR{4 \qquad S = 9}}
   \pstree{\TR{6}}{%
         \TR{1 \qquad S = 7} \TR{2 \qquad S = 8} \TR{3 \qquad S = 9}\TR{4 \qquad S = 10}}}
