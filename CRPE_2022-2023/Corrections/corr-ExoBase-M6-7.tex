\ \\ [-5mm]
\begin{enumerate}
   \item On a $\pi\times\text{diamètre} =54\pi\text{ cm} \approx169,646\text{ cm}$. \\
   {\bf La circonférence de la roue vaut environ 169,6 cm.}
   \item
   \begin{enumerate}
      \item La voiture parcourt : \begin{tabular}[t]{rl}
         110 km & en une heure ; \\
         11\,000\,000 cm & en 3\,600 secondes ; \\ [1mm]
         $\dfrac{11\,000\,000}{3\,600}\text{ cm} =\dfrac{27\,500}{9}\text{ cm}$ & en une seconde. \\
      \end{tabular} \\ [1mm]
      Or, le périmètre de la roue fait 54$\pi$ cm donc en une seconde, on a $\dfrac{\dfrac{27\,500}{9}}{54\pi}\text{ tours} \approx18,01\text{ tours}$. \\
      {\bf La roue fait environ 18 tours par seconde.}
      \item En une seconde, la caméra fait 24 images et la roue 18 tours, donc entre deux images la roue fait \\
      $\dfrac{18}{24}$ tour = 0,75 tour. \\ [1mm]
      {\bf Le pneu aura fait 0,75 tour entre deux images.}
   \end{enumerate}
   \item Le nombre de tours est proportionnel à la vitesse, pour s'en convaincre, on peut d'écrire la formule du nombre de tours $T$ en fonction de la vitesse $v$ en km/h : \\ [1mm]
   $T =\dfrac{\dfrac{\text{vitesse en m/s}}{\text{périmètre en m}}}{\text{vitesse de défilement en images/s}} =\dfrac{\dfrac{\dfrac{100\,000\,v}{3\,600}}{54\pi}}{24} =\dfrac{100\,000\,v}{3\,600\times54\pi\times24} \approx 0,0068\,v$. \\ [2mm]
    La roue semble ne pas tourner lorsqu'elle fait un cinquième de tour complet puisque, dans l'énoncé, il est précisé que la roue possède cinq rayons (on considère qu'elle contient cinq rayons identiques et que de plus, les portions entre deux rayons sont aussi identiques, ce qui semble être un implicite de l'énoncé). \\
   On sait d'après la question précédente qu'une vitesse de 110 km/h correspond à 0,75 tour entre \\ [1mm] deux images, donc pour avoir 0,2 tour, on effectue le calcul $\dfrac{110\times0,2}{0,75} \approx29,3$. \\
   {\bf La roue semble immobile à une vitesse d'environ 29 km/h ou tout multiple de 29 km/h. cohérent avec la réalité.} \\ [5mm]
   {\it Remarque. Cette question peut paraître ambigüe : on peut considérer que la roue est immobile si elle fait un tour entier en imaginant, par exemple, des défauts ou des rayons non distribués de manière homogène. \\
      Dans ce cas, la roue semble ne pas tourner lorsqu'elle fait un tour complet, on sait d'après la question précédente qu'une vitesse de 110 km/h correspond à 0,75 tour entre deux images, donc pour avoir un tour complet, on effectue le calcul $\dfrac{110}{0,75} \approx146,7$. \\ [1mm]
      La roue semble immobile à une vitesse d'environ 147 km/h.}
\end{enumerate}
