\ \\ [-5mm]
\begin{enumerate}
   \item
   \begin{enumerate}
      \item On a successivement : \\
      \begin{itemize}
         \item $x =5$ ;
         \item Étape 1 = $6\times x =6\times5 =30$ ;
         \item Étape 2 = Étape 1 $+10 =30+10 =40$ ;
         \item Résultat = Étape 2$\div2 =40\div2 =20$.
      \end{itemize}
      \bm{En choisissant 5 comme nombre de départ, on obtient 20.}
      \item On a successivement : \\
      \begin{itemize}
         \item $x =7$ ;
         \item Étape 1 = $6\times x =6\times7 =42$ ;
         \item Étape 2 = Étape 1 $+10 =42+10 =52$ ;
         \item Résultat = Étape 2$\div2 =52\div2 =26$.
      \end{itemize}
      \bm{En choisissant 7 comme nombre de départ, on obtient 26.}
   \end{enumerate}
   \item En remontant les étapes du programme, on obtient successivement :
   \begin{itemize}
      \item Résultat = 8 ;
      \item Étape 2 = Résultat$\times2 =8\times2 =16$ ;
      \item Étape 1 = Étape 2 $-10 =16-10 =6$ ;
      \item $x =$ Étape 1$\div6 =6\div6 =1$.
   \end{itemize}
   \bm{Pour obtenir 8 comme résultat final, le nombre de départ doit être 1.}
   \item On a successivement : \\
   \begin{itemize}
      \item $x$ ;
      \item Étape 1 = $6\times x =6x$ ;
      \item Étape 2 = Étape 1 $+10 =6x+10$ ;
      \item Résultat = Étape 2$\div2 =(6x+10)\div2 =3x+5$.
   \end{itemize}
   \bm{Si $x$ est le nombre de départ, le résultat obtenu est $3x+5$.}
   \item Le programme de Maxime donne le résultat $5(x+2) =5x+10$ pour tout nombre $x$ choisi. \\
   On résout donc l'équation : $5x+10 =3x+5 \iff 5x-3x =5-10 \iff 2x =-5 \iff x =-\dfrac52$. \\
   \bm{Si Julie et Maxime choisissent le nombre $-2,5$, ils obtiennent le même résultat.}
\end{enumerate}
