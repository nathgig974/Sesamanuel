\ \\ [-5mm]
\begin{enumerate}
   \item
   \begin{enumerate}
      \item En saisissant 10, on obtient successivement :
      \begin{itemize}
         \item $x =10$ ;
         \item A $= x-8 =10-8 =2$ ;
         \item B = A$\times3 =2\times3 =6$ ;
         \item C =  B $+24 =6+24 =30$ ;
         \item D = C $+x =30+10 =40$.
      \end{itemize}
      Donc, {\blue si le nombre choisi et 10, le résultat affiché est 40}.
      \item En saisissant $-2$, on obtient successivement :
      \begin{itemize}
         \item $x =-2$ ;
         \item A $= x-8 =-2-8 =-10$ ;
         \item B = A$\times3 =-10\times3 =-30$ ;
         \item C = B $+24 =-30+24 =-6$ ;
         \item D = C $+x =-6-2 =-8$.
      \end{itemize}
      Donc, {\blue si le nombre choisi et $-2$, le résultat affiché est $-8$}.
   \end{enumerate}
   \item On obtient successivement :
      \begin{itemize}
         \item $x$ ;
         \item A $= x-8$ ;
         \item B = A$\times3 =(x-8)\times3 =3x-24$ ;
         \item C = B $+24 =3x-24+24 =3x$ ;
         \item D = C $+x =3x+x =4x$.
      \end{itemize}
      Donc, {\blue il faut mettre l'expression $4x$ dans le carré blanc}.
\end{enumerate}
