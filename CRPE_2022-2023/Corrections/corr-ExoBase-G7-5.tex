\ \\ [-5mm]
\begin{enumerate}
   \item \textcolor{A2}{$\bullet$} programmer le déplacement d’un personnage sur un plan ;
   \begin{itemize}
      \item se repérer dans l'espace ;
      \item résoudre un problème utilisant un repérage de l'espace.
   \end{itemize}
   \item
   \begin{enumerate}
      \item {\bf Oriane} : elle a trouvé 6 quilles ce qui est le maximum de quilles possibles de récupérer en 20 secondes. Son chemin semble correct dans sa tête, cohérent au niveau des instructions \og av \fg{}, \og rq \fg{} et le temps de 20 secondes est respecté mais les instructions d'orientation (droite et gauche) sont souvent fausses. \\
      {\bf Samuel} : il donne un premier programme juste permettant de récupérer 2 quilles (ligne 1). Il maîtrise sur cet exemple les diverses instructions de déplacement et d'orientation. Il semble ensuite qu'il recommence du point de départ et construit son déplacement sans anticipation, ce qui lui donne in fine le même parcours que précédemment. Enfin, il tente un troisième parcours pour lequel on ne sait pas vraiment d'où il part ni ce qu'il fait. Il a utilisé plus de 20 déplacements et une nouvelle instruction : \og av4 \fg.
      \item {\bf Oriane} a des difficultés d'orientation : elle mélange les repérages absolus et relatifs, c'est donc vers cette remédiation qu'il faut se tourner, on peut lui proposer par exemple :
      \begin{itemize}
         \item une activité déconnectée réelle en lui faisant faire les déplacements de manière effective sur sur un grand quadrillage dans la cours de l'école (jeu du robot idiot) ;
        \item une activité sur ordinateur ou tablette afin de percevoir l'effet d'une erreur d'instruction en direct (ScratchJr, Scratch, Lightbot, Gcompris labyrinthe\dots).
     \end{itemize}
     {\bf Samuel} semble avoir des difficultés d'anticipation, on peut lui proposer de tracer le déplacement qu'il souhaite obtenir sur le quadrillage, puis de coder ce déplacement tout en dénombrant le temps écoulé et le nombre de quilles récupérées.
   \end{enumerate}
\end{enumerate}
