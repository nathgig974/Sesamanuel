{\bf SITUATION 1 :} \\
\begin{enumerate}
   \item {\it Voir tableau de comparaison page suivante.} \\
   On peut évoquer plusieurs raisons à la non réussite de l'élèves à la question \ding{204} :
   \begin{itemize}
      \item dans la question \ding{203}, les fractions peuvent être classées facilement dans l'ordre croissant puisqu'elles ont le même dénominateur. Il suffit ensuite de les mettre dans le même ordre sur l'axe gradué étant donné qu'il y a autant de fractions que de lettres. Dans la question \ding{204}, la question est beaucoup plus ouverte ;
      \item la reproduction de la droite elle-même sur du papier millimétré dans la question \ding{204} peut poser problème, d'autant plus que l'origine n'est pas représentée. Dans la question \ding{203}, la droite graduée est déjà tracée ;
      \item dans la question \ding{204}, l'élève doit constamment \og jongler \fg{} entre les différentes écritures, alors que la question \ding{203} ne comprend qu'une sorte d'écriture ;
      \item les deux représentations de 3 et 4 dans l'exercice \ding{204} possèdent des dénominateurs différents, ce qui peut perturber l'élève ;
      \item dans l'exercice \ding{204}, les graduations des centièmes ne sont pas clairement marquées, il faut se servir du papier millimétré comportant de multiples graduations alors que dans la question \ding{203}, les graduations sont claires.
   \end{itemize}
   \item Un nombre décimal est un nombre pouvant s'écrire sous la forme d'une fraction décimale, c'est à dire une fraction dont le dénominateur est une puissance de 10 : 1, 10, 100, 1000\dots \\
\end{enumerate}

\Coupe
   \begin{Ltableau}{1\linewidth}{2}{p{7.65cm}|p{7.65cm}}
      \hline
      Question \ding{203} & Question \ding{204} \\
      \hline
      \multicolumn{2}{|c|}{Comparaison de la présentation} \\
      \hdashline
      Un droite graduée est tracée, les graduations jusqu'au centième bien représentées. \newline
      La droite comporte l'origine et l'unité. \newline
      Des lettres indiquent des fractions à placer, il y a autant de lettres que de fractions. \newline
      Les nombres à placer sont tous écrits sous la forme d'un fraction décimale de dénominateur 100.
      & La droite doit être reproduite sur papier millimétré. \newline
      \newline
      L'origine n'est pas présente, le premier entier visible est 3. \newline
      3 est associé à $\dfrac{30}{10}$ et 4 à $\dfrac{400}{100}$. \newline
      Les nombres à placer sont des nombres décimaux écrits de différentes façons. \\
      \hline
      \multicolumn{2}{|c|}{Comparaison de la tâche demandée} \\
      \hdashline
      Il s'agit d'associer une fraction à une lettre de la droite graduée. L'élève doit ensuite donner l'écriture décimale de chacune des fractions.
      & Il faut placer des fractions après avoir reproduit la droite. Il n'y a rien à faire après avoir placé les nombres. \\
      \hline
   \end{Ltableau}

{\bf SITUATION 2 }: \\
\begin{enumerate}
   \item {\bf Lara} n'obtient pas le bon dénominateur : elle écrit $1\,000$ au lieu de 100. Cela provient peut-être du fait du codage usuel de $a+\dfrac{b}{10}+\dfrac{c}{100} =\overline{a,bc}$ qu'elle a vu en classe, le $1\,000$ provenant de la suite $1 ; 10 ; 100 ; 1\,000$ ou de la multiplication de $10$ par $100$ ? Ensuite elle se trompe en enlevant la virgule. Il s'agit peut-être d'une manière implicite de comparer les parties décimales \og comme si un nombre décimal était composé de deux nombres entiers séparés par une virgule \fg. Sa conclusion ne justifie pas pourquoi Max à tort.
   \item {\bf Clément} semble avoir acquis la compétence \og passer du développement en factions décimales d'un nombre à son écriture décimale \fg.
   \item Elle utilise la règle suivante : pour comparer deux nombres décimaux, on compare tout d'abord les parties entières. Si l'une est plus grande que l'autre, il en est de même pour le nombre décimal. Si les parties entières sont égales, on compare le chiffre des dixièmes : le nombre le plus grand est celui qui a le plus grand chiffre des dixièmes. On continue ainsi de suite pour chacun des rangs successifs.
\end{enumerate}
