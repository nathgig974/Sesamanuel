\ \\ [-5mm]
   \begin{enumerate}
      \item
         \begin{itemize}
            \item Calcul du volume d'aluminium nécessaire pour fabriquer le corps de la canette : \\
               $130\,\mu\um{} =130\times10^{-6}\um{} =130\times10^{-4}\,\ucm{} =\ucm{0,013}$. \\
               $\mathcal{V} =\text{surface}\times\text{épaisseur} =\ucmq{268,42}\times\ucm{0,013} \approx\ucmc{3,49}$. \\
            \item Calcul de la masse d'aluminium nécéssaire pour fabrique le corps de la canette : \\ [1mm]
               la masse volumique de l'aluminium vaut $\mu =\dfrac{\ukg{2700}}{\umc{1}} =\dfrac{\ug{2700000}}{\ucmc{1000000}} =2,7\text{ g/cm}^3$. \\ [1mm]
               Un volume de \ucmc{3,49} d'aluminium a une masse $m$ en gramme égale à $3,49\times2,7 \approx9,423$.
            \item Calcul de la masse totale d'aluminium nécéssaire pour la canette entière : \\
               $M =$ masse du corps + masse de l'anneau + masse de soudure \\
               $M \approx\ug{9,4}+\ug{1,4}+\ug{1,9} \approx\ug{12,7}$. \\
               {\blue Il faut environ \ug{12,7} d'aluminium pour fabriquer une canette classique}.
         \end{itemize}
      \item Pour un vélo de \ukg{9}, soit \ug{9000}, on effectue le calcul suivant : \\
         $\ug{9000}\div\ug{12,7} \approx 708,66$. \\
         {\blue Il faudrait environ 709 canettes classiques pour fabriquer ce type de vélo}.
   \end{enumerate}
