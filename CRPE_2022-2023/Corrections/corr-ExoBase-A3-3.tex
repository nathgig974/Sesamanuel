Procédons étape par étape :
   \begin{itemize}
      \item Marie-Capucine a précisément l'âge médian, 7 ans :
      \begin{tabular}{|*{7}{C{0.4}|}}
         \hline
         \textcolor{white}{7} & \textcolor{white}{7} & \textcolor{white}{7} & 7 & \textcolor{white}{7} & \textcolor{white}{7} & \textcolor{white}{7} \\
         \hline
      \end{tabular}
      \smallskip
      \item Les jumeaux ont 8 ans :
       \begin{tabular}{|*{7}{C{0.4}|}}
         \hline
         \textcolor{white}{7} & \textcolor{white}{7} & \textcolor{white}{7} & 7 & 8 & 8 & \textcolor{white}{7} \\
         \hline
      \end{tabular}
      \smallskip
      \item l'âge modal est 5 ans, c'est l'âge le plus représenté :
       \begin{tabular}{|*{7}{C{0.4}|}}
         \hline
         5 & 5 & 5 & 7 & 8 & 8 & \textcolor{white}{7} \\
         \hline
      \end{tabular}
      \smallskip
      \item l'âge moyen est de 8 ans. Soit $a$ l'âge de l'aîné, on a : \\ [1mm]
      $\dfrac{5+5+5+7+8+8+a}{7} =8 \iff \dfrac{38+a}{7} =8 \iff 38+a =8\times7 \iff a =56-38 \iff a =18$. \\ [1mm]
    \end{itemize}
   \bm{Mon aîné à 18 ans.} \\
