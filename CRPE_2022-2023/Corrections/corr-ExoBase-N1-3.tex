\ \\ [-5mm]
   \begin{enumerate}
      \item Il s'agit d'un système {\blue positionnel de base 3}.
      \item Le premier nombre ayant quatre symboles est \ding{115}\,\ding{72}\,\ding{72}\,\ding{72} qui correspond à $1\times3^3 ={\blue 27}$.
      \item \ding{108}\,\ding{115}\,\ding{72}\,\ding{115}\,\ding{72} correspond au nombre $2\times3^4+1\times3^3+0\times3^2+1\times3^1+0\times3^0 ={\blue 192}$.
      \item On commence par écrire 143 en base 3, par exemple grâce aux divisions euclidiennes successives : \\
         $\opidiv[remainderstyle.2=\blue]{143}{3} \qquad \opidiv[remainderstyle.2=\blue]{47}{3} \qquad \opidiv[remainderstyle=\blue]{15}{3} \qquad \opidiv[remainderstyle=\blue]{5}{3} \qquad \opidiv[remainderstyle=\blue]{1}{3}$ \\ [1mm]
      On a donc $143 =\overline{12022}^3$ ce qui correspond à {\blue \ding{115}\,\ding{108}\,\ding{72}\,\ding{108}\,\ding{108}} en numération trimontoise.
      \item On peut effectuer la soustraction suivante : \begin{tabular}[t]{lllll}
         & & & \footnotesize\ding{115} & \footnotesize\ding{115}\ding{115}\ding{115} \\
         & \ding{115} & \ding{72} & \cancel{\ding{108}} & \ding{72} \\ [1mm]
         $-$ & & & & \ding{115} \\ [1mm]
         \cline{2-5}
         & \ding{115} & \ding{72} & \ding{115} & \ding{108} \\
      \end{tabular} \\ [1mm]
      Donc, le nombre qui vient juste avant \ding{115}\,\ding{72}\,\ding{108}\,\ding{72} est {\blue \ding{115}\,\ding{72}\,\ding{115}\,\ding{108}}
   \end{enumerate}
