\ \\ [-5mm]
   \begin{enumerate}
      \item La tente 1 est une {\blue pyramide} à base carrée, la tente 2 est un {\blue cube}, la tente 3 est un {\blue prisme} à base hexagonale et la tente 4 est un {\blue cône.}
      \item
         \begin{enumerate}
            \item La tente 2 est un cube, sa hauteur est {\blue $c =\um{1,9}$} et la tente 3 est un prisme de hauteur {\blue $a =\um{1,8}$}.
            \item On appelle $S$ le sommet du cône et $A$ un point du cercle de base, dans le triangle $SOA$ rectangle en $O$ (la hauteur $h$ est perpendiculaire à la base), on a : $\tan(\widehat{OSA}) =\dfrac{OA}{h}$ d'où $h =\dfrac{r}{\tan(\alpha)} =\dfrac{\um{1,5}}{\tan{\udeg{35}}} \approx\um{2,14}$. \\ [1mm]
               La tente 4 a pour {\blue hauteur \um{2,14}} environ.
            \item On appelle $S$ le sommet de la pyramide et $A$ un sommet du carré de base, la diagonale du carré a pour mesure $c\sqrt{2}\,\um{}$ et donc sa demi-diagonale mesure $c\sqrt{2}/2\,\um{}$. Dans le triangle $SOA$ rectangle en $O$ (la hauteur est perpendiculaire à la base), on a d'après le théorème de Pythagore, avec des mesure en \um{} : \\
               $SA^2 =SO^2+OA^2 \iff \ell^2 =h^2+\left(\dfrac{c\sqrt{2}}{2}\right)^2 \iff h^2 =3,2^2-(1,85\sqrt2)^2 \iff h^2 =3,395$. \\ [1mm]
               La tente 1 a pour {\blue hauteur \um{1,84}} environ.
         \end{enumerate}
      \setcounter{enumi}{2}
      \item
         \begin{itemize}
            \item $\mathcal{V}_{\text{tente 1}} =\dfrac13\times\mathcal{A}_{\text{base}}\times h=(\um{3,7})^2\times\um{1,84} \approx\umc{8,40} \approx{\blue\umc{8}}$. \smallskip
            \item $\mathcal{V}_{\text{tente 2}} =c^3=(\um{1,9})^3 \approx\umc{6,86} \approx{\blue \umc{7}}$.
            \item $\mathcal{V}_{\text{tente 3}} =\mathcal{A}_{\text{base}}\times a=6\sqrt3\,\umq{}\times\um{1,8} \approx\umc{18,71} \approx{\blue \umc{19}}$. \smallskip
            \item $\mathcal{V}_{\text{tente 4}} =\dfrac13\times\mathcal{A}_{\text{base}}\times h=\pi\times(\um{1,5})^2\times\um{2,14} \approx\umc{5,04} \approx{\blue \umc{5}}$. \smallskip
         \end{itemize}
      \item M. Mathrice était hébergé dans la {\blue tente 4}, la seule ayant une hauteur supérieure à sa taille.
   \end{enumerate}
