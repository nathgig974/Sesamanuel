\ \\ [-5mm]
   \begin{enumerate}
      \item $m =\dfrac{4\times0+6\times0,3+4\times1,3+4\times1,7+3\times2,5+3\times7+2\times13+1\times21+2\times28+1\times42}{4+6+4+4+3+3+1+1+2+1}$ \\ [2mm]
         $m =\dfrac{187,3}{30} \approx6,2433$. \\ [2mm]
         {\blue Il a plu en moyenne environ \umm{6,2} au mois d'avril 2016}.
      \item Il y a 30 valeurs, il faut donc prendre un nombre entre la 15\up{e} valeur (qui est \umm{1,7} et la 16\up{e} valeur (qui est également \umm{1,7} donc, {\blue la médiane vaut \umm{1,7}}. \\
         La moitié des précipitations journalière sont inférieures ou égales à \umm{1,7} et la moitié des précipitations journalières sont supérieures ou égales à \umm{1,7}.
      \item $42-0 =42$ donc, {\blue l'étendue vaut \umm{42}}.
      \item La hauteur des précipitations est supérieure ou égale à \umm{13} {\blue pendant 6 jours}, ce qui correspond à un \\ [1mm]
         pourcentage de $\dfrac{6}{30}\times100 ={\blue 20\,\%}$. \\ [1mm]
      \item Il a plu \umm{187,3} d'eau durant le mois d'avril 2016. Ceci sur une surface totale de \um{3200} par \um{50}, soit $\um{3200}\times\um{50} =\umq{160000}$. \\
         Ce qui donne un volume de $\umq{160000}\times\um{0,1873} =\umc{29968}$. \\
         Or, pour transformer un volume en capacité, on utilise généralement la correspondance suivante :   \udmc{1} =\ul{1}. \\
         Soit $\umc{29968} =\udmc{29968000} \approx \ul{30000000}$. \\
         {\blue Au cours du mois d'avril, il est tombé \udmc{29968} soit près de \ul{30000000} d'eau sur l'aéroport de Rolland Garros}. \\ [5mm]
   \end{enumerate}
