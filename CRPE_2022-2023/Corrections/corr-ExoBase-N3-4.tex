\ \\ [-5mm]
   \begin{enumerate}
      \item
      \begin{enumerate}
         \item $(x+1)(x-1) =x^2-1$ et $(x+2)(x-2) =x^2-4$ donc, \\
            $(x+1)(x-1)-(x+2)(x-2) =(x^2-1)-(x^2-4)$ \\
            \hspace*{4.35cm} $=x^2-1-x^2+4 ={\blue 3}$.
         \item pour calculer $297\times295-298\times294$, on utilise le résultat précédent appliqué à $x =296$ et on trouve {\blue $297\times295-298\times294 =3$}.
      \end{enumerate}
      \setcounter{enumi}{1}
      \item Si l'on observe les quatre résultats, il semble que la proposition 1 soit exacte, prouvons-le : \\
         $a$ et $b$ sont deux nombres consécutifs (supposons $a<b$), donc $b =a+1$. \\
         Leur somme est égale à $a+b =a+(a+1) =2a+1$ ; \\
         la différence de leurs carrés est égal à $b^2-a^2 =(a+1)^2-a^2 =a^2+2a+1-a^2 =2a+1$, \\
            la proposition est donc vérifiée. \\
           {\blue Si $a$ et $b$ sont deux nombres consécutifs, alors leur somme est égale à la différence de leurs carrés}.
   \end{enumerate}
