\ \\ [-5mm]
\begin{enumerate}
   \item {\bf Production de Nicolas.}
   \begin{itemize}
      \item Démarche utilisée : somme des parties entières, somme des parties décimales sous forme de fractions décimales puis transformation en écriture décimale.
      \item Compétence acquise : calcul du périmètre d'un triangle (domaine des grandeurs et mesures).
      \item Erreurs : erreur dans le calcul de la somme des nombres entiers (16 unités au lieu de 15 unités), erreur dans le calcul de la somme des fractions décimales (11 dixièmes au lieu de 12 dixièmes). Erreur de codage : pour 11 dixièmes, il a ajouté une unité et un centième.
   \end{itemize}
   \smallskip
    {\bf Production de Thomas.}
   \begin{itemize}
       \item Démarche utilisée : somme des parties entières, somme des parties décimales sous forme de fractions décimales puis écriture sous la forme unités et dixièmes.
      \item Compétences acquises : calcul du périmètre d'un triangle (domaine des grandeurs et mesures), somme de nombres entiers et somme de fractions décimales (domaine du calcul).
      \item Erreur : l'écriture n'est pas optimisée, Thomas n'a pas transformé son résultat en une écriture décimale, il n'a pas \og vu \fg{} que 12 dixièmes est égal à une unité et 2 dixièmes.
   \end{itemize}
   \smallskip
   {\bf Production d'Amina.}
   \begin{itemize}
       \item Démarche utilisée : somme des parties entières de proche en proche, somme des parties décimales sous forme de fractions décimales de proche en proche puis écriture sous la forme unités et dixièmes et enfin codage sous forme décimale.
      \item Compétences acquises : calcul du périmètre d'un triangle (domaine des grandeurs et mesures), somme de nombres entiers et somme de fractions décimales (domaine du calcul), codage d'un nombre décimal (domaine des nombres).
      \item Erreur : erreur d'écriture mathématique dans ses sommes (statut du signe égal à revoir).
   \end{itemize}
   \item L'enseignant peut demander à Thomas s'il peut décomposer 12 dixièmes pour l'associer à ses 15 unités, ou lui demander de placer sur une droite graduée son nombre obtenu.
\end{enumerate}
