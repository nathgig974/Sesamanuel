\ \\ [-5mm]
   \begin{enumerate}
      \item Le salaire minimal est \ueuro{1488,11} et le salaire maximal est \ueuro{2192,48}. Or, $2\,192,48-1\,488,11 =704,37$ donc, {\blue l'étendue de cette série est de \ueuro{704,37}}.
      \item La série comporte 13 valeurs, le salaire médian est donc le 7\up{e} lorsqu'on les ordonne, par exemple du plus petit au plus grand. On a : $1\,488,11 < 1\,539,45 <1\,593,38 < 1\,593,38 < 1\,864,37 < 1\,864,37 < \fbox{1\,864,37}\dots$ \\
         {\blue Le salaire médian est \ueuro{1\,864,37}}. \smallskip
      \item $\dfrac{1\,938,36+1\,488,11+1\,994,38+2\,048,37+2\,192,48+1\,998,93+1\,539,45+1\,948,37+3\times1\,864,37+2\times1\,593,38}{13}$ \\
         $=\dfrac{23\,928,32}{13} =1\,840,64$. Donc, {\blue Le salaire moyen est de $\ueuro{1840,64}$}.
      \item Coût global d'un salarié de l'emballage et de l'organisation des livraisons : $\dfrac{\ueuro{1864,37}}{0,78}\times1,45 \approx\ueuro{3465,82}$. \\
         {\blue Le coût d'un tel salarié pour l'entreprise est de \ueuro{3465,82}}.
      \item
         \begin{enumerate}
            \item Une augmentation de 3\,\% correspond à un coefficient multiplicateur de $1+\dfrac{3}{100} =1,03$. \\ [1mm]
               $\ueuro{1488,11}\times1,03 \approx\ueuro{1532,75}$. {\blue Le salaire net après augmentation sera d'environ \ueuro{1532,75}}. \smallskip
            \item Coût global de ce salarié après augmentation : $\dfrac{\ueuro{1532,75}}{0,78}\times1,45 \approx\ueuro{2849,34}$. \\ [1mm]
              {\blue Le coût d'un tel salarié après augmentation est de \ueuro{2849,34}}. \smallskip
           \item Coût global de ce salarié avant augmentation : $\dfrac{\ueuro{1488,11}}{0,78}\times1,45 \approx\ueuro{2766,36}$. \\ [1mm]
              Pourcentage d'augmentation : $\dfrac{\ueuro{2849,34}-\ueuro{2766,36}}{\ueuro{2766,36}}\times100 \approx 3\,\%$. {\blue Le coût a augmenté d'environ 3\,\%}.
        \end{enumerate}
  \end{enumerate}
