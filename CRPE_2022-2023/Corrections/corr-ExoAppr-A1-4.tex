   On peut matérialiser la situation par un arbre pondéré : \\ [5mm]
\begin{minipage}{8cm}
\textcolor{B2}{sec}\pstree[treemode=R,nodesep=3pt,levelsep=3cm,treesep=1.8cm]{\Tp}{%
      \pstree{\TR{sec}\naput{$\frac56$}}{%
         \TR{sec}\naput{$\frac56$} \TR{\textcolor{B2}{humide}}\nbput{$\frac16$}}
      \pstree{\TR{humide}\nbput{$\frac16$}}{%
         \TR{sec}\naput{$\frac13$} \TR{\textcolor{B2}{humide}}\nbput{$\frac23$}}}
\end{minipage}
\begin{minipage}{8cm}
 Dans un arbre pondéré, la probabilité d'une issue est calculée en multipliant les probabilités de chaque éventualité. On obtient deux branches, donc \\ [1mm]
$\mathcal{P} =\dfrac56\times\dfrac16+\dfrac16\times\dfrac23 =\dfrac{5}{36}+\dfrac{2}{18}$ \\ [1mm]
 $\phantom{\mathcal{P}} =\dfrac{5}{36}+\dfrac{4}{36} =\dfrac{9}{36} =\dfrac14$. \\ [5mm]
   On trouve bien une probabilité de $\dfrac14 =0,25$ donc, \\
   \bm{l'affirmation est vraie.}
\end{minipage}
