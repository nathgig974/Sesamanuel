\ \\ [-5mm]
\begin{enumerate}
   \item Pour {\bf Adama}, il s'agit de la division euclidienne usuelle sans pose des soustractions intermédiaires. \\
   Un avantage est un gain de temps si l'élève est expert en calcul mental ainsi qu'un gain de place. \\
   Un inconvénient est sa complexité : il faut effectuer divers calculs mentalement, elle est donc source d'erreurs et de surcharge cognitive. \\
   Pour {\bf Anaïs}, il s'agit d'une procédure avec écriture de la table de multiplication du diviseur et de résultats intermédiaires reprenant à chaque étape la totalité du dividende. \\
   Quelques avantages : elle donne du sens à l'opération et elle évite la surcharge cognitive en ayant écrit les multiples de 37. De plus, un quotient non optimal lors d'une étape peut être rattrapé lors des étapes suivantes. \\
   Un inconvénient est la lourdeur de l'écriture.
   \item {\bf Marie} se trompe au niveau du quotient : elle oublie de mettre un \og zéro \fg{} entre le 1 et le 4. Cette erreur vient probablement du fait que dans 17, elle ne peut pas mettre 37, elle abaisse alors le chiffre suivant, sans penser à caractériser cette impossibilité par un 0 qui correspond au nombre de centaines du quotient. \\
   {\bf Kévin} pose son opération en effectuant les soustractions intermédiaires. Il ne calcule pas à l'avance les multiplications et lorsqu'il se trompe, il barre sa réponse et reprend ensuite jusqu'à trouver une solution convenable. Lors de sa dernière étape, le calcul de $9\times37$ donne un résultat trop grand pour être soustrait, il tente donc la multiplication par 7 ce qui fonctionne. Cependant, il ne s'est pas aperçu que le reste obtenu est  supérieur au diviseur. La longueur de l'opération est peut-être en cause, et le calcul du répertoire multiplicatif de 37 aurait pu l'aider à calculer cette division de manière moins aléatoire.
\end{enumerate}
