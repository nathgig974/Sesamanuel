\ \\ [-2.5mm]
\begin{enumerate}
   \item
   \begin{LCtableau}{0.7\linewidth}{7}{c}
      \hline
      Récipient & R1 & R2 & R3 & R4 & R5 & R6 \\
      \hline
      Jauge & E & C & F & A & B & D \\
      \hline
      Courbe & 2 & 1 & 5 & 4 & 6 & 3 \\
      \hline
   \end{LCtableau}
   \item Le volume d'un cylindre de hauteur $h$ et de rayon de base $r$ est : $\mathcal{V} =\pi\times r^2\times h$. \\
   Le diamètre de R2 est de 16 cm, donc son rayon vaut 8 cm ; \\
   le volume est de 10 litres, soit \udmc{10}, ou encore \ucmc{10000}. On a alors : \\
   $\ucmc{10000} =\pi\times 8^2\times \Ucmc{h} \iff h =\dfrac{\ucmc{10000}}{\Ucmq{64\pi}} \approx\ucm{49,74}$. \\ [1mm]
   \bm{Le récipient R2 mesure environ 50 centimètres de haut.}
   \item Le cylindre R2 se remplit de manière homogène, donc, $\mathcal{V}' =\dfrac23\times\mathcal{V} =\dfrac23\times10\,\text{L} \approx 6,67$ L. \\
   \bm{À l'instant $t$, le récipient R2 contient environ 6,7 litres d'eau.}
   \item La courbe correspondant à R2 est la courbe 1 : c'est une fonction linéaire puisque le débit est constant. \\
   Sur l'axe des abscisses, 15 carreaux représentent le temps pour remplir les réservoirs en entier, donc 10 carreaux représentent les deux tiers du temps $\left(\dfrac23\times15 =10\right)$. \\
   À cet endroit, la courbe représentative de R2 est au dessus de la courbe représentative de R6 d'où : \\
   \bm{aux deux tiers du temps, la hauteur d'eau dans le récipient R2 est supérieure à celle du récipient R6.}
\end{enumerate}
