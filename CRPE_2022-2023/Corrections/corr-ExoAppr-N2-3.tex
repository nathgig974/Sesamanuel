\ \\ [-5mm]
\begin{enumerate}
   \item La notion abordée est la soustraction, l'élève doit effectuer une soustraction par la méthode de son choix (développer l'algorithme) : par un calcul en ligne ou par un calcul en colonnes (soustraction classique, par emprunts ou comme addition à trous), et doit avoir des connaissances en calcul mental (répertoire additif).  \item
   \begin{enumerate}
      \item {\bf Antoine} utilise une procédure de calcul en ligne par soustractions successives. Il part du nombre le plus grand et décompose le nombre à retirer : il commence par enlever le plus grand multiple de 100 s'il existe, puis le plus grand multiple de 10 et enfin les unités, sauf pour le dernier calcul. Ensuite, il écrit le calcul en ligne, les résultats sont justes mais mal écrits d'un point de vu mathématique (statut du signe \og = \fg).\\
      {\bf Barbara} effectue des soustractions posées en colonnes grâce à la méthode par emprunts. \\
      {\bf Clara} effectue des calculs en ligne, elle soustrait pour chaque rang le plus petit nombre au plus grand, en commençant par le chiffre le plus à gauche (on le devine grâce à la dernière opération).
      \item {\bf Barbara} utilise la méthode par emprunts : si besoin, elle emprunte une unité au rang supérieur, qu'elle donne au rang actuel en cassant cette unité en dix. \\
      {\bf Dominique} effectue des soustractions par la méthode classique basée sur le principe des différences constantes : lorsque la valeur du chiffre du haut est inférieure à celle du chiffre du bas, il ajoute 10 unités du rang en haut et une unité du rang supérieur en bas.
      \item {\bf Barbara} effectue correctement les opérations a., b. et d. Par contre, elle se trompe dans l'opération c., erreur classique de la méthode lorsque le nombre le plus grand comporte un ou plusieurs \og 0 \fg. Elle a besoin d'emprunter une dizaine, mais 800 ne comporte pas de chiffre des dizaines, il faut donc emprunter une centaine, il en reste bien 7 et on a 10 dizaines. Elle semble ensuite reporter un \og 1 \fg{} au lieu d'ôter la dizaine du départ. Elle effectue donc $11-5$ au lieu de $9-5$. \\
      {\bf Clara} n'a aucun résultat de juste car sa procédure n'a aucun sens.
      \item Pour Clara, il faut revoir la soustraction posée au niveau de son sens. Pour cela, on peut revenir aux bases et conceptualiser la soustraction grâce à un matériel pédagogique (pièces et billets par exemple).
   \end{enumerate}
\end{enumerate}
