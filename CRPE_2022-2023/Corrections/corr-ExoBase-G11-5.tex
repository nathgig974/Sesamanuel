\ \\ [-5mm]
   \begin{enumerate}
      \item Pour déterminer la pente $p$ du câble égale à $\dfrac{\text{BB'}}{\text{AB}}$, on peut utiliser par exemple des formules de \\ [1mm]
         trigonométrie : $\sin(\alpha) =\dfrac{\text{BB'}}{\text{AB}} =\dfrac{\um{2620}-\um{2100}}{\um{2480}} =\dfrac{\um{520}}{\um{2480}} \iff \alpha =\arcsin\left(\dfrac{520}{2\,480}\right) \approx\udeg{12,10}$. \\ [1mm]
         La pente est donc égale à $\dfrac{\text{BB'}}{\text{AB'}} =\tan(\widehat{\alpha}) =\tan(\udeg{12,10}) \approx0,2144$. \\ [1mm]
         {\blue La pente du câble est de 21,44\,\% environ}.
      \item
      \begin{enumerate}
         \item Dans la triangle ACC' rectangle en C', on a : $\sin(\alpha) =\dfrac{\text{CC'}}{\text{AC}} \iff \sin(\udeg{12,10}) =\dfrac{\text{CC'}}{\um{2000}}$ \\ [1mm]
         $\iff \text{CC'} =\um{2000}\times\sin(\udeg{12,10}) \approx\um{419,24}$. \\
         {\blue La longueur du segment [CC'] est de 419 mètres environ}.
         \item Le point C' est à l'altitude de départ, c'est-à-dire \um{2100}. Or, \um{2100} + \um{419} = \um{2519} ; donc, \\
         {\blue le point C est à une altitude de \um{2519} environ.}
      \end{enumerate}
      \setcounter{enumi}{2}
      \item
      \begin{enumerate}
         \item On a AC $=\um{2480}-\um{480} =\um{2000}$ et E est le milieu de [AC], donc EC $=\um{2\,000}\div2 =\um{1000}$. \\
         {\blue EC = 1000 m.}
         \item Soit $t$ le temps mis par la cabine pour aller de E à C, on utilise la formule : \\ [1mm]
         $v =\dfrac{d}{t} \iff \ums{5} =\dfrac{\um{1000}}{t}$ \\
         \hspace*{0.9cm} $\iff t = \dfrac{\um{1000}}{\ums{5}}$ \\ [1.5mm]
         \hspace*{0.9cm} $\iff t  =\us{200}$. \\
         Or, $\us{200} =\us{180}+\us{20}$, donc, \\
         {\blue il faudra 3 minutes et 20 secondes à la cabine pour parcourir la distance EC}.
      \end{enumerate}
   \end{enumerate}
