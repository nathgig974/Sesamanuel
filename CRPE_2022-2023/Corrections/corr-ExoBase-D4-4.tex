\ \\ [-5mm]
   \begin{enumerate}
      \item La médiane correspond à un âge compris entre le 400\up{e} et le 401\up{e} âge classés dans l'ordre croissant. \\
         Il y a 243 chefs de moins de 45 ans ($11+84+148$), donc moins de 400 et 297 de plus de 55 ans, donc moins de 400 également. Par conséquent, {\blue la médiane se situe bien dans l'intervalle [ 45 ; 55 [.}
      \item Il y a 243 chefs de moins de 45 ans. Il faut donc déterminer la 157\up{e} valeur ($400-243$) et la 158\up{e} valeur ($401-243$) du tableau. \\ \smallskip
         \qquad
         \begin{LCtableau}{0.8\linewidth}{11}{c}
            \hline
            âge & 45 & 46 & 47 & 48 & 49 & 50 & 51 & 52 & 53 & 54 \\
            \hline
            effectif & 18 & 21 & 24 & 31 & 30 & 31 & 30 & 27 & 28 & 20 \\
            \hline
            e.c.c. & 18 & 39 & 63 & 94 & 124 & 155 & 185 & \dots & & \\
            \hline
         \end{LCtableau}
         \smallskip
         La 157\up{e} et la 158\up{e} valeur correspondent à un âge de 51 ans  donc, {\blue la médiane est de 51 ans.} \\ [1mm]
      On pouvait également déterminer la 400\up{e} et la 401\up{e} en considérant le tableau entier et en prenant en compte la répartition plus précise du tableau de la question 2).
   \end{enumerate}
