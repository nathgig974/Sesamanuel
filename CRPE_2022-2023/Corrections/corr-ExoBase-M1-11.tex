Soit $n$ le nombre de caisses à transporter, $t_1$, le temps mis (en minutes) pour ranger les caisses à une \og vitesse \fg{} de 50 caisses par heure, c'est à dire $\dfrac56$ de caisse par minute et $t_2$ le temps mis à une \og vitesse \fg{} de 60 caisses par heure, c'est-à-dire une caisse par minute.
\begin{center}
\begin{pspicture}(0,-1)(6,2)
   \pcline{|->}(0,0)(5,0)
   \lput*{:U}{$t_2$}
   \rput(5,-0.4){11h15}
   \pcline{|->}(0,1)(6,1)
   \lput*{:U}{$t_1$}
   \rput(6,0.6){11h30}
   \psline[linestyle=dashed](0,-0.5)(0,1.5)
\end{pspicture}
\end{center}
   On a : $t_1-t_2 =15 \quad ; \quad \dfrac56 =\dfrac{n}{t_1} \quad \text{ et } \quad 1 =\dfrac{n}{t_2}$, \quad ou encore $n =\dfrac56t_1$ et $n =t_2$. \\ [1mm]
   On obtient le système : $\syst{t_1-t_2 &=&15}{\dfrac56t_1 &=&t_2} \iff \syst{t_1 &=&t_2+15}{5t_1 &=&6t_2} \iff \syst{t_1 &=&t_2+15}{5(t_2+15) &=&6t_2}$ \\ [1mm]
   \hspace*{2.5cm} $\iff \syst{t_1 &=&t_2+15}{5t_2+75 &=&6t_2} \iff \syst{t_1 &=&t_2+15}{75 &=&6t_2-5t_2} \iff \syst{t_1 &=&75+15=90}{t_2 &=&75}$ \\ [3mm]
  On a, par exemple, $t_2 =\umin{75}$, soit 1 h 15 min et 11 h 15 min - 1 h 15 min = 10 h 00. \\
  Donc, {\blue le magasinier a commencé son travail à 10h00.} \\
