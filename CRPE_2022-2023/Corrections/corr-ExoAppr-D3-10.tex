\ \\ [-5mm]
\begin{enumerate}
   \item
   \begin{enumerate}
      \item La médiane correspond à un âge compris entre le 400\up{ième} et le 401\up{ième} âge classés dans l'ordre croissant. \\
      Il y a 243 chefs de moins de 45 ans ($11+84+148$), donc moins de 400 et 297 de plus de 55 ans, donc moins de 400 également. Par conséquent, \bm{la médiane se situe bien dans l'intervalle [ 45 ; 55 ].}
      \item Il y a 243 chefs de moins de 45 ans. Il faut donc déterminer la 157\up{ième} valeur ($400-243$) et la 158\up{ième} valeur ($401-243$) du tableau. \\
      \bigskip
      \qquad
      {\renewcommand{\arraystretch}{1.5}
      \begin{LCtableau}{0.8\linewidth}{11}{c}
         \hline
         \text{âge} & 45 & 46 & 47 & 48 & 49 & 50 & 51 & 52 & 53 & 54 \\
         \hline
         \text{effectif} & 18 & 21 & 24 & 31 & 30 & 31 & 30 & 27 & 28 & 20 \\
         \hline
         \text{e.c.c.} & 18 & 39 & 63 & 94 & 124 & 155 & 185 & \dots & & \\
         \hline
      \end{LCtableau}}
      \medskip
      La 157\up{ième} et la 158\up{ième} valeur correspondent à un âge de 51 ans  donc, \bm{la médiane est de 51 ans.} \\ [1mm]
      On pouvait également déterminer la 400\up{ième} et la 401\up{ième} en considérant le tableau entier et en prenant en compte la répartition plus précise du tableau de la question 1)\,b).
   \end{enumerate}
   \item Si l'on regarde les médianes, la moitié des chefs d'exploitation de Cowville ont 47 ans ou moins alors que la moitié des chefs de Pigville ont 57 ans ou moins. \\
   On peut également remarquer que la médiane du premier groupe est inférieure au troisième quartile du second groupe, donc, on peut considérer que \bm{l'interprétation du journaliste est correcte.}
\end{enumerate}
