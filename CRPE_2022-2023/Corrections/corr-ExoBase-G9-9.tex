\ \\ [-5mm]
   \begin{enumerate}
      \item On a par exemple la figure suivante : \\
         \begin{pspicture}(-2,0)(5,4)
            \pstGeonode[CurveType=polygon,PosAngle={180,0,45,90,135}](0.5,0.5){A}(3,1){B}(3,3){C}(2,3){D}(0,2){E}
            \pstLineAB[linecolor=B2]{A}{C}
            \pstLineAB[linecolor=B2]{A}{D}
         \end{pspicture} \\
         Tout pentagone convexe $ABCDE$ peut être décomposé en trois triangles, par exemple  $BAC$, $CAD$ et $DAE$. \\
         Or, la sommes des angles d'un triangle vaut $180$\degre. \\
         La somme des angles du pentagone vaut donc $3\times\udeg{180} =\udeg{540}$. \\
         L'affirmation est {\blue vraie.}
      \item
         \begin{itemize}
            \item Le triangle ACD est isocèle rectangle en C donc, $\widehat{\text{CDA}} =\udeg{45}$.
            \item Le triangle ABD est isocèle en B donc, $\widehat{\text{ADB}} =\widehat{\text{DAB}} =\udeg{50}$.
            \item Dans le triangle BDE, la somme des angles faisant \udeg{180}, on a $\widehat{\text{BDE}} =\udeg{180}-\udeg{90}-\udeg{25} =\udeg{65}$.
         \end{itemize}
         On a alors : $\widehat{\text{CDA}}+\widehat{\text{ADB}}+\widehat{\text{BDE}} =\udeg{45}+\udeg{50}+\udeg{65} =\udeg{160}\not =\udeg{180}$. \\
         {\blue Les points C, D et E ne sont pas alignés}.
   \end{enumerate}
