\ \\ [-5mm]
\begin{enumerate}
   \item Le prisme ADD'C'CB est composé de :
   \begin{itemize}
      \item 2 triangles isométriques BCC' et ADD' rectangles respectivement en C et D ;
      \item 3 rectangles ABCD, DCC'D' et ABC'D'.
   \end{itemize}
   \item On a par exemple, dans le triangle BCC' rectangle en C : CB = 3 cm et CC' = 4 cm, donc le troisième côté BC' mesure 5 cm car (3, 4, 5) est un triplet pythagoricien. \\
   {\psset{unit=0.55}
   \begin{pspicture}(-1,-4)(13,4.5)
      \pstGeonode[CurveType=polygon,PosAngle={135,135,-135,-135}](0,1){D}(3,1){A}(3,-1){B}(0,-1){C}
      \pstGeonode[CurveType=polygon,PointName={D',D,C,C'},PosAngle={45,45,-45,-45}](8,1){E}(12,1){F}(12,-1){G}(8,-1){H}
      \pstGeonode[PointName={D,C},PosAngle={90,-90}](4.8,3.4){I}(4.8,-3.4){J}
      \pstLineAB{A}{E}
      \pstLineAB{A}{I}
      \pstLineAB{I}{E}
      \pstRightAngle{A}{I}{E}
      \pstLineAB{B}{H}
      \pstLineAB{B}{J}
      \pstLineAB{J}{H}
      \pstRightAngle{B}{J}{H}
      \psframe[fillstyle=solid,fillcolor=lightgray](3,-1)(8,1)
   \end{pspicture}
   \begin{pspicture}(2,-4)(15,4.5)
      \pstGeonode[CurveType=polygon,PosAngle={135,-135,-45,45}](3,1){A}(3,-1){B}(8,-1){C'}(8,1){D'}
      \pstGeonode[CurveType=polygon,PointName={D,A,B,C},PosAngle={90,45,-45,-90}](12,1){E}(15,1){F}(15,-1){G}(12,-1){H}
      \pstGeonode[PointName={D,C},PosAngle={90,-90}](4.8,3.4){I}(4.8,-3.4){J}
      \pstLineAB{E}{D'}
      \pstLineAB{A}{I}
      \pstLineAB{I}{D'}
      \pstRightAngle{A}{I}{D'}
      \pstLineAB{H}{C'}
      \pstLineAB{B}{J}
      \pstLineAB{J}{C'}
      \pstRightAngle{B}{J}{C'}
      \psframe[fillstyle=solid,fillcolor=lightgray](3,-1)(8,1)
   \end{pspicture}}
\end{enumerate}
