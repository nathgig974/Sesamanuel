   {\bf Partie 1}. \\
   \begin{enumerate}
      \item Pour passer du polygone 1 au polygone 2, on effectue {\blue la rotation de centre A et d'angle \udeg{60} dans le sens indirect}.
      \item Pour passer du polygone 2 au polygone 3, on effectue {\blue la symétrie dont le centre est le milieu de [BC]}.
      \item Pour passer du polygone 3 au polygone 4, on effectue {\blue la translation de vecteur $\overrightarrow{CD}$, ou qui transforme le point $C$ en le point $D$}. \bigskip
   \end{enumerate}
   {\bf Partie 2}. \\
   {\psset{unit=0.58}
   \begin{pspicture}(-5,0)(18,17.5)
      \psgrid[subgriddiv=1,gridlabelcolor=white](1,0)(15,18)
      \psline(1,0)(15,14)
      \pspolygon[linewidth=1.5pt](3,7)(4,8)(6,8)(7,9)(6,10)(5,10)(4,9)(3,10)
      \psline[linewidth=1.5pt]{->}(3,13)(5,17)
      \uput[dr](5,9){$\mathcal{P}$}
      \psdots(8,9)(7,6)(12,11)
      \uput[ul](8,9){O}
      \uput[ul](3,13){A}
      \uput[ul](5,17){B}
      \uput[dr](7,6){E}
      \uput[ul](12,11){F }
      \pspolygon[linewidth=1.5pt,linecolor=B2](13,11)(12,10)(10,10)(9,9)(10,8)(11,8)(12,9)(13,8)
      \uput[dr](10,9){\textcolor{B2}{$\mathcal{P}_1$}} %P1
      \pspolygon[linewidth=1.5pt,linecolor=A1](8,2)(9,3)(9,5)(10,6)(11,5)(11,4)(10,3)(11,2)
      \uput[dr](9,5){\textcolor{A1}{$\mathcal{P}_2$}} %P2
      \pspolygon[linewidth=1.5pt,linecolor=blue](5,11)(6,12)(8,12)(9,13)(8,14)(7,14)(6,13)(5,14)
      \uput[dr](7,13){\textcolor{blue}{$\mathcal{P}_3$}} %P3
      \pspolygon[linewidth=1.5pt,linecolor=violet](6,2)(5,3)(5,5)(4,6)(3,5)(3,4)(4,3)(3,2)
      \uput[dr](4,5){\textcolor{violet}{$\mathcal{P}_4$}} %P4
   \end{pspicture}}
