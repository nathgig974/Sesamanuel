\ \\ [-5mm]
   \begin{enumerate}
      \item
         \begin{enumerate}
            \item La note minimale est 5 et l'étendue est 14 donc, la note maximale est $5+14 =19\neq20$. \\
               {\blue L'affirmation est fausse}.
            \item La moyenne entre la note minimale et la note maximale est $\dfrac{5+19}{2} =12$ qui est exactement la moyenne de la classe donc, cette moyenne ne va pas changer. \\
               {\blue L'affirmation est fausse}.
            \item La médiane est 11 donc, la moitié des élèves ont une note supérieure ou égale à 11. \\
               {\blue L'affirmation est fausse}. \medskip
         \end{enumerate}
      \setcounter{enumi}{1}
      \item $m_B =\dfrac{16\times11,7+11\times10,3}{16+11} =\dfrac{300,5}{27} \approx11,13$. \\ [1mm]
         {\blue La moyenne des notes des élèves de la classe B à ce contrôle est d'environ 11,1}. \smallskip
      \item Soit $m_C$ la moyenne de la classe C. \\ [1mm]
         $\dfrac{24\times12+32\times m_C}{24+32} =11,2 \iff \dfrac{288+32m_C}{56} =11,2$ \\ [2mm]
         \hspace*{3.8cm} $\iff 288+32m_c =11,2\times56 =627,2$ \\ [1mm]
         \hspace*{3.8cm} $\iff 32m_C =627,2-288 =339,2$ \\ [2mm]
         \hspace*{3.8cm} $\iff m_C =\dfrac{339,2}{32} =10,6$. \\ [2mm]
         {\blue La moyenne des notes pour la classe C est de 10,6}.
   \end{enumerate}
