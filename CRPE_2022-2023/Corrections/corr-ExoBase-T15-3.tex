\ \\ [-5mm]
   \begin{enumerate}
      \item On peut résumer dans un tableau les valeurs de a, b et n : \\  [1mm]
      \qquad
      {\hautab{1.5}
         \begin{tabular}{r|C{0.8}|C{0.8}|C{0.8}|}
            \cline{2-4}
            & a & n & b \\
            \cline{2-4}
            valeurs initiales & 5 & 0 & 1 \\
            \cline{2-4}
            valeurs après le premier passage & \textcolor{blue}{5} & \textcolor{blue}{1} & \textcolor{blue}{5} \\
            \cline{2-4}
            valeurs après le second passage & \textcolor{blue}{5} & \textcolor{blue}{2} & \textcolor{blue}{25} \\
            \cline{2-4}
         \end{tabular}
      } \medskip
   \item À chaque boucle :
       -- il n'y a aucune action sur $a$ qui reste donc égal à 5 ; \\
       -- n est incrémenté de 1 ; \\
       -- b est multiplié par a, donc par 5. \\
       D'où, {\blue ce programme donne les puissances successives de 5 de $5^1$ à $5^{10}$.}
   \end{enumerate}
