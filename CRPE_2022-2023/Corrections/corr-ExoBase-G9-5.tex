\ \\ [-5mm]
   \begin{enumerate}
      \item Les diagonales du quadrilatère QMNP sont [MN] et [PQ]. Ces diagonales sont de même longueur et se coupent en O, le milieu de [MN] et de [PQ].  Ce procédé permet de construire {\blue une famille de rectangles}. \\
         Cas particulier : lorsque les diagonales sont perpendiculaires, on obtient un {\blue carré}.
      \item Les diagonales du quadrilatère MSNR sont [MN] et [RS]. Ces diagonales se coupent en O = I, milieu de [MN] et de [RS]. Ce procédé permet de construire {\blue une famille de parallélogrammes}. \\
         Cas particulier : lorsque les diagonales sont perpendiculaires, on obtient un {\blue losange}.
      \item Programme de construction : \\
         \begin{itemize}
            \item Construire le segment [MN] de longueur \ucm{9} ainsi que le point O' sur ce segment situé à \ucm{2,5} de N ;
            \item tracer le segment [RS] de longueur \ucm{5} et de milieu O' ;
            \item tracer le quadrilatère MSNR.
         \end{itemize}
         {\psset{unit=0.9}
         \begin{pspicture}(-5,0.5)(9,5.5)
            \pspolygon[linecolor=B2](1,1)(5.5,4.6)(9.6,3.5)(8.9,1)
            \pspolygon[linecolor=A1](9.2,3.6)(9.1,3.2)(9.5,3.1)(9.6,3.5)
            \psline(1,1)(9.6,3.5)
            \pscircle(7.2,2.8){2.5}
            \psline(5.5,4.6)(8.9,1)
            \rput[bl](0.5,0.9){M}
            \rput[bl](9.82,3.64){N}
            \rput[bl](7.1,3.2){O'=I}
            \rput[bl](5.26,4.9){R}
            \rput[bl](9.08,0.72){S}
         \end{pspicture}}
         \bigskip \\
         On a I = O' donc, dans le quadrilatère MSNR, O'R = O'N = O'S et O' est le milieu de [RS] d'où : N est sur le cercle de diamètre [RS] ce qui implique que le triangle SNR est rectangle en N. \\
         {\blue Le quadrilatère MSNR possède donc un angle droit en N}.
      \item Programme de construction : \\
         \begin{itemize}
            \item Construire le segment [MN] de longueur 9 cm ainsi que le point O' sur ce segment situé à \ucm{2,5} de N ;
            \item tracer le cercle de diamètre [MN] et le cercle de centre O' et de  rayon \ucm{4,5} : placer P sur l'un des points d'intersection de ces cercles ;
            \item construire Q sur la droite (PO') tel que PQ = \ucm{9} ;
            \item tracer le quadrilatère MQNP. \\
         \end{itemize}
      {\psset{unit=0.6}
      \begin{pspicture}(-10,-2.2)(12,8)
         \pspolygon[linecolor=A1](4.6,6.3)(5.1,6)(5.4,6.5)(4.9,6.8)
         \psline(1,1)(9.6,3.7)
         \pscircle(5.3,2.35){4.5}
         \pscircle(7.2,2.95){4.5}
         \pspolygon[linecolor=B2](1,1)(4.9,6.8)(9.6,3.7)(9.5,-0.9)
         \psline(4.9,6.8)(9.5,-0.9)
         \rput[bl](0.2,0.85){M}
         \rput[bl](9.8,3.9){N}
         \rput[bl](7.2,3.2){O'}
          \rput[bl](5,7.2){P}
        \rput[bl](9.3,-1.7){Q}
      \end{pspicture}}
      \bigskip \\
      Le point P est situé sur le cercle de diamètre [MN], donc le triangle MNP est rectangle en P. \\
      {\blue Le quadrilatère MPNQ possède donc un angle droit en P}.
   \end{enumerate}
