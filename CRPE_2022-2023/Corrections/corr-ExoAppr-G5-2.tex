\ \\ [-5mm]
\begin{enumerate}
   \item On effectue plusieurs tracés pour obtenir la figure suivantes : \\
   {\psset{linewidth=0.5pt}
   \begin{pspicture}(-1,-0.5)(12,4.76)
      \pscircle[linestyle=dashed,linecolor=B2](2,2){2}
      \pspolygon(0,2)(0.17,2.81)(4,2)(0.85,3.63)
      \pspolygon(0,2)(1.98,4)(4,2)(3.14,3.64)
      \pspolygon(0,2)(3.76,2.96)(4,2)(3.96,2.42)
      \pspolygon(0,2)(0.18,1.17)(4,2)(1.3,0.13)
      \pspolygon(0,2)(2.69,0.12)(4,2)(3.67,0.9)
      \psdots[dotstyle=x,dotsize=2mm](0.17,2.81)(0.85,3.63)(1.98,4)(3.14,3.64)(3.76,2.96)(3.96,2.42)(0.18,1.17)(1.3,0.13)(2.69,0.12)(3.67,0.9)
      \psdots(0,2)(4,2)
      \rput[bl](-0.5,2){A}
      \rput[bl](4.2,2){B}
      \rput(-0.9,4){\textcolor{G1}{a)}}
      \rput(10,2.5){\begin{minipage}{9cm} \textcolor{G1}{b)} Soient A et B deux points du plan et M un troisième point du plan tel que l'angle $\widehat{\text{AMB}}$ soit un angle droit, alors M appartient au cercle de diamètre [AB]. \end{minipage}}
   \end{pspicture}}
   \item Les constructions incorrectes sont les constructions (1), (5) et (6). \\
   $\bullet$ {\bf Construction (1) :} tous les points placés au-dessus de (AB) sont bien placés mais les points placés en dessous de (AB) sont placés sur un demi-cercle ayant un diamètre dont A est une extrémité mais dont l'autre extrémité est le point M situé \og juste au-dessus \fg{} de B. Il est probable qu'ayant retourné son équerre afin de placer des points dans le demi-plan inférieur, l'élève a commis l'erreur de prendre un second point fixe autre que B, assez proche cependant de celui-ci. \\
   $\bullet$ {\bf Construction (5) :} l'élève a placé ses dix points sur deux droites parallèles aux bords verticaux de la feuille, mais également perpendiculaires au segment [AB] tracé. On peut émettre plusieurs hypothèses correspondant chacune à une compréhension incorrecte de l'énoncé :
   \begin{itemize}
      \item l'élève a confondu les termes et les concepts de \og perpendiculaire \fg{} et de \og verticale \fg{}, et considéré que la consigne de l'énoncé \og la droite (AM) et la droite (BM) sont perpendiculaires \fg{} signifiait que ces droites devaient être verticales ;
      \item il a interprété la consigne \og la droite (AM) et la droite (BM) sont perpendiculaires \fg{} en la complétant de sorte que pour lui elle devienne : \og la droite (AM) et la droite (BM) sont perpendiculaires à (AB) \fg{};
      \item il a mal utilisé son équerre et l'a placée de manière à avoir l'angle droit \og en A \fg{} et \og en B \fg{}.
   \end{itemize}
    De plus, ce que dit cet élève est en partie inexact : les dix points ne sont pas tous sur une même droite, mais chaque groupe de cinq points se trouve sur une même droite. \\
   $\bullet$ {\bf Construction (6) :} les dix points sont alignés sur une droite, apparemment horizontale. Ces points ne respectent pas la consigne. Il est possible que cet élève ait essayé de faire comme sur l'image en plaçant de la même façon son équerre et en la gardant bien fixement posée sur sa feuille, il a alors pu placer dix points le long du bord inférieur de son équerre. C'est donc peut-être une interprétation erronée de la consigne \og faire comme Louis \fg, ainsi que du modèle que donne l'image, qui est à l'origine de cette erreur.
   \item $\bullet$ {\bf Construction (3) :} l'élève a commencé par placer 9 points au dessus de (AB), comme le suggère l'image, puis a situé le dixième et dernier point en dessous, à égale distance de A et de B. Est-ce le sentiment de manquer de place pour placer un point de plus qui l'a amené à passer de l'autre côté de (AB) ? Le fait qu'il ait bien perçu que les points étaient sur \og un rond \fg, il l'a écrit, peut faire penser qu'il lui a fallu choisir un endroit pour placer un dernier point, le dixième\dots \\
   $\bullet$ {\bf Construction (6) :} l'image du manuel montre une façon de placer son équerre. L'énoncé n'indique pas que l'équerre peut, et doit changer de position et, manifestement, cet élève l'a gardée dans cette position fixe sur sa feuille.
\end{enumerate}
