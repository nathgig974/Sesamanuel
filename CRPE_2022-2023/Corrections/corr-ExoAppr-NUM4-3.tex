\ \\ [-5mm]
\begin{enumerate}
   \item Il s'agit d'une situation de partage équitable pour laquelle on recherche \og la valeur de chaque part \fg, il s'agit donc d'une division-partition.
   \item Il faudrait pour cela que le dividende soit un nombre entier. Étant donné qu'il est exprimé en cm, il suffit de l'exprimer en mm. On aurait alors un ruban de 1\,376 mm à partager en 8 morceaux de même longueur, soit $1\,736\div8$, exprimée en mm : $\opdiv{1376}{8}$. \quad Chaque morceau mesure 172 mm, soit 17,2 cm.
   \item Au CM2, la procédure experte de la division d'un nombre décimal par un nombre entier est au programme :
   $\opdiv[shiftdecimalsep=divisor]{137,6}{8}$.
   \item Un nombre décimal (positif) peut s'écrire sous la forme $\dfrac{p}{10^q}$ où $p$ et $q$ sont des nombres entiers positifs. \\
   Or, $\dfrac{p}{10^q}\div8 =\dfrac{p}{10^q\times2^3} =\dfrac{p\times5^3}{10^{q}\times10^3} =\dfrac{125p}{10^{q+3}}$ qui est un nombre décimal. \\ [1mm]
   Le quotient d'un décimal par 8 est donc toujours un décimal.
\end{enumerate}
