\ \\ [-5mm]
\begin{enumerate}
   \item Lorsqu'une aiguille parcourt la totalité du cadran, elle parcourt 360\degre. Donc, l'écart entre deux graduations des heures vaut 30\degre ($360\div12 =30$) et l'écart entre deux graduations des minutes vaut 6\degre ($360\div60 =6$). \\
   On considère que la grande aiguille est celle des minutes, et la petite aiguille celle des heures.
   \begin{enumerate}
      \item
      \begin{minipage}{3cm}
      {\psset{unit=0.5}
      \begin{pspicture}(-2.5,-2)(2.5, 2.5)
         \pscircle[linewidth=1mm](0,0){2}
         \multido{\i=0+6}{60}{\psline[linecolor=A1](1.8;\i)(2;\i)} %minutes
         \multido{\i=0+30}{12}{\psline[linecolor=B2](1.7;\i)(2;\i)} %heures
         \multido{\i=60+-30,\n=1+1}{12}{\rput(1.5;\i){\scriptsize\n}} %écritures
         \psline[linewidth=1mm,linecolor=B2]{->}(0,0)(1;-150)
         \psline[linewidth=1mm,linecolor=A1]{->}(0,0)(1.3;90)
         \psdot(0,0)
         \psarc[linecolor=J1]{<->} (0,0){1}{90}{-150}
         \rput(-0.5,0.25){\textcolor{J1}{\scriptsize 120\degre}}
      \end{pspicture}}
      \end{minipage}
      \begin{minipage}{12.5cm}
         À 8 h 00, la grande aiguille est sur le 12 et la petite aiguille est sur le 8. \\
         Il y a donc 4 graduations des heures entre les deux. Or, $4\times 30^\circ =120^\circ$. Donc, \\
         \bm{l'angle formé par les deux aiguilles lorsqu'il est 8 h 00 est de 120\degre.}
         \end{minipage}
         \item
         \begin{minipage}{3cm}
         {\psset{unit=0.5}
         \begin{pspicture}(-2.5,-2)(2.5, 2.5)
            \pscircle[linewidth=1mm](0,0){2}
            \multido{\i=0+6}{60}{\psline[linecolor=A1](1.8;\i)(2;\i)} %minutes
            \multido{\i=0+30}{12}{\psline[linecolor=B2](1.7;\i)(2;\i)} %heures
            \multido{\i=60+-30,\n=1+1}{12}{\rput(1.5;\i){\scriptsize\n}} %écritures
            \psline[linewidth=1mm,linecolor=B2]{->}(0,0)(1;135)
            \psline[linewidth=1mm,linecolor=A1]{->}(0,0)(1.3;-90)
            \psdot(0,0)
            \psarc[linecolor=J1]{<->} (0,0){1}{135}{-90}
            \rput(-0.5,-0.25){\textcolor{J1}{\scriptsize 135\degre}}
          \end{pspicture}}
         \end{minipage}
         \begin{minipage}{12.5cm}
            À 10 h 30, la grande aiguille est sur le 6 et la petite aiguille est juste entre le 10 et le 11, il y a donc 4,5 graduations des heures entre les deux. Or, $4,5\times30^\circ =135^\circ$. Donc, \\
            \bm{l'angle formé par les deux aiguilles lorsqu'il est 10 h 30 est de 135\degre.}
         \end{minipage}
         \item
         \begin{minipage}{3cm}
         {\psset{unit=0.5}
         \begin{pspicture}(-2.5,-2.5)(2.5, 2.5)
            \pscircle[linewidth=1mm](0,0){2}
            \multido{\i=0+6}{60}{\psline[linecolor=A1](1.8;\i)(2;\i)} %minutes
            \multido{\i=0+30}{12}{\psline[linecolor=B2](1.7;\i)(2;\i)} %heures
            \multido{\i=60+-30,\n=1+1}{12}{\rput(1.5;\i){\scriptsize\n}} %écritures
            \psline[linewidth=1mm,linecolor=B2]{->}(0,0)(1;-100)
            \psline[linewidth=1mm,linecolor=A1]{->}(0,0)(1.3;-30)
            \psdot(0,0)
            \psarc[linecolor=J1]{<->} (0,0){1}{-100}{-30}
            \rput(0.35,-0.55){\textcolor{J1}{\scriptsize 70\degre}}
         \end{pspicture}}
         \end{minipage}
         \begin{minipage}{12.5cm}
            À 06 h 20, la grande aiguille est sur le 4 et la petite aiguille est entre le 6 et le 7, à un tiers du 6, il y a donc 2 graduations des heures plus un tiers entre les deux. \\
            Or, $2\times30^\circ+1\div3\times 30^\circ =60^\circ+10^\circ =70^\circ$. Donc, \\
            \bm{l'angle formé par les deux aiguilles lorsqu'il est 06 h 20 est de 70\degre.}
         \end{minipage}
      \end{enumerate}
      \item À chaque fois que l'aiguille des minutes parcourt 5 minutes (30\degre), l'aiguille des heures parcourt 2,5\degre ($30\div12 =2,5$). On choisit alors l'angle le plus petit entre les deux aiguilles. \\
      Dans le tableau suivant, on a indiqué l'angle le plus petit des aiguilles avec le \og 0 \fg. \\
      \medskip
      \qquad
      \begin{CLtableau}{0.9\linewidth}{9}{c}
         \hline
         heure & 00h05 & 00h10 & 00h15 & 00h20 & 00h25 & 00h30 & 00h35 & 00h40 \\
         \hline
         aiguille des minutes &  30\degre  & 60\degre & 90\degre & 120\degre & 150\degre & 180\degre & 150\degre & 120\degre \\
         \hline
         aiguille des heures & 2,5\degre & 5\degre & 7,5\degre & 10\degre & 12,5\degre & 15\degre & 17,5\degre & 20\degre \\
         \hline
         angle entre les deux & 27,5\degre & 55\degre & 82,5\degre & 110\degre & 137,5\degre & 165\degre & 167,5\degre & \textcolor{B2}{140\degre} \\
         \hline
      \end{CLtableau}
      \smallskip
   \bm{Il est minuit et 40 minutes.}
   \end{enumerate}
