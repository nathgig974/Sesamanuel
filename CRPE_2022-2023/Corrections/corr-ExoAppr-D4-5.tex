\ \\ [-5mm]
\begin{enumerate}
   \item Pour une masse de 70 g, le ressort du peson s'allongera de $7\times0,5$ cm = 3,5 cm. Sa longueur totale sera alors de 14 cm + 3,5 cm = \bm{17,5 cm.}
   \item Si le ressort mesure 28 cm, il s'est allongé de 14 cm. Or, 14 cm $=28\times0,5$ cm. Ce qui correspond à une masse de  $28\times10$ g = \bm{280 g.}
   \item Si la longueur du ressort était proportionnelle à la masse suspendue, elle serait nulle pour une masse nulle. Ce n'est pas le cas puisque pour une masse nulle, le ressort mesure 14 cm. \\
   \bm{Ces deux grandeurs ne sont pas proportionnelles.}
\end{enumerate}
