\ \\ [-5mm]
\begin{enumerate}
   \item La valeur 18 correspond à \bm{l'espacement minimal en mètre entre deux véhicules à une vitesse de 3 m/s pour 4 sièges par véhicule}.
   \item On peut choisir les formules suivantes : \Cell{\texttt{=\$B3*(4+E\$2/2)}} ou \Cell{\texttt{=B3*(4+4/2)}}
   \item La cellule E3 nous indique que, pour un véhicule à 4 places se déplaçant à 2,3 m/s, l'espacement minimal est de 13,8 m. Avec la formule, on obtient alors : $D =3\,600\times4\times\dfrac{2,3}{13,8} =2\,400$. \\
   \bm{L'affirmation est cohérente avec les données de l'énoncé}.
   \item La cellule E6 nous indique que, pour un véhicule à 4 places se déplaçant à 2 m/s, l'espacement minimal est de 12 m. Avec la formule, on obtient alors : $D =3\,600\times4\times\dfrac{2}{12} =2\,400$. \\
   La cellule E13 nous indique que, pour un véhicule à 4 places se déplaçant à 3 m/s, l'espacement minimal est de 18 m. Avec la formule, on obtient alors : $D =3\,600\times4\times\dfrac{3}{18} =2\,400$. \\ [1mm]
   \bm{Pour des véhicules à 4 sièges, le débit est identique que la vitesse soit de 2~m/s ou de 3~m/s}.
   \item On a $D =3\,600\,n\dfrac{V}{E}$ où $E =V\left(4+\dfrac{n}{2}\right)$ soit $\dfrac{E}{V} =4+\dfrac{n}{2}$, ou encore $\dfrac{V}{E} =\dfrac{1}{4+\dfrac{n}{2}}$. \\
   Alors, $D =\dfrac{3\,600\,n}{4+\dfrac{n}{2}} =\dfrac{7\,200\,n}{8+n}$. \\ [1mm]
   \bm{Le débit $D$ s'exprime donc uniquement en fonction du nombre de sièges $n$}. \\
   Ce qui confirme le résultat trouvé dans la question 4 : en effet, avec $n=4$, on trouve $D =\dfrac{7\,200\times4}{8+4} =2\,400$.
\end{enumerate}
