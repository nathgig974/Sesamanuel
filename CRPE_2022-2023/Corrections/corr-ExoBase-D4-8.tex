\ \\ [-5mm]
   \begin{enumerate}
      \item
         \begin{enumerate}
            \item On obtient le tableau suivant : \\ [1mm]
               {\hautab{1.5}
               \begin{LCtableau}{0.9\linewidth}{13}{c}
                  \hline
                  Durée en min& 5 & 10 & 15 & 20 & 25 & 30 & 35 & 40 & 45 & 50 & 55 & 60 \\
                  \hline
                  Effectif & 1 & 3 & 2 & 2 & 4 & 7 & 5 & 4 & 3 & 2 & 0 & 3 \\
                  \hline
                  Fréquence en $\%$ & 2,8 & 8,3 & 5,6 & 5,6 & 11,1 & 19,4 & 13,9 & 11,1 & 8,3 & 5,6 & 0 & 8,3 \\
                  \hline
               \end{LCtableau}}
            \item $60-5 =55$, donc, l'étendue vaut {\blue $e =55$ minutes}. \medskip
            \item $m =\dfrac{1\times5+3\times10+2\times15+2\times20+4\times25+7\times30+5\times35+4\times40+3\times45+2\times50+3\times60}{1+3+2+2+4+7+5+4+3+2+0+3}$ \\ [1mm]
               $m =\dfrac{1\,165}{36} \approx32,35$. \\ [2mm]
               Il y a 36 valeurs dans cette liste, la médiane est donc une valeur située entre la 18\up{e} et la 19\up{e} valeur lorsqu'elles sont ordonnées. Or, ces valeurs valent toutes les deux \umin{30} donc, la médiane est égale à \umin{30}. \\
               {\blue La moyenne vaut \umin{32,4} et la médiane vaut \umin{30}}.
         \end{enumerate}
   \setcounter{enumi}{1}
   \item
      \begin{enumerate}
         \item Tableau par classes : \\ [1mm]
            {\hautab{1.5}
            \begin{lctableau}{0.9\linewidth}{5}
               \hline
               Durée en min & $]\,0\,;\,15\,]$ & $]\,15\,;\,30\,]$ & $]\,30\,;\,45\,]$ & $]\,45\,;\,60\,]$ \\
               \hline
               Effectif & 6 & 13 & 12 & 5 \\
               \hline
            \end{lctableau}}
         \item On trouve $\overline{m} =\dfrac{6\times7,5+13\times22,5+12\times37,5+5\times52,5}{36} \approx$ {\blue 29,2 minutes.} \\ [1mm]
            Ce résultats est assez différent de celui trouvé dans la  question 2)c) car la moyenne est faite à partir de la répartition par classe, on perd donc en précision, puisqu'alors on ne s'occupe plus de la valeur exacte, mais de l'appartenance à un intervalle plus grand.
         \end{enumerate}
   \end{enumerate}
