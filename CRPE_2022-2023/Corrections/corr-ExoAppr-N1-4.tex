\ \\ [-5mm]
\begin{enumerate}
   \item C'est l'aspect cardinal du nombre qui est ici mobilisé (construire le nombre pour exprimer les quantités) : en effet, l'élève doit comprendre qu'un objet est une unité, en dehors de toute considération de forme, d'utilité, puis construire le principe cardinal, c'est à dire concevoir que le dernier mot-nombre de la liste ayant servi à énumérer désigne, à lui seul, le nombre total d'éléments de la collection. \\
   À ce titre, l'objectif pour l'élève est de ramener autant de biscuits qu'il y a de poupées ou d'assiettes.
   \bigskip
   \item On remarque tout d'abord que tous les élèves ont rempli le contrat et sont arrivés au résultat escompté, mais avec des procédures différentes.
   \begin{itemize}
      \item L'\textbf{élève A} va chercher les biscuits un à un, jusqu'à ce qu'il ait rempli toutes les assiettes. On ne peut pas affirmer qu'il ait dénombré les assiettes ou les biscuits, mais plutôt qu'il a fait une \og correspondance terme à terme \fg{} entre les assiettes et les biscuits.
      \item L'\textbf{élève B} a dénombré les assiettes, ce qu'il modélise par des doigts levés : il est probable qu'il ait effectué une correspondance terme à terme entre les assiettes et les doigts, ou qu'il ait compté en même temps qu'il levait ses doigts un à un. Il effectue probablement la même procédure pour déterminer le nombre de biscuits à rapporter.
      \item L'\textbf{élève C} est probablement l'élève ayant le mieux compris le principe cardinal : il a entendu ou/et vu qu'il y avait 3 assiettes (par subitisation ou dénombrement), il ramène donc 3 biscuits, il n'a pas besoin à ce stade d'aide particulière (doigts, correspondance\dots{}).
      \item L'\textbf{élève D} ramène tous les biscuits de la cuisine, il distribue les biscuits en en mettant un par assiette, puis il ramène les biscuits \og en trop \fg{}. \\
   \end{itemize}
   \bigskip
   \item Pour les élèves A et D, on peut observer qu'ils utilisent une méthode de \og remplissage \fg{} des assiettes sans utiliser les caractéristiques des nombres. Il effectuent tous les deux plusieurs voyages pour arriver au bon résultat. \\
   Pour qu'ils engagent une procédure inhérente à la construction du nombre, on pourrait préciser dans l'énoncé que seul un voyage est autorisé. Ils devront alors réfléchir à la manière de ramener 3 biscuits \og du premier coup \fg.
\end{enumerate}
