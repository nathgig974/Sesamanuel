   Pour les figures 1) et 2), il y a une seule configuration possible, à isométries près. Pour les figures 3) et 4), il y a une infinité de configurations qui dépendant de l'endroit où l'on place le troisième point.
   \begin{pspicture*}(0,-1)(3.5,3.5)
      \pspolygon[fillstyle=solid,fillcolor=lightgray!50](0,0)(3,0)(3,3)(0,3)
      \psset{linecolor=gray}
      \pscircle(3,0){3}
      \psline(0,0)(0,4)
      \psline(3,0)(3,4)
      \psline(-1,3)(3,3)
      \rput(-0.3,-0.3){A}
      \rput(3.3,-0.3){B}
      \rput(3.3,3.3){C}
      \rput(-0.3,3.3){D}
   \end{pspicture*}
   \begin{pspicture*}(-4,-3)(3.5,3)
      \pspolygon[fillstyle=solid,fillcolor=lightgray!50](-3,0)(0,-2)(3,0)(0,2)
      \psset{linecolor=gray}
      \psline(-3,0)(3,0)
      \psline(0,-3)(0,3)
      \pscircle(0,0){2}
      \rput(-3.3,0){A}
      \rput(3.3,0){C}
      \rput(0.3,-2.3){B}
      \rput(0.3,2.3){D}
   \end{pspicture*}
   \begin{pspicture*}(-0.5,-0.5)(4.5,2.75)
      \pspolygon[fillstyle=solid,fillcolor=lightgray!50](0,0)(3,0)(3.9,1.8)(0.9,1.8)
      \psset{linecolor=gray}
      \pscircle(0,0){2}
      \pscircle(3,0){2}
      \pscircle(0.9,1.8){3}
      \rput(-0.3,0){A}
      \rput(3.3,0){B}
      \rput(1,2.2){D}
      \rput(4.2,2.1){C}
   \end{pspicture*}
