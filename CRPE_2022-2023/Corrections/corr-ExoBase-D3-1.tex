   Toutes ces situations sont des situations d'équiprobabilité. \\
   \begin{enumerate}
      \item La probabilité d'obtenir pile est toujours la même, quel que soit le nombre de lancer donc, \\
         {\blue l'affirmation 1 est vraie}.
       \item On obtient deux boules vertes en tirant une boule verte dans la première urne, et une autre boule verte dans la deuxième urne. Or, la probabilité d'obtenir une boule vert est de $\dfrac14$ pour chacune des urnes donc, la probabilité d'obtenir deux boules vertes est de $\dfrac14\times\dfrac14 =\dfrac{1}{16}$. \\ [1mm]
         {\blue L'affirmation 2 est fausse}.
      \item On obtient une somme de 3 en ayant soit \og 1 \fg{} sur un dé et \og 2 \fg{} sur l'autre dé, soit deux possibilités (1 et 2 ou 2 et 1) alors qu'il y a une seule possibilité pour obtenir 2 (1 et 1) donc, les probabilités ne sont pas égales. \\
         {\blue L'affirmation 3 est fausse}.
   \end{enumerate}
