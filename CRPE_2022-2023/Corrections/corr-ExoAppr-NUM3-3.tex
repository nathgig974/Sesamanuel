\ \\ [-5mm]
\begin{enumerate}
   \item {\bf Élève 1 : } il effectue une addition itérée de 6 (\oe ufs), treize fois, qu'il pose en colonne. Le résultat est 78, on ne sait pas si l'élève s'est trompé dans son opération ou s'il s'est trompé en notant le résultat (échange de la dizaine et de l'unité). \\
   L'élève 1 sait résoudre un problème dans le champ additif, et il sait modéliser correctement la situation par une opération mathématique correcte : l'addition itérée. \\
   {\bf Élève 2 : } il modélise la situation par un schéma dans lequel il dessine des paquets de 6 \oe ufs en prenant soin d'indiquer le nombre d'\oe ufs à côté de chaque paquet. Puis il compte les \oe ufs un à un, ou six par six jusqu'à obtenir 78, qui est un résultat juste. \\
   L'élève 2 sait résoudre un problème dans le champ additif, il sait modéliser correctement la situation par un schéma et utiliser ce schéma pour dénombrer. \\
   {\bf Élève 3 : } il effectue la multiplication de 13 (boites) par 6 (\oe ufs) et obtient un résultat juste. Il s'agit ici de la procédure experte. \\
   L'élève 3 sait résoudre un problème dans le champ multiplicatif, il sait modéliser correctement la situation et maitrise la technique opératoire de la multiplication posée en colonne. \\
   \item {\bf Élève 4 : } il effectue la multiplication de 13 (boites) par 6 (\oe ufs), il sait donc résoudre un problème dans le champ multiplicatif en posant la bonne opération experte. \\
   Le résultat de $3\times6$ est juste : 18, il pose bien ses deux chiffres dans l'opération mais ensuite semble effectuer la somme de 1 et de 1, ceci est probablement dû au fait que les deux nombres n'ont pas le même nombre de chiffres, et après avoir utilisé une fois le \og 6 \fg{}, il additionne la retenue à la dizaine comme il le ferait dans une addition. \\
   {\bf Élève 5 : } il sait effectuer une addition posée en colonnes. \\
   Par contre, il ne sait pas modéliser la situation par la bonne opération. \\
   {\bf Élève 6 : } il sait schématiser la situation et poser le calcul adéquat. \\
   Il regroupe les \og 6 \fg{} par paquets de 4 pour obtenir 24, qu'il indique à côté de chaque paquet, ce qui est juste. Enfin, il effectue l'opération $24+24+6$ en colonnes, mais son \og 6 \fg{} n'est pas bien placé et il le calcule comme étant un chiffre de la colonne des dizaines, son résultat final est donc faux.
   \item Pour l’élève 5, on peut lui proposer de schématiser la situation, et de commencer à dénombrer les \oe ufs, il se rendrait vite compte que 19 est trop petit. Ensuite, on pourrait lui proposer d'écrie le calcul de plusieurs façons différentes (somme, produit) afin de lui faire découvrir quelle procédure est la plus rapide.
   \item Pour les élèves 1 et 6, il suffirait de donner un nombre de boites bien plus élevé (par exemple, 67), pour que la procédure par schématisation soit longue et fastidieuse et que les élèves soient contraints à réfléchir à une autre méthode, plus rapide.
\end{enumerate}
