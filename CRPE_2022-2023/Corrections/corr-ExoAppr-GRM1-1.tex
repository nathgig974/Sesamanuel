\ \\ [-5mm]
\begin{enumerate}
   \item
   \begin{enumerate}
      \item Difficultés de l'exercice 1 :
      \begin{itemize}
         \item les élèves peuvent ne pas bien estimer la masse des objets parce qu'ils ne les ont jamais soupesés ;
         \item l'élève peut avoir du mal à imaginer les masses autre que 1 g ou 1 kg indiquées au début de l'exercice ;
         \item certains objets proposent des masses avec des unités différentes (grammes et kilogrammes).
      \end{itemize}
      \item Difficultés de l'exercice 2-a :
      \begin{itemize}
         \item difficulté dans la conversion des unités (1 kg 250 g correspondent à 1 250 g ou inversement) ;
         \item méconnaissance des signes > et < ;
         \item représentations des masses à l'aide de deux objets différents (balance électronique et balance de Roberval).
      \end{itemize}
   \end{enumerate}
   \item
   \begin{enumerate}
      \item L'élève n'apporte aucune importance à l'unité et au fait que 1 kg représente 1 000 g : il associe entre eux les différents nombres représentant les masses. Par exemple, 2 kg 300 g = 2 300 g.
      \item Si on a un nombre non composé de 4 chiffres, ce procédé ne fonctionne pas. \\
      Par exemple, 23 456 g donnerait 2 kg et 3 456 g et 123 g ferait 1 kg et 23 g.
   \end{enumerate}
   \item
   \begin{enumerate}
      \item On peut utiliser deux masses marquées de 1 kg et quatre masses de 10 g et les poser sur la balance digitale, pour que l'élève s'aperçoive qu'en fait, la masse totale est de 2 040 grammes et non pas 240 grammes.
      \item L'enseignant peut par exemple reprendre avec l'élève les masses marquées et lui faire tenir dans une main une masse de 1 g, et dans l'autre celle de 1 kg afin qu'il s'aperçoive de la grande différence de poids, puis effectuer avec lui des conversions toujours en utilisant la manipulation de masses marquées.
   \end{enumerate}
\end{enumerate}
