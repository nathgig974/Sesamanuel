\ \\ [-5mm]
   \begin{enumerate}
      \item Patron à main levée d'un cylindre : \\
         \begin{pspicture}(-6,-3)(2,2.5)
            \psline[linecolor=blue](-2,-1)(2,-1)
            \psline[linecolor=blue](-2,1)(2,1)
            \psline[linecolor=red](-2,-1)(-2,1)
            \psline[linecolor=red](2,-1)(2,1)
            \pscircle[linecolor=blue](0,1.64){0.64}
            \pscircle[linecolor=blue](0,-1.64){0.64}
         \end{pspicture}
      \item On a $\mathcal{V} =\pi\times r^2\times h =908$ donc, {\blue $h =\dfrac{908}{\pi\,r^2}$} \, avec des longueurs exprimées en cm. \smallskip
      \item
         \begin{enumerate}
            \item La formule de l'aire d'un disque de rayon $r$ étant $\mathcal{A} =\pi\times r^2$, la formule entrée en C2 puis recopiée vers le bas correspond à la {\blue proposition 2}.
            \item L'aire totale de la boite de conserve est la somme de l'aire des deux disques et de l'aire latérale donc, on peut proposer la formule : {\blue \texttt{=2*C2+D2}}.
            \item En analysant la colonne E, on observe que les valeurs de l'aire totale sont décroissantes pour des valeurs entières du rayon allant de 2 à 6, puis sont croissantes pour des valeurs de $r$ allant de 6 à 16 donc : {\blue on peut conjecturer que la valeur du rayon est dans l'intervalle ]\,5\,;\,7\,[}.
         \end{enumerate}
   \end{enumerate}
