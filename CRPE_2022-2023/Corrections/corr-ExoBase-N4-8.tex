   Étudions une à une les données de l'exercice :
   \begin{itemize}
      \item 111 est un multiple du nombre entier positif $A$. \\
         Autrement dit, $A$ est un diviseur de 111. Or, $111 =3\times37$ et les diviseurs positifs de 111 sont 1, 3, 37 et 111. \\
         $A$ peut donc prendre les valeurs de 1, 3, 37 ou 111.
      \item $A-B$ est un nombre entier positif ou nul divisible par 10. \\
         Il existe un entier $n$ positif ou nul tel que $A-B =10\,n$, soit $B =A-10\,n$ avec $B\geq0$, c'est à dire $A-10\,n\geq0$ ou encore $0\leqslant n\leqslant \dfrac{A}{10}$. Résumons dans un tableau les possibilités pour $A, n$ et $B$ : \\ [1mm]
         \begin{tabular}{|C{0.8}|c|C{0.8}|C{0.8}|}
            \hline
            \cellcolor{lightgray!50}$A$ & \cellcolor{lightgray!50}inégalité & \cellcolor{lightgray!50}$n$ & \cellcolor{lightgray!50}$B$ \\
            \hline
            1 & $0\leqslant n\leqslant 0,1$ & 0 & {\blue 1} \\
            \hline
            3 & $0\leqslant n\leqslant 0,3$ & 0 & 3 \\
            \hline
            \multirow{4}{*}{37} & \multirow{4}{*}{$0\leqslant n\leqslant 3,7$} & 0 & 37 \\
            & & 1 & {\blue 27} \\
            & & 2 & 17 \\
            & & 3 & 7 \\
            \hline
         \end{tabular}
         \qquad
         \begin{tabular}{|C{0.8}|c|C{0.8}|C{0.8}|}
            \hline
            \cellcolor{lightgray!50}$A$ & \cellcolor{lightgray!50}inégalité & \cellcolor{lightgray!50}$n$ & \cellcolor{lightgray!50}$B$ \\
            \hline
            \multirow{12}{*}{111} & \multirow{12}{*}{$0\leqslant n\leqslant 11,1$} & 0 & 111 \\
            & & 1 & 101 \\
            & & 2 & 91 \\
            & & 3 & 81 \\
            & & 4 & 71 \\
            & & 5 & 61 \\
            & & 6 & 51 \\
            & & 7 & 41 \\
            & & 8 & 31 \\
            & & 9 & 21 \\
            & & 10 & 11 \\
            & & 11 & {\blue 1} \\
            \hline
         \end{tabular}
         \bigskip
      \item $B$ est le cube d'un nombre entier. \\
         Les cinq premiers cubes sont $0, 1, 8, 27$ et $256$. Parmi les solutions trouvées dans le deuxième item, on a trois couples de solutions : $(A,B) =(1,1)$ \quad ; \quad $(A,B) =(37,27)$ \quad et \quad $(A,B) =(111,1)$. \\
         Les valeurs possibles pour  $A$ et $B$ sont donc $\left\{\begin{tabular}{l}
           {\blue  $A=1$ et $B=1$} \\ {\blue $A =37$ et $B=27$} \\ {\blue $A =111$ et $B =1$}  \\
         \end{tabular}\right.$
   \end{itemize}
