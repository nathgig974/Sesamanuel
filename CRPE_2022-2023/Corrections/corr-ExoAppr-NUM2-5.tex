\ \\ [-5mm]
\begin{enumerate}
   \item On peut citer les objectifs d'apprentissage suivants :
   \begin{itemize}
      \item un objectif général : résolution d'un problème (en groupe) ;
      \item un objectif disciplinaire : travailler le champ additif (addition à cinq nombres) et une approche de la division partition.
   \end{itemize}
   \item Analyse des productions des groupes 1 et 2. \\
   \begin{enumerate}
      \item Au niveau des stratégies :
      \begin{itemize}
         \item pour le {\bf groupe 1}, les élèves ont additionné toutes les dizaines des nombres en présence, il y en a quatre ce qui fait 40, puis ils ont ajouté à ce nombre les unités restantes : 2 pour Lisa, 1 pour Camille, 9 pour Ilyes et 3 pour Nora. Ils ont trouvé 55 (cependant, l'écriture mathématique est incorrecte puisque les calculs sont faits les uns derrière les autres et le signe \og = \fg{} n'a pas son sens mathématique usuel) ;
         \item pour le {\bf groupe 2}, ils additionnent les nombres en présence par deux : tout d'abord Lisa et Luc, puis Camille et Ilyes. Ensuite, ils additionnent les deux résultats trouvés, et enfin ils additionnent le dernier nombre, celui de Nora.
      \end{itemize}
      \item Point commun et différences :
      \begin{itemize}
         \item {\bf point commun} : les deux groupes ont additionné les bonbons puis les ont puis \og partagés \fg{} grâce à un schéma ;
         \item {\bf différences} : au niveau de l'addition, le groupe 1 a séparé les dizaines et les unités alors que le groupe 2 a additionné les nombres un par un. \\
         Au niveau du partage, le groupe 1 a écrit la division $55:5$ et on a l'impression qu'il a trouvé la solution mentalement puis l'a vérifiée par un schéma (les lignes verticales formées par les points modélisant les bonbons semblent faites en une seule fois) alors que dans le groupe 2, il est possible qu'ils aient distribué un par un les bonbons aux cinq élèves jusqu'à 55 sans se poser la question de l'opération en jeu. La vérification se fait alors à la fin par un calcul en ligne.
      \end{itemize}
   \end{enumerate}
   \item Difficultés rencontrées par le groupe 3 :
   \begin{itemize}
      \item la première difficulté a été d'effectuer l'addition en ligne des cinq nombres : les élèves sont capables de faire des additions en ligne à deux chiffres (Lisa-Luc et Camille-Ilyes), mais ensuite ont du mal à trouver la somme totale qui est fausse. Ils ont fini par modéliser la situation par un schéma et compter un à un les bonbons ;
      \item la seconde difficulté provient de la résolution du problème : ils n'ont pas répondu à la question (parce qu'ils ne savaient pas le faire, ou parce qu'ils n'ont pas compris la consigne ?) et se sont arrêté au résultat 55 qu'ils ont encadré.
   \end{itemize}
\end{enumerate}
