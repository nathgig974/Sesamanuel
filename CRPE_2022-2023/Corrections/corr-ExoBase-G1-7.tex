\ \\ [-5mm]
   \begin{enumerate}
      \item \textcolor{G1}{$\bullet$} Pour construire un angle de \udeg{60}, il suffit de construire un triangle équilatéral ;
         \begin{itemize}
            \item  pour construire un angle de \udeg{45}, on peut construire un angle droit grâce à une perpendiculaire, puis construire sa bissectrice. \\ [3mm]
            \end{itemize}
            {\psset{algebraic=true,unit=1.34}
            \begin{pspicture*}(-3,-1)(7,5.5)
               \pspolygon[linecolor=gray](0,0)(6,0)(3,5.2)
               \psline[linecolor=gray](6,-1)(6,6)
               \psplot[linecolor=gray]{-1}{7}{(-4.24--0.71*x)/-0.71}
               \psline[linecolor=B2](2.2,-1)(2.2,6)
               \pscircle(3,0){3}
               \psplot[linecolor=A1]{-1}{7}{--x}
               \psplot[linecolor=A1]{-1}{7}{(--16-2.6*x)/4.5}
               \psline[linecolor=G1](3,-0.5)(3,5.7)
               \pspolygon[linecolor=B2](0,0)(6,0)(2.2,3.8)
               \rput(-0.3,0){A}
               \rput[bl](6.1,0.2){B}
               \rput[bl](2.3,4){C}
               \rput[bl](0.6,0.2){\textcolor{B2}{60\degre}}
               \rput[bl](4.6,0.2){\textcolor{B2}{45\degre}}
              \rput[bl](2.28,-0.3){H}
               \rput[bl](3.08,-0.3){I}
               \rput[bl](1.5,2.8){L}
               \rput[bl](3.08,3.12){K}
               \rput[bl](2.28,2.32){O}
            \end{pspicture*}}
         \item
            \begin{itemize}
               \item I est le centre du cercle de diamètre [AB], il est donc situé au milieu de [AB], à égale distance de A et de B ;
               \item K est situé sur le cercle de diamètre [AB], le triangle ABK est inscrit dans un cercle ayant pour diamètre un de ses côtés, c'est donc un triangle rectangle en K. \\
                  Dans ce triangle, sachant que la somme des angles mesure 180\degre, que $\widehat{\text{ABK}}=45$\degre et $\widehat{\text{BKA}}=90$\degre, alors $\widehat{\text{KAB}}=180^\circ-90^\circ-45^\circ =45^\circ$. \\
                  Le triangle ABK est donc rectangle et isocèle en K puisqu'il possède deux angles de même mesure.
            \end{itemize}
            On a alors KA = KB, et IA = IB, donc, {\blue la droite (KI) est la médiatrice du segment [AB].}
         \item
            \begin{itemize}
                \item Le triangle ABK étant rectangle en K, la droite (AK) est perpendiculaire à la droite (BC). \\
                   (AK) est donc la hauteur issue de A du triangle ABC ;
                \item le triangle ABL étant rectangle en L (il est inscrit dans le cercle de diamètre [AB]), la droite (BL) est perpendiculaire à la droite (AC). (BL) est donc la hauteur issue de B du triangle ABC ;
                \item ces deux hauteurs se coupent en l'orthocentre noté O. La troisième hauteur du triangle ABC est la droite (CH), qui passe donc elle aussi par O.
            \end{itemize}
            D'où : {\blue les points C, O, H sont alignés.}
         \item
            \begin{itemize}
               \item (KI) est la médiatrice du segment [AB], donc, (KI) est perpendiculaire à (AB) ;
               \item (CH) est une hauteur du triangle ABC, donc, (CH) est perpendiculaire à (AB) ;
            \end{itemize}
            or, deux droites perpendiculaires à une même droite sont parallèles entre elles, donc, les droites (CH) et (KI) sont parallèles. D'après la question 2, O est un point de la droite (CH), d'où  (CO) et (KI) sont parallèles, d'où : \\
            {\blue le quadrilatère IKCO est un trapèze.}
   \end{enumerate}
