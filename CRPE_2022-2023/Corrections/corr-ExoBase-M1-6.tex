\ \\ [-5mm]
   \begin{enumerate}
      \item
         \begin{enumerate}
            \item Le diamètre est lu grâce à la côte \og R15 \fg, il s'agit donc d'une jante de diamètre 15 pouces. \\
               Or, $15\times\ucm{2,54} =\ucm{38,1}$. {\blue Le diamètre de la jante vaut \ucm{38,1}}. \\
            \item La hauteur du pneu peut-être calculée grâce à la côte \og 195/65 \fg. \\
               Donc, la hauteur du pneu vaut 65 \% de sa largeur (\umm{195}). Or, $\dfrac{65}{100}\times\umm{195} =\umm{126,75}$. \\
               {\blue La hauteur du pneu est \ucm{12,675}}. \\
            \item $\text{Diamètre de la roue} = \text{diamètre de la jante} + 2\times\text{hauteur du pneu}$ \\
               \hspace*{3.38cm} $=\ucm{38,1}+2\times\ucm{12,675}=\ucm{63,45}$. \\
               {\blue Le diamètre total de la roue est \ucm{63,45}}. \\
         \end{enumerate}
      \setcounter{enumi}{1}
      \item Les informations inscrites sur le pneu sont au nombre de cinq :
         \begin{itemize}
            \item largeur : $\ucm{20,5} = {\bf 205} \umm{}$ ;
            \item hauteur : $\text{hauteur du pneu} = \dfrac{\text{diamètre total du pneu}-\text{diamètre de la jante}}{2}$ \\ [1mm]
               \hspace*{4.7cm} $=\dfrac{\ucm{63,19}-\ucm{40,64}}{2} =\ucm{11,275}$. \\ [1mm]
               Or, une mesure de \ucm{11,275} représente un pourcentage de $\dfrac{\ucm{11,275}}{\ucm{20,5}}\times100 =\bf{55}\,\%$ par rapport à une mesure de \ucm{20,5} ;
            \item diamètre : le diamètre de la jante vaut \ucm{40,64}, qu'il faut convertir en pouce. Or, un pouce vaut \ucm{2,54} et $40,64\div2,54=16$ donc, l'inscription est {\bf R16} ;
            \item indice du poids : la charge maximale est de \ukg{412} ce qui correspond à l'indice {\bf 77} ;
            \item indice de vitesse : la vitesse maximale est de \ukmh{270} ce qui correspond à l'indice {\bf W}. \\
         \end{itemize}
         {\blue Les informations inscrites sur ce pneu sont : 205/55 R16 77 W}.
   \end{enumerate}
