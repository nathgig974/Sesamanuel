\ \\ [-5mm]
   \begin{enumerate}
      \item $x$ peut prendre toutes les valeurs {\blue entre 0 et 8 inclus}.
      \item L'aire de la surface grisée, notée $\mathcal{A}_G$ est égale à $\mathcal{A}(\text{DEFM})+\mathcal{A}(\text{BGFH})$. \\
         $\mathcal{A}_G =x^2+(20-x)(8-x) = x^2+160-20x-8x+x^2$, soit : \\
         {\blue $\mathcal{A}_G =2x^2 - 28x +160$}.
      \item $2(x - 7)^2 + 62 =2(x^2-14x+49)+62$ \\
         \hspace*{2.45cm} $=2x^2-28x+98+62$ \\
         \hspace*{2.45cm} $=2x^2-28x+160$. \\
         D'où {\blue $2x^2-28x+160 =2(x-7)^2+62$}
      \item L'aire de la partie grisée est minimale lorsque l'expression $2(x-7)^2+62$ est minimale. \\
         Le minimum est atteint lorsque $2(x-7)^2 =0$, c'est à dire lorsque $x-7 =0$, soit $x =7$. \\
         {\blue L'aire de la partie grisée est minimale pour $x =\ucm{7}$.}
      \item L'aire de la partie grisée est égale à \ucmq{112} lorsque $2(x-7)^2+62 =112$ \\
         $\iff 2(x-7)^2 =50 \iff (x-7)^2=25 \iff x-7=5$ ou $x-7=-5 \iff x=12$ ou $x=2$. \\
         Or, $x$ est compris entre 0 et 8, donc : {\blue l'aire de la partie grisée est égale à \ucmq{112} lorsque $x=\ucm{2}$.}
\end{enumerate}
