\ \\ [-5mm]
\begin{enumerate}
   \item En descente, le batelier parcourt 120 km en $n$ jours, donc \bm{$\dfrac{120}{n}$ km} en un jour. \\
   En remontée, le batelier parcourt 120 km en $n+1$ jours, donc \bm{$\dfrac{120}{n+1}$ km} en un jour. \\
   \item Comme la distance parcourue quotidiennement lors de la remontée est inférieure de 6 km à celle parcourue quotidiennement lors de la descente, on a bien \bm{$\dfrac{120}{n+1} =\dfrac{120}{n} - 6$}.
   \item $\dfrac{120}{n+1} =\dfrac{120}{n} - 6 \iff \dfrac{120}{n+1} =\dfrac{120-6n}{n}$ \\ [1mm]
   \hspace*{2.85cm} $\iff (120-6n)(n+1) =120n$ \\
   \hspace*{2.85cm} $\iff \cancel{120n}+120-6n^2-6n =\cancel{120n}$ \\
   \hspace*{2.85cm} $\iff 6n^2+6n =120$ \\
   \hspace*{2.85cm} $\iff n^2+n =20 \iff$ \bm{$n (n + 1) = 20$}.
   \item $n$ est un nombre entier, il faut donc trouver deux nombres consécutifs dont le produit vaut 20. \\
   On trouve $n =4$ puisque $4\times5 =20$. \\
   \bm{Le batelier descend la rivière en 4 jours et la remonte en 5 jours}.
\end{enumerate}
