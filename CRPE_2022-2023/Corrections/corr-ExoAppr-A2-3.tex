\ \\ [-5mm]
   \begin{enumerate}
      \item La probabilité d'atteindre la couronne extérieure est proportionnelle à l'aire de cette zone. On peut, par exemple, déterminer l'aire de la couronne ainsi que l'aire de la cible.
      \begin{itemize}
         \item Aire de la cible en cm$^2$ : $\mathcal{A} =\pi\times15^2 =225\pi$.
         \item Aire cumulée de la couronne blanche et du disque noir en cm$^2$ : $\mathcal{A}_1 =\pi\times10^2 =100\pi$.
         \item Aire de la couronne extérieure en cm$^2$ : $\mathcal{A}_2 =\mathcal{A} -\mathcal{A}_1 =225\pi-100\pi =125\pi$.
      \end{itemize}
      \smallskip
      Probabilité d'atteindre la couronne quadrillée : $\mathcal{P}_q =\dfrac{125\pi}{225\pi} =\dfrac{125}{225} =\dfrac59$. \\ [1mm]
      \bm{La probabilité d'atteindre la couronne extérieure est de $\dfrac59$.}
      \item Calculons la probabilité d'atteindre le coeur de la cible : $\mathcal{P}_c =\dfrac{25\pi}{225\pi} =\dfrac{25}{225} =\dfrac19$. \\ [5pt]
      Un tireur débutant atteint la cible une fois sur deux, et dans le cas où il l'atteint, il atteint le c\oe ur une fois sur 9. \\
      On a donc $\mathcal{P} =\dfrac12\times\dfrac19 =\dfrac{1}{18}$. \\
      \bm{La probabilité que le tireur débutant atteigne le c\oe ur de la cible est donc de $\dfrac{1}{18}$.}
   \end{enumerate}
