\ \\ [-5mm]
\begin{enumerate}
   \item On a $d = e$  et $c = d+e$, donc : $d = e = \dfrac12c$ \; puis \; $b = d + c = \dfrac12c + c = \dfrac32c$ \; et enfin \; $a = b + c = \dfrac32c + c = \dfrac52c$. \\
   \bm{$a = \dfrac52c$ \, ; \, $b = \dfrac32c$ \, ; \, $d = e = \dfrac12c$}
   \smallskip
   \item L'aire du rectangle vaut : \\
   $a^2+b^2+c^2+d^2+e^2 =\left(\dfrac52c\right)^2+\left(\dfrac32c\right)^2+c^2+\left(\dfrac12c\right)^2+\left(\dfrac12c\right)^2$ \\ [1mm]
   \hspace*{3.05cm} $=\dfrac{25}{4}c^2+\dfrac94c^2+c^2+\dfrac14c^2+\dfrac14c^2$ \\ [1mm]
   \hspace*{3.05cm} $=\dfrac{40}{4}c^2=10c^2$. \\
   On a alors $10c^2 =\ucmq{3610} \iff c^2 =\dfrac{\ucmq{3610}}{10} =\ucmq{361}$, donc $c=\ucm{19}$. \\
   La largeur vaut $b+c =\dfrac32\times\ucm{19}+\ucm{19} = 47,5$ cm, la longueur $a+b =\dfrac52\times\ucm{19}+\dfrac32\times\ucm{19}= \ucm{76}$. \\ [1mm]
   \bm{La feuille a pour dimensions 76 cm par 47,5 cm.}
   \item La plaque métallique est homogène, ce qui signifie que la masse est proportionnelle à sa surface. \\
   On sait que $a=\dfrac52c$ et $b=\dfrac32c$, donc $a=\dfrac53b$. \\ [1mm]
   Or, l'aire de A vaut $a^2 =\dfrac{25}{9}b^2$, ce qui signifie que la masse de A se calcule ainsi : $\dfrac{25}{9}\times\ug{100} \approx\ug{277,78}$. \\ [1mm]
   \bm{La masse de la pièce A est d'environ 277,8 grammes.}
   \item On a $a =\dfrac52c$, donc $c=\dfrac25a$ et $c^3=\left(\dfrac25\right)^3a^3 =\dfrac{8}{125}a^3$. D'où le volume du cube est de $\dfrac{8}{125}\times\umc{2} =\ucmc{0,128}$. \\ [1mm]
   \bm{Le volume du cube C est de \udmc{128}.}
\end{enumerate}
