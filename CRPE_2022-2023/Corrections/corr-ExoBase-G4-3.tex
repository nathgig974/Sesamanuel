\ \\ [-5mm]
   \begin{enumerate}
      \item On obtient les deux vues de droite, puis de dessus suivantes : \\ \medskip
         \begin{tabular}{C{6}C{1}C{8}}
            \begin{pspicture}(-2,-0.5)(4,6)
               \psframe(4,4)
               \psframe(2,0)(4,2)
               \psframe(0,4)(2,6)
               \psframe(0,2)(-2,4)
            \end{pspicture}
            & &
            \begin{pspicture}(0,-0.5)(8,6)
               \psframe(0,2)(2,4)
               \psframe(2,2)(6,6)
               \psframe(6,4)(8,6)
               \psframe(4,0)(6,2)
            \end{pspicture} \\
         \end{tabular}
      \item
      \begin{enumerate}
         \item Le polynôme P possède {\blue 9 faces}.
         \item Un exemple de patron à l'échelle 1/2 :
            \begin{pspicture}(-4,0)(8,6.7)
               \psframe(0,2)(2,4)
               \psframe(2,0)(4,2)
               \psframe(6,2)(8,4)
               \pspolygon(2,2)(4,2)(4,7)(3,7)(3,6)(2,6)(2,5)(3,5)(3,4)(2,4)
               \psline(4,2)(6,2)
               \psline(4,6)(3,6)(3,5)(5,5)(5,4)(7,4)
               \psline(8,4)(8,6)(7,6)(7,5)(6,5)(6,4)
            \end{pspicture}
      \end{enumerate}
   \end{enumerate}
