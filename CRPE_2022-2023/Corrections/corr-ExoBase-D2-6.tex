\ \\ [-5mm]
\begin{enumerate}
   \item Arthur parcourt 6 kilomètres en 1 heure, soit 6 000 mètres en 60 minutes, ou encore 100 mètres en 1 minute. \\
      {\blue Arthur avance à une vitesse de 100 m/min.} \\
      Boz parcourt 24 kilomètres en 1 heure, soit 24 000 mètres en 60 minutes, ou encore 400 mètres en 1 minute. \\
      {\blue Boz avance à une vitesse de 400 m/min.}
   \item
      \begin{enumerate}
         \item On note O, origine du repère, l'endroit où se trouve Arthur au départ, à 9 heures.
            \begin{itemize}
               \item Sur l'axe des abscisses, on représente la distance des robots au point O. À 9 heures, Arthur est en O alors que Boz est à 300 mètres du point O. On choisit (par exemple) une échelle de 1 cm pour 20 mètres en abscisse.
               \item Sur l'axe des ordonnées, on représente le temps écoulé, en minutes, à partir de 9 h 00. Pour parcourir les 300 mètres, le robot le plus lent : Arthur, mettra 3 minutes. On choisit 1 cm pour 30 secondes en ordonnée.
               \item Courbe d'Arthur : sa vitesse est uniforme, sa courbe est donc une droite passant par l'origine et par le point de coordonnées (100;1) puisqu'il a une vitesse de 100 m/min.
               \item Courbe de Boz : sa vitesse est uniforme également, sa courbe est donc une droite passant par son point de départ représenté en (300;0). Sa vitesse est de 400 m/min, il aura donc parcouru 200 mètres en 0,5 minute, graphiquement, cela correspond au point de coordonnées (100;0,5).
            \end{itemize}
            {\psset{unit=0.9cm}
            \begin{pspicture}(-0.9,-1.5)(16,7.7)
               \psgrid[subgriddiv=10, gridlabels=0, gridwidth=0.4pt, subgridwidth=0.4pt,gridcolor=brown!80,subgridcolor=brown!40](-1,-1)(16,7)
               \psaxes[dx=5,Dx=100,dy=2]{->}(0,0)(16,7)
               \rput(14,-1){distance en mètres de O}
               \rput(2.7,6.5){minutes écoulées depuis 9 h 00}
               \psline[showpoints=true,linecolor=B2](0,0)(5,2)(10,4)(15,6)
               \psline[showpoints=true,linecolor=A1](15,0)(10,0.5)(5,1)(0,1.5)
               \rput(14,0.8){\textcolor{A1}{déplacement de Boz}}
               \rput(14,6.5){\textcolor{B2}{déplacement d'Arthur}}
               \psline[linestyle=dashed](3,1.2)(3,0)
               \psline[showpoints=true,linestyle=dashed]{*->}(3,1.2)(0,1.2)
               \rput(-0.5,1.2){\fbox{0,6}}
            \end{pspicture}}
         \item Graphiquement, il suffit de trouver le point d'intersection des courbes de déplacement des deux robots, puis de lire son ordonnée : on trouve 0,6 minutes, soit 36 secondes. \\
            {\blue Les deux robots se rencontreront à 9 h 00 min 36 s.}
         \end{enumerate}
         \smallskip
      \setcounter{enumi}{2}
      \item On sait que $v=\dfrac{d}{t}$, donc $d=v\times t$ où on exprimera $v$ en mètre par minute, $d$ en mètre et $t$ en minute. \\ [1mm]
         Pour Arthur, on a $d_A =100\,t_A$ ; pour Boz, on a $d_B=400\,t_B$. \\
         Au point de rencontre, le même temps se sera écoulé, donc $t=t_A=t_B$ ; \\
         de plus, la distance totale parcourue par les deux robots sera de 300 mètres, soit : \\
         $d_A+d_B=300 \iff 100t+400t =300$ \\
         \hspace*{2.1cm} $\iff 500t=300$ \\
         \hspace*{2.1cm} $\iff t=\dfrac{300}{500} =0,6$. \\ [1mm]
         On obtient un temps écoulé de 0,6 minute, soit $0,6\times\us{60} = \us{36}$. \\ [1mm]
         {\blue Les deux robots se rencontreront à 9 h 00 min 36 s.}
   \end{enumerate}
