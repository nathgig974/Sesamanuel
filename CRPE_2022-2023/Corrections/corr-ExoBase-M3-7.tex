\ \\ [-5mm]
\begin{enumerate}
   \item L'échelle 1/1\,000\up{e} signifie que \ucm{1000} = \um{10} dans la réalité sont représentés par \ucm{1} sur le plan. \\
    {\psset{unit=1}
      \begin{pspicture}(-3,-0.5)(7.5,3.5)
         \pspolygon(0,0)(5,0)(7,0)(5,3)(0,3)
         \psline[linestyle=dashed](5,0)(5,3)
         \psframe(0,0)(0.3,0.3)
         \psframe(0,3)(0.3,2.7)
         \psframe(5,0)(5.3,0.3)
         \rput(-0.3,-0.3){D}
         \rput(-0.3,3.3){A}
         \rput(5,-0.3){E}
         \rput(7.3,-0.3){C}
         \rput(5.3,3.3){B}
      \end{pspicture}}
   \item
   \begin{enumerate}
      \item Dans le trapèze ABCD, calculons BC. Le triangle BEC est rectangle en E, on utilise le théorème de Pythagore avec des mesures de longueur en mètre. \\
      BC$^2$ = BE$^2$ + EC$^2 \iff \text{BC}^2 = 30^2+20^2 =1\,300 \iff \text{BC} =10\sqrt{13}$. \\
      Le périmètre du jardin vaut alors : AB + BC + CD + DA = $\um{50}+10\sqrt{13}\,\um{}+\um{70}+\um{30} =(150+10\sqrt{13})\,\um{}$. \\
      À ce périmètre, il faut enlever la largeur du portail, donc, la longueur de la clôture est de \bm{$(146,90+10\sqrt{13})\,\um{} \approx\um{183}$}
      \item
      \begin{itemize}
         \item Aire de potager : $\mathcal{A}_p =\dfrac{\text{EC}\times\text{EB}}{2} =\dfrac{\um{20}\times\um{30}}{2} =$ \bm{$\umq{300}$}.
         \item Aire des plantations florales : $\mathcal{A}_f =\dfrac12\times\pi\left(\dfrac{\text{AB}}{2}\right)^2=\dfrac12\times\pi\left(\dfrac{\um{50}}{2}\right)^2 =\dfrac{625}{2}\pi\,\umq{} \approx$ \bm{$\umq{982}$}.
         \item Aire du gazon : $\mathcal{A}_g =\text{DE}\times\text{DA}-\mathcal{A}_f =\um{50}\times\um{30}-\dfrac{625}{2}\pi\,\umq{} =\umq{1500}-\dfrac{625}{2}\pi\,\umq{} \approx $ \bm{$\umq{518}$}.
      \end{itemize}
   \end{enumerate}
\end{enumerate}
