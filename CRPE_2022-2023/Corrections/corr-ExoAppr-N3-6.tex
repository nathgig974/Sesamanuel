\ \\ [-5mm]
   \begin{enumerate}
      \item
      \begin{itemize}
         \item 7 n'est divisible que par 1 et 7, c'est un nombre premier ;
         \item 13 n'est divisible que par 1 et 13, c'est un nombre premier ;
         \item $57 =3\times19$ est divisible par 1, 3, 19 et 57, ce n'est pas un nombre premier ;
         \item 61 n'est divisible que par 1 et 61, c'est un nombre premier.
      \end{itemize}
      \bm{Parmi ces quatre nombres, seul 57 n'est pas un nombre premier.}
      \item
      \begin{enumerate}
         \item $3\,737 =37\times100+37 =37\times101$, donc il est divisible par d'autres nombres que 1 et lui même. \\
         \bm{3\,737 n'est pas un nombre premier.}
         \item $\overline{abab} =\overline{ab}\times100+\overline{ab} =\overline{ab}\times101$, il est donc divisible par d'autres nombres que 1 et lui même. \\
         \bm{Un nombre du type $\overline{abab}$ n'est pas un nombre premier.}
      \end{enumerate}
      \item
      \begin{enumerate}
         \item $\overline{abc}+\overline{abb}+\overline{acc} =(100a+10b+c)+(100a+10b+b)+(100a+10c+c)$ \\
         \hspace*{3.1cm} $=300a+21b+12c$ \\
         \hspace*{3.1cm} $=3(100a+7b+4c)$ \\
          \bm{La somme de $\overline{abc}, \overline{abb}$ et $\overline{acc}$ est un nombre divisible par 3.}
         \item Soit $\overline{mnp}$ le nombre recherché. \\
         $\overline{cba}+\overline{bba}+\overline{mnp} =(100c+10b+a)+(100b+10b+a)+(100m+10n+p)$ \\
         $\phantom{\overline{cba}+\overline{bba}+\overline{mnp}} =100c+120b+2a+100m+10n+p$ \\
         $\phantom{\overline{cba}+\overline{bba}+\overline{mnp}} =3(33c+40b+33m+3n)+(c+2a+m+n+p)$ \\
         Il reste à choisir $m, n$ et $p$ pour que $c+2a+m+n+p$ soit divisible par 3 : on a déjà deux $a$ et un $c$, on peut donc prendre un $a$ supplémentaire et deux autres $c$. On choisit un nombre comportant les chiffes $a, c$ et $c$, par exemple $\overline{cac}$ (mais on pourrait également choisir $\overline{acc}$ ou $\overline{cca}$) et on vérifie qu'il convient : \\
         $\overline{cba}+\overline{bba}+\overline{cac} =(100c+10b+a)+(100b+10b+a)+(100c+10a+c)$ \\
         $\phantom{\overline{cba}+\overline{bba}+\overline{cac}} =201c+120b+12a$ \\
         $\phantom{\overline{cba}+\overline{bba}+\overline{cac}} =3(67c+40b+4a)$ qui est bien multiple de 3. \\
          \bm{La somme de $\overline{cba}, \overline{bba}$ et de $\overline{cac}$ est un nombre divisible par 3.}
      \end{enumerate}
   \end{enumerate}
