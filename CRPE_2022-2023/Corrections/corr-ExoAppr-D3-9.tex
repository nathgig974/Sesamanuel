\ \\ [-5mm]
\begin{enumerate}
   \item $m =\dfrac{4\times0+\dots+1\times42}{4+\dots+1} =\dfrac{187,3}{30} =6,2433 \approx 6,2$. \\ [1mm]
   \bm{Il a plu en moyenne environ 6,2 mm au mois d'avril 2016}.
   \item Il y a 30 valeurs, il faut donc prendre un nombre entre la 15\up{ième} valeur (qui est 1,7 mm) et la 16\up{ième} valeur (qui est également 1,7 mm) donc, \bm{la médiane vaut 1,7 mm}. \\
   La moitié des précipitations journalière sont inférieures ou égales à 1,7 mm et la moitié des précipitations journalières sont supérieures ou égales à 1,7 mm.
   \item $42-0 =42$ donc, \bm{l'étendue vaut 42}.
      \item La hauteur des précipitations est supérieure ou égale à 13 mm \bm{pendant 6 jours}, ce qui correspond à un pourcentage de $\dfrac{6}{30}\times100 =$\bm{$20\,\%$}. \\ [1mm]
      \item Il a plu 187,3 mm d'eau durant le mois d'avril 2016. Ceci sur une surface totale de 3 200 m par 50 m, soit $3\,200\text{ m}\times50\text{ m} =160\, 000\text{ m}^2$. \\
      Ce qui donne un volume de $160\,000\text{ m}^2\times0,1873\text{ m} =29\,968\text{ m}^3$. \\
      Or, pour transformer un volume en capacité, on utilise généralement la correspondance suivante : \\
      1 dm$^3$ = 1 L. \\
      Soit $29\,968\text{ m}^3 =29\,968\,000\text{ dm}^3\approx 30\,000\,000\text{ L}$. \\
      \bm{Au cours du mois d'avril, il est tombé \udmc{29968} soit près de 30\,000\,000 L d'eau sur l'aéroport de Rolland Garros}. \\ [5mm]
\end{enumerate}
