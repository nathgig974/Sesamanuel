\ \\ [-5mm]
   \begin{enumerate}
      \item
         \begin{enumerate}
            \item $\bullet$ Calcul de la longueur DE, en cm : nous sommes en présence d'un pavé, donc le triangle DHE est rectangle en H. D'après le théorème de Pythagore, $\text{DE}^2 =\text{DH}^2+\text{HE}^2 =12^2+9^2 =225$, donc : $\text{DE} =15$. \\
               $\bullet$ Calcul de la longueur DG : le triangle DHG rectangle en H est isométrique au triangle DHE rectangle en H donc : {\blue DG = DE = \ucm{15}}.
            \item {\blue Le triangle DGF est rectangle en G et le triangle DEF est rectangle en E.} car l'arête [FG] est perpendiculaire à la face CDHG, elle est alors perpendiculaire à la droite (GD) contenue dans ce plan.
            \item Exemple de patron à l'échelle 1/3 : \\
            {\small
               \begin{pspicture}(-7,-4)(8,7.7)
                  \pspolygon(0,0)(3,0)(3,3)(0,3)
                  \pspolygon(0,0)(-4,0)(0,3)(0,8)(3,3)(8,0)(3,0)(0,-4)
                  \rput(-0.3,-0.3){H}
                  \rput(3.3,-0.3){E}
                  \rput(3.3,3.3){F}
                  \rput(-0.3,3.3){G}
                  \rput(-4.3,0){D}
                  \rput(-0.3,-4){D}
                  \rput(8.3,0){D}
                  \rput(-0.3,8){D}
               \end{pspicture}
            }
         \end{enumerate}
      \setcounter{enumi}{1}
      \item
         \begin{enumerate}
            \item La pyramide DJKLM est une réduction de la pyramide DEFGH. La base de cette dernière étant un carré, il en est de même pour la base de la pyramide réduite. {\blue JKLM est un carré}.
            \item Les longueurs JK et JM sont égales puisque JKLM est un carré. \\ [1mm]
               Calculons le coefficient de réduction de la pyramide : $\dfrac{\text{DJ}}{\text{DH}} =\dfrac{\ucm{12}-\ucm{2}}{\ucm{12}} =\dfrac{10}{12} =\dfrac56$. \\ [1mm]
               Donc, la longueur du côté du carré JKLM mesure : $\dfrac56\times\ucm{9}  =\ucm{7,5}$. On a {\blue JK = JM = \ucm{7,5}}.
            \item Avec des mesures de longueur en cm et des mesures de volumes en \ucmc{}, on a :
               \begin{itemize}
                  \item volume $B$ du sable blanc : $B =\dfrac13\times\text{aire du carré JKLM}\times\text{ hauteur} =\dfrac13\times7,5^2\times10 =187,5$ ;
                  \item volume $V$ de la pyramide DHEFG : c'est un agrandissement de coefficient $\dfrac65$ de la pyramide DJKLM, donc : $V =\left(\dfrac65\right)^3\times B =\dfrac{6^3}{5^3}\times187,5 =324$ ; \smallskip
                  \item volume $R$ du sable rouge : $R =V-B =324-187,5 =136,5$.
            \end{itemize}
            {\blue Le sable blanc occupe un volume de \ucmc{187,5} et le sable rouge \ucmc{136,5}}.
      \end{enumerate}
   \end{enumerate}
