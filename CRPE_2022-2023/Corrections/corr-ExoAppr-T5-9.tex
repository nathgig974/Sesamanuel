\ \\ [-5mm]
\begin{enumerate}
   \item Dans \texttt{B2}, on peut écrire : \cell{\texttt{=2*PI()*B1$\wedge$2+660/B1}}
   \item Les valeurs minimales pour l'aire dans le tableau sont 264,40 et 264,41, elles correspondent à \\
   {\bf un rayon compris entre 3,7 cm et 3,8 cm.}
   \item Soit $h$ la hauteur d'une canette en centimètre, une capacité de 33 cL = 0,33 L correspond à un volume de 0,33 dm$^3$ =330 cm$^3$. \\
   On a alors $\mathcal{V}_{\text{canette}} =\pi r^2\times h \iff 330\text{ cm}^3 = \pi (3,7\text{ cm})^2\times h \iff h =\dfrac{330\text{ cm}^3}{\pi\times13,69\text{ cm}^2} \approx7,67\text{ cm}$. \\
  {\bf Une canette de 33 cL de rayon 3,7 cm a pour hauteur 7,7 cm environ.}
\end{enumerate}
