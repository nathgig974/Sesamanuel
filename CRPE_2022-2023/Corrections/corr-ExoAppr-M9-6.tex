\ \\ [-5mm]
\begin{enumerate}
   \item
   \begin{enumerate}
      \item Sur l'un des côtés, on peut mettre côte à côte le carré et le triangle puisque la mesure du côté du carré est de 4 cm et celle du côté du triangle est de 3 cm. Or, 4 cm + 3 cm = 7 cm. \\
      Pour l'un des côtés adjacents, la somme des mesures des côtés de deux carrés côte à côte fait 6 cm, ce qui est bien inférieur à 7 cm. \\
      La hauteur du triangle de côté 4 cm fait $4\dfrac{\sqrt3}{2}\text{ cm} =2\sqrt{3} \ucm{} \approx 3,46$ cm, ce qui est bien inférieur à 3,5 cm. \\ [1mm]
      On peut donc bien découper deux triangles \og l'un au dessus de l'autre \fg{}, comme sur la {\it figure 2}. \\
      \bm{Il es possible de découper deux carrés et deux triangles comme indiqué sur la figure 2.}
      \item Aire du carré de côté 3 cm : $\mathcal{A}_c =(3\text{ cm})^2 =9$ cm$^2$. \\
      Aire du triangle équilatéral de côté 4 cm en cm$^2$ : $\mathcal{A}_t =\dfrac{4\times\cancel{2}\sqrt3}{\cancel{2}} =4\sqrt3 \approx6,9$. \\ [1mm]
      \bm{Les périmètres sont égaux, mais les aires ne sont pas égales.}
   \end{enumerate}
   \item Si $x$ et $y$ sont solutions du problème, alors :
      \begin{itemize}
         \item le périmètre du carré et du triangle sont égaux, ce qui se traduit par $4\times x =3\times y$, soit \bm{$4x-3y =0$} ;
         \item la somme des longueurs d'un côté du carré et d'un côté du triangle est de 7 cm, ce qui se traduit par \bm{$x+y =7$} ;
         \item deux carrés doivent tenir sur un même côté du grand carré de 7 cm de côté, ce qui se traduit par \bm{$2x\leqslant7$} ;
         \item la somme des longueurs de deux hauteurs du triangle est inférieure à 7 cm, ce qui se traduit par $\cancel{2}\times\dfrac{y\sqrt3}{\cancel{2}}\leqslant7$ \bm{$y\sqrt3\leqslant7$}.
      \end{itemize}
   $\begin{cases} x+y =7 \\ 4x-3y =0  \\ \end{cases}
   \iff \begin{cases} x =7-y \\ 4(7-y)-3y =0 \end{cases}
   \iff \begin{cases} x =7-y \\ 28-7y =0 \end{cases}
   \iff \begin{cases} x =7-y \\ y =4 \end{cases}
   \iff \begin{cases} x =3 \\ y =4 \end{cases}$. \\ [1mm]
   \bm{L'unique solution du problème est le couple $(3,4)$.}
\end{enumerate}
