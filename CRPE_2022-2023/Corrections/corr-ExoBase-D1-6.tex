\ \\ [-5mm]
   \begin{enumerate}
      \item En D4, on entre : {\blue\texttt{=B4*C4}}
      \item en D11, on entre : {\blue\texttt{=SOMME(D4:D9)}}
      \item Cette formule est une formule conditionnelle : si le total HT est inférieur à 5 000 \euro, la remise sera de 5\,\% (0,05), sinon, la remise sera de 10\,\% (0,10).
      \item En D12, on entre {\blue\texttt{=D11*B12}}
      \item Non, car sinon, le calcul de la TVA se ferait sur la remise. En D13, on entre {\blue\texttt{=D11*B13}} \\
   Remarque : la recopie aurait été possible si on avait fixé la cellule D11, au moins au niveau de la ligne avec par exemple la formule {\blue\texttt{=D\$11*B12}}
      \item En D15, on entre {\blue\texttt{=D11-D12+D13}}
      \item Si on réduit le nombre d'ordinateurs à 3, les cellules suivantes changent : D7 (somme payée pour les ordinateurs), D11 (prix HT), B12 (pourcentage de la remise), D12 (remise), D13 (montant de la TVA) et D15 (prix total). Le prix à payer est de {\blue \ueuro{5 684,16}}.
   \end{enumerate}
