\ \\ [-5mm]
\begin{enumerate}
   \item Dans le triangle ABC rectangle en A, on applique le théorème de Pythagore avec des mesures en cm : \\
   $\text{BC}^2=\text{BA}^2+\text{AC}^2 =8^2+6^2 =64+36 =100 \Longrightarrow$ \bm{BC =10 cm}.
   \item Dans le triangle ABC rectangle en A, on utilise le cosinus : \\
   $\cos(\widehat{ABC}) =\dfrac{BA}{BC} =\dfrac{8}{10} =\dfrac45 \Longrightarrow \widehat{ABC} =\cos^{-1}\left(\dfrac45\right) \approx$ \bm{37\degre}.
   \item Le quadrilatère ADEF possède trois angles droits, c'est donc un rectangle. Or, les diagonales d'un rectangle sont de même longueur d'où : \bm{AE = DF}.
\end{enumerate}
