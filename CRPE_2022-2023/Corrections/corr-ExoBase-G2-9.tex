   Dans tout l'exercice, les longueurs sont exprimées en \ucm{} et les aires en \ucmq{}. \\
   \begin{enumerate}
      \item Dans le triangle $ABC$, on a d'une part : $AB^2+BC^2 =6^2+8^2 =36+64 =100$ \\
         et d'autre part : $AC^2 =10^2 =100$. \\
         On a alors $AB^2+BC^2 =AC^2$, et d'après la réciproque du théorème de Pythagore, on peut affirmer que le triangle $ABC$ est rectangle en $B$. \\
         Par conséquent, {\blue les droites $(AB)$ et $(BC)$ sont perpendiculaires}.
      \item Dans le triangle $ABD$ rectangle en $B$, on applique le théorème de Pythagore : \\
         $AD^2 =AB^2+BD^2 \iff 8^2 =6^2+BD^2$. \\
         D'où : $BD^2 =64-36 =28 \Longrightarrow BD =\sqrt{28} =2\sqrt7 \approx5,2915$. \\
         Conclusion : {\blue la longueur $BD$ vaut exactement $2\sqrt7\ucm{}$, soit approximativement \ucm{5,3}}.
      \item Les droites $(AB)$ et $(CE)$ dont toutes les deux perpendiculaires à la droite $(BC)$, elles sont donc parallèles entre elles. On a alors : $B, D, C$ et $A, D, E$ alignés dans cet ordre ; les droites $(AB)$ et $(CE)$ sont parallèle. \\ [1mm]
         D'après le théorème de Thalès et sa conséquence, $\dfrac{DC}{DB} =\dfrac{DE}{DA} =\dfrac{CE}{BA} \iff \dfrac{8-2\sqrt7}{2\sqrt7} =\dfrac{DE}{8} =\dfrac{CE}{6}$. \\
         On cherche $CE$ : $CE =\dfrac{6\times(8-2\sqrt7)}{2\sqrt7} \approx3,071$. {\blue La longueur $CE$ vaut environ \ucm{3,1}}. \smallskip
      \item Si on choisit $[CE]$ comme base du triangle $ACE$, alors sa hauteur associée est $[BC]$. \\ [1mm]
         On a alors : $\mathcal{A}(ACE) =\dfrac{CE\times BC}{2} \approx\dfrac{3,071\times8}{2} \approx12,284$. {\blue L'aire du triangle $ACE$ vaut environ \ucmc{12,3}}.
   \end{enumerate}
