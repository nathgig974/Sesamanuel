   {\bf Aire des figures :} l'aire du carré vaut $8\,u.l.\times8\,u.l. =64\,u.a.$ et l'aire du rectangle vaut $13\,u.l.\times5\,u.l. =65\,u.a.$ \\
      On a donc \og 64 = 65 \fg{} et on a \og gagné \fg{} un carré unité entre les deux configurations. \\
   {\bf Explication du paradoxe :} dans le deuxième puzzle, on se place au niveau du trapèze gris et du triangle bleu que l'on schématise.
      \begin{center}
      {\psset{unit=0.5}
      \small
         \begin{pspicture}(0,-1)(13,4.5)
            \psgrid[subgriddiv=1,gridlabels=0](0,0)(13,5)
            \pspolygon[linewidth=0.8mm,linecolor=B1](0,0)(5,0)(5,3)(0,5)
            \psline[linewidth=0.8mm,linecolor=B1](5,0)(13,0)(5,3)
            \rput(-0.5,-0.5){$C$}
            \rput(5,-0.5){$B$}
            \rput(13.5,0){$A$}
            \rput(-0.5,5){$E$}
            \rput(5.5,3.5){$D$}
         \end{pspicture}
      }
      \end{center}
      On a : $\dfrac{AB}{AC} =\dfrac8{13}$ et $\dfrac{BD}{CE} =\dfrac35$. Or, $\dfrac8{13}\neq\dfrac35$, donc, l'une des hypothèses du théorème de Thalès n'est pas vérifiée.
      \medskip
      \begin{itemize}
         \item Les droites $(CE)$ et $(BD)$ sont parallèles puisque le quadrilatère $CBDE$ est un trapèze.
         \item Le triangle $DBA$ est rectangle en $B$ donc, $(BD) \perp (BA)$ et le quadrilatère $CBDE$ est rectangle en $B$ et $C$ donc, $(BD) \perp (BC)$, d'où $(BA)$ est parallèle à $(BC)$ et les points $A, B, C$ sont alignés dans cet ordre.
         \item Par conséquent, la seule condition non vérifiée est que les points $A, D, E$ ne sont pas alignés, et le carré unité supplémentaire se situe \og autour \fg{} de la diagonale $AE$.
      \end{itemize}
      {\bf Visuel :} si on affine le contour des figures, on \og voit \fg{} bien où se situe le problème.
      \begin{center}
      {\psset{unit=0.9}
         \begin{pspicture}(0,0)(13,5)
            \pspolygon[fillstyle=solid,fillcolor=FondTableaux,linewidth=0.1mm](0,0)(5,0)(5,3)(0,5)
            \pspolygon[fillstyle=solid,fillcolor=G1,linewidth=0.1mm](13,0)(13,5)(8,5)(8,2)
            \pspolygon[fillstyle=solid,fillcolor=BleuOuv,linewidth=0.1mm](5,0)(13,0)(5,3)
            \pspolygon[fillstyle=solid,fillcolor=B2,linewidth=0.1mm](0,5)(8,5)(8,2)
            \psgrid[subgriddiv=1,gridlabels=0](0,0)(13,5)
         \end{pspicture}
      }
      \end{center}
