\ \\ [-5mm]
\begin{enumerate}
   \item {\bf Compétences} acquises par Raphaëlle  et de Terry :
   \begin{itemize}
      \item ajouter des grandeurs, en particulier des aires : ils savent déterminer une aire complexe par ajout d'aires de figures plus simples ;
      \item mesurer des grandeurs, en particulier des aires : ils savent déterminer l'aire d'un triangle rectangle, à partir de l'aire d'un carré ;
      \item manipuler des nombres décimaux (pour Raphaëlle) : en particulier, calcul de la moitié et du quart d'un entier, addition de nombres décimaux ; et des fractions (pour Terry) : transformation de 43 triangles valant un quart de \ucmq{} en \ucmq{10} plus trois quarts de \ucmq{}.
   \end{itemize}
   {\bf Erreurs} commises par Raphaëlle et Terry :
   \begin{itemize}
      \item le raisonnement de Raphaëlle est correct, mais elle fait une erreur de dénombrement des petits triangles (il lui manque un demi carré). De plus, certaines unités sont manquantes, en particulier dans la réponse ;
      \item la réponse de Terry est juste mais manque d'éléments de rédaction.
   \end{itemize}
   \item \textcolor{A1}{$\bullet$} Les deux élèves comptent correctement des petits triangles et ont conscience qu'il faut diviser ce nombre par 4 pour obtenir l'aire de la figure, ils semblent avoir acquis la compétence \og mesurer une aire \fg ;
   \begin{itemize}
      \item les deux élèves entreprennent une division mais ne vont pas suffisamment loin dans sa résolution : Clément effectue une division euclidienne mais fait une erreur d'interprétations du résultat obtenu : le quotient est 10, le reste 3, ce qu'il écrit 10,3. Il manque a Cloé une dernière étape afin de trouver le résultat exact de 10,75, certainement parce que son résultat intermédiaire (10,7) ne correspond pas à la valeur trouvée de 10,3 ;
      \item Clément utilise la division comme outil pour trouver la réponse alors que Cloé utilise la division comme élément de vérification, elle a dénombré 10 carrés entiers et 3 petits triangles qu'elle interprète comme $10 + 0,3$ ;
      \item le résultat trouvé est le même pour Clément et Cloé (mais il est faux).
   \end{itemize}
\end{enumerate}
