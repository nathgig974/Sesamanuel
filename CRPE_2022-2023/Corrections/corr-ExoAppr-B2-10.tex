\ \\ [-5mm]
\begin{enumerate}
   \item Volume du cylindre en mètre cube : $V_{\text{cylindre}} = \pi\times1,3^2\times2,4$ ; \\
   Volume du cône en mètre cube : $V_{\text{cône}} = \dfrac13\times\pi\times1,3^2\times1,6$ ; \\
   Volume du silo en mètre cube : $V =V_{\text{cylindre}}+V_{\text{cône}} =4,056\pi+\dfrac13\times2,704\pi \approx 15,5739$. \\
   \bm{Le silo a un volume d'environ 15,57 m$^3$.}
   \item Quantité de farine en litre pour 48 vaches pendant 90 jours : $C_{\text{vaches}} =90\times48\times3\text{ L} =12\,960\text{ L}$. \\
    Capacité en litre du silo : $C_{\text{silo}} \approx\dfrac67\times15,5739\text{ m}^3 \approx\dfrac67\times15\,573,9\text{ dm}^3 \approx\dfrac67\times15\,573,9\text{ L} \approx 13\,349\text{ L}$. \\
   On a $C_{\text{silo}}>C_{\text{vaches}}$ donc, \bm{l'éleveur aura suffisamment de farine pour nourrir ses 48 vaches pendant 90 jours.}
   \item Remarque : A et D  sont les centres des bases des solides de révolution, donc (SA) est orthogonale à la base du cône et (AD) est orthogonale à la base du cylindre. On a alors (SA) perpendiculaire à (AB) et (AD) perpendiculaire à (DC) et à (AB). \\
   D'où ABHS et ABCD sont des rectangles et leurs côtés sont deux à deux égaux et parallèles. \\
   Les points H, M, N et H, B, C sont alignés dans cet ordre. \\ [1mm]
   On calcule les rapports $\dfrac{\text{HN}}{\text{HM}}$ et $\dfrac{\text{HC}}{\text{HB}}$ avec les valeurs suivantes : \\ [1mm]
   HN = SN $-$ SH = SN $-$ AB = 3,3 m $-$ 1,3 m = 2 m. \\
   HM = SM $-$ SH = SM $-$ AB = 2,1 m $-$ 1,3 m = 0,8 m. \\
   HC = HB + BC = SA + AD = 1,6 m + 2,4 m = 4 m. \\
   HB = SA = 1,6 m. \\ [1mm]
   D'où $\dfrac{\text{HN}}{\text{HM}} =\dfrac{2\text{ m}}{0,8\text{ m}} =2,5$ et $\dfrac{\text{HC}}{\text{HB}} =\dfrac{4\text{ m}}{1,6\text{ m}} =2,5$. \\ [1mm]
   Les rapports sont égaux, donc d'après la réciproque du théorème de Thalès, les droites (NC) et (MB) sont parallèles. \bm{Les deux échelles sont parallèles.}
\end{enumerate}
