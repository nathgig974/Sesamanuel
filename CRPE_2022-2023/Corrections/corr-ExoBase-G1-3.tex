\ \\ [-5mm]
\begin{enumerate}
   \item Les intentions du maître sont de :
   \begin{itemize}
      \item faire émerger les caractéristiques d'un patron de cube ;
      \item faire reconnaître un patron de cube parmi différentes représentations ;
      \item faire travailler le passage du plan (le patron) à l'espace.
   \end{itemize}
   \item Un \og patron \fg{} d'un solide est une surface plane d'un seul tenant qui, par pliage, permet de reconstituer le solide, sans recouvrement de ses faces.
   \item
   \begin{itemize}
      \item L'assemblage {\bf c} n'est pas un patron du cube parce qu'il comporte sept carrés au lieu de six.
      \item Pour l'assemblages {\bf b}, les deux carrés \og du dessus \fg{} vont se chevaucher lorsqu'on va reconstituer le solide.
      \item Pour l'assemblage {\bf d}, les quatre carrés en haut à droite forment un carré ce qui est impossible puisque la somme des angles issus du sommet commun vaut $4\times90^\circ = 360^\circ$, les quatre faces sont dans le même plan.
   \end{itemize}
   \item {\bf Concernant la présentation :}
   \begin{itemize}
      \item dans l'exercice 1, on a en plus la présence d'un cube représenté en perspective ;
      \item l'exercice 1 comporte cinq assemblages alors qu'il y en a sept dans l'exercice 2 ;
      \item dans l'exercice 2, les \og traits de pliage \fg{} sont représentés par des pointillés alors que ce sont des traits pleins dans l'exercice 1 ;
      \item les figures sont plus grandes dans l'exercice 1 que dans l'exercice 2 ;
      \item dans l'exercice 1, l'un des assemblages est formé de sept carrés alors que tous ceux de l'exercice 2 n'en possèdent que six.
   \end{itemize}
   {\bf Concernant la consigne :}
   \begin{itemize}
      \item le statut des objets : dans l'exercice 1, on parle d'assemblages de carrés, alors que dans l'exercice 2 on parle de figures ;
      \item la formulation : dans l'exercice 1, la question est fermée, il s'agit de trouver les deux seuls patrons de cube. Dans l'exercice 2, le nombre de solutions n'est pas indiqué ;
      \item la réponse attendue : dans l'exercice 1, les élèves doivent reproduire deux assemblages, alors que dans l'exercice 2, ils doivent reproduire sur papier quadrillé les figures permettant de construire le cube et vérifier en essayant de le construire effectivement.
   \end{itemize}
   {\bf Concernant la vérification :}
   \begin{itemize}
      \item dans l'exercice 1 aucune vérification n'est évoquée ;
      \item à contrario, dans l'exercice 2, l'élève doit vérifier ses propositions par reconstruction du cube à partir des patrons. La reproduction de ces patrons est facilitée par l'utilisation du papier quadrillé.
   \end{itemize}
   \item Dans l'exercice 2, il est demandé à l'élève d'anticiper le résultat d'une construction à partir du patron du solide en repérant les éventuelles superpositions de faces. Dans l'exercice 3, il s'agit de situer des faces voisines du cube les unes par rapport aux autres sur le patron et dans une représentation en perspective. La résolution de l'exercice 3 nécessite le passage d'une représentation du cube en dimension 2 à une représentation du cube en dimension 3. Cet exercice peut donc être considéré comme un prolongement de l'exercice 2, à condition que les élèves aient au préalable travaillé la lecture de représentations en perspective.
   \item À l'issue de l'exercice 2, l'élève possède un patron de cube à partir de la figure E, patron qu'il peut manipuler pour reconstruire le dé. Dans l'exercice 4, il s'agit de repérer les faces opposées et la vérification se fait par construction. Ainsi, proposer l'exercice 4 à la suite de l'exercice 2, permet une continuité de l'apprentissage : reconnaître des patrons du cube, puis travailler plus précisément sur les positions relatives des différentes faces d'un de ces patrons.
\end{enumerate}
