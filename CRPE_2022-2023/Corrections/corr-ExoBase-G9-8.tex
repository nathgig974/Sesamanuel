\ \\ [-5mm]
   \begin{enumerate}
      \item Exemple de figure de l'exercice :
         {\psset{algebraic=true,unit=0.9}
         \begin{pspicture*}(-7,-5.7)(8,6)
            \pspolygon(6,0)(2,5)(-4,5)(0,0)
            \pspolygon(6,0)(2,5)(0,0)(4,-5)
            \pscircle[linecolor=A1](4.,2.5){3.2015621187164243}
            \psplot[linecolor=B2]{-5}{8}{(--12.4-2.1*x)/5.17}
            \psline[linecolor=B2](2,-6)(2,6)
            \psplot[linecolor=B2]{-5}{8}{--0.8*x}
            \rput[bl](-0.6,-0.1){A}
            \rput[bl](6.4,0){B}
            \rput[bl](2.1,5.3){C}
            \rput[bl](2.1,0.1){D}
            \rput[bl](0.4,2.4){E}
            \rput[bl](2.1,1.9){H}
            \rput[bl](-4.6,5.1){M}
            \rput[bl](3.9,-5.6){N}
         \end{pspicture*}}
      \item
      \begin{itemize}
         \item D est un point du cercle de diamètre [BC] ; donc, le triangle BCD est rectangle en D et la droite (CD) est perpendiculaire à la droite (AB). D'où, (CD) est la hauteur issue de C du triangle ABC ;
         \item E est un point du cercle de diamètre [BC] ; donc, le triangle BCE est rectangle en E et la droite (BE) est perpendiculaire à la droite (AC). D'où, (BE) est la hauteur issue de B du triangle ABC ;
      \end{itemize}
      or, les hauteurs d'un triangle sont concourantes au point H appelé orthocentre. La troisième hauteur étant (AH), elle est perpendiculaire à la droite (BC). \\
      {\blue Les droites (AH) et (BC) sont perpendiculaires.}
   \item Voir figure.
   \item
      \begin{itemize}
         \item Le quadrilatère BCMA est un parallélogramme, donc, les droites (BC) et (AM) sont parallèles, et les segments [BC] et [AM] sont de même longueur ;
         \item le quadrilatère BCAN est un parallélogramme, donc, les droites (BC) et (NA) sont parallèles, et les segments [BC] et [NA] sont de même longueur ;
      \end{itemize}
      les droites (AM) et (NA) sont toutes les deux parallèles à (BC), elles sont donc parallèles entre elles et contiennent un point en commun ; par conséquent, elles sont confondues et les points M, A et N sont alignés dans cet ordre. \\
      De plus, les longueurs des segments [AM] et [NA] sont égales à la longueur du segment [BC], soit AM = AN, ce qui signifie que soit M et N sont confondus, soit A est au milieu de [MN]. Or, M et N ne peuvent pas être confondus par construction de BC{\bf MA} et BC{\bf AN}. \\
      Conclusion : {\blue le point A est le milieu du segment [MN].}
   \end{enumerate}
