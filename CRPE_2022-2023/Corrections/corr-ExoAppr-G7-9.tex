\ \\ [-5mm]
\begin{enumerate}
   \item $\text{vitesse} =\dfrac{\text{distance}}{\text{temps}}$ donc : $v =\dfrac{200\text{ m}}{19,19\text{ s}} =\dfrac{0,2\text{ km}}{19,19\div3\,600\text{ h}} \approx37,52\text{ km/h}.$ \\ [1mm]
  {\bf La vitesse d'Usain Bolt sur 200 m est d'environ 37,5 km/h.} \\ [1mm]
   \item $v =\dfrac{d}{t} \iff t =\dfrac{d}{v} =\dfrac{42,195\text{ km}}{37,52\text{ km/h}} \approx1,125\text{ h}$. \\ [1mm]
   Or, 1,125 h = 1 h + 0,125 h \\
   \hspace*{1.8cm} = 1 h + 0,125$\times$60 min \\
   \hspace*{1.8cm} = 1 h + 7,5 min \\
   \hspace*{1.8cm} = 1h + 7 min + 30 s. \\
   {\bf À cette allure, Usain Bolt mettrait 1 h 7 min 30 s pour arriver au bout d'un marathon.} \\ [1mm]
   \item $p =\dfrac{t_{\text{Bolt}}-t_{\text{Johnson}}}{t_{\text{Johnson}}}\times100 =\dfrac{19,19\text{ s}-19,32\text{ s}}{19,32\text{ s}}\times100 \approx-0,67$. \\ [1mm]
   {\bf Usain Bolt a réduit le temps de Michael Johnson de 0,67\,\%.}
   \end{enumerate}
