Pour aller de Nantes à Paris, l'hélicoptère se déplace vers l'est de 120 ce qui correspond à deux cases, donc une case correspond à un déplacement de 60. \\
De la même manière, un déplacement vers le nord d'un case correspond à 60, les cases sont donc carrées. Donc, pour aller de Nantes à Lyon :
\begin{itemize}
   \item il faut tout d'abord se déplacer de trois cases vers l'est, soit $3\times60 =180$ ;
   \item  puis il faut s'orienter vers de sud, donc tourner à droite d'un angle de 90\degre ;
   \item enfin, il faut se déplacer vers le sud d'une case, soit de 60.
\end{itemize}
{\bf Pour aller de Nantes à Lyon, il faut remplacer les nombres 120, 270 et 60 par 180, 90 et 60.}
