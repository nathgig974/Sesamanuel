\ \\ [-5mm]
\begin{enumerate}
   \item 0,5 litre de boisson A contient $\dfrac{10}{100}\times0,5 \text{ L } =0,05 \text{ L}$ de jus d'orange. \\ [1mm]
      1,25 litres de boisson B contient $\dfrac{5}{100}\times1,25 \text{ L} = 0,0625 \text{ L}$ de jus d'orange. \\ [1mm]
      {\blue C'est la bouteille B qui contient la plus grande quantité de jus d'orange.}
   \item Dans 20 cL de boisson A, on a $0,1\times20 \text{ cL } =2 \text{ cL}$ de jus d'orange. \\
      Dans 30 cL de boisson B, on a $0,05\times30 \text{ cL } =1,5 \text{ cL}$ de jus d'orange. \\
      Donc, dans le mélange de 20 cL + 30 cL = 50 cL de boisson, il y a 2 cL + 1,5 cL = 3,5 cL de jus d'orange. \\ [1mm]
      Ce qui représente un pourcentage de $\dfrac{3,5\text{ cL}}{50\text{ cL}}\times100 =7\,\%$. \\ [1mm]
      D'où : {\blue il y a 7\,\% de jus d'orange dans le mélange.}
   \item Les capacités sont données en cL. \\
      Soit $x$ la contenance de la boisson A et $(40-x)$ la contenance de la boisson B. \\
      La quantité de jus d'orange dans le mélange est de : \\
      $0,1\times x+0,05\times(40-x) =0,1x+2-0,05x$ \\
      \hspace*{3.6cm} $=0,05x+2$, \\ [1mm]
      soit un taux en pourcentage de : \\ [1mm]
      $\dfrac{0,05x+2}{40}\times100 =(0,05x+2)\times2,5$. \\ [1mm]
      \hspace*{2.35cm} $=0,125x+5$. \\
      Ce taux doit être égal à 8\,\%, d'où l'équation : \\
      $0,125x+5 =8 \iff 0,125x= 8-5 =3$ \\ [1mm]
      \hspace*{2.2cm} $\iff x =\dfrac{3}{0,125} =24$. \\ [1mm]
      On a alors $40-x =40-24 =16$. \\
      {\blue Pour avoir 8\,\% de jus d'orange dans le mélange, il faut 24 cL de boisson A et 16 cL de boisson B.}
   \item Dans la case C14, on a la formule \cell{\texttt{=(0,1*A14+0,05*B14)/40}}. \\
      On a A14 = 12 et B14 = 28 donc, le nombre obtenu dans la cellule C14 représente le taux de jus d'orange dans un mélange composé de 12 cL de boisson A et de 28 cL de boisson B. \\
      On obtient $(0,1\times12+0,05\times28)\div40 =2,6\div40 = 0,065$, soit 6,5\,\%. \\
      {\blue Si on mélange 12 cL de boisson A avec 28 cL de boisson B, on obtient une concentration de 6,5\,\% de jus d'orange dans le mélange.}
\end{enumerate}
