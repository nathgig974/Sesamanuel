\ \\ [-5mm]
   \begin{enumerate}
      \item On note $\mathcal{A}_i$ l'aire initiale et $\mathcal{A}_f$ l'aire finale après transformation. Soit $\ell$ et $L$ respectivement la largeur et la longueur d'un rectangle. L'aire initiale du rectangle est $\mathcal{A}_i =\ell L$. \\
         Une réduction de 20\% correspond à un coefficient multiplicateur de $1-\dfrac{20}{100} =0,8$ et une réduction de 10\% correspond à un coefficient multiplicateur de $1-\dfrac{10}{100} =0,9$. \\
          D'où $\mathcal{A}_f =0,8\ell\times0,9L =0,72\ell L =\left(1-\dfrac{28}{100}\right)\mathcal{A}_i$ ce qui correspond à une diminution de 28\%. \\
         {\blue L'affirmation est vraie.}
      \item On note $P_i$ le périmètre initial et $P_f$ le périmètre final après transformation. En reprenant les coefficients multiplicateurs de l'item précédent, on trouve : \\
         $P_i =2(6\text{ cm}+9\text{ cm}) =30$ cm et $P_f =2(0,8\times6\text{ cm}+0,9\times9\text{ cm}) =25,8$ cm. \\
         Or, une diminution de 15\% donnerait un périmètre $P_f =0,85\times29\text{ cm} =24,65$ cm ce qui n'est pas le cas. \\
          {\blue L'affirmation est fausse.}
       \item Un baisse de 30\,\% suivie d'une baisse de 20\,\% correspond à un coefficient multiplicateur de \\ [1mm]
          $\left(1-\dfrac{30}{100}\right)\times\left(1-\dfrac{20}{100}\right) =0,7\times0,8 =0,56 =1-\dfrac{44}{100}$. Soit une baisse de 44\,\%. \\ [1mm]
          {\blue L'affirmation est fausse.}
       \item Une baisse de 30\,\% correspond à un coefficient multiplicateur de $1-\dfrac{30}{100} =0,7$ et une hausse de 50\,\% correspond à un coefficient de $1+\dfrac{50}{100} =1,5$. Le coefficient résultant de ces deux variations vaut $0,7\times1,5 =1,05 =1+\dfrac{5}{100}$ ce qui correspond à une augmentation de 5\,\%. \\ [1mm]
         {\blue L'affirmation est vraie.}
      \item Une augmentation de 5\,\% correspond à un coefficient multiplicateur de $1+\dfrac{5}{100} =1,05$. \\
         Au bout de 15 ans, le coefficient sera de $1,05^{15} \approx2,08 >2$. \\ [1mm]
         {\blue L'affirmation est vraie.}
      \item Soit $x$ le prix initial, $x\times\dfrac{30}{100} =48 \iff x =160$. Le prix initial est de 160 \euro{}, auquel on enlève la réduction \\ [1mm]
   de 48 \euro{} ce qui donne bien 112 \euro. \\ [1mm]
       {\blue L'affirmation est vraie.} \\ [3mm]
   \end{enumerate}
