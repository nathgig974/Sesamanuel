\ \\ [-5mm]
\begin{enumerate}
   \item On peut, par exemple, utiliser une méthode basée sur la décomposition de la figure : l'aire du polygone $ABCDEF$ est la somme des aires du carré $BCDF$ et des deux triangles $DEF$ et $ABF$. \\
   On notera $\mathcal{A}(P)$ l'aire du polygone $P$. \\
   On a alors, avec des mesures de longueurs en $u.\ell.$ et des mesures d'aires en $u.a.$ : \\
   $\mathcal{A}(BCDF) =5\times5 = 25$ \quad ; \quad $\mathcal{A}(DEF) =\dfrac{5\times5}{2} =\dfrac{25}{2}$ \quad et \quad $\mathcal{A}(ABF) =\dfrac{5\times4}{2} =10$. \\ [1mm]
   Donc, $\mathcal{A}(ABCDEF) =25+\dfrac{25}{2}+10 = \dfrac{95}{2} =47,5$. \\ [1mm]
\bm{L'aire du polygone $ABCDEF$ est de $47,5\,u.a.$}
   \item On a : $i =37$ et $b =23$ donc, $\mathcal{A} =37+\dfrac{23}{2}-1 =\dfrac{95}{2} =47,5$. \\ [1mm]
   \bm{En utilisant la formule de Pick, on retrouve l'aire calculée précédemment.}
   \item Modélisation des aires : \\
   {\psset{unit=0.6}
   \begin{pspicture}(5,-0.5)(18.5,11.5)
      \pstGeonode[CurveType=polygon,PosAngle={45,180,-135,-45,45,45}](9,10){A}(7,6){B}(7,1){C}(12,1){D}(17,6){E}(12,6){F}
   \pspolygon[fillstyle=solid,fillcolor=lightgray](9,10)(7,6)(7,1)(12,1)(12,6)
     \psgrid[griddots=1,gridlabels=0,subgriddiv=1,gridwidth=0.5mm](6,0)(18,11)
   \end{pspicture}
   \begin{pspicture}(5,-0.5)(18,11.5)
      \pstGeonode[CurveType=polygon,PosAngle={45,180,-135,-45,45,45}](9,10){A}(7,6){B}(7,1){C}(12,1){D}(17,6){E}(12,6){F}
   \pspolygon[fillstyle=solid,fillcolor=lightgray](12,1)(17,6)(12,6)
      \psgrid[griddots=1,gridlabels=0,subgriddiv=1,gridwidth=0.5mm](6,0)(18,11)
   \end{pspicture}} \\
   Pour le polygone $ABCDF$, on trouve $i =27$ et $b=18$ donc $\mathcal{A}_1 =27+\dfrac{18}{2}-1 =35$ ; \\ [1mm]
   pour le polygone $DEF$, on trouve $i =6$ et $b=15$ donc $\mathcal{A}_2 =6+\dfrac{15}{2}-1 =\dfrac{25}{2}$. \\ [1mm]
   $\mathcal{A}_1+\mathcal{A}_2 =35+\dfrac{25}{2} =\dfrac{95}{2}.$ \\ [1mm]
   \bm{La somme des résultats obtenus est égale au résultat trouvé à la question 1).}
   \item Expression de $b$ : sur une longueur du rectangle, on a ($L+1$) points sur le bord ; sur une largeur du rectangle, on a ($\ell+1$) points sur le bord ; donc, sur le bord entier, on a $2\times(L+1)+2\times(\ell+1)-4$ points (puisqu'en procédant ainsi, on aura compté deux fois les 4 points situés dans les coins du rectangle). \\
   Cela donne $b =2L+\cancel{2}+2\ell+\cancel{2}-\cancel{4}=2(L+\ell)$. \\
   Expression de $i$ : à l'intérieur du rectangle, on a ($L-1$) points dans la longueur et ($\ell-1$) points dans la largeur ; ce qui nous donne $(L-1)\times(\ell-1)$ points à l'intérieur du rectangle. \\
   \bm{On obtient $b =2(L+\ell)$ et $i =(L-1)\times(\ell-1)$.} \\
   On calcule d'une part : $i+\dfrac{b}{2}-1 =(L-1)\times(\ell-1)+\dfrac{\cancel{2}(L+\ell)}{\cancel{2}}-1 =L\times\ell-L-\ell+1+L+l-1 =L\times\ell$. \\ [1mm]
   Et d'autre part, l'aire $\mathcal{A}$ d'un rectangle de côtés $L$ et $\ell$ est donnée par la formule : $\mathcal{A} =L\times\ell$. \\
   \bm{L'aire $\mathcal{A}$ du rectangle vérifie $\mathcal{A} =i+\dfrac{b}{2}-1$.} \\
   La formule de Pick est ainsi démontrée pour un tel rectangle.
\end{enumerate}
