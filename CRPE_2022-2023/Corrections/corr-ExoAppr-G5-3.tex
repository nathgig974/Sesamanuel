{\bf Situation de recherche dirigée au cycle 2 : LES POLYMINOS} \\
\begin{enumerate}
   \item Avec trois triangle, on obtient une seule solution :
   {\psset{unit=0.8}
   \begin{pspicture}(-1,0)(4,2)
      \pspolygon(0,0)(2,0)(1,1.73)
      \pspolygon(2,0)(3,1.73)(1,1.73)
      \pspolygon(3,1.73)(2,0)(4,0)
   \end{pspicture}} \\
   Avec quatre triangles, on obtient trois solutions, que l'on construit à partir de la solution précédente à laquelle on ajoute un triangle équilatéral : \\
   {\psset{unit=0.8}
   \begin{pspicture}(0,-0.5)(6,1.5)
      \pspolygon(0,0)(2,0)(1,1.73)
      \pspolygon(2,0)(3,1.73)(1,1.73)
      \pspolygon(3,1.73)(2,0)(4,0)
      \pspolygon[linecolor=B2](4,0)(5,1.73)(3,1.73)
   \end{pspicture}
   \qquad
   \begin{pspicture}(0,-0.5)(5,3.5)
      \pspolygon(0,0)(2,0)(1,1.73)
      \pspolygon(2,0)(3,1.73)(1,1.73)
      \pspolygon(3,1.73)(2,0)(4,0)
      \pspolygon[linecolor=B2](3,1.73)(2,3.46)(1,1.73)
   \end{pspicture}
   \qquad
   \begin{pspicture}(0,-2.5)(4.5,1.5)
      \pspolygon(0,0)(2,0)(1,1.73)
      \pspolygon(2,0)(3,1.73)(1,1.73)
      \pspolygon(3,1.73)(2,0)(4,0)
      \pspolygon[linecolor=B2](2,0)(0,0)(1,-1.73)
   \end{pspicture}}
   \item L'enseignant devra préciser ce qu'il appelle \og assembler les triangles exactement par un côté \fg{} en disant que deux sommets doivent également coïncider, et en présentant un exemple et des contre exemples avec des dessins. Il peut également préciser ce qu'il appelle \og des formes géométriques différentes \fg. \\
   {\psset{unit=0.8}
   \begin{pspicture}(0,0)(3,2)
      \pspolygon(0,0)(2,0)(1,1.73)
      \pspolygon(2,0)(3,1.73)(1,1.73)
   \end{pspicture}
   correct ;
   \begin{pspicture}(-2,0)(2.5,3)
      \pspolygon(0,0)(2,0)(1,1.73)
      \pspolygon[linecolor=B2](1.34,1.14)(0.36,2.86)(2.3,2.84)
   \end{pspicture}
   et
   \begin{pspicture}(-0.5,0)(4.2,2)
      \pspolygon(0,0)(2,0)(1,1.73)
      \pspolygon[linecolor=B2](4,0)(2,0)(3,1.73)
   \end{pspicture}
   incorrects}
    \item Les élèves peuvent rencontrer des difficultés à identifier les ressemblances et les différences des assemblages qu'ils réalisent. Il serait bon de laisser du papier calque à disposition. Mais cette difficulté, naturelle en cycle 2, n'a pas à être levée individuellement ou collectivement dans la phase de recherche, elle sera prise en compte au moment de la mise en commun des différentes propositions des élèves. \\
Les élèves peuvent également avoir une certaine difficulté à organiser leur recherche, ici le maître peut apporter une aide individualisée en proposant à certains élèves de partir de l'assemblage qu'ils ont réalisé avec 3 triangles et de positionner le 4\up{ème} triangle aux différents endroits possibles.
   \item L'enseignant a choisi de mener cette recherche en groupe restreint et non en classe entière sans doute pour pouvoir apporter une aide individualisée à ses élèves et peut-être pour favoriser les échanges entre les élèves.
   \item Les élèves peuvent élaborer un document présentant les assemblages obtenus, soit en les dessinant à main levée ou sur une feuille pointée en réseau triangulaire, soit par collage. Sur ce document, il serait intéressant que les élèves écrivent une phrase liée aux objectifs que l'enseignant a défini pour cette séance.
\end{enumerate}

\medskip
{\bf Situation de recherche dirigée au cycle 3 : LES POLYGONES} \\
\begin{enumerate}
   \item \textcolor{A2}{$\bullet$} Question I : identifier une figure parmi d'autres figures.
   \begin{itemize}
      \item Questions a), c) et d ) de la question II : reconnaître et nommer les figures usuelles.
   \end{itemize}
   \item L'enseignant veut s'assurer que ses élèves identifient bien les propriétés spécifiques des triangles rectangles isocèles (angle droit, égalité des longueurs de 2 côtés), indépendamment de leur taille.
   \item L'élève a raison car un carré est un losange particulier, qui est aussi un parallélogramme particulier.
   \item La figure a est un rectangle, la figure d est un triangle rectangle isocèle, et la figure i est un parallélogramme. Cette question permet de mettre en évidence la capacité des élèves à envisager un assemblage de figures comme une figure unique. On pourrait également accepter que les figures b et f sont des trapèzes, même si ce n'est pas au programme.
\end{enumerate}
