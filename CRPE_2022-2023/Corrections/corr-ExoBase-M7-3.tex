\ \\ [-5mm]
\begin{enumerate}
   \item \textcolor{A1}{$\bullet$} mesurer une longueur en utilisant une unité adéquat ;
   \begin{itemize}
      \item mesurer une longueur en utilisant une unité adéquat ;
      \item utiliser un instrument de mesure : ici à priori la règle graduée ;
      \item calculer le périmètre d'une figure ;
      \item additionner des longueurs en utilisant l'unité choisie (conversions ou addition de nombres décimaux).
   \end{itemize}
   \item \textcolor{A1}{$\bullet$} difficultés dans l'organisation de la tâche : choix des outils, ordre des mesures, présentation des calculs\dots ;
   \begin{itemize}
      \item difficultés de mesure : utilisation de la règle graduée, choix de l'unité, objet à étudier \og qui bouge \fg ;
      \item difficulté de calcul : somme de nombres décimaux ou de mesures avec des unités différentes (cm, mm).
   \end{itemize}
   \item
   \begin{enumerate}
      \item {\bf Corantin} reproduit la lettre sur une feuille à taille réelle, puis mesure chacun des segments qui la compose à l'aide (à priori) d'une règle graduée. Ses mesures sont exprimées en cm et mm et semblent cohérentes. Ensuite, il effectue une addition en colonnes de toutes les mesures en additionnant d'une part les mesures en mm, et d'autre part celles en cm. L'une des mesures a été inversée (5 cm 2 mm au lieu de 2 cm 5 mm), probablement parce que sa mesure sur la figure est \og à l'envers \fg{}. Il obtient 31 cm 13 mm au lieu de 29 cm 13 mm. Il ne fait pas la conversion 13 mm = 1 cm 3 mm. Enfin, il transforme son résultat en 24 cm et 21 mm : peut-être a-t-il vu son erreur d'inversion et a-t-il voulu soustraire 7 cm comme résultat de 5 cm + 2 mm ce qui donne 24 cm, puis ajouter 7 mm comme résultat de 2 cm + 5 mm, avec une erreur de calcul supplémentaire puisqu'il obtient 21 mm au lieu de 20 mm. \\
      Il sait mesurer une longueur en utilisant une unité adéquat, utiliser une règle graduée, calculer le périmètre d'une figure mais n'a pas acquis la compétence d'addition de mesures et ne maîtrise pas bien les unités. \\
      {\bf César} schématise à main levée sa lettre sur sa feuille en la cotant avec des mesures prises sur le gabarit. Ses mesures sont exprimée en cm en utilisant des nombres décimaux et semblent précises et cohérentes. Puis il effectue une addition en colonne de nombre décimaux, il obtient 23,7 cm au lieu de 26,5 cm mais la raison de l'erreur est difficile à déterminer ? Enfin, il écrit son résultat sous la figure en cm, en se trompant : ça peut être un oubli, ou une erreur de conversion. \\
      Il sait mesurer une longueur en utilisant une unité adéquat, utiliser une règle graduée, calculer le périmètre d'une figure mais n'a pas acquis complètement la compétence d'addition de nombres décimaux. \\
      {\bf Clarisse} reproduit la lettre sur une feuille à taille réelle, puis mesure chacun des segments qui la compose à l'aide (à priori) d'une règle graduée. Ses mesures sont exprimées en mm et semblent cohérentes. Ensuite, elle effectue une addition en colonne terme par terme, son résultat est juste mais l'écriture mathématique n'est pas correcte : elle aurait dû poser à chaque fois l'addition des deux termes au lieu d'additionner chaque terme bout à bout. Enfin, elle convertit son résultat en cm. \\
         Elle sait mesurer une longueur en utilisant une unité adéquat, utiliser une règle graduée, calculer le périmètre d'une figure, additionner des mesures, effectuer es conversions de longueur, mais n'a pas bien acquis le sens du signe opératoire \og = \fg.
   \item César et Corantin ont tous les deux fait des erreurs dans la technique experte de l'addition : erreur sur les nombres entiers pour Corantin, sur les nombres décimaux pour César. On pourra donc proposer une remédiation du côté de la technique experte pour tous les deux, en utilisant les variables didactiques suivantes : nombre de nombres dans l'opération, types de nombres (entiers, décimaux). Pour Corantin, on pourra également travailler les conversions, et notamment la relation \og 1 cm = 1 mm \fg{}.
   \end{enumerate}
\end{enumerate}
