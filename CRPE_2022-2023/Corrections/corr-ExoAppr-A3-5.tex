\ \\ [-5mm]
\begin{enumerate}
   \item Le dernier arrivé a mis 4,2 minutes de plus que le premier arrivé, soit 12,5 min + 4,2 min = 16,7 min. \\
   \bm{Le dernier arrivé a mis 16,7 minutes}.
   \item La somme des 200 performances en minutes se calcule en multipliant le nombre d'élèves par la moyenne, soit $200\times15,4 =3\,080$. \bm{La somme des 200 performances est de 3\,080 minutes}.
   \item Il y a 200 enfants, donc le premier quartile englobe les 50 premiers enfants, d'où \bm{Ariane a mis entre 12,5 minutes et 14,8 minutes}.
   \item Le temps moyen est de 15,4 min et la médiane est de 15,7 min donc, au moins la moitié des élèves ont mis plus de 15,7 min, ce qui est plus que la moyenne. \bm{L'affirmation est vraie}.
\end{enumerate}
