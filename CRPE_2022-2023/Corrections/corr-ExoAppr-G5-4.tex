\ \\ [-5mm]
\begin{enumerate}
   \item L'objectif de cette activité est : tracer, sur papier quadrillé, la figure symétrique d'une figure donnée par rapport à une droite donnée.
   \item Globalement, les procédure possibles pour un élève de cycle 3 sont :
   \begin{itemize}
      \item le pliage ;
      \item l'utilisation d'un papier calque ;
      \item l'utilisation de la règle et du compas ;
      \item l'utilisation du quadrillage : construire le symétrique de chacun des points de la figure initiale puis tracer les segments les rejoignant ;
      \item l'utilisation du quadrillage : construire le symétrique d'un point et à partir de celui-ci reconstruire la figure initiale de proche en proche.
\end{itemize}
    \item
    \begin{itemize}
       \item {\bf Élève A} : \\
       1) l'élève a, semble-t-il, commencé par reporter les points symétriques (présence des points), cette étape est correcte. Puis il les a reliés et c'est à ce moment qu'il a commis une erreur dans la manière de les relier entre eux (ordre erroné) ; \\
       2) un seul point est bien placé et l'axe oblique est pris en compte mais la figure n'est pas correcte : il semble avoir compté le nombre de côtés de carreaux qu'il a reportés suivant des segments obliques. Il n'a pas terminé sa figure.
       \item {\bf Élève B} : \\
       1) trois points sont bien placés mais la figure n'est pas conforme (non conservation des aires). \\
       Pour placer le quatrième point, l'élève semble utiliser le bord de la feuille et non la distance à l'axe ; \\
       2) la figure est superposable mais il a tracé la figure \og en face \fg{} d'un point fictif, il s'agit d'une symétrie centrale (étudiée en 5\up{ème}).
       \item {\bf Élève C} : \\
       1) la figure est correctement construite, apparement l'élève a commencé par construire chacun des points du polygone (présence des marques), puis les a reliés ; \\
       2) la figure est bien superposable mais l'élève semble avoir construit le symétrique à partir d'un axe vertical. On peut également penser, qu'il a placé dans un premier temps les points (présence de marques), par translation, puis il a inversé la figure.
       \item {\bf Élève D} : \\
       1) la figure ressemble, mais n'est pas isométrique et aucun élément n'est juste par rapport à l'exercice demandé. Il semble que cet élève utilise une translation un peu approximative (les sommets ne sont pas tous sur des n\oe uds du quadrillage) ; \\
       2) la figure est superposable mais il a utilisé une translation et non une symétrie : il a déterminé la distance horizontale jusqu'à l'axe de symétrie en fonction des carreaux, distance qu'il a reportée pour déterminer le point de départ de la figure symétrique.
    \end{itemize}
    \item Ici, seule la règle est disponible pour l'élève, l'objectif implicite du maître est donc certainement qu'il utilise l'une des deux dernières procédures avec utilisation du quadrillage et de la règle comme instrument de tracé.
 \end{enumerate}
