\ \\ [-5mm]
\begin{enumerate}
   \item L'exercice 1 est réussi à 65\,\% et le deuxième à 42\,\%. Il est plus difficile de placer les chiffres 0 dans l'écriture du second nombre que dans celui du premier : à l'oral, on entend distinctement les nombres qui constituent les tranches de trois chiffres pour le premier nombre, mais pas pour le second où il faut juxtaposer des 0 en début de tranche afin que cette tranche contienne trois chiffres.
   \item
   \begin{enumerate}
      \item Les élèves ont probablement inversé mentalement les termes [cent] et [cinq]. Ceux-ci apparaissant en fin de nombre, ils étaient peut-être moins concentrés et en surcharge cognitive.
      \item Le système de  numération ne semble pas complètement acquis. En effet, il écrit les nombres \og comme il les entend \fg{} : deux millions (qu'il écrit 20000 en omettant deux chiffres 0), trois cent (qu'il écrit correctement 300), quarante (qu'il écrit correctement 40), mille cent cinq (qu'il n'écrit pas correctement du tout, mais qu'il écrit tout de même avec les chiffres 0, 1 et 5 entendus ou pressentis dans la désignation orale).
   \end{enumerate}
   \item Pour 17 200 058, l'écriture respecte la convention usuelle d'écriture par tranches de trois chiffres. Mais, par contre, il est à noter une erreur de position des chiffres dans la seconde tranche (200 au lieu de 002), ce qui ne respecte pas la convention usuelle d'écriture d'un nombre. La convention du maître n'est pas respectée car sinon, on aurait lu \og deux-cent-mille \fg{} et non \og deux-mille \fg{} comme dans l'énoncé. \\
   Pour 17 2 58, l'élève ne respecte pas la convention usuelle d'écriture par tranches de trois chiffres. Par contre, la convention du maître est respectée scrupuleusement : les mots [millions] et [mille] sont remplacés par des espaces.
   \item Pour rejeter la réponse 17 200 058, une simple re-lecture du nombre devrait permettre de se rendre compte que l'oralisation du \og deux-cent \fg{} ne devrait pas paraître. Cet argument ne devrait pas permettre de rejeter la réponse 17 2 58, pour des élèves qui vont donner au premier espace le sens de million et pour le second le sens de mille.
   \item Tout nombre se lisant sans le mot-nombre [mille] : par exemple \og douze-millions-quatorze \fg{}, l'élève pourra écrire 12 14 et s'apercevoir qu'il manque au moins la tranche des mille.
\end{enumerate}
