Il y a équiprobabilité des faces du dé. \\
   \begin{enumerate}
      \item La probabilité d'obtenir 3 avec le dé vert est de $\dfrac26 ={\blue \dfrac13}$. \smallskip
      \item On peut modéliser la situation par un tableau à double entrée pour obtenir les 36 résultats possibles : \\ [1mm]
         \hspace*{4cm}
         {\small\hautab{1.5}
         \begin{cltableau}{0.4\linewidth}{7}
            \hline
            {\tiny \!\!\!\!bleu$\backslash$vert} & 0 & 1 & 1 & 2 & 3 & 3 \\
            \hline
            0 & 0 & 1 & 1 & 2 & 3 & 3 \\
            \hline
            1 & 1 & 2 & 2 & 3 & 4 & 4 \\
            \hline
            2 & 2 & 3 & 3 & 4 & 5 & 5 \\
            \hline
            2 & 2 & 3 & 3 & 4 & 5 & 5 \\
            \hline
            3 & 3 & 4 & 4 & 5 & 6 & 6 \\
            \hline
            3 & 3 & 4 & 4 & 5 & 6 & 6 \\
            \hline
         \end{cltableau}}
         \begin{enumerate}
            \item On peut obtenir les sommes suivantes : {\blue 0 ; 1 ; 2 ; 3 ; 4 ; 5 ; 6}.
            \item L'élève passe son tour s'il obtient une face vide sur chacun des dés, donc une somme égale à 0. \\
               {\blue La probabilité qu’il doive passer son tour est de $\dfrac{1}{36}$}.
            \item {\blue La probabilité qu’il doive avancer de 3 cases est de} $\dfrac{9}{36}$ {\blue=$\dfrac14$}. \smallskip
            \item Il suffit de compter les occurrences dans le tableau. Pour chaque valeur, on obtient : \\ [1mm]
               {\blue $\mathcal{P}(0) =\dfrac{1}{36} \; ; \; \mathcal{P}(1) =\dfrac{3}{36} \; ; \; \mathcal{P}(2) =\dfrac{5}{36} \; ; \; \mathcal{P}(3) =\dfrac{9}{36} \; ; \; \mathcal{P}(4) =\dfrac{8}{36} \; ; \; \mathcal{P}(5) =\dfrac{6}{36} \; ; \; \mathcal{P}(6) =\dfrac{4}{36}$}.
            \item Le résultat du dé vert est strictement supérieur à celui du dé bleu avec une probabilité de {\blue $\dfrac{12}{36}$}.
         \end{enumerate}
      \setcounter{enumi}{2}
      \item Après deux tours on s'arrête sur la case \fbox{10} après s'être arrêté sur la case \fbox{4} si on a une somme de 6 donc, $\mathcal{P} =\dfrac{4}{36} =${\blue $\dfrac{1}{9}$}.
   \end{enumerate}
