\ \\ [-5mm]
\begin{enumerate}
   \item
   \begin{enumerate}
      \item Les méthodes (M1 et M2) et les erreurs (E1 et E2) sont récapitulées dans le tableau ci-dessous : \\ [1mm]
         {\renewcommand{\arraystretch}{1.5}
         \begin{CLtableau}{1\linewidth}{3}{c}
             \hline
             &
             Configuration 1
             &
             Configuration 2 \\
             \hline
             M1
             &
             \multicolumn{2}{c|}{\bf Procédure par subitisation (reconnaissance perceptive immédiate d'une quantité)} \\
             &
             Procédure possible pour des petites quantités (nombres inférieurs à 5). Ici, l'élève \og voit \fg{} un certain nombre d'éléphants.
             &
             Pour la carte C, cette procédure devrait être immédiate car les nombres sont représentés sous forme de configurations géométriques particulières : celles des constellations du dé, qui sont une des représentations des nombres vues dès la PS. \\
             \hline
             M2
             &
             \multicolumn{2}{c|}{\bf Procédure par comptage un à un.} \\
             &
             \multicolumn{2}{p{14cm}|}{Utilisation de la comptine numérique : suite de mots-nombre mis en correspondance un à un avec les éléments de la collection considérée, le dernier mot-nombre utilisé indiquant la quantité (principe cardinal).} \\
             \hline
             E1
             &
             Procédure par subitisation : l'élève peut se tromper dans la reconnaissance des quantités, surtout si celles-ci dépassent 3.
             &
             Erreur liée au matériel : le fait de ne pas avoir de boite ne favorise pas le rangement des jetons, l'élève va disposer ses jetons sous sa carte, mais certains jetons peuvent se mélanger avec la représentation du nombre précédent.
             \\
             \hline
             E2
             &
             \multicolumn{2}{c|}{\bf Erreurs de comptage.} \\
             &
             \multicolumn{2}{p{14cm}|}{$\bullet$ L'élève dénombre deux fois le même objet ou il en oublie un (principe d'adéquation unique). \newline
             $\bullet$ L'élève se trompe dans l'ordre de la comptine numérique (principe d'ordre stable).} \\
             \hline
          \end{CLtableau}}
       \bigskip
       \item Louise et Kévin ont compris l'objectif du maître, à savoir de réaliser des collections de cardinaux identiques à ceux de la carte. Cependant, leurs procédures semblent différentes :
       \begin{itemize}
          \item Louise dispose ses jetons les uns en dessous des autres. On peut imaginer qu'elle a tout d'abord dénombré les jetons de la carte (par subitisation ou comptage), puis qu'elle a réalisé une collection de jetons de même cardinal en les plaçant un à un.
          \item Kévin semble procéder par correspondance terme à terme en posant un jeton sur chaque point du dé. Cette procédure n'utilise pas le dénombrement mais est tout à fait efficiente, même si, avec des jetons beaucoup plus gros que les points de la carte, l'organisation matérielle pourra poser problème pour les deux dernières constellations (3 et 5).
       \end{itemize}
    \end{enumerate}
    \item
    \begin{itemize}
       \item Une facilité : les jetons représentant les animaux sont bien rangés en regard de la collection de la carte témoin, ne dépassent pas et donc ne se mélangeront pas.
       \item Une difficulté : il sera difficile pour l'élève de vérifier son résultat pour des quantités supérieures à deux puisqu'alors les jetons vont se recouvrir ou se chevaucher.
    \end{itemize}
 \end{enumerate}
