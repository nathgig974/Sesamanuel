\ \\ [-5mm]
\begin{enumerate}
   \item La compétence générale est \og Résoudre un problème de proportionnalité dans le champ multiplicatif \fg{}, à partir à chaque fois d'une valeur unitaire. \\
   Une compétence mathématiques est la maitrise des nombres décimaux et en particulier savoir diviser et multiplier un nombre par 10 et par 100.
   \item
   \begin{enumerate}
      \item {\bf Théo} utilise une règle valable avec les nombres entiers : \og pour multiplier un nombre par 100, on ajoute deux zéros à la fin du nombre \fg{}. \\
      Cependant, cette règle ne s'applique pas aux nombres décimaux, puisqu'ajouter des \og zéros \fg{} à la partie décimale ne change rien à la valeur du nombre.
      \item {\bf Eugénie} a compris que multiplier par 100 revenait à multiplier chaque rang du nombre par 100, c'est-à-dire décaler chaque chiffre de deux rangs vers la gauche dans le tableau de numération. \\
      Cette procédure est meilleure mathématiquement et didactiquement beaucoup plus pertinente que la précédente !
   \end{enumerate}
\end{enumerate}
