   Soit N $=\overline{cdu}$ le nombre recherché.
   \begin{itemize}
      \item La somme de ses chiffres est 16 donc, $c+d+u =16$. \quad{\it (1)} \\
      \item Si l'on intervertit le chiffre des centaines et celui des dizaines, le nombre augmente de 450 :
      \begin{align*}
         \overline{dcu} =\overline{cdu}+450 & \iff 100d+10c+\cancel{u} =100c+10d+\cancel{u}+450 \\
         & \iff 100d-10d-100c+10c =450 \\
         & \iff 90d-90c=450 \\
         & \iff 9d-9c =45 \\
         & \iff d-c =5 \\
         & \iff d=c+5 \quad {\it (2)}
      \end{align*}
   \end{itemize}

\Coupe

   \begin{itemize}
      \item Si l'on intervertit le chiffre des centaines et celui des unités, il augmente de 198 :
      \begin{align*}
         \overline{udc} =\overline{cdu}+198 & \iff 100u+\cancel{10d}+c =100c+\cancel{10d}+u+198 \\
         & \iff 100u-u-100c+c =198 \\
         & \iff 99u-99c =198 \\
         & \iff u-c =2 \\
         & \iff u =c+2 \quad {\it (3)}
      \end{align*}
   \end{itemize}
   Dans {\it (1)}, on remplace $d$ et $u$ obtenus en {\it (2)} et {\it (3)} :
   \begin{align*}
   c+(c+5)+(c+2)=16 & \Longleftrightarrow c+c+5+c+2=16 \\
   & \Longleftrightarrow 3c=9 \\
   & \Longleftrightarrow c=3
   \end{align*}
   On a alors d'après {\it (2)} : $d=3+5=8$ et d'après {\it (3)} : $u=3+2=5$. \\
   Le nombre recherché est {\blue $N = 385$}. \\
