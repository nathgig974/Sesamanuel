\ \\ [-5mm]
\begin{enumerate}
   \item On peut résumer dans un tableau les valeurs de a, b et n : \\  [1mm]
   \qquad
   {\renewcommand{\arraystretch}{1.2}
   \begin{tabular}{r|c|c|c|}
      \cline{2-4}
      & a & n & b \\
      \cline{2-4}
      valeurs initiales & 5 & 0 & 1 \\
      \cline{2-4}
      valeurs après le premier passage & 5 & 1 & 5 \\
      \cline{2-4}
      valeurs après le second passage & 5 & 2 & 25 \\
      \cline{2-4}
   \end{tabular}}
   \item À chaque boucle, il n'y a aucune action sur $a$ qui reste donc égal à 5, $n$ est incrémenté de 1 et b est multiplié par a, donc par 5. D'où, \bm{ce programme donne les puissances successives de 5 de $5^1$ à $5^{10}$.}
   \end{enumerate}
