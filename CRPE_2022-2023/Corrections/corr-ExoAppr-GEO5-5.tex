\ \\ [-5mm]
\begin{enumerate}
   \item Cette séquence peut être proposée en deuxième année du cycle 3 (CM2) où on retrouve les compétences suivantes :
      \begin{itemize}
         \item utiliser les instruments pour tracer des droites parallèles et perpendiculaires ;
        \item vérifier la nature d'une figure en ayant recours aux instruments ;
         \item reproduire une figure complexe (sur papier uni, quadrillé ou pointé), à partir d'un dessin ;
         \item rédiger un programme de construction.
      \end{itemize}
   \item
   Tableau comportant deux compétences pour chaque activité. \\ [1mm]
   \begin{CLtableau}{0.926\linewidth}{3}{c|C{5}|C{8.5}}
      \hline
      & Compétence générale  & compétence \og instrumentale \fg \\
      \hline
      1 & Reconnaître, reproduire un carré & Vérifier qu'un angle est droit en utilisant l'équerre ou un gabarit \\
      \hline
      2 & Reproduire une figure sur papier pointé, à partir d'un dessin & Utiliser la règle et l'équerre pour construire des figures planes usuelles \\
      \hline
      3 & Rédiger un programme de construction & Utiliser les instruments (règle, équerre) pour reproduire une figure sur papier uni à partir d'un modèle \\
      \hline
   \end{CLtableau}
   \item Plusieurs logiques peuvent être mises en évidence :
   \begin{itemize}
      \item utilisation progressive d'instruments dans le cadre d'une difficulté croissante. D'abord règle sur un support pointé (activité 2), puis règle et équerre sur du papier uni (activité 3), et enfin règle, équerre et compas sur du papier uni (activité 4) ;
      \item volonté de faire acquérir aux élèves des compétences spatiales dans une activité de reproduction de dessins. Reconstitution de figures à partir de pièces (activité 1), reproduction d'une figure avec l'aide d'un papier pointé (activité 2), et enfin, reproduction d'une figure sur papier uni à l'aide d'instruments de géométrie et formulation de l'action (activités 3 et 4).
    \end{itemize}
   \item
   \begin{enumerate}
      \item La règle permet de vérifier des alignements d'objets, de tracer des segments en joignant 2 points, et de réaliser des alignements. L'équerre permet de vérifier si un angle est droit, de tracer une perpendiculaire à un segment en un point donné. On remarque qu'aucun instrument n'est dévolu à la comparaison de longueurs : est-ce pour faire réfléchir l'élève à une procédure de comparaison non numérique ? (par exemple : étalon).
      \item Il y a plusieurs \og familles \fg{} de difficultés :
      \begin{itemize}
         \item prendre l'information nécessaire pour la reproduction du dessin qui n'est pas codée explicitement : égalité de longueurs de segments, angles droits, alignements ;
         \item déterminer l'ordre des constructions à réaliser ;
         \item utiliser des instruments, en particulier l'équerre.
      \end{itemize}
      \item
      \begin{itemize}
         \item Difficulté à trouver l'ordre des différentes étapes de la construction.
         \item Difficulté à décrire une construction, en particulier dans l'élaboration de phrases souvent très complexes pour des enfants de cet âge.
         \item Difficulté à utiliser du vocabulaire géométrique efficace (perpendiculaire à un segment en un point\dots) des désignations géométriques (nommer un point, désigner un segment, une droite).
      \end{itemize}
      \item
      \begin{itemize}
         \item Reprendre collectivement la construction de la figure complexe en pointant la nécessité de bien l'analyser (ce que permet l'activité 2) et faire réaliser avec soin les différents tracés de perpendiculaires.
         \item Reprendre collectivement la rédaction du programme en introduisant le vocabulaire géométrique approprié utilisable dans de futures activités identiques.
   \end{itemize}
   \end{enumerate}
\end{enumerate}
