\ \\ [-5mm]
   \begin{enumerate}
      \item
      \begin{enumerate}
         \item le point C$_1$ est sur le segment [AB], donc entre A et B. On a alors : BC$_1$ = BA $-$ AC$_1$. \\
         D'où les équivalences : BC$_1$ = 2AC$_1 \iff$ BA $-$ AC$_1$ = 2AC$_1 \iff$ BA = 3AC$_1 \iff$ AC$_1 = \dfrac13$ AB. \\
         Or, AB = 6 cm, donc : \bm{la longueur du segment [AC$_1$] est de 2 cm.}
         \item Si BC$_2$ = 2 AC$_2$ et C$_2$ distinct de C$_1$, alors BC$2$ est plus grand que AC$_2$ et donc le point C$_2$ est en dehors du segment [AB], du côté de A. On a alors : BC$_2$ = BA + AC$_2$. D'où les équivalences : \\
         BC$_2$ = 2AC$_2 \iff$ BA + AC$_2$ = 2AC$_2 \iff$ BA = AC$_2$. Donc : \bm{la longueur du segment [AC$_2$] est de 6 cm.}
         \item Une figure possible :
      \end{enumerate}
  \end{enumerate}
   \Coupe

{\psset{unit=0.65}
\begin{pspicture}(-4,-6.5)(16,5)
\psgrid[subgriddiv=10, gridlabels=0, gridwidth=0.4pt, subgridwidth=0.4pt,gridcolor=brown!80,subgridcolor=brown!40](-1,-6)(16,6)
   \psline(-1,0)(16,0)
   \psline[linewidth=0.05cm](6,0)(12,0)
   \psline[linewidth=0.05cm](6,0)(6.75,2.9)(12,0)
   \psline[linewidth=0.05cm](6,0)(6.75,-2.9)(12,0)
   \psdots(6,0)(12,0)(0,0)(8,0)
   \psdots[linecolor=A1](6.75,-2.9)(6.75,2.9)
   \psdots[linecolor=B2](2.75,3.8)(2.75,-3.8)
   \pscircle[linecolor=B2](6,0){5}
   \psarc[linecolor=B2](12,0){10}{143}{-143}
   \pscircle[linecolor=A1](6,0){3}
   \psarc[linecolor=A1](12,0){6}{48}{-48}
   \rput[bl](6.1,0.3){A}
   \rput[bl](12,0.3){B}
   \rput[bl](8,0.3){C$_1$}
   \rput[bl](0,0.3){C$_2$}
   \rput[bl](2,4){\textcolor{B2}{$C_3$}}
   \rput[bl](2,-4.4){\textcolor{B2}{$C_4$}}
   \rput[bl](6.3,-3.7){\textcolor{A1}{$C_6$}}
   \rput[bl](6.3,3.4){\textcolor{A1}{$C_5$}}
\end{pspicture}}
\ \\
\begin{enumerate}
\setcounter{enumi}{1}
      \item Si AC $=x$, alors BC $=2x$. De plus, si C existe, A, B et C ne sont pas alignés et forment donc un triangle. \\
      \begin{enumerate}
         \item Si $x=9$, alors AC = 9 cm et BC = 18 cm. Or, AB = 6 cm, et $6+9 =15$ donc, BA + AC < BC, ce qui signifie qu'on ne peut pas construire un triangle. \bm{Il n'existe pas de point C pour $x=9$.}
         \item Si $x=5$, alors AC = 5 cm et BC = 10 cm. Or, AB = 6 cm, et $6+5 =11$ donc, BA + AC > BC, ce qui signifie que l'on peut construire un triangle. Dans ce cas, C est sur le cercle de centre A de rayon 5 cm et sur le cercle de centre B de rayon 10  cm. Ces deux cercles sont sécants en deux points C$_3$ et C$_4$. \\
         \bm{Il existe deux points C pour $x=5$.}
      \end{enumerate}
      \item D'après le théorème de Pythagore, dans le triangle ABC rectangle en C, on a : \\
      $\text{AB}^2 =\text{AC}^2+\text{BC}^2 \iff 6^2 =x^2+(2x)^2 \iff 36 =5x^2 \iff x^2 =\dfrac{36}{5} \Longrightarrow x=\dfrac{6}{\sqrt5} =\dfrac{6}{5}\sqrt5$. \\
       \bm{Pour $x=\dfrac65\sqrt5$, le triangle ABC est rectangle den C.}
       \item Vérifions les cas où le triangle peut-être isocèle :
          \begin{itemize}
            \item Supposons qu'il soit isocèle en C, alors AC = BC $\iff x =2x \iff x=0$, ce qui est impossible.
            \item Supposons qu'il soit isocèle en A, alors CA = BA = 6 cm et BC = 12 cm. \\
            Mais dans ce cas, BA + AC = BC, donc A, B et C sont alignés, ce qui est contraire à l'énoncé.
             \item Supposons qu'il soit isocèle en B, alors AB = CB = 6 cm et AC = 3 cm. \\
            Dans ce cas, BA + AC > BC, donc, il existe deux points C$_5$ et C$_6$ tels que le triangle ABC soit isocèle en B.
         \end{itemize}
         \bm{Il existe une seule valeur de $x$ (qui vaut 3) pour laquelle le triangle ABC est isocèle.}
\end{enumerate}
