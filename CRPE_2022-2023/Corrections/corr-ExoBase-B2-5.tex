   {\bf Remarque de départ :} les points $I, J, K, L$ sont les milieux respectifs des arêtes $[SA], [SB], [SC], [SD]$. \\
   Donc, d'après le théorème de la droite des milieux appliqué aux triangles $SAB, SBC, SCD$ et $SDA$, on a $IJ =\dfrac12 AB, JK =\dfrac12 BC, KL =\dfrac12 CD, LI =\dfrac12DA$ et de plus, $IJ // AB, JK // BC, KL // CD$ et $LI // DA$. \\ [2mm]
   \begin{enumerate}
      \item D'après la remarque, le quadrilatère $ABCD$ est un  agrandissement du carré $IJKL$ de rapport 2, l'aire est donc multipliée par $2\times2 =4$. \\
      \bm{L'affirmation est vraie.}
      \item Le triangle $SAB$ est un agrandissement de coefficient $2$ du triangle $SIJ$. \\
      On a alors : $\mathcal{A}(SAB) =4\times \mathcal{A}(SIJ)$. \\
      Or, $\mathcal{A}(IJBA) =\mathcal{A}(SAB)-\mathcal{A}(SIJ) =\mathcal{A}(SAB)-\dfrac14\mathcal{A}(SAB) =\dfrac34\mathcal{A}(SAB)$. \\
      \bm{L'affirmation est fausse.}
      \item La pyramide $SABCD$ est un agrandissement de coefficient $2$ de la pyramide $SIJKL$ donc : \\
      $\mathcal{V}(SABCD) =2^3\times \mathcal{V}(SIJKL) =8\times\mathcal{V}(SIJKL)$. \\
      \bm{L'affirmation est fausse.}
      \item $\mathcal{V}(ABCDIJKL) =\mathcal{V}(SABCD)-\mathcal{V}(SIJKL) =\mathcal{V}(SABCD)-\dfrac18\mathcal{V}(SABCD) =\dfrac78\mathcal{V}(SABCD)$. \\
      Soit $\mathcal{V}(SABCD) =\dfrac87\mathcal{V}(SABCD)$. \\ [1mm]
      \bm{L'affirmation est vraie.}
   \end{enumerate}
