\ \\ [-5mm]
\begin{enumerate}
   \item
   \begin{enumerate}
      \item {\bf Programme de construction :}
      \begin{itemize}
         \item Tracer un segment [BD] de longueur quelconque ;
         \item construire sa médiatrice (d) qui le coupe en son milieu O ;
         \item tracer le cercle de centre O passant par B et D. Ce cercle coupe (d) en deux points dont l'un est le point A ;
         \item choisir le point C sur (d) distinct de A et O ;
         \item relier les points A, B, C et D dans et ordre.
      \end{itemize}
      \begin{pspicture}(-5,-0.75)(7,4.75)
         \pstGeonode[PosAngle={-90,90,45}](2,0){D}(2,4){B}(6,2){C}
         \pstLineAB{D}{B}
         \pstMediatorAB[PosAngle=135]{D}{B}{O}{A}
         \pstLineAB[nodesepA=-1,nodesepB=-5]{A}{O}
         \pstCircleAB{B}{D}
         \pstSegmentMark[SegmentSymbol=MarkHash]{O}{B}
         \pstSegmentMark[SegmentSymbol=MarkHash]{O}{D}
         \pstRightAngle{B}{O}{C}
         \psset{linecolor=B2}
         \pstLineAB{A}{B}
         \pstLineAB{B}{C}
         \pstLineAB{C}{D}
         \pstLineAB{D}{A}
         \pstRightAngle{B}{A}{D}
         \pstSegmentMark{A}{B}
         \pstSegmentMark{D}{A}
         \pstSegmentMark[SegmentSymbol=MarkCros]{B}{C}
         \pstSegmentMark[SegmentSymbol=MarkCros]{D}{C}
      \end{pspicture}
      \item {\bf Programme de construction :} \\
      \begin{itemize}
         \item Tracer un segment [BD] de longueur quelconque ;
         \item construire sa médiatrice (d) ;
         \item tracer le cercle de centre O passant par B et D ;
         \item choisir deux points distincts A et C sur (d) qui ne soient pas sur [BD] ni sur le cercle ;
         \item relier les points A, B, C et D dans et ordre.
      \end{itemize}
      \begin{pspicture}(-5,-0.75)(7,4.75)
         \pstGeonode[PosAngle={-90,90,45,135}](2,0){D}(2,4){B}(6,2){C}(1,2){A}
         \pstLineAB{D}{B}
         \pstMediatorAB[PosAngle=135,PointName=none,PointSymbol=none]{D}{B}{O}{M}
         \pstLineAB[nodesepA=-1,nodesepB=-5]{M}{O}
         \pstSegmentMark[SegmentSymbol=MarkHash]{O}{B}
         \pstSegmentMark[SegmentSymbol=MarkHash]{O}{D}
         \pstCircleAB{B}{D}
         \pstRightAngle{B}{O}{C}
         \psset{linecolor=B2}
         \pstLineAB{A}{B}
         \pstLineAB{B}{C}
         \pstLineAB{C}{D}
         \pstLineAB{D}{A}
         \pstSegmentMark[SegmentSymbol=MarkHashhh]{A}{B}
         \pstSegmentMark[SegmentSymbol=MarkHashhh]{D}{A}
         \pstSegmentMark[SegmentSymbol=MarkCros]{B}{C}
         \pstSegmentMark[SegmentSymbol=MarkCros]{D}{C}
      \end{pspicture}
   \end{enumerate}
   \item
   \begin{enumerate}
      \item Tous les carrés sont des isocerfvolants : \bm{vrai}, puisque tout carré ABCD possède un angle droit en A et sa diagonale (AC) est un axe de symétrie.
      \item Tous les rectangles sont des isocerfvolants : \bm{faux}, puisque la diagonale d'un rectangle ABCD n'est pas, en général, un axe de symétrie (sauf dans le cas particulier d'un carré).
      \item Tous les isocerfvolants dont les diagonales se coupent en leur milieu sont des carrés : \bm{vrai}, puisque si les diagonales se coupent en leur milieu, c'est un parallélogramme qui en plus possède un angle droit, c'est donc un rectangle, dont une diagonale est un axe de symétrie, c'est un carré.
   \end{enumerate}
  \end{enumerate}
