\ \\ [-5mm]
   \begin{enumerate}
      \item Une vitesse de 90 km/h correspond à : $\dfrac{\um{90}}{\us{3600}} =25$ m/s. \\ [1mm]
         Sachant que la constante vaut $k =0,14$, en utilisant la formule donnée on obtient en mètre : \\
         $d_A =25\times0,75+0,14\times25^2 =106,25$. \\
         {\blue Un véhicule circulant sur route mouillée à 90 km/h mettra 106,25 mètres pour s'arrêter}. \\
      \item Pour un conducteur vigilant sur route sèche, on a $t_R =0,75$ et $k =0,14$, d'où : \\
   $d_A =0,75\,v+0,073\,v^2$. \\
          L'expression de la distance en fonction de la vitesse n'est pas une fonction linéaire à cause de la présence du \og $v^2$ \fg{} (il s'agit d'une fonction du second degré représentée par une parabole) donc, \\
         {\blue la distance d'arrêt sur route sèche n'est pas proportionnelle à la vitesse.}
   \item
      \begin{enumerate}
          \item Un véhicule roulant à 110 km/h s'arrête en {\blue \um{101}}.
          \item La distance de freinage à 80 km/h est de {\blue \um{41}}.
          \item Un véhicule qui roule à 130 km/h met {\blue \us{6,76}} pour s'arrêter.
          \item Un distance de réaction de \um{25} correspond à une vitesse de {\blue 120 km/h.}
          \item Un conducteur roulant à 27,8 m/s mettra \um{85,4} pour s'arrêter, donc {\blue il évitera l'obstacle}.
      \end{enumerate}
   \end{enumerate}
