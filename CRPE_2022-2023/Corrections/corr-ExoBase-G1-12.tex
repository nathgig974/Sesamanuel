\ \\ [-5mm]
\begin{enumerate}
   \item I, J et K sont les milieux respectifs des segments [FE], [FG] et [FB] qui sont des arêtes du cube. \\
   On a donc FI = FJ = FK = 3 cm. De plus, F est un sommet du cube ce qui signifie que les angles $\widehat{\text{IFK}}, \widehat{\text{KFJ}}$ et $\widehat{\text{JFI}}$ sont droits. Les triangles IFK, KFG et JFI sont donc isométriques et IK = KJ = JI d'où : \\
   \bm{Le triangle IJK est un triangle équilatéral.} \\
   \item On a $V_{\text{tétraèdre}} =\dfrac13\times\mathcal{A}_{\text{FIJ}}\times\text{ FK} =\dfrac13\times\dfrac{3\text{ cm}\times3\text{ cm}}{2}\times3\text{ cm} =4,5\text{ cm}^3$. \\ [1mm]
   \bm{Le volume du tétraèdre FIJK est de 4,5 cm$^3$.} \\
   \item Patron à l'échelle 3/4 : \\
   {\psset{unit=0.75}
   \begin{pspicture}(-6,-0.75)(7.5,6.5)
      \psline(0,3)(6,3)
      \psline(3,6)(3,0)(0,3)(3,6)(6,3)(7.1,7.1)(3,6)
      \rput(3.3,2.7){F}
      \rput(-0.3,3){I}
      \rput(3,-0.3){K}
      \rput(3,6.3){J}
      \rput(6.3,2.7){K}
      \rput(7.4,7.4){I}
      \psframe(3,3)(3.3,3.3)
   \end{pspicture}}
   \item On peut démontrer de manière analogue à la question 1. que toutes les arêtes du cuboctaèdre sont de même longueur et donc que tous les tétraèdre ôtés sont isométriques.
   \begin{enumerate}
      \item $\mathcal{V}_{\text{cuboctaèdre}} =\mathcal{V}_{\text{cube}}-8\mathcal{V}_{\text{tétraèdre}} =(6\text{ cm})^3-8\times4,5\text{ cm}^3 =180\text{ cm}^3$. \\ [1Mm]
      \bm{Le volume du cuboctaèdre est de 180 cm$^3$.} \\
      \item Le cuboctaèdre possède 3 arêtes à chaque \og coin \fg{} ôté du cube, soit $8\times3$ arêtes = 24 arêtes. \\
      Calculons par exemple IK : dans le triangle IFK rectangle en F, d'après le théorème de Pythagore avec des mesures en centimètre on a $\text{IK}^2 =\text{IF}^2+\text{FK}^2 =3^2+3^2 =18$, donc $\text{IK} =\sqrt{18} =3\sqrt2$. \\
   D'où, la longueur totale des arêtes vaut $24\times3\sqrt2 =72\sqrt2 \approx101,82$ et \\
   \bm{la longueur totale des arêtes du cuboctaèdre vaut environ 101,8 cm.}
   \end{enumerate}
\end{enumerate}
