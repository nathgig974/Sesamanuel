\ \\ [-5mm]
\begin{enumerate}
   \item Procédure utilisées pour résoudre le problème.
   \begin{itemize}
      \item {\bf Production n\degres1} : procédure progressive dans le champ additif (additions). \\
      L'élève additionne le nombre 12 cinq fois à la suite, il trouve 60. Ce nombre étant supérieur à 56, il barre un \og 12 \fg{} et recalcule la somme qui vaut 48. Ce nombre est inférieur à 56. Ensuite, il effectue la soustraction de 48 à 56 pour trouver ke nombre de billes restantes. Il lui reste à compter le nombre de \og 12 \fg{} additionnés pour conclure qu'il peut faire 4 paquets de 12 billes et qu'il en reste 8.
      \item {\bf Production n\degres2} : procédure progressive dans le champ additif (soustractions). \\
      L'élève soustrait 12 à 56 pour trouver 44, puis il réitère ce procédé jusqu'à trouver un nombre inférieur à 12 (on note que, même si la procédure est correcte, l'écriture mathématique en colonnes n'est pas valable à partir du deuxième calcul). Il note à côté chaque paquet de 12 qu'il soustrait, il en trouve 4 et il reste 8 billes.
      \item {\bf Production n\degres3} : procédure progressive dans le champ multiplicatif (multiplications). \\
      Cet élève calcule le répertoire multiplicatif de 12 jusqu'à obtenir un encadrement de 56, il semble effectuer les calculs en additionnant à chaque fois 12, étant donné son erreur à $12\times4$ (il a un écart de 2 avec le résultat réel, cet écart est répercuté sur le dernier calcul). Sa réponse est cohérente.
   \end{itemize}
   \item Les procédures mises en \oe uvre sont acceptable puisqu'elles ne demandent pas trop de calculs : tous les élèves effectuent des calculs de proche en proche qui aboutissent assez rapidement au résultat. Cependant, si on choisissait un nombre bien plus grand (par exemple 1\,246), les calculs seront trop longs et fastidieux et les élèves devraient penser à une autre procédure. On peut également faire varier le nombre de billes dans chaque sac (par exemple 57) pour que les calculs soient moins évident à calculer mentalement. \\
   Cette situation est donc intéressante dans un premier temps et doit évoluer par la suite avec des nombres plus grands afin de transiter petit à petit vers la méthode experte qui consiste à effectuer une division euclidienne pour une division-quotition.
   \item On peut citer les connaissances ou capacités suivantes :
   \begin{itemize}
      \item connaissance : connaître ses tables de multiplication et d'addition ;
      \item capacité : effectuer une multiplication, une soustraction (pour les calculs intermédiaires) ;
      \item capacité : savoir faire une approximation afin de trouver  le nombre de chiffres au quotient ainsi que le meilleur nombre au quotient par exemple par essai-erreur.
   \end{itemize}
\end{enumerate}
