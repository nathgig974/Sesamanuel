\ \\ [-5mm]
\begin{enumerate}
   \item Le calcul en ligne se pratique en amont du calcul posé (implicitement le calcul en colonnes), puis les deux pratiques sont menées en étroite relation. Elles sont complémentaires l'une de l'autre pour diverses raisons que l'on peut résumer dans un tableau : \\
   \medskip
   \begin{tabular}{|p{7cm}|p{7cm}|}
      \hline
      Calcul en ligne.
      &
      Calcul posé. \\
      \hline
      Procédure basée sur la {\bf réflexion}. \newline
      Les élèves travaillent les propriétés de notre numération ainsi que le sens des opérations et utilisent une procédure en fonction des nombres en jeu.
      &
      Procédure \og {\bf clé en main} \fg{}. \newline
      Mise en place d'un algorithme : succession d’étapes utilisées dans un certain ordre et de la même manière indépendamment des nombres en jeu. \\
      \hline
      Procédure de {\bf construction}. \newline
      Il développe des habiletés calculatoires des élèves ainsi que la construction et la connaissance de faits numériques et de propriétés élémentaires.
      &
      Procédure d'{\bf utilisation}. \newline
      Utilisation de ces faits numériques (par exemple les tables de multiplications) pour mener à bien des calculs posés. \\
      \hline
      Procédure {\bf rapide}. \newline
      Le calcul en ligne se travaille également en relation avec le calcul mental et permet d'effecteur des calculs assez rapidement tant qu'ils sont simples.
      &
      Procédure {\bf sûre}. \newline
      N'importe quelle opération classique peut être effectuée mais on ne peut pas agir sur le nombre d'étapes de l'algorithme afin de le réduire. \\
      \hline
   \end{tabular}
   \bigskip
   \item
   \begin{itemize}
      \item Décomposition additive \og dizaine-unité \fg{} puis commutativité de l'addition : \\
      $28+17 =20+8+10+7 =20+10+8+7 =30+15 =45.$
      \smallskip
      \item Complément à la dizaine la plus proche puis commutativité : \\
      $28+17 =30-2+20-3 =30+20-2-3 =50-5 =45$.
      \smallskip
      \item Calcul de proche en proche en utilisant la décomposition additive : \\
      $28+17 = 28+10+7 =38+7 =45$ ou $28+17 = 28+2+15 =30+15 =45$. \\
   \end{itemize}
   \item En cycle 2, une partie du calcul mental s'effectue sur les nombres 1, 2, 5, 10\dots{}, on a donc par exemple les trois procédures suivantes : \\
   \medskip
   \begin{tabular}{|p{7cm}|p{7cm}|}
      \hline
      Connaissances. & Propriétés. \\
      \hline
      \multicolumn{2}{|c|}{$14\times5 =14\times10\div2 =140\div2 =70$} \\
      \hline
      Multiplication par 10. \newline Moitié (de 10 et de 140).
      &
      Associativité de la multiplication. \\
      \hline
      \multicolumn{2}{|c|}{$14\times5 =2\times7\times5 =2\times35 =70$} \\
      \hline
      Table du 5. \newline Double.
      &
      Associativité et commutativité de la multiplication. \\
      \hline
      \multicolumn{2}{|c|}{$14\times5 =10\times5+4\times5 =50+20 =70$} \\
      \hline
      Décomposition additive \og dizaine-unité \fg.\newline
      Table du 5.
      &
      Distributivité de la multiplication par rapport à l'addition. \\
      \hline
   \end{tabular}
\end{enumerate}
