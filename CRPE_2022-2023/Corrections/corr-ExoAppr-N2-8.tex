\begin{itemize}
   \item {\bf L'affirmation 1 est fausse :} \\
   $(x - 7)(x + 4) =(x -7)(16 - x) \iff \underline{(x - 7)}(x + 4) -\underline{(x -7)}(16 - x) =0$ \\
   \hspace*{4.6cm} $\iff \underline{(x - 7)}[(x + 4) -(16 - x)] =0$ \\
   \hspace*{4.6cm} $ \iff (x - 7)(2x-12) =0$ \\
   \hspace*{4.6cm} $ \iff x-7 =0 \text{ ou } 2x-12 =0$ \\
   \hspace*{4.6cm} $ \iff x =7 \text{ ou } x =6$.
   \item {\bf L'affirmation 2 est vraie :} \\
   les trois quarts des adhérents ont moins de 18 ans, donc un quart a plus de 18 ans. \\
   Parmi ceux-ci, le tiers a plus de 25 ans, donc deux tiers ont entre 18 et 25 ans. \\
   Or, deux tiers d'un quart, c'est deux douzièmes, soit un sixième. \\
   Ce qui correspond au calcul : $\left(1-\dfrac13\right)\times\left(1-\dfrac34\right) =\dfrac23\times\dfrac14 =\dfrac2{12} =\dfrac16$.
   \item {\bf L'affirmation 3 est fausse :} \\
   Prenons les nombres 1 et 1. L'inverse de la somme de ces deux nombres vaut $\dfrac12$ et la somme des inverses de ces deux nombres 2. Ceci constitue un contre-exemple.
\end{itemize}
