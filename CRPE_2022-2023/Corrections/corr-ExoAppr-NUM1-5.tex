\ \\ [-5mm]
\begin{enumerate}
   \item Un élève peut utiliser les procédures suivantes :
   \begin{itemize}
      \item correspondance terme à terme avec une autre collection constituée de trois objets ;
      \item perception visuelle globale de trois objets (ou subitisation) ;
      \item comptage-dénombrement de la quantité : un et un (deux) et encore un ça fait trois.
   \end{itemize}
   \bigskip
   \item On peut, par exemple, proposer les activités suivantes :
   \begin{itemize}
      \item La salade de fruits ({\it atelier dirigé proposé par Marine V., PES, ESPE Réunion 2017-2018}). \\ [1mm]
      \begin{tabular}{C{5}cC{4.5}cC{4.05}}
         \includegraphics[height=3.3cm]{Nombres_et_calculs_did/Images/quatre_boite}
         &&
         \includegraphics[height=3.3cm]{Nombres_et_calculs_did/Images/quatre_contenu}
         &&
         \includegraphics[height=3.3cm]{Nombres_et_calculs_did/Images/quatre_maison} \\
         L'élève possède une boite opaque avec 2 ouvertures par lesquelles il doit mettre 4 fruits à choisir parmi des pommes ou des oranges ; \newline
         &&
         il ouvre la boite, puis les dénombre : par exemple \og il y a une pomme et trois oranges, 1 et encore 3 ça fait 4 \fg ;
         &&
         il dispose ses fruits dans la \og maison du 4 \fg. \newline
         Puis, il répète l'opération avec d'autres fruits. \\ [3mm]
      \end{tabular}
      \item Les albums à calculer de {\it Rémi Brissaud, collection \og J’apprends les maths \fg, éditions Retz}. \\
      Ils s'utilisent progressivement et comporte trois types d'activité. \\ [3mm]
      \begin{tabular}{C{5}C{5}C{5}}
         \includegraphics[width=4.8cm]{Nombres_et_calculs_did/Images/quatre_souris}
         &
         \includegraphics[width=4.8cm]{Nombres_et_calculs_did/Images/quatre_souris_g}
         &
         \includegraphics[width=4.8cm]{Nombres_et_calculs_did/Images/quatre_souris_d} \\
         Dans l'image, il y a 4 souris : 3 sont dans des trous de gruyère disposés comme les points du dé. Une est dans l'herbe sur l'autre page. 3 et encore 1, ça fait 4.
         &
         Il y a 4 souris mais on ne les voit pas toutes. \newline
         Combien y a-t-il de souris dans l'herbe sous le rabat ?
         &
         Il y a 4 souris mais on ne les voit pas toutes. \newline
         Combien y a-t-il de souris dans le gruyère sous le rabat ? \\
      \end{tabular}
   \end{itemize}
   \bigskip
   \item On peut citer plusieurs intérêts :
   \begin{itemize}
      \item travailler d'autres constellation moins classiques ;
      \item travailler le lien entre les différentes représentations des nombres : pour passer d'un nombre à son successeur, on ajoute 1 point à la droite de la configuration précédente sans en changer la disposition (sur un dé classique, le passage de 3 à 4 demande de modifier la disposition des points intégralement. Ici, il suffit de compléter le 3 par un point sur le coin vide);
      \item travailler les décompositions de manière visuelle : par exemple, 5 c'est 3 et encore 2 (si on lit verticalement) ou c'est 2 et encore 2 et encore 1 (si on lit horizontalement), ou encore c'est 4 et encore 1.
   \end{itemize}
\end{enumerate}
