\ \\ [-5mm]
   \begin{enumerate}
      \item
         \begin{enumerate}
            \item $0,127 =\dfrac{127}{1\,000} ={\blue \dfrac{127}{10^3}}$. \smallskip
            \item $\dfrac14 =\dfrac{25}{100} ={\blue \dfrac{25}{10^2}}$. \medskip
         \end{enumerate}
      \setcounter{enumi}{1}
      \item
         \begin{itemize}
            \item Élève A : le nombre {\blue$\dfrac13 =1,333\dots$} qui s'écrit avec une virgule suivie d'un nombre infini de 3 n'est pas un nombre décimal.
            \item Élève B : le nombre {\blue$0,127 =\dfrac{127}{1\,000}$} est un nombre décimal dont le numérateur vaut 1\,000.
            \item Élève C : le nombre entier {\blue1} peut s'écrire {\blue$\dfrac{10}{10}$}, c'est donc aussi un nombre décimal. \smallskip
         \end{itemize}
      \item
         \begin{itemize}
            \item $2,48 =\dfrac{248}{100} =\dfrac{248}{10^2}$ {\blue est un nombre décimal}. \smallskip
            \item $\dfrac{7}{25} =\dfrac{28}{100} =\dfrac{28}{10^2}$ {\blue est un nombre décimal}. \smallskip
            \item $12 =\dfrac{12}{1} =\dfrac{12}{10^0}$ {\blue est un nombre décimal}. \smallskip
            \item $\dfrac{7}{9} =\dfrac{7}{3^2}$ est sous une forme irréductible et comporte uniquement une puissance de 3 au dénominateur, \\ [1mm]
               {\blue ce n'est pas un nombre décimal}. \smallskip
            \item $\dfrac{49}{14} =\dfrac{7\times7}{7\times2} =\dfrac72 =\dfrac{35}{10}$ {\blue est un nombre décimal}. \medskip
         \end{itemize}
      \item {\blue Le produit de deux nombres décimaux est un nombre décimal}. \\
        Justification : soit $a =\dfrac{p}{10^n}$ et $b =\dfrac{q}{10^m}$ avec $p$ et $q$ des nombres entiers et $n$ et $m$ des nombres entiers positifs. \\ [1mm]
        $a\times b =\dfrac{p}{10^n}\times\dfrac{q}{10^m} =\dfrac{p\times q}{10^n\times10^m} =\dfrac{pq}{10^{n+m}}$ est bien un nombre décimal puisque $pq$ est un entier comme \\ [1mm]
        produit d'entiers et $n+m$ un entier naturel positif comme somme d'entiers positifs.
      \item {\blue Le quotient de deux nombres décimaux n'est pas toujours un nombre décimal}. \\
         Contre-exemple : $\dfrac{\dfrac{1}{10}}{\dfrac{3}{10}} =\dfrac{1}{10}\times\dfrac{10}{3} =\dfrac13$ n'est pas un nombre décimal puisque son dénominateur ne comporte pas uniquement des puissances de 2 et de 5.
   \end{enumerate}
