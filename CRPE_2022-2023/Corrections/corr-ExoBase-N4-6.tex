\ \\ [-5mm]
   \begin{enumerate}
      \item Pour déterminer les entiers naturels jusqu'à 99 divisibles par 7, il suffit de prendre les multiples de 7 : \\
      on trouve {\blue 0 ; 7 ; 14 ; 21 ; 28 ; 35 ; 42 ; 49 ; 56 ; 63 ; 70 ; 77 ; 84 ; 91 et 98}.
      \item
         \begin{enumerate}
            \item On procède de la même manière que les exemples :
               \begin{multicols}{3}
                  $\opsub[voperation=top]{40}{12}|\,6$
                  \psarc(0,0){0.25}{-90}{90}
                  \psline{->}(0,-0.25)(-0.4,-0.25)
                  \rput(0.6,0){$\times2$} \\
                  28 est divisible par 7 \\
                  donc, 406 l'est aussi. \\
                  $\opsub[voperation=top]{89}{10}|\,5$
                  \psarc(0,0){0.25}{-90}{90}
                  \psline{->}(0,-0.25)(-0.4,-0.25)
                  \rput(0.6,0){$\times2$} \\
                  79 n'est pas divisible par 7 \\
                  donc, 895 non plus. \\
                  $\opsub[voperation=top]{390}{12}|\,6$
                  \psarc(0,0){0.25}{-90}{90}
                  \psline{->}(0,-0.25)(-0.4,-0.25)
                  \rput(0.6,0){$\times2$} \\
                   378 est divisible par 7 \\
                   donc, 3 906 l'est aussi. \\
               \end{multicols}
               {\blue Les nombres 406 et 3 906 sont divisibles par 7, mais 895 ne l'est pas.}
            \item Soit E le nombre dont on souhaite savoir s'il est divisible par 7. On soustrait le double de la valeur du chiffre des unités de E au nombre de dizaines de E. Si le nombre obtenu est divisible par 7, alors E est divisible par 7. Si le nombre obtenu n'est pas divisible par 7, alors E n'est pas divisible par 7.
         \end{enumerate}
      \setcounter{enumi}{2}
      \item
         \begin{enumerate}
            \item {\blue $273 =10\times27+3$} et {\blue $1\,856 =10\times185+6$.}
            \item Soit $n$ le nombre obtenu après procédure, alors {\blue $n=v-2u$.}
            \item Si $n$ est divisible par 7, alors il existe un entier $k$ tel que $n=7k$, c'est-à-dire $v-2u =7k$, ou encore $v=2u+7k$. Or, $E =10v+u =10(2u+7k)+u =21u+70k =7(3u+10k)$ qui est multiple de 7. \\
               {\blue Si le nombre $n$ est divisible par 7, alors le nombre $E$ de départ est divisible par 7.}
      \end{enumerate}
   \end{enumerate}
