\ \\ [-5mm]
\begin{enumerate}
   \item Compétences mises en jeu : être capable d'écrire, ranger les nombres entiers naturels inférieurs à 100 et avoir compris l'organisation, l'aspect algorithmique de notre numération positionnelle de base 10.
   \item Procédures utilisables par l'élève, par exemple :
   \begin{itemize}
      \item l'élève peut placer dans les cases tous les nombres manquants, et ensuite relier chaque nombre à la case représentant le même nombre ;
      \item il peut également réciter \og mentalement \fg{} la comptine numérique en indiquant à chaque fois le case correspondante, et placer au fur et à mesure les nombres qu'ils récite ;
      \item il peut utiliser les compléments à 5 ou à 10 et avancer ou reculer par rapport à une case déjà remplie (par exemple 59, c'est $60-1$, donc une case avant la case 60\dots).
   \end{itemize}
   \item Erreurs prévisibles, par exemple :
   \begin{itemize}
      \item le nombre 48 peut ne pas avoir été placé, car l'élève compte à partir de 50 ;
      \item les nombres peuvent être associés à une case située \og entre \fg{} deux nombres déjà placés (par exemple 64 entre 60 et 65), sans lien avec le nombre de cases ;
      \item l'élève peut avoir placé par exemple 59 juste après 55, c'est à dire dans l'ordre (de plus petit au plus grand), sans se soucier la suite des nombres de 1 en 1 ;
      \item il a pu associer à chaque nombre la case marquée \og la plus proche \fg{} (par exemple, relier 54 à 55).
   \end{itemize}
\end{enumerate}
