\ \\ [-5mm]
\begin{enumerate}
   \item 117 est divisible au moins par 9 puisque la somme de ses chiffres est égale à 9, donc, 117 n'est pas un nombre premier. \\
      {\blue L'affirmation est fausse}.
   \item
      $(n+2)^2 - (n-2)^2 =(n^2+4n+4)-(n^2-4n+4)$ \\
      \hspace*{3.15cm} $=\cancel{n^2}+4n+\cancel{4}-\cancel{n^2}+4n-\cancel{4}$ \\
      \hspace*{3.15cm} $=8n$. \\
      \begin{enumerate}
         \item D'après le résultat obtenu après développement et réduction, \\
            {\blue l'affirmation est vraie}.
         \item Pour $n =1$ par exemple, $(n+2)^2 - (n-2)^2 =8$ qui n'est pas un multiple de 32 donc, \\
            {\blue l'affirmation est fausse}.
      \end{enumerate}
   \setcounter{enumi}{2}
   \item Le nombre recherché est multiple de 2 et de 3, mais pas de $2^2$ et $3^2$. Il suffit donc de choisir un nombre dont la décomposition en produit de facteurs premiers commence par $2\times3\times\dots$. Par exemple : $2\times3\times5 =30$. \\
      {\blue L'affirmation est vraie}.
   \item Une série de perles \og jaune-rouges-blanches \fg{} comporte 6 perles. Or, $147 =6\times24+3$ donc, la 147\up{e} perle sera de la même couleur que la 3\up{e} perle de la série, soit rouge. \\
      {\blue L'affirmation est vraie}.
   \item $126 =2^1\times3^2\times7^1$ donc, 126 possède $(1+1)\times(2+1)\times(1+1)$ diviseurs, soit 12 diviseurs qui sont : 1 ; 2 ; 3 ; 6 ; 7 ; 9 ; 14 ; 18 ; 21 ; 42 ; 63 ; 126. \\
      {\blue L'affirmation est fausse}.
\end{enumerate}
