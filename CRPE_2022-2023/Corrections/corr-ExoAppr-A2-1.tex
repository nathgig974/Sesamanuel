Martin répond au hasard, nous sommes bien dans une situation d'équiprobabilité.   \begin{itemize}
      \item Première affirmation : la probabilité d'obtenir une réponse juste est $\dfrac13$ pour la première question, $\dfrac13$ pour la deuxième question, et $\dfrac13$ pour la troisième question. \\
      Donc, la probabilité que toutes les réponses soient justes est de $\dfrac13\times\dfrac13\times\dfrac13 =\dfrac{1}{27}$. \\ [1mm]
      \bm{L'affirmation 1 est vraie.}
   \smallskip
   \item Deuxième affirmation : la probabilité d'obtenir une réponse fausse est de $\dfrac23$ pour chaque question. \\ [1mm]
   Donc, la probabilité que toutes les réponses soient fausses est de $\dfrac23\times\dfrac23\times\dfrac23 =\dfrac{8}{27}$. \\ [1mm]
   \bm{L'affirmation 2 est fausse.}
   \end{itemize}
