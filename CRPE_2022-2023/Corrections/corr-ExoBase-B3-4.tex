On dit que deux patrons sont différents s'ils ne peuvent pas se superposer, ni par rotation, ni par retournement. On peut, par exemple, trouver tous les patrons comprenant quatre faces à la suites, puis trois, puis deux.
\begin{center}
\og {\bf 4 à la suite} \fg{}: \\
{\psset{unit=0.6}
\begin{pspicture}(0,0)(4,4.5) %1
   \facer{0}{0} \facec{1}{0} \facec{1}{1} \facec{1}{2} \facec{1}{3} \facer{2}{0}
\end{pspicture}
\begin{pspicture}(0,0)(4,4.5) %2
   \facer{0}{0} \facec{1}{0} \facec{1}{1} \facec{1}{2} \facec{1}{3} \facer{2}{1}
\end{pspicture}
\begin{pspicture}(0,0)(4,4.5) %3
   \facer{0}{0} \facec{1}{0} \facec{1}{1} \facec{1}{2} \facec{1}{3} \facer{2}{2}
\end{pspicture}
\begin{pspicture}(0,0)(4,4.5) %4
   \facer{0}{0} \facec{1}{0} \facec{1}{1} \facec{1}{2} \facec{1}{3} \facer{2}{3}
\end{pspicture}
\begin{pspicture}(0,0)(4,4.5) %5
   \facer{0}{1} \facec{1}{0} \facec{1}{1} \facec{1}{2} \facec{1}{3} \facer{2}{1}
\end{pspicture}
\begin{pspicture}(0,0)(4,4.5) %6
   \facer{0}{1} \facec{1}{0} \facec{1}{1} \facec{1}{2} \facec{1}{3} \facer{2}{2}
\end{pspicture}

\og {\bf 3 à la suite} \fg{}: \hspace*{7cm} \og {\bf 2 à la suite} \fg{}: \\
\begin{pspicture}(0,0)(4,4.5) %7
   \facec{1}{1} \facec{1}{2} \facec{1}{3} \face{2}{0} \face{2}{1} \facer{0}{3}
\end{pspicture}
\begin{pspicture}(0,0)(4,4.5) %8
   \facec{1}{1} \facec{1}{2} \facec{1}{3} \face{2}{0} \face{2}{1} \facer{0}{2}
\end{pspicture}
\begin{pspicture}(0,0)(3,4.5) %9
   \facec{1}{1} \facec{1}{2} \facec{1}{3} \face{2}{0} \face{2}{1} \facer{0}{1}
\end{pspicture}
\begin{pspicture}(0,0)(4,4) %10
   \facec{1}{1} \facec{1}{2} \facec{1}{3} \face{2}{0} \face{2}{1} \facer{2}{-1}
\end{pspicture}
\begin{pspicture}(0,0)(4,3.5) %11
   \facec{0}{0} \facec{1}{0} \face{1}{1} \face{2}{1} \face{2}{2} \face{3}{2}
\end{pspicture}}
\end{center}
