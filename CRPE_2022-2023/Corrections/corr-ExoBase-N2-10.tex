\ \\ [-5mm]
   \begin{enumerate}
      \item \\
         \begin{minipage}{8cm}
            On trouve : $\opdiv[maxdivstep=7,voperation=top]{22}{7}$ \medskip
         \end{minipage}
         \begin{minipage}{6cm}
            Étant donné que l'on obtient 1 comme reste à l'ordre 7, tout comme au premier ordre, la suite des décimales va se répéter de manière identique toutes les 6 décimales. Donc, {\blue $\dfrac{22}{7} =3,\overline{142857}$}
         \end{minipage}
      \item
         \begin{enumerate}
            \item D'une part, $100x-x =$ {\blue  $99x$} et d'autre part, $100x-x =27,\overline{27}-0,\overline{27} =$ {\blue $27$}. \smallskip
            \item Donc, $99x =27$, soit $x =\dfrac{27}{99}$ ou encore : {\blue $x =\dfrac{3}{11}$}. \smallskip
         \end{enumerate}
      \setcounter{enumi}{2}
      \item Soit $y =19,\overline{78}$, alors $100y-y =1978,\overline{78}-19,\overline{78} =1959$. Donc, $99y =1959$, soit $y =\dfrac{1959}{99} =${\blue $\dfrac{653}{33}$}. \smallskip
      \item Soit $z =0,999\dots =0,\overline{9}$, alors $10z-z =9,\overline{9}-0,\overline{9} =9$. Donc, $9z =9$, soit : {\blue $z =0,999\dots =1$.}
   \end{enumerate}
