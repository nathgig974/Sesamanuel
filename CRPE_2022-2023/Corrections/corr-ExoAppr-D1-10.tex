\ \\ [-5mm]
\begin{enumerate}
   \item Pour l'abscisse $r =1,5$ cm, on lit une aire environ égale à \bm{450 cm$^2$.}
   \item Pour une ordonnée $\mathcal{A} =300$ cm$^2$, on lit deux valeurs pour le rayon : \bm{$r_1 \approx2,5$ cm et $r_2 \approx 5,25$ cm.}
   \item Pour une canette classique de rayon 3,3 cm, on lit une aire d'environ 270 cm$^2$ et pour une canette slim de rayon 2,8 cm, on lit une aire d'environ 285 cm$^2$. donc, \\
   \bm{c'est la canette classique qui demande le moins de surface de métal.}
   \item La surface minimale est atteinte pour un rayon d'environ \bm{3,75 cm.}
\end{enumerate}
