\ \\ [-5mm]
   \begin{enumerate}
      \item
         \begin{enumerate}
            \item Les deux fonctions sont affines, donc représentées par des droites. On peut lire l'ordonnée à l'origine de chacune des droites :
               \begin{itemize}
                  \item l'ordonnée à l'origine de $f$ vaut 284, qui correspond à l'ordonnée à l'origine de la droite $C_1$ ;
                  \item l'ordonnée à l'origine de $g$ vaut 115, qui correspond à l'ordonnée à l'origine de la droite $C_2$.
               \end{itemize}
               {\blue La fonction $f$ est représentée par la courbe $C_1$ et $g$ par $C_2$}.
            \item Pour 6 palettes, c'est la courbe $C_2$ qui a l'ordonnée la plus petite, donc la fonction $g$, ce qui correspond à {\blue la société B}.
            \item Graphiquement, la courbe représentative de $g$ est située en dessous de celle de $f$ pour un nombre de palettes inférieur à 10. Donc, {\blue entre 0 et 9 palettes, il est préférable de choisir la société B, et à partir de 10 palettes il vaut mieux choisir la société A}.
         \end{enumerate}
      \setcounter{enumi}{1}
      \item $f(x) =g(x) \iff 12x+284 =29x+115$ \\
         \hspace*{2.13cm} $\iff 284-115 =29x-12x$ \\
         \hspace*{2.13cm} $\iff 169 =17x$ \\ [1mm]
         \hspace*{2.13cm} $\iff {\blue x =\dfrac{169}{17}} x \approx9,94$ \\ [1mm]
         $x$ étant un nombre entier, {\blue la société B est bien préférable pour un nombre de palettes compris entre 0 et 9}.
      \end{enumerate}
