\ \\ [-5mm]
\begin{enumerate}
   \item
   \begin{enumerate}
      \item Le diamètre est lu grâce à la côte \og R15 \fg, il s'agit donc d'une jante de diamètre 15 pouces. \\
      Or, $15\times2,54\text{ cm} =38,1\text{ cm}$. \\
      \bm{Le diamètre de la jante vaut 38,1 cm.} \\
      \item La hauteur du pneu peut-être calculée grâce à la côte \og 195/65 \fg. \\
      Donc, la hauteur du pneu vaut 65 \% de sa largeur (195 mm). Or, $\dfrac{65}{100}\times195\text{ mm} =126,75\text{ mm}$. \\
      \bm{La hauteur du pneu est 12,675 cm.} \\
      \item $\text{Diamètre de la roue} = \text{diamètre de la jante} + 2\times\text{hauteur du pneu}$ \\
      \hspace*{3.38cm} $= 38,1\text{ cm} + 2\times12,675\text{ cm}=63,45\text{ cm}$. \\
      \bm{Le diamètre total de la roue est 63,45 cm.} \\
   \end{enumerate}
   \item Les informations inscrites sur le pneu sont au nombre de cinq :
   \begin{itemize}
      \item largeur : 20,5 cm = {\bf 205} mm ;
      \item hauteur : $\text{hauteur du pneu} = \dfrac{\text{diamètre total du pneu}-\text{diamètre de la jante}}{2}$ \\ [1mm]
      \hspace*{4.7cm} $=\dfrac{63,19\text{ cm}-40,64\text{ cm}}{2} =11,275\text{ cm}$. \\ [1mm]
      Or, une mesure de 11,275 cm représente un pourcentage de $\dfrac{11,275\text{ cm}}{20,5\text{ cm}}\times100 =\bf{55}\,\%$ par rapport à une mesure de 20,5 cm ;
      \item diamètre : le diamètre de la jante vaut 40,64 cm, qu'il faut convertir en pouce. Or, un pouce vaut 2,54 cm et $40,64\div2,54=16$ donc, l'inscription est {\bf R16} ;
      \item indice du poids : la charge maximale est de 412 kg ce qui correspond à l'indice {\bf 77} ;
      \item indice de vitesse : la vitesse maximale est de 270 km/h ce qui correspond à l'indice {\bf W}. \\
   \end{itemize}
   \bm{Les informations inscrites sur ce pneu sont : 205/55 R16 77 W}.
\end{enumerate}
