\ \\ [-5mm]
  \begin{enumerate}
      \item La mesure de la taille des dalles doit diviser à la fois la largeur de l'allée de \ucm{120}, la longueur totale du sol de \ucm{1040} ($\ucm{800}+2\times\ucm{120}$) et la largeur totale du sol de 740 cm ($\ucm{500}+2\times\ucm{120}$). Or, \\
      \begin{colitemize}{3}
         \item $120 =2^3\times3\times5$
         \item $1\,040 =2^4\times5\times13$
         \item $740 =2^2\times5\times37$
      \end{colitemize}
      Les diviseurs communs à ces trois nombres sont 1;\,2;\,4;\,5;\,10;\,20. \\
      {\blue Les longueurs possibles des dalles sont \ucm{1} ;  \ucm{2} ; \ucm{4} ; \ucm{5} ; \ucm{10} et \ucm{20}.}
      \item
      \begin{enumerate}
         \item On peut considérer que l'on a deux types d'allées : deux allées A de \ucm{1040} par \ucm{120} et deux allées B de \ucm{500} par \ucm{120}.
         \begin{itemize}
            \item A : on peut poser $\ucm{1040}\div20 \text{ cm/dalle} =52$ dalles dans la longueur et $\ucm{120}\div20\text{ cm/dalle} =6$ dalles dans la largeur, ce qui donne donc 312 dalles sur toute la surface ($52\times6$).
            \item B : on peut poser $\ucm{500}\div20 \text{ cm/dalle}=25$ dalles dans la longueur et $\ucm{120}\div20\text{ cm/dalle} =6$ dalles dans la largeur, ce qui donne donc $150$ dalles sur toute la surface ($25\times6$).
         \end{itemize}
         Au total, on a donc $2\times312$ dalles + $2\times150$ dalles = 924 dalles. \\
         {\blue M. Durand aura besoin de 924 dalles de 20 cm de côté pour couvrir son pourtour de piscine.}
         \item On a besoin de 16 dalles de 5 cm de côté pour couvrir une dalle de 20 cm de côté (la longueur du côté étant divisée par 4, son aire est divisée par $4^2$). Il faut donc 16 fois plus de dalles de 5 cm pour couvrir la même surface. Or, $924\times16 =14\,784$, d'où : \\
         {\blue M. Durand aura besoin de 14\,784 dalles de 5 cm de côté pour couvrir son pourtour de piscine.}
      \end{enumerate}
   \end{enumerate}
