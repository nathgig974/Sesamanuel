\ \\ [-5mm]
\begin{enumerate}
   \item Figure à l'échelle 1/2. \\
   \begin{pspicture}(-5,-2)(4.5,3.3)
   {\psset{unit=0.5}
      \pstGeonode[PosAngle={-135,-45,45,135},CurveType=polygon](0,0){A}(8,0){D}(8,5){C}(0,5){B}
      \pstGeonode[PosAngle=90](2,5){P}(-0.56,1.93){R}(5.44,-3.07){H}
      \pstLineAB[linecolor=B2]{P}{D}
      \pstLineAB[linecolor=B2]{P}{R}
      \pstLineAB[linecolor=B2]{R}{H}
      \pstLineAB[linecolor=B2]{H}{D}
      \pstRightAngle{P}{C}{D}
      \pstRightAngle[linecolor=G1]{P}{D}{H}
      \pstLineAB[linecolor=A1]{P}{H}
      \psarc[linecolor=B1](8,0){4}{190}{270}}
   \end{pspicture}
   \item{\bf Calcul de DP :} dans le triangle DCP rectangle en C, on utilise le théorème de Pythagore avec des mesures en cm : $\text{DP}^2 =\text{DC}^2+\text{CP}^2 =5^2+(8-2)^2 =25+36 =61 \Longrightarrow \text{DP} =\sqrt{61}$ cm. \\
      {\bf Calcul de PH :} le triangle PDH est rectangle en H, d'après le théorème de Pythagore, on a \\
      $\text{PH}^2 =\text{PD}^2+\text{DH}^2 =61+16 =77 \Longrightarrow$ \bm{PH $=\sqrt{77}$ cm.}
   \item Le volume d'un prime se calcule en multipliant l'aire de sa base par sa hauteur. \\ [1mm]
   Calcul de l'aire de la base : $\mathcal{A} =\dfrac{(\text{BP}+\text{AD})\times\text{AB}}{2} =\dfrac{(2\text{ cm}+8\text{ cm})\times5\text{ cm}}{2} =25\text{ cm}^2.$ \\ [1.5mm]
   Calcul du volume du prisme : $\mathcal{V} =\mathcal{A}\times\text{AE} =25\text{ cm}^2\times4\text{ cm} =100\text{ cm}^3$. \\ [1mm]
   \bm{Le volume du prisme ABPDEFRH est de 100 cm$^3$.}
\end{enumerate}
