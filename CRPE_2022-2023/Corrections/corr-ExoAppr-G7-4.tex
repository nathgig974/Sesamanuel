\ \\ [-5.5mm]
   \begin{enumerate}
      \item On a $v =\dfrac{d}{t}$ donc, $t =\dfrac{d}{v} =\dfrac{150\times10^6\,\ukm{}}{3\times10^8\,\ums{}} =\dfrac{150\times10^9\,\um{}}{3\times10^8\,\ums{}} =\us{500}$. \\ [1mm]
      Or, $\us{500} =8\times\us{60}+\us{20} =\umin{8}+\us{20}$ donc : \\
      \bm{un signal lumineux émis par le Soleil met \umin{8} et \us{20} pour parvenir à la Terre.}
      \item $d =v\times t$ donc : $d =3\times10^5\,\ukms{}\times365,25\times24\times\us{3600} \approx9,47\times10^{12}\,\ukm{}$. \\
      \bm{Une année -- lumière vaut environ $9,47\times10^{12}\,\ukm{}$.}
      \item
      \begin{enumerate}
         \item 1 UA = $150\times10^6\,\ukm{}$ donc, $4,5\times10^9\,\ukm{}$ correspondent à $\dfrac{4,5\times10^9\,\ukm{}\times1\,\text{UA}}{150\times10^6\,\ukm{}} =30\,\text{UA}$. \\ [1mm]
         \bm{Neptune est située à 30 UA du Soleil.}
         \item 30 UA dans la réalité correspondent à \um{1} sur la maquette. \\
         Donc, 1 UA dans la réalité correspond à $\dfrac{1}{30}\um{} \approx\um{0,033} \approx\ucm{3,3}$ sur la maquette. \\ [1mm]
         \bm{Il faudra placer la Terre à environ \ucm{3,3} du Soleil.}
      \end{enumerate}
   \end{enumerate}
