\ \\ [-5mm]
\begin{enumerate}
   \item
   \begin{enumerate}
      \item On effectue la division euclidienne de \ucm{418} par \ucm{29} :
      $$\opidiv{418}{29}$$
      donc, \bm{le carreleur pourra poser au maximum 14 dalles dans la largeur de la pièce.}
      \item On effectue la division euclidienne de \ucm{567} par \ucm{29} :
      $$\opidiv{567}{29}$$
      donc, \bm{le carreleur pourra poser au maximum 19 dalles dans la longueur de la pièce.}
      \item Dans la largeur, il restera 12 cm à combler pour 15 joints ($14+1$), soit  0,8 cm par joint ($12\div15 =0,8$) ; dans la longueur, il restera 16 cm à combler pour $20$ joints ($19+1$), soit 0,8 cm par joint ($16\div20$) d'où : \\
      \bm{les joints autour des dalles auront tous une largeur de 8 mm.}
   \end{enumerate}
   \item Une dalle et un joint mesurent 36,6 cm = 366 mm.
   $$\text{Pour la largeur :} \quad \opidiv{4180}{366}$$
   $$\text{Pour la longueur :} \quad  \opidiv{5670}{366}$$
   donc, le carreleur devra poser 11 dalles entières dans la largeur plus une qu'il devra couper et 15 dalles entières dans la longueur plus une qu'il devra couper. \\
    Or, $12\times16 =192$ donc, \bm{le nombre de dalles nécessaire est de 192.}
   \item
   \begin{itemize}
      \item Premier cas : $14\times19 =266$, il a besoin de 266 dalles à 2,30 \euro{} chacune, soit un montant total de : \\
      $266\times2,30$ \euro{}  $=611,80$ \euro{}.
      \item Second cas : il a besoin de 192 dalles à 3,10 \euro{} chacune, soit un montant de : $192\times3,10$ \euro{} $=595,20$ \euro{}.
   \end{itemize}
   \bm{Le choix le moins coûteux est le deuxième choix.}
\end{enumerate}
