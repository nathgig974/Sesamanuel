\ \\ [-5mm]
   \begin{enumerate}
      \item Dans le triangle ABC rectangle en A, on applique le théorème de Pythagore avec des mesures en cm : \\
         $\text{BC}^2=\text{BA}^2+\text{AC}^2 =8^2+6^2 =64+36 =100 \Longrightarrow {\blue BC = \ucm{10}}$.
   \item Dans le triangle ABC rectangle en A, on utilise le cosinus : \\ [1mm]
      $\cos(\widehat{ABC}) =\dfrac{BA}{BC} =\dfrac{8}{10} =\dfrac45 \Longrightarrow \widehat{ABC} =\arccos\left(\dfrac45\right) \approx {\blue \udeg{37}}$. \smallskip
   \item Le quadrilatère ADEF possède trois angles droits, c'est donc un rectangle. Or, les diagonales d'un rectangle sont de même longueur d'où :
      {\blue AE = DF}.
   \end{enumerate}
