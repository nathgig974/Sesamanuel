\ \\ [-5mm]
\begin{enumerate}
   \item La figure suivante est à l'échelle 1\slash3. \\
   \begin{pspicture}(-5,-3.5)(5,2.5)
   {\psset{unit=0.33}
      \pstGeonode[PosAngle={180,0,0,180,0,45,45}]{A}(14,0){B}(14,9){F}(0,3){E}(14,-9){G}(3.5,0){P}(8,0){M}
      \pstLineAB{A}{B}
      \pstLineAB{B}{F}
      \pstLineAB{F}{E}
      \pstLineAB{E}{A}
      \pstLineAB[linewidth=0.5mm,linecolor=B1]{E}{G}
      \pstLineAB[linestyle=dashed,linecolor=A1]{E}{M}
      \pstLineAB[linestyle=dashed,linecolor=A1]{M}{G}
      \pstRightAngle[linecolor=G1]{E}{A}{B}
      \pstRightAngle[linecolor=G1]{A}{B}{F}
      \pstSegmentMark{B}{F}
      \pstSegmentMark{B}{G}}
   \end{pspicture}
   \item Le point $P$ est situé sur le segment [$EG$], par construction. On a donc $EP+PG =EG$. \\
   Tout point $M$ du plan non situé sur le segment [$EG$] vérifie \\
   $EM+MG > EG$, soit $EM+MG > EP+PG$. \\
   Si le point $M$ appartient au segment [$EG$], on a $EM+MG = EP+PG$. \\
Pour tout point $M$ de [$AB$], on a \bm{$EM+MG \geqslant EP+PG$ avec égalité lorsque $M$ est placé en $P$.} \\
   $G$ est le symétrique de $F$ par rapport à $B$ donc, $B$ est le milieu du segment [$FG$]. \\
   La droite ($AB$) est perpendiculaire à la droite ($FB$). Donc, ($AB$) est la médiatrice du segment [$FG$]. \\
   Le point $M$ étant situé sur la médiatrice de [$FG$], il est à égale distance des points $F$ et $G$, c'est-à-dire $MF =MG$ et on a alors $EM+MG = EM+MF$. \\
   La valeur minimale de $EM+MF$ correspond donc à la valeur minimale de $EM+MG$, qui est $EP+PG$. \\
   \bm{La valeur $EM+MF$ est minimale lorsque $M$ est placé en $P$.}
   \item Les droites ($EA$) et ($FB$) sont toutes deux perpendiculaires à la même droite ($AB$), elles sont donc parallèles entre elles. \\
      Les points $F, B$ et $G$ sont alignés donc, les droites ($EA$) et ($BG$) sont parallèles. \\
      Les points $E, P, G$ et $A, P, B$ sont alignés dans le même ordre. \\
      On peut donc appliquer le théorème de Thalès et sa conséquence : $\dfrac{PE}{PG} =\dfrac{PA}{PB}  =\dfrac{EA}{GB}$. \\
      Or, $P\in[AB]$ donc $AP+PB =AB \iff PB =14-AP$. \\
      Avec $EA =3$ et $GB =FB =9$, on obtient bien \quad \bm{$\dfrac{AP}{14-AP} =\dfrac{3}{9}$.}
      $\dfrac{AP}{14-AP} =\dfrac{3}{9} \iff 9\times AP =3\times(14-AP) \iff 9AP+3AP =42 \iff AP =\dfrac{42}{12} =\dfrac{7}{2}.$ \\ [1mm]
      \bm{La mesure de $AP$ est de $3,5$\,cm.}
   \item D'après le question 2), la valeur minimale de $EM+MF$ est atteinte lorsque $M$ est en $P$. Calculons $EP+PF$.
   \begin{itemize}
      \item Calcul de $EP$. Dans le triangle $EAP$ rectangle en $A$, d'après le théorème de Pythagore : \\
      $EP^2 = EA^2 + AP^2=3^2+\left(\dfrac72\right)^2=9+\dfrac{49}{4}=\dfrac{85}{4}$, \quad soit $EP =\dfrac{\sqrt{85}}{2}$.
      \item Calcul de $PF$. Dans le triangle $PBF$ rectangle en $B$, d'après le théorème de Pythagore : \\
      $PF^2 =PB^2+BF^2$ avec $PB =AB-AP =14-\dfrac72 =\dfrac{21}{2}$. \\
      D'où  $PF^2=\left(\dfrac{21}{2}\right)^2+9^2 =\dfrac{441}{4}+81=\dfrac{765}{4}$, \quad soit $PF=\dfrac{3\sqrt{85}}{2}$. \\
   \end{itemize}
   Pas conséquent, $EP+PF =\dfrac{\sqrt{85}}{2}$ + $\dfrac{3\sqrt{85}}{2} =2\sqrt{85}\approx 18,44$. \\
   \bm{La valeur minimale de $EM+MF$ est de $2\sqrt{85}$ cm $\approx 18,4$ cm.}
   \end{enumerate}
