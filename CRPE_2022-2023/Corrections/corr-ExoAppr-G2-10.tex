\ \\ [-5mm]
\begin{enumerate}
   \item Les points M et B sont des points du cercle $\mathcal{C}$ de centre O donc : OM = OB. \\
   M est un point de la médiatrice de [OB] donc, MO = MB. \\
   On en déduit que OM = OB = MB et \bm{le triangle OMB est équilatéral.}
   \item le segment [AB] est un diamètre du cercle de centre O de rayon $r$ donc, OA = OB = $r$. \\
   S est le symétrique de M par rapport à O donc, [SM] est un autre diamètre de ce cercle et OM = OS = $r$. \\
   Le quadrilatère AMBS a ses diagonales de même longueur et qui se coupent en leur milieu, \bm{c'est un rectangle.}
   \item Dans le triangle AMB rectangle en M, on utilise le théorème de Pythagore (les mesures sont en cm) : \\
   $\text{AB}^2 =\text{AM}^2+\text{MB}^2 \iff \text{AM}^2 =10^2-5^2 =75$, soit $AB =\sqrt{75} =5\sqrt3$. \\
   L'aire du rectangle AMBS vaut alors $\mathcal{A} =\text{AM}\times\text{MB} =5\sqrt3\text{ cm}\times5\text{ cm} =25\sqrt3\text{ cm}^2$. \\
   \bm{Le rectangle AMBS a une aire de $25\sqrt3$ cm$^2$.} \\
   \item On a déjà démontré en question 1) que MO = MB. De plus, N est sur la médiatrice de [OB] donc NO = NB. \\
   M et N étant des points du cercle, on a ON = OM, d'où BM = MO = ON = NB et les quatre côtés sont de même mesure. Par conséquent, \bm{Le quadrilatère OMBN est un losange.}
\end{enumerate}
