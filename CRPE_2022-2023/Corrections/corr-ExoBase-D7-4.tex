\ \\ [-5mm]
   \begin{enumerate}
      \item Les boules sont indiscernables au toucher, nous sommes donc dans un cas d'équiprobabilité. \\
         Il y a 7 boules bleues pour un total de $25$ boules ($3+4+5+7+6 =25)$. D'où : $\mathcal{P} =\dfrac{7}{25}$. \\
         {\blue La probabilité de tirer une boule bleue est de $\dfrac7{25}$.}
      \item Soit $n$ le nombre de boules bleues à ajouter, la probabilité de tirer une boule bleue est alors de $\mathcal{P} =\dfrac{7+n}{25+n}.$ \\
         $\dfrac{7+n}{25+n} \geqslant 0,4 \iff
            7+n \geqslant 10+0,4n \iff
            n-0,4n \geqslant 10-7 \iff
            0,6n \geqslant 3 \iff
            n \geqslant \dfrac{3}{0,6} \iff
            n \geqslant 5$. \\ [1mm]
         {\blue Il faut ajouter au minimum 5 boules bleues} avant le tirage pour que la probabilité de tirer une boule bleue soit supérieure ou égale à 0,4.
      \item Soit $m$ le nombre de boules rouges à ajouter, la probabilité de tirer une boule bleue est alors de $\mathcal{P} =\dfrac{7}{25+m}.$  \\
         $\dfrac{7}{25+m} \leqslant 0,2 \iff
          7 \leqslant 5+0,2m \iff
          -0,2m \leqslant 5-7 \iff
          -0,2m \leqslant -2 \iff
          m \geqslant \dfrac{-2}{-0,2} \iff
          m \geqslant 10$. \\ [1mm]
         {\blue Il faut ajouter au minimum 10 boules rouges} avant le tirage pour que la probabilité de tirer une boule bleue soit inférieure ou égale à 0,2.
   \end{enumerate}
