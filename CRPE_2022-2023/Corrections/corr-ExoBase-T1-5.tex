\ \\ [-5mm]
   \begin{enumerate}
      \item Cette situation permet de travailler le champ multiplicatif puisqu'il s'agit d'une activité d'agrandissement. Plus particulièrement, la situation se situe dans le thème transversal de la proportionnalité dans le domaine des grandeurs et mesures.
      \item Analyse des trois premières affiches :
         \begin{itemize}
            \item {\bf Affiche 1.} Les élèves ont \og vu \fg{} que 6 cm, c'est 4 cm + 2 cm. Ils ont donc ajouté 2 cm à toutes les mesures du côté droit et du bas du carré. Pour le haut et le côté gauche, ils ont ajouté 3 cm, peut-être parce que les valeurs d'origine à augmenter (5 cm et 6 cm) sont plus grandes, donc il faut ajouter un peu plus que 2 cm. On remarque toutefois que l'agrandissement est le même pour toutes les mesures d'un même côté, y compris pour 7 cm qui est plus grand que 6 cm. \\
            Réussites : ils ont su modéliser la situation par un schéma, ils ont repéré qu'il ne fallait pas toujours ajouter le même nombre. \\
            Erreurs : ils ont commis l'erreur classique consistant à penser qu'augmenter une mesure, c'est ajouter un même nombre, c'est à dire effectuer une addition alors que l'on se situe dans le champ multiplicatif. Les valeurs à l'intérieur du carré n'ont pas été agrandies (oubli ?).
           \item {\bf Affiche 2.} Les élèves ont remarqué qu'ajouter 2 cm à 4 cm, c'est ajouter deux fois 1 cm qui est le quart de 4~cm, c'est à dire, en langage mathématique : 6 cm = 4 cm + $\dfrac14\times$ 4 cm. Ils utilisent les propriétés de linéarité mixte (additive et multiplicative). \\
           Réussites : ils ont su modéliser la situation en donnant toutes les mesures nécessaires pour l'agrandissement du puzzle, ils ont trouvé les bons résultats par un raisonnement tout à fait pertinent. \\
           Erreur : la phrase d'explication est un peu litigieuse car elle laisse penser qu'il faut ajouter le quart de la valeur d'origine (par exemple 4 cm + 1 cm = 5 cm) puis multiplier ce nombre par 2 (2$\times$5 cm = 10 cm). Il aurait été préférable d'écrire par exemple \og il faut prendre le quart de chaque nombre, le multiplier par 2, puis ajouter le nombre obtenu au nombre de départ \fg.
            \item {\bf Affiche 3.} Ils ont utilisé les propriétés de linéarité multiplicative appliquées aux valeurs données dans l'énoncé (4 cm devient 6 cm) en prenant les deux coefficients scalaires $\dfrac12$  et 3 : 6, c'est 3 fois 2 et 2, c'est 4 divisé par 2. Ensuite, ils ont appliqué ces coefficients à chacune des mesures du dessin. \\
      Réussites : explication claire de la procédure, calculs corrects sans erreur d'écriture y compris pour les nombres décimaux. \\
      Pas d'erreur.
         \end{itemize}
      \item Procédures utilisées pour l'affiche 4 :
         \begin{itemize}
            \item \bm{$4\to6$} : reprise de l'énoncé.
            \item \bm{$6\to9$} : on peut penser que les élèves ont ajouté la moitié de 6 à 9 qui est 3 en vertu de ce qu'ils ont fait par la suite.
            \item \bm{$7\to10,5$} : pas d'explication.
            \item \bm{$2\to3$} : par analogie avec la correspondance $4\to6$, 2 étant la moitié de 4, il faut calculer la moitié de 6 qui est 3. Ils ont utilisé implicitement la propriété de linéarité multiplicative de la proportionnalité.
            \item \bm{$5\to7,5$} : 5 c'est 4 + 1 soit 4 + 2$\div$2 donc, les élèves utilisent une procédure mixte utilisant la linéarité additive et multiplicative. Ils appliquent alors cette procédure pour passer de 5 à 7,5 : pour 4 cm, l'agrandissement vaut 6 cm et pour 2 cm, il est de 3 cm, donc pour 1 cm il est de 1,5 cm et pour 5 cm on a donc 6 cm + 1,5 cm = 7,5 cm.
            \item \bm{$9\to13,5$} : même procédure que précédemment en utilisant la décomposition $9 = 4 + 4 + 1$ soit $4 + 4 + 2\div2$. Pour 4 cm on a 6 cm et pour 2 cm on a 3 cm, donc pour 9 cm, on a 6 cm + 6 cm + 3 cm $\div$ 2 = 13,5 cm.
         \end{itemize}
   \end{enumerate}
