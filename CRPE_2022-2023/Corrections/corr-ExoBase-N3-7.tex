\ \\ [-5mm]
   \begin{enumerate}
      \item
         \begin{enumerate}
            \item {\hautab{1.6}
               $\syst{\dfrac12 & = \dfrac36}{\dfrac23 & = \dfrac46}$ \qquad donc, \qquad $\dfrac12<\dfrac23$. \\ [1mm]
               $\syst{\dfrac{12}{13} & = \dfrac{12\times14}{13\times14} & =\dfrac{168}{182}}{\dfrac{13}{14} & =\dfrac{13\times13}{14\times13} & =\dfrac{169}{182}}$ \qquad donc, \qquad $\dfrac{12}{13}<\dfrac{13}{14}$. \\ [1mm]
               $\syst{\dfrac{176}{177} & =\dfrac{176\times178}{177\times178} & =\dfrac{31\,328}{31\,506}}{\dfrac{177}{178} & =\dfrac{177\times177}{178\times177} & =\dfrac{31\,329}{31\,506}}$ \qquad donc, \qquad $\dfrac{176}{177}<\dfrac{177}{178}$. \\ [1mm]
               On peut conjecturer que pour tout entier naturel non nul, {\blue $\dfrac{n-1}{n}<\dfrac{n}{n+1}$.}} \\ [1mm]
            \item $\dfrac{n-1}{n} =\dfrac{(n-1)(n+1)}{n(n+1)} =\dfrac{n^2-1}{n(n+1)}$ \quad et \quad $\dfrac{n}{n+1} =\dfrac{n\times n}{(n+1)n} =\dfrac{n^2}{n(n+1)}$. \\ [1mm]
               On a $\dfrac{n^2-1}{n(n+1)}<\dfrac{n^2}{n(n+1)}$, d'où la conjecture. \\ [1mm]
            \item On choisit $n =987\,654\,322$ et on applique le résultat démontré à la question précédente : \\ [2mm]
               {\blue $\dfrac{987\,654\,322}{987\,654\,323}>\dfrac{987\,654\,321}{987\,654\,322}$.} \\ [1mm]
         \end{enumerate}
      \item On a $1,117<\dfrac{p}{1789}<1,118 \iff 1998,313<p<2000,102$ et $p$ est un entier donc, \\ [1mm]
         {\blue les valeurs possibles pour $p$ sont 1999 et 2000.}
   \end{enumerate}
