\ \\ [-5mm]
   \begin{enumerate}
      \item La région correspondant à l'indice 1 est constituée de deux demi-plans dont les frontières sont des droites parallèles, de part et d'autre de L et situées à 500 mètres de L.
      \item La région correspondant à l'indice 2 est l'extérieur du cercle de centre E et de rayon 800 mètres.
      \item La région correspondant à l'indice 3 est l'intérieur du cercle de centre B et de rayon 300 mètres.
      \item La région correspondant à l'indice 4 est la médiatrice du segment [SP].
      \item L'échelle est 1/10\,000\up{e} : 1 cm sur la figure représente 10 000 cm dans la réalité, c'est-à-dire 100 m.
   \end{enumerate}
   Sur la figure suivante, le trésor se trouve sur le segment rouge. \\
   \begin{center}
      \begin{pspicture*}(0,-4.5)(16,8)
      \psset{algebraic=true}
         \psplot{-1}{15}{(--27.33--2.04*x)/7.78}
         \rput[bl](4.1,5){L}
         \psdots[dotstyle=+](10.1,1)(4.3,-0.5)(13.7,-3.8)(9,-2.15)
         \rput[bl](10.3,0.6){B}
         \rput[bl](4.5,-0.9){E}
         \rput[bl](13.9,-4.3){S}
         \rput[bl](9.1,-2){P}
         \psline[linewidth=2mm,linecolor=B2](12.3,-0.3)(13,1.7)
         \psset{linecolor=A1}
         \psplot{-1}{15}{(-12.89--2.04*x)/7.78}
         \psplot{-1}{15}{(--67.54--2.04*x)/7.78}
        \pscircle(4.3,-0.5){8}
         \pscircle(10.1,1){3}
         \psplot{-12.3}{24.2}{(-58.55--4.73*x)/1.62}
      \end{pspicture*}
   \end{center}
