\ \\ [-5mm]
   \begin{enumerate}
      \item On considère le triangle AOI rectangle en I et on pose OA = OC = $x$. On a alors OI = $x-5$. \\
         On applique le théorème de Pythagore dans ce triangle avec des mesures en mètre. \\
         AO$^2$ = AI$^2$ + IO$^2 \iff x^2 = 12^2+(x-5)^2$ \\
         \hspace*{2.45cm} $\iff \cancel{x^2} =144+ \cancel{x^2}-10x+25$ \\
         \hspace*{2.45cm} $\iff 10x =169$ \\
         \hspace*{2.45cm}  $\iff x =16,9$. {\blue Le rayon OA de l'arche est \um{16,9}}.
      \item On considère le schéma de coupe suivant dans lequel on a représenté la moitié de la figure puisqu'il y a symétrie par rapport à (CO). Pour maximiser ses chances de passer sous l'arche, la péniche doit être centrée. \\
         \begin{minipage}{7cm}
         {\psset{unit=0.4}
         \small
            \begin{pspicture}(-1,-1.5)(14,18)
               \pspolygon(0,11.9)(12,11.9)(12,0)
               \psframe(12,15.9)(11.7,15.6)
               \psline(12,11.9)(12,16.9)
               \psarc(12,0){16.9}{90}{135.2}
               \psframe[fillstyle=solid,fillcolor=lightgray!50](6,11.9)(12,15.9)
               \psline(12,0)(6,15.9)
               \rput(9,14.3){Demi}
               \rput(9,13.6){Péniche}
               \rput(12,-0.5){O}
               \rput(6,16.5){F}
               \rput(12.5,15.9){J}
               \rput(12.5,11.9){I}
               \rput(12.5,17.4){C}
               \rput(-0.5,11.9){A}
               \rput(4.7,6){16,9 m}
               \rput(9.6,16.35){6 m}
               \rput(12.8,13.9){4 m}
            \end{pspicture}}
         \end{minipage}
         \qquad
         \begin{minipage}{8cm}
            Pour savoir si la péniche passe sous l'arche, il faut vérifier que la distance OF (F étant le coin supérieur gauche de la péniche vue de face) est inférieure au rayon de l'arche (\um{16,9}). \\ [2mm]
            On a : OJ = OI + IJ \\
            \hspace*{1.35cm} = (OC $-$ IC) + IJ \\
            \hspace*{1.35cm} = ($\um{16,9}-\um{5}$) + \um{4} \\
            \hspace*{1.35cm} = \um{15,9}. \\ [2mm]
            Dans le triangle OJF rectangle en J, on utilise le théorème de Pythagore avec des mesures en mètre : \\
            $\text{OF}^2 = \text{OJ}^2 + \text{JF}^2$ \\
            \hspace*{0.65cm}  $= 15,9^2 + 6^2$ \\
            \hspace*{0.65cm}  $= 288,81$ \\
            Donc, OF $\approx$ \um{16,99} > \um{16,9}.
         \end{minipage}
      {\blue Cette péniche ne pourra pas passer sous l'arche sans dommage}.
   \end{enumerate}
