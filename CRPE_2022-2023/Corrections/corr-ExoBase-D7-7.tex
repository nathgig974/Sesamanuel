   Dans cet exercice, on suppose que les billets sont vendus au hasard, et donc que nous sommes dans une situation d'équiprobabilité. \\ [1mm]
   \begin{enumerate}
      \item $p_1 =\dfrac{\text{nombre de billets permettant de gagner une télévision}}{\text{nombre total de billets}} =\dfrac{2}{300} =\dfrac{1}{150}$. \\ [2mm]
         {\blue La probabilité de gagner une télévision est de $\dfrac{1}{150}$}. \\ [2mm]
      \item $p_2 =\dfrac{\text{nombre de billets permettant de gagner un bon de réduction}}{\text{nombre total de billets}} =\dfrac{5+10}{300} =\dfrac{15}{300} =\dfrac{1}{20}$. \\ [2mm]
         {\blue La probabilité de gagner un bon de réduction est de $\dfrac{1}{20}$}. \smallskip
   \item
      \begin{enumerate}
         \item Calcul des dépenses $D$ de l'organisateur : \\
            $D =2\times\ueuro{500}+5\times\ueuro{100}+10\times\ueuro{50} +20\times\ueuro{0,50} =\ueuro{2010}$. \\
            Or, $2\,010\div300 \approx6,7$ donc, si l'organisateur vend 300 billets, \\
            {\blue il devra les vendre au minimum à \ueuro{6,70} pour ne pas perdre d'argent}. \\
         \item Soit $n$ le nombre de billets à ajouter aux 300 billets. On a l'équation suivante : \\
            $(300+n)\times\ueuro{2} \geq\ueuro{2010} \iff 600+2n\geq2\,010$ \\
            \hspace*{3.7cm} $\iff 2n \geq1410$ \\
            \hspace*{3.7cm} $\iff n\geq705$. \\
            {\blue À \ueuro{2}, l'organisateur doit ajouter au moins 705 billets pour ne pas perdre d'argent}.
      \end{enumerate}
   \end{enumerate}
