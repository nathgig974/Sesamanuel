On peut modéliser la situation par un tableau à double entrée. \\ [2mm]
\begin{minipage}{7cm}
      {\renewcommand{\arraystretch}{1.5}
      \begin{cltableau}{0.9\linewidth}{7}
         \hline
          & \bf 1 & \bf 2 & \bf 3 & \bf 4 & \bf 5 & \bf 6 \\
          \hline
          \bf 1 & 2 & 3 & 4 & 5 & 6 & 7 \\
          \hline
          \bf 2 & 3 & 4 & 5 & 6 & 7 & 8 \\
          \hline
          \bf 3 & 4 & 5 & 6 & 7 & 8 & 9 \\
          \hline
          \bf 4 & 5 & 6 & 7 & 8 & 9 & 10 \\
          \hline
          \bf 5 & 6 & 7 & 8 & 9 & 10 & 11 \\
          \hline
          \bf 6 & 7 & 8 & 9 & 10 & 11 & 12 \\
          \hline
      \end{cltableau}}
\end{minipage}
\begin{minipage}{9cm}
   On recense les résultats dans le tableau ci-contre.
   \smallskip
   \begin{itemize}
      \item Probabilité d'obtenir un résultat pair : $\mathcal{P}_1 =\dfrac{18}{36} =\dfrac12$ ; \\ [1mm]
probabilité d'obtenir un résultat impair : $\mathcal{P}_2 =\dfrac{18}{36} =\dfrac12$ ; \\ [1mm]
Les probabilités d'obtenir un résultat pair ou un résultat impair sont égales, donc, \bm{la première assertion est vraie.}
      \item Probabilité d'obtenir une somme égale à 7 : $\mathcal{P}_3 =\dfrac{6}{36} =\dfrac16$ ; \\ [1mm]
probabilité d'obtenir une somme égale à 5 : $\mathcal{P}_4 =\dfrac{4}{36} =\dfrac19$ ; \\ [1mm]
donc, \bm{la deuxième assertion est fausse.}
   \end{itemize}
\end{minipage}
