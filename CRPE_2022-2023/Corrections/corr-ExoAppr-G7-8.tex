\ \\ [-5mm]
\begin{enumerate}
   \item
   \begin{itemize}
      \item Calcul du volume d'aluminium nécessaire pour fabriquer le corps de la canette : \\
      $130\,\mu\text{m} =130\times10^{-6}\text{ m} =130\times10^{-4}\text{ cm} =0,013\text{ cm}$. \\
      $\mathcal{V} =\text{surface}\times\text{épaisseur} =268,42\text{ cm}^2\times0,013\text{ cm} \approx 3,49\text{ cm}^3$. \\
      \item Calcul de la masse d'aluminium nécéssaire pour fabrique le corps de la canette : \\ [1mm]
   la masse volumique de l'aluminium vaut $\mu =\dfrac{2\,700\text{ kg}}{1\text{ m}^3} =\dfrac{2\,700\,000\text{ g}}{1\,000\,000\text{ cm}^3} =2,7\text{ g/cm}^3$. \\ [1mm]
   Un volume de 3,49 cm$^3$ d'aluminium a une masse $m$ en gramme égale à $3,49\times2,7 \approx9,423$.
      \item Calcul de la masse totale d'aluminium nécéssaire pour la canette entière : \\
      $M =$ masse du corps + masse de l'anneau + masse de soudure \\
      $M \approx$ 9,4 g + 1,4 g + 1,9 g $\approx$ 12,7 g. \\
      {\bf Il faut environ 12,7 g d'aluminium pour fabriquer une canette classique.}
   \end{itemize}
   \item Pour un vélo de 9 kg, soit 9\,000 g, on effectue le calcul suivant : \\
   9\,000 g $\div$ 12,7 g $\approx$ 708,66. \\
   {\bf Il faudrait environ 709 canettes classiques pour fabriquer ce type de vélo.}
\end{enumerate}
