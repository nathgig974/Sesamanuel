On a la figure suivante : \\
\begin{minipage}{6.5cm}
{\psset{unit=0.9}
\begin{pspicture}(0,-0.8)(5,5.7)
   \pstGeonode[PosAngle={-135,-45,45,135},CurveType=polygon](0,0){A}(6.2,0){B}(6.2,5.2){C}(0,5.2){D}
   \pstLineAB[linecolor=B1]{A}{C}
\end{pspicture}}
\quad
\end{minipage}
\begin{minipage}{9.5cm}
   \begin{enumerate}
      \item Le triangle ABC étant rectangle en B, on peut appliquer le théorème de Pythagore : \\
      $\text{AC}^2 =\text{AB}^2+\text{BC}^2 \iff \text{AC}^2 =6,2^2+5,2^2$ \\
      $\iff \text{AC}^2 =38,44+27,04 =65,48$ \\
      $\Longrightarrow \text{AC} =\sqrt{65,48} \approx8,09$ cm \\
      \bm{La diagonale de sa console mesure environ 8,1 cm.}
      \item 3 pouces correspondent à 8,1 cm ; donc, 1 pouce correspond à 8,1 cm $\div$3 = 2,7 cm. \\
      \bm{Un pouce vaut environ 2,7 cm.}
      \item $10,1\times2,7 = 27,27$ donc, \bm{la diagonale d'un netbook de 10,1 pouces est d'environ 27,3 cm.}
   \end{enumerate}
\end{minipage}
