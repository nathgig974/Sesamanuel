\ \\ [-5mm]
\begin{enumerate}
      \item
      \begin{itemize}
         \item Énoncé 1 : \bm{faux.} \\
         Contre-exemple : $2x=1$ est un entier naturel, mais $x=0,5$ n'en est pas un (la moitié d'un entier naturel n'est pas forcément un entier naturel. Pour cela, il faudrait qu'il soit pair).
          \item Énoncé 2 : \bm{vrai.} \\ [1mm]
          Démonstration : $\dfrac{x}{2}=n\in\N$, donc, $x=2n\in\N$ (le double d'un entier naturel reste un entier naturel).
         \item Énoncé 3 : \bm{faux.} \\
         Contre-exemple : $x+1=0$ est un entier naturel, mais $x=-1$ est un entier relatif, non naturel.
      \end{itemize}
      \item Soit $x$, $y$ et $z$ les trois nombres recherchés, les données de l'énoncé nous permettent d'obtenir le système à trois équations et trois inconnues suivant : \\ [2mm]
         $\Syst{x+y\phantom{+z} & =78 & (L1)}{x\phantom{+y}+z & =59 & (L2)}{\phantom{x+}y+z & =43 & (L3)} \iff \Syst{x+y\phantom{+z} & =78 & (L1)}{\phantom{x+}y-z & =19 & (L1-L2)}{\phantom{x+}y+z & =43 & (L3)}$ \\ [3mm]
         \hspace*{3.8cm} $\iff \Syst{x+y\phantom{+z} & =78 & (L1)}{\phantom{x+}y-z & =19& (L2)}{\phantom{x+y}2z & =24 & (L3-L2)}$ \\ [3mm]
         \hspace*{3.8cm} $\iff \Syst{x & =78-(19+12)}{y & =19+12 \phantom{000001}}{z & =12 \phantom{0000000000}} \iff \Syst{x & =47}{y & =31}{z & =12}$. \\ [2mm]
         Les trois nombres recherchés sont \bm{12, 31 et 47.}
   \end{enumerate}
