Dans cet exercice, on suppose que les billets sont vendus au hasard, et donc que nous sommes dans une situation d'équiprobabilité. \\ [3mm]
\begin{enumerate}
   \item $p_1 =\dfrac{\text{nombre de billets permettant de gagner une télévision}}{\text{nombre total de billets}} =\dfrac{2}{300} =\dfrac{1}{150}$. \\ [2mm]
   \bm{La probabilité de gagner une télévision est de 1/150.} \\ [2mm]
   \item $p_2 =\dfrac{\text{nombre de billets permettant de gagner un bon de réduction}}{\text{nombre total de billets}} =\dfrac{5+10}{300} =\dfrac{15}{300} =\dfrac{1}{20}$. \\ [2mm]
   \bm{La probabilité de gagner un bon de réduction est de 1/20.}
   \item
   \begin{enumerate}
      \item Calcul des dépenses $D$ de l'organisateur : \\
      $D =2\times500\text{ \euro}+5\times100\text{ \euro}+10\times50\text{ \euro} +20\times0,50\text{ \euro} =2\,010\text{ \euro}$. \\
      Or, $2\,010\div300 \approx6,7$ donc, si l'organisateur vend 300 billets, \\
      \bm{il devra les vendre au minimum à 6,70 \euro{} pour ne pas perdre d'argent.} \\
      \item Soit $n$ le nombre de billets à ajouter aux 300 billets. On a l'équation suivante : \\
      $(300+n)\times 2\text{ \euro{}} \geq2\,010\text{ \euro{}} \iff 600+2n\geq2\,010$ \\
      \hspace*{3.7cm} $\iff 2n \geq1410$ \\
      \hspace*{3.7cm} $\iff n\geq705$. \\
      \bm{À 2 \euro{}, l'organisateur doit ajouter au moins 705 billets pour ne pas perdre d'argent.}
   \end{enumerate}
\end{enumerate}
