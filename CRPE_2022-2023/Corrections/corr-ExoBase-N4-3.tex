\ \\ [-5mm]
   \begin{enumerate}
      \item
         \begin{enumerate}
            \item $N =a\times100+b\times10+c$. Si le nombre formé par le chiffre des dizaines et le chiffre des unités est divisible par 4, alors il existe un entier $m$ tel que $b\times10+c =4\times m$. \\
                On a alors $N =a\times100+b\times10+c =a\times4\times25+4\times m =4\times(a\times25+m)$ qui est bien un multiple de 4. \\
                {\blue Si le nombre formé par le chiffre des dizaines et le chiffre des unités est divisible par 4, $N$ est multiple de 4}.
            \item Prenons le nombre 116 : il se termine par \og 16 \fg{} qui est divisible par 8, mais $116 =8\times14+4$ n'est pas divisible par 8 donc, {\blue cette règle ne fonction pas pour 8}.
         \end{enumerate}
      \setcounter{enumi}{1}
      \item $N-(a+b+c) =a\times100+b\times10+c-a-b-c =a\times99+b\times9 ={\blue 9\times(a\times11+b)}$. \\
      Si $a+b+c$ est divisible par 9, alors il existe un entier $m$ tel que $a+b+c =9\times m$. \\
      On a alors $N =9\times(a\times11+b)+(a+b+c) =9\times(a\times11+b)+9\times m =9\times(11a+b+m)$. \\
         {\blue Si $a+b+c$ est divisible par 9, $N$ l'est aussi}.
   \end{enumerate}
