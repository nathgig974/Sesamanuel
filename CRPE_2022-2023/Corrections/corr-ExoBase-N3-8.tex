\ \\ [-5mm]
   \begin{enumerate}
      \item Ce problème se modélise par l'équation suivante : \\
         $(n-2)^2+(n-1)^2+n^2 =(n+1)^2+(n+2)^2$ \\
         $\iff \cancel{n^2}-4n+\cancel{4}+\cancel{n^2}-2n+\cancel{1}+n^2 =\cancel{n^2}+2n+\cancel{1}+\cancel{n^2}+4n+\cancel{4}$ \\
         $\iff n^2-12n =0$. \\
         {\blue Résoudre ce problème revient à résoudre l'équation $n^2-12n =0$.}
      \item $n^2-12n =0 \iff n(n-12) =0$. \\
         Les solutions de cette équation sont les solutions de $n =0$ et $n-12 =0$ soit : \\
         {\blue les solutions de l'équation $n^2-12n =0$ sont $0$ et $12$.}
      \item $n$ est la mesure en centimètre du côté du carré du milieu, donc $0$ ne peut pas être une solution valable puisqu'alors la mesure de chacun des côtés des carrés gris serait négative ou nulle. \\
         {\blue Seule la solution $12$ peut être retenue comme solution à  ce problème.}
      \item Avec $n =12$, les mesures successives en centimètres des côtés des carrés sont $10, 11, 12, 13$ et $14$. \par
         À l'échelle $\dfrac15$, on divise chaque mesure par $5$ et on obtient les mesures : 2 cm ; 2,2 cm ; 2,4 cm ; 2,6 cm et 2,8 cm ce qui donne la figure suivante : \\
         \begin{pspicture}(-2,0)(12,3)
            \psframe(0,0)(2,2)
            \psframe(2,0)(4.2,2.2)
            \psframe(4.2,0)(6.6,2.4)
            \psframe(6.6,0)(9.2,2.6)
            \psframe(9.2,0)(12,2.8)
         \end{pspicture}
   \end{enumerate}
