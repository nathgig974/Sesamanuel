\ \\ [-5mm]
\begin{enumerate}
   \item
      \begin{enumerate}
         \item La distance entre $A$ et $B$ est de 204,4 km dans la réalité, il est représenté par un segment [$AB$] de mesure 7,3 cm. Il s'agit d'une situation de proportionnalité que l'on peut résoudre en effectuant, par exemple, un tableau de proportionnalité et en utilisant le coefficient de proportionnalité ($\div2\,800\,000$) ou une règle de trois. \\ [1mm]
         \qquad
         \begin{CLtableau}{0.7\linewidth}{4}{c}
            \hline
            Distance dans la réalité en cm & 20\,440\,000 & 21\,000\,000 & 14\,560\,000 \\
            \hline
            Distance sur le triangle en cm & 7,3 & 7,5 & 5,2 \\
            \hline
         \end{CLtableau}
         \bm{Le segment [$AC$] mesure 7,5 cm et le segment [$BC$] mesure 5,2 cm.}
         \item
         \ \\ [-8mm]
         \begin{minipage}{7cm}
            Pour construire le triangle $ABC$, on commence par exemple par construire le segment [$AC$] de longueur 7,5 cm. \\
            Le point $B$ est situé à l'une des intersections entre le cercle de centre $A$ de rayon 7,3 cm et le cercle de centre $C$ de rayon 5,2 cm.
         \end{minipage}
         \hspace*{2cm}
         \begin{minipage}{7cm}
         {\psset{unit=0.8}
         \begin{pspicture}(0,-0.8)(8,4.5)
            \pspolygon(0,0)(7.5,0)(5.5,4.8)
            \psarc[linecolor=lightgray](0,0){7.3}{30}{50}
            \psarc[linecolor=lightgray](7.5,0){5.2}{100}{120}
            \psframe(5.5,0)(5.3,0.2)
            \psline(5.5,4.8)(5.5,0)
            \rput(-0.3,-0.3){A}
            \rput(7.8,-0.3){C}
            \rput(5.5,5.1){B}
            \rput(5.5,-0.3){D}
         \end{pspicture}}
         \end{minipage}
         \item 7,3 cm sur le dessin représentent 20\,440\,000 cm dans la réalité, donc 1 cm sur le dessin représente $20\,440\,000\text{ cm}\div7,3 =2\,800\,000$ cm dans la réalité.
         \bm{L'échelle utilisée est au 1/2\,800\,000\up{e}.} \\
      \end{enumerate}
   \item
   \begin{enumerate}
        \item La distance la plus courte est atteinte lorsque la droite ($BD$) est perpendiculaire à la droite ($AC$). Le point $D$ représente donc le pied de la hauteur issue de $B$ dans le triangle $ABC$ et \bm{$(BD)$ est la hauteur issue de $B$ dans le triangle $ABC$.} \\
        \item Avec des mesures en cm, on a : $BC^2 =AB^2+AC^2-2\times AC\times AD \iff$ \\
        $ 5,2^2 =7,3^2+7,5^2-2\times7,5\times AD \iff 27,04 =53,29+56,25-15\times AD \iff 15AD =82,5 \iff AD =5,5.$ \\
         \bm{Le segment [$AD$] mesure 5,5 cm.}
         \item Les points $A, D$ et $C$ sont alignés dans cet ordre, on a alors : \\
         $AD+DC =AC \iff DC =7,5\text{ cm}-5,5\text{ cm} =2\text{ cm}$. \\
         Dans le triangle $ABD$ rectangle en $D$, on utilise le théorème de Pythagore avec des mesures (positives) en cm :
         $AB^2 =AD^2+DB^2 \iff BD^2 =7,3^2-5,5^2 =23,04 \iff BD =\sqrt{23,04} =4,8$. \\
         \bm{La longueur $CD$ vaut 2 cm et la longueur $BD$ vaut 4,8 cm.}
         \end{enumerate}
    \end{enumerate}
