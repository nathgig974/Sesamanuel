\ \\ [-5mm]
\begin{enumerate}
   \item \textcolor{A2}{$\bullet$} Le patron doit avoir autant de faces que l’objet réel ;
   \begin{itemize}
      \item les formes des faces du patron doivent être les mêmes que celles de l’objet réel ;
      \item le patron doit être d'un seul tenant ;
      \item les faces ne doivent pas se recouvrir. \\
   \end{itemize}
   \item
   \begin{enumerate}
      \item \textcolor{A2}{$\bullet$} des gabarits de faces sont donnés, mais en nombre supplémentaire. Cela oblige l'élève à analyser le solide afin de ne choisir que les faces convenables ;
      \begin{itemize}
         \item les gabarits des faces du cube tronqué sont donnés. Cela libère les élèves de la complexité du tracé de certaines faces. Une fois ces gabarits trouvés, l'élève pourra mieux se concentrer sur l’analyse du solide, le nombre et la nature de ses faces et la disposition à adopter pour obtenir un patron ;
         \item chaque gabarit est en un seul exemplaire. Si le maître fournissait les sept gabarits correspondant aux sept faces, la tâche des élèves serait réduite à les disposer correctement. \\
      \end{itemize}
      \item \textcolor{A2}{$\bullet$} si la difficulté porte sur le choix des faces, le maître peut proposer aux élèves de manipuler le cube tronqué, de le comparer aux gabarits qu'il possède ;
      \begin{itemize}
         \item si la difficulté porte sur l'assemblage des gabarits, il peut proposer de donner plusieurs gabarits de même forme afin qu'il puisse avoir autant de faces que de gabarits ;
         \item il peut aussi proposer d'encrer les faces du cube tronqué afin de créer un patron sur une feuille et de s'inspirer de ce patron pour tracer le sien. \\
      \end{itemize}
   \end{enumerate}
   \item La difficulté particulière de cette activité vient du fait que le solide à construire n’est pas un objet matériel, on doit imaginer qu’il bouche le trou. Cela entraîne entre autre l’impossibilité de le tourner dans tous les sens pour l’observer (et par exemple de placer la pyramide dans sa position la plus usuelle, sommet en haut). \\
   \item On pourrait proposer la trace écrite suivante : \og Le patron d’un solide est une surface plane d’un seul tenant qui, par pliage, permet de reconstituer le solide sans recouvrement de ses faces. \fg{}
\end{enumerate}
