\ \\ [-5mm]
   \begin{enumerate}
      \item
      {\blue La proposition A est vraie} \\
         Soit $n$ un nombre se terminant par 2, il s'écrit $n=10k+2$ où $k$ est le nombre de dizaines de $n$. \\
         On a alors : $n^2 =(10k+2)^2$ \\
         \hspace*{2.1cm} $=100k^2+40k+4$ \\
         \hspace*{2.1cm} $=10(10k^2+4k)+4$. \\
         On reconnait l'écriture d'un nombre décimal où le chiffre des unités est 4. \\
      {\blue La proposition B est fausse.} \\
       Il suffit de donner un contre-exemple : $14^2=196$ ne se termine pas par 16. \\
      \item $n =\overline{a5} =10a+5$. Donc, $n^2=(10a+5)^2$ \\
      \hspace*{4.35cm} $=100a^2+100a+25$ \\
      \hspace*{4.35cm} $ =100(a^2+a)+25$. \\
      On en déduit les choses suivantes :
      \begin{itemize}
         \item le maximum de $n^2$ est $100(9^2+9)+25 =9025$ qui s'écrit avec {\blue 4 chiffres} ; \\
         \item $n^2$ est composé de $a^2+a =${\blue $a(a+1)$} centaines additionné de 25, donc {\blue il se termine par 25}. \\
      \end{itemize}
   \end{enumerate}
