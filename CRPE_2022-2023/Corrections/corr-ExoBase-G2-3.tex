   {\bf Remarque initiale :} soit $N =2n+1$ un nombre impair, alors $N^2 =4n^2+4n+1$ est encore un nombre impair. \\
   Soit $N'=2n'+1$ un autre nombre impair, alors $N'^2 =4n'^2+4n'+1$ est un nombre impair. \\
   La somme $N+N'=4n^2+4n'^2+4n+4n'+2 =2(2n^2+2n'^2+2n+2n'+1)$ est un nombre pair. \\ [2mm]
   {\bf Démonstration :} on effectue une démonstration par l'absurde en supposant que la conclusion est fausse. Le contraire de \og l’un au moins de ces trois nombres est pair \fg{} est \og les trois nombres sont impairs \fg. \\
   Supposons alors que $a$, $b$ et $c$ soient tous les trois impairs, alors $a^2$, $b^2$ et $c^2$ le sont également. \\
   Or, le triangle dont les longueurs sont $a$, $b$ et $c$ est rectangle, donc $a^2 + b^2 = c^2$, ce qui signifie que $c^2$ est un nombre pair d'après la remarque initiale. L'hypothèse de départ est donc fausse : $a$, $b$ et $c$ ne peuvent pas être tous les trois impairs. D'où : {\blue l'un des trois nombres $a, b, c$ au moins est pair.}
