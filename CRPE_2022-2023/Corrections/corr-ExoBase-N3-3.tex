\ \\ [-5mm]
   \begin{enumerate}
      \item Si on multiplie le chiffre des unités de chaque terme, on trouve $6\times8 =48$, donc, le chiffre des unités de A est 8. Or, le résultat trouvé sur la calculatrice donne 3 500 000 820 000 000 et se termine par 0, d'où : \\
         {\blue la valeur affichée sur la calculatrice n'est pas la valeur exacte}.
      \item On a : $50\,000\,000<50\,000\,006<60\,000\,000$ et $70\,000\,000<70\,000\,008<80\,000\,000$ \\
         Donc, ces nombres étant positifs, on peut écrire :$(5\times10^7)\times(7\times10^7)<\text{A}<(6\times10^7)\times(8\times10^7)$ : \\
         {\blue $35\times10^{14} < \text{A} < 48\times10^{14}$}, ce qui veut dire que {\blue A possède 16 chiffres}.
      \item A $=(5 \times 10^7+6)\times(7\times10^7+8)$ \\
         \hspace*{0.7cm} $=(5\times10^7)\times(7\times10^7)+(5\times10^7)\times8+6\times(7\times10^7)+6\times8$ \\
         \hspace*{0.7cm} $=5\times7\times10^{7+7}+5\times8\times10^7+6\times7\times10^7+6\times8$ \\
         \hspace*{0.7cm} $=35\times10^{14}+40\times10^7+42\times10^{7}+48$ \\
         \hspace*{0.7cm} $=3\,500\,000\,000\,000\,000+820\,000\,000+48$ \\
         {\blue A $=3\ 500\,000\,820\,000\,048$.}
      \item $48\,506\times505 =24\,495\,530$ \qquad ; \qquad $557\times505 =281\,285$ \qquad ; \qquad $48 506\times149 =7\,227\,394$ \qquad ; \qquad $557\times149 =82\,993$. \\
       Donc, $48\,506\,557\times505\,149 =(48\,506\times10^3+557)\times(505\times10^3+149)$ \\
         \hspace*{3.85cm} $=48\,506\times505\times10^3\times10^3+48\,506\times149\times10^3+557\times505\times10^3+557\times149$ \\
         \hspace*{3.85cm} $=24\,495\,530\times10^6+7\,227\,394\times10^3+281\,285\times10^3+82\,993$ \\
         \hspace*{3.85cm} $=24\,495\,530\,000\,000+7\,227\,394\,000+281\,285\,000+82\,993$ \\
         { \blue B $=24\,503\,038\,761\,993$}.
   \end{enumerate}
