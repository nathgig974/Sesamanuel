   Dans cet exercice, la personne est choisie au hasard, nous sommes donc dans une situation d'équiprobabilité.
   \begin{enumerate}
      \item Un donneur universel est un donneur de groupe O et de rhésus négatif. Or, cela concerne 6\,\% de la population française donc, {\blue la probabilité d'être donneur universel est de 0,06.}
      \item Un receveur universel est un receveur de groupe AB et de rhésus positif. Or, cela concerne 3\,\% de la population française donc, {\blue la probabilité d'être receveur universel est de 0,03.}
      \item Pour être donneur à une personne de groupe B et de rhésus positif, il faut être O$+$, O$-$, B$+$ ou B$-$, ce qui représente 36\,\% + 6\,\% + 9\,\% + 1\,\% soit 52\,\% de la population française donc, {\blue la probabilité de pouvoir donner a une personne de groupe B rhésus positif est de 0,52.}
      \item Parmi les personnes de groupe O, seules les personnes de rhésus négatif sont donneurs universels, ce qui représente du probabilité de $\dfrac{6\,\%}{36\,\%+6\,\%} =\dfrac17 \approx 0,143$. \\ [2mm]
         {\blue Pour une personne du groupe O, la probabilité d'être donneur universel est d'environ 0,14.}
      \item 43\,217\,325 personnes peuvent donner leur sang, et parmi elles, 6\,\% sont donneurs universels, soit $0,06\times43\,217\,325\text{ personnes} =2\,593\,039,5\text{ personnes}$. \\
         {\blue Au 1\up{er} janvier 2016, il y avait environ 2\,593\,040 donneurs universels.} \\ [2mm]
      \item $\dfrac{43\,217\,325}{66\,627\,602}\times100 \approx64,86.$ \\ [2mm]
         {\blue Au 1\up{er} janvier 2016, environ 65\,\% de la population pouvait donner son sang.}
   \end{enumerate}
