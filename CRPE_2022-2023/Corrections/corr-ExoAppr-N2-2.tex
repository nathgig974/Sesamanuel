\ \\ [-5mm]
\begin{enumerate}
   \item La principale compétence mathématique évaluée dans cet exercice est : \og Résoudre un problème relevant de la soustraction. \fg{}. Selon la typologie de {\it G. Vergnaud}, il s'agit d'un problème du champ additif de type \og transformation d'un état \fg, dans lequel l'état initial et l'état final sont connus ; on recherche la transformation. \\
   La compétence secondaire est : \og Effectuer une soustraction \fg{}. À la fin du CE1, les élèves doivent connaître et utiliser les techniques opératoires de l'addition et de la soustraction (sur des nombres inférieurs à 1\,000).
   \item Si $x$ désigne le nombre de coureurs ayant abandonné, on peut écrire :
   \begin{itemize}
      \item $108-x = 85$ ;
      \item $85+x =108$ ;
      \item $108-85 = x$.
   \end{itemize}
   \item
   \begin{enumerate}
      \item On peut distinguer :
      \begin{itemize}
         \item les procédures {\bf non apparentes} de Melvin et de Nabila (le premier ne donne pas de réponse, la deuxième donne la réponse exacte sans aucune justification) ;
         \item les procédures traduisant une {\bf mauvaise compréhension de la situation} : Camille qui fait une modélisation erronée de la situation ;
         \item les procédures utilisant la {\bf schématisation} : celles de Driss et de Siham, dont la représentation est plus aboutie ;
         \item les procédure utilisant le {\bf décomptage} : celle d'Hilda ;
         \item les procédures de {\bf recherche du complément} : par sauts successifs pour Amandine et Gabrielle, par une addition à trous pour Cédric (qui fait une erreur de calcul) ;
         \item les procédures {\bf soustractives} d'Houssan et Benyamine qui se trompent dans leur calcul.
      \end{itemize}
   \end{enumerate}
\end{enumerate}
\Coupe
\begin{enumerate}
\setcounter{enumi}{2}
   \item
   \begin{enumerate}
   \setcounter{enumii}{1}
   \item {\bf Erreur d'Houssan :} on peut envisager deux hypothèses principales :
  \begin{itemize}
   \item dans chaque colonne, il calcule l'écart entre le plus grand et le plus petit chiffre : $8 - 5 = 3$ , $8 - 0 =8$ et $1 - 0 =1$ ;
   \item au lieu de soustraire, il additionne 108 et 85 et oublie la retenue.
   \end{itemize}
   {\bf Erreur de Benyamine :} pas de problème apparent pour les unités. Pour la suite du calcul, il est difficile de recréer la chronologie. Ne pouvant ôter 8 de 0, il ôte 8 de 10 et place le 1 en bas (conservation des écarts). \\
   La présence du 7 indique très vraisemblablement qu'il a retranché 1 de 8 pour obtenir le 7. Mais ce 1 n'est pas le 1 entouré du bas, puisqu'il trouve 33 et non 133. Il semble donc que le 1 ajouté en haut soit traité doublement : pour la conservation des écarts d'abord, mais aussi par retrait au terme du bas.
   \end{enumerate}
   \item Les codes attendus.
   \begin{itemize}
      \item On mettra le code 1 à Amandine, Siham et Nabila qui ont donné la réponse attendue : 23.
      \item Cédric devrait recevoir le code 8 ou 9 selon si l'on considère que faire une addition à trou rentre dans la \og mise en \oe uvre de l'addition \fg.
      \item Camille devrait recevoir le code 8 si on se réfère à la procédure utilisée, mais au regard de sa réponse (193 coureurs), on pourrait penser au code 9.
      \item Les consignes de codage sont ambiguës : elles ne permettent pas de distinguer l'utilisation erronée de l'addition et l'utilisation pertinente de l'addition à trous. Le codage proposé ne permet pas de distinguer les erreurs relatives à la procédure de résolution du problème de celles relatives à la procédure de calcul. \\
   \end{itemize}
\end{enumerate}
