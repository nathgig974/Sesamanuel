\ \\ [-5mm]
   \begin{enumerate}
      \item
         \begin{enumerate}
            \item $\dfrac23\times\us{3600} =\us{2400}$ donc, {\blue deux tiers d'heure font 2 400 secondes.} \smallskip
            \item $1,2\times\us{3600} =\us{4320}$ donc, {\blue 1,2 heure fait 4 320 secondes.}
         \end{enumerate}
      \setcounter{enumi}{1}
      \item
         \begin{enumerate}
            \item Il s'agit, en fait, de convertir 5 532 en base 60 : \\
               $5\,532 =1\times60^2+32\times60^1+12\times60^0 =1\times3\,600+32\times60+12$
               donc, {\blue 5 532 s = 1 h 32 min 12 s.}
            \item 1,87 heure = 1 heure et 0,87 heure. \\
               Or, $0,87\times\umin{60} =\umin{52,2}$. D'où 1,87 heure = 1 heure 52 minutes et 0,2 minutes. \\
               De plus, $0,2\times\us{60} =\us{12}$. Conclusion : {\blue 1,87 h = 1 h 52 min 12 s}.
         \end{enumerate}
       \setcounter{enumi}{2}
       \item La grande aiguille parcourt  \udeg{360} en une heure, soit 3\,600 secondes, soit  \udeg{1} en 10 secondes, ou encore  \udeg{54} en 540 secondes, ce qui correspondent à 9 minutes. \\
         {\blue La grande aiguille d'une montre met 9 minutes pour parcourir un angle de  \udeg{54}}.
      \item La petite aiguille parcourt  \udeg{360} en une 12 heures, soit 720 minutes, soit \udeg{1} en 2 minutes, donc  \udeg{68} en 136 minutes, ce qui correspondent à 2 heures et 16 minutes. \\
         {\blue L'heure indiquée est 14 h 16 min}.
      \item
         \begin{enumerate}
            \item Il y a 7 heures de décalage horaire entre Houston et Paris. Donc, quand il est 23 h 00 à Paris, il est 16 h 00 à Houston. L'arrivée se faisant à 03 h 00, la durée du vol est de 8 h + 3 h = 11 h. \\
               {\blue La durée du vol entre Paris et Houston est de 11 heures}.
            \item Arnaud reste à Houston pendant une heure, donc jusqu'à 04 h 00, puis repart pour Rio par un vol qui dure 10 h 00, il arrivera donc à 14 h 00 heure de Houston. Or, Rio est à $+3$ heures de Houston, conclusion : \\
               {\blue Arnaud arrivera à 17 h 00 le lendemain de son départ à Rio de Janeiro}.
         \end{enumerate}
   \end{enumerate}
