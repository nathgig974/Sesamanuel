\ \\ [-5mm]
   \begin{enumerate}
      \item
      \begin{enumerate}
         \item Le système Cincofiles est un système de numération positionnel de base 5. \\
         Dans notre système positionnel de base 10, \, {\huge $\square \, \square \, \square$} est représenté par le nombre $4\times5^2+4\times5^1+4\times5^0 =4\times25+4\times5+4 ={\blue 124}.$
      \end{enumerate}
   \end{enumerate}

\Coupe

   \textcolor{G1}{\bf b)} On effectue les divisions euclidiennes successives par 5 : \\
      $\opidiv[remainderstyle.2=\textcolor{red}]{273}{5}$ \quad $\opidiv[remainderstyle.2=\textcolor{red}]{54}{5}$ \quad $\opidiv[remainderstyle=\textcolor{red}]{10}{5}$ \quad $\opidiv[remainderstyle=\textcolor{red}]{2}{5}$. \\
      Donc, 273 s'écrit $\overline{2043}^5$ en base 5, ce qui est codé par {\blue {\Large$\wedge$}\pscircle[fillstyle=solid,fillcolor=blue](0.28,0.15){1.3mm}\qquad{\huge $\square$}\,\rotatedown{$\Delta$}} \\
   \begin{enumerate}
   \setcounter{enumi}{1}
      \item
      \begin{enumerate}
          \item On doit enlever une unité au nombre \rotatedown{$\Delta$} {\huge $\square$}\pscircle[fillstyle=solid,fillcolor=black](0.28,0.15){1.3mm} \\
            Or, le chiffre du premier rang (rang des unités) est nul, on va donc \og prendre \fg{} une unité au rang 2 (que l'on pourrait appeler rang des quinaires) que l'on va ajouter au rang 1, auquel on peut cette fois-ci supprimer une unité, il restera donc {\huge $\square$} alors qu'au deuxième rang  {\huge $\square$} sera devenu $\rotatedown{\Delta}$. \\
Le nombre recherché est donc {\blue \rotatedown{$\Delta$}\,\rotatedown{$\Delta$}\,{\huge $\square$}}.
            \item On ajoute une unité au premier rang {\huge $\square$}, qui devient alors \pscircle[fillstyle=solid,fillcolor=black](0.28,0.15){1.3mm} \quad\; et qui nous donne une unité de rang 2 en plus (la retenue). Le chiffre de ce rang devient \, \pscircle[fillstyle=solid,fillcolor=black](0.28,0.15){1.3mm} \qquad avec une unité supplémentaire au rang 3 : {\Large$\wedge$} devient $\rotatedown{\Delta}$. \\
            Le nombre suivant \og {\Large$\wedge$} {\huge$\square \, \square$} \fg{} est donc {\blue $\rotatedown{\Delta}\pscircle[fillstyle=solid,fillcolor=blue](0.28,0.15){1.3mm}\quad\,\pscircle[fillstyle=solid,fillcolor=blue](0.28,0.15){1.3mm}\quad\,$}
      \end{enumerate}
   \end{enumerate}
