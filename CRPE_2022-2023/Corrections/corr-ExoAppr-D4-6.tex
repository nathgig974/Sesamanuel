\ \\ [-5mm]
   \begin{enumerate}
      \item Graphiquement, le tarif 2 devient plus avantageux lorsque sa courbe représentative (qui est une droite affine) se situe en dessous de celle du tarif 1. Cependant, le graphique présenté n'est pas suffisamment précis et est assez litigieux car il est difficile de voir ce qu'il se passe pour 12 voyages. Seul un calcul rapide lève cette ambiguïté et donne 148,8 \euro{} pour le tarif 1 et 149,04 \euro{} pour le tarif 2. \\
         \bm{À partir du 13\up{e} voyage, le tarif 2 devient plus avantageux.}
         \item Pour le tarif 1, un aller simple coûte 12,40 \euro, donc $x$ allers simples coûtent, en \euro : \bm{$f(x) =12,4x$.}
         \item Pour le tarif 2, la réduction appliquée est de 20\% par voyage, ce qui correspond à un coefficient multiplicateur de $1-\dfrac{20}{100} =0,8$. \\
         Donc, un aller simple coûte $0,8\times12,40\text{ \euro} =9,92$ \euro. $x$ allers simples coûtent alors $9,92x$. \\
         À ce prix, il faut ajouter la carte d'abonnement de 30 \euro{} donc : \bm{$g(x) =9,92x+30$.}
      \item On cherche la valeur de $x$ pour laquelle $g(x)<f(x)$ :
         \begin{align*}
         9,92x+30<12,4x & \iff 30<12,4x-9,92x \\
         & \iff 30<2,48x \\
         & \iff 30\div2,48<x \\
         & \iff x>12,097
         \end{align*}
         $x$ étant un nombre entier, on choisit $x =13$. \bm{Le tarif 2 devient plus avantageux à partir du 13\up{e} aller simple.}
      \end{enumerate}
