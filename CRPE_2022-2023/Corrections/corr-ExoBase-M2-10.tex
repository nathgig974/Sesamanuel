\begin{itemize}
   \item Volume d'un tube creux : il correspond au volume d'un tube plein de rayon 6 cm, auquel on enlève le volume d'un tube plein de rayon 4 cm. \\
   -- Volume du cylindre de 6 cm de rayon en cm$^3$ : $\mathcal{V}_1= 3,14\times6^2\times75 =8\,478$. \\
   -- Volume du cylindre de 4 cm de rayon  en cm$^3$ : $\mathcal{V}_2= 3,14\times4^2\times75 =3\,768$. \\
   Donc, le volume du tube creux est de $\ucmc{8478}-\ucmc{3768} = \ucmc{4710}$.
   \item Masse d'un tube creux : la masse volumique du tube est de 2,7g\,\slash cm$^3$. \\
   Or, $2,7\times4\,710 =12\,717$, donc, la masse du tube creux est de 12 717 g = 12,717 kg.
   \item Nombre de tubes à transporter : la charge utile du camion est de 14 tonnes, soit 14 000 kg. \\ [1mm]
   Or, $\dfrac{14\,000}{12,717} \approx1\,100,89$ donc, {\blue le camion peut transporter au maximum 1 100 tubes creux en aluminium.}
\end{itemize}
