\ \\ [-5mm]
\begin{enumerate}
   \item La principale notion travaillée ici est la \textbf{proportionnalité} dans le cadre de la résolution d'un problème.
   \item Procédures des élèves A, B et C.
   \begin{itemize}
      \item L'\textbf{élève A} utilise une procédure mixte de linéarité : il utilise la propriété additive de la linéarité pour dire que 9 personnes, c'est 6 personnes plus 3 personnes, et la propriété multiplicative de la linéarité pour dire que 3 personnes, c'est la moitié de 6 personnes. Pour chacune de ces quantités, il associe la quantité d'ingrédients correspondant. Sa procédure est juste, son résultat également. On pourrait éventuellement lui demander de préciser les unités dans ses calculs.
      \item L'\textbf{élève B} utilise le passage par l'unité : il calcule la masse de farine pour une personne puis, il multiplie par 9 qui est le nombre de personnes. La procédure est correcte, mais il trouve une valeur approchée, donc son résultat est erroné (le problème ici est que la valeur obtenue par la division de 250 par 6 est une valeur approchée puisque $\frac{250}{6}$ est un nombre rationnel non décimal). Pour les pincées de sel, il commence par poser sa division mais ne va pas au bout : son résultat est juste, mais l'opération est fausse, peut-être parce qu'il l'a en fait trouvé par une autre méthode. Il n'a pas le temps de calculer toutes les quantités.
      \item L'\textbf{élève C} semble vouloir utiliser le passage par l'unité en posant la division euclidienne de 250 par 6 sans résultats intermédiaires. Pourtant, il pose la virgule et met des pointillés pour exprimer le fait que \og ça continue \fg{} et il obtient 41 qui correspond à la masse de farine en gramme pour une personne. Ensuite, on veut la masse pour 9 personnes alors qu'on a à l'origine la masse pour 6 personnes, soit 3 personnes de plus, donc il additionne la masse de farine à trois fois le quotient entier (41) trouvé grâce à son opération ce qui est un raisonnement juste mais approximatif. Il trouve 373 qui est une valeur assez proche du résultat, mais fausse.
   \end{itemize}
   \item 300 est un multiple de 6, ce qui est avantageux au niveau du calcul : pour une procédure par passage à l'unité, le résultat de la division est un entier, donc l'élève B par exemple arriverait à un résultat juste. Mais ces nombres permettent aussi une autre procédure, celle de la recherche du coefficient de proportionnalité permettant de passer de 6 personnes à une masse de 300 g qui est un coefficient relativement simple (50). Ce coefficient pouvait être trouvé par un calcul mental, par un procédure type essais-erreurs ou par une division.
\end{enumerate}
