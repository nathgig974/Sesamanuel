\ \\ [-5mm]
   \begin{itemize}
      \item Énoncé 1 : {\blue faux.} \\
      Contre-exemple : $2x=1$ est un entier naturel, mais $x=0,5$ n'en est pas un (la moitié d'un entier naturel n'est pas forcément un entier naturel. Pour cela, il faudrait qu'il soit pair).
       \item Énoncé 2 : {\blue vrai.} \\ [1mm]
       Démonstration : $\dfrac{x}{2}=n\in\N$, donc, $x=2n\in\N$ (le double d'un entier naturel reste un entier naturel).
      \item Énoncé 3 : {\blue faux.} \\
      Contre-exemple : $x+1=0$ est un entier naturel, mais $x=-1$ est un entier relatif, non naturel (c'est donc vrai pour tout nombre entier naturel non nul).
   \end{itemize}
