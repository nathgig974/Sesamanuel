\ \\ [-5mm]
   \begin{enumerate}
      \item On effectue les divisions successives par 2 : \\ [1mm]
      $\opidiv[remainderstyle.2=\textcolor{blue}]{60}{2}$ \quad $\opidiv[remainderstyle.2=\textcolor{blue}]{30}{2}$ \quad $\opidiv[remainderstyle=\textcolor{blue}]{15}{2}$ \quad $\opidiv[remainderstyle=\textcolor{blue}]{7}{2}$ \quad $\opidiv[remainderstyle=\textcolor{blue}]{3}{2}$ \quad $\opidiv[remainderstyle=\textcolor{blue}]{1}{2}$. \\ [1mm]
         Donc, 60 s'écrit {\blue $\overline{111100}^2$} en base 2. \\
      \item $b =1\times2^6+0\times2^5+1\times2^4+0\times2^3+1\times2^2+0\times2^1+1\times2^0$, ce qui donne {\blue $b = 85$}.
      \item L'écriture de $b$ en base 2 se termine par 1 donc, {\blue $b$ est impair}.
      \item Lorsque l'on multiplie ce nombre par 2, il suffit de lui ajouter un 0 à la fin de son écriture : \\
         {\blue $2b =\overline{10101010}^2$} \\
         {\blue $4b =\overline{101010100}^2$} \\
         {\blue $8b =\overline{1010101000}^2$}.
       \item On effectue l'addition posée de manière classique, tout en sachant que l'on est en base 2. \\ \hspace*{1cm} +\begin{tabular}{cccccccc}
            \tiny +1 & \tiny +1 & \tiny +1 & \tiny +1 & \tiny +1 & & & \\
            & & 1 & 1 & 1 & 1 & 0 & 0 \\
            & 1 & 0 & 1 & 0 & 1 & 0 & 1 \\
            \hline
            1 & 0 & 0 & 1 & 0 & 0 & 0 & 1 \\
         \end{tabular}
   \qquad Ce qui donne {\blue $a+b =\overline{10010001}^2$}.
   \end{enumerate}
