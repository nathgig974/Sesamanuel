\ \\ [-5mm]
   \begin{enumerate}
      \item
         \begin{enumerate}
            \item Pour le rugby, le ratio gains - pertes est de 9 : 6, ou encore 3 : 2. \\
               Pour le judo, le ratio gains - pertes est de 12 : 8, soit 3 : 2. \\
               Pour le handball, le ratio gains - pertes de 10 : 5, ou 2 : 1. \\
               Conclusion : {\blue le rugby et le judo ont le même ratio gains - pertes}.
            \item Pour le handball, le ratio gains - matchs joués est de 10 : 15 équivalent à {\blue 2 : 3}. \\ [1mm]
               La fraction de matchs gagnés est de $\dfrac{10}{15} = \blue \dfrac23$. \\ [1.5mm]
               Le pourcentage de matchs gagnés est de $\dfrac{10}{15}\times100 \approx {\blue 67\,\%}$. \smallskip
         \end{enumerate}
      \setcounter{enumi}{1}
      \item Le ratio 3 : 5 signifie que lorsqu'un joueur gagne \ueuro{3}, l'autre gagne \ueuro{5} pour une somme de \ueuro{8}. \\
         S'ils gagnent \ueuro{64}, c'est à dire 8 fois plus, {\blue l'un gagnera \ueuro{24} et l'autre \ueuro{40}}. \\
      \item Le jus d'orange, d'ananas et de pomme sont dans le ratio 2 : 3 : 4. Donc, pour \ul{2} de jus d'orange, il faut \ul{3} de jus d'ananas et \ul{4} de jus de pomme ce qui donne \ul{9} de cocktail. \\
         Pour \ul{45}, soit 5 fois plus, il faut {\blue \ul{10} de jus d'orange, \ul{15} de jus d'ananas et \ul{20} de jus de pomme}.
   \end{enumerate}
