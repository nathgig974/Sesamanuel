   On peut matérialiser la situation par un arbre pondéré : \\ [7mm]
   \hspace*{3cm}
   {\blue{sec}}\pstree[treemode=R,nodesep=3pt,levelsep=4cm,treesep=1.8cm]{\Tp}{%
      \pstree{\TR{sec}\naput{$\frac56$}}{%
         \TR{sec}\naput{$\frac56$} \TR{\blue humide}\nbput{$\frac16$}}
      \pstree{\TR{humide}\nbput{$\frac16$}}{%
         \TR{sec}\naput{$\frac13$} \TR{\blue humide}\nbput{$\frac23$}}} \\ [7mm]
      Dans un arbre pondéré, la probabilité d'une issue est calculée en multipliant les probabilités de chaque éventualité. On obtient deux branches, donc \\ [2mm]
      $\mathcal{P} =\dfrac56\times\dfrac16+\dfrac16\times\dfrac23 =\dfrac{5}{36}+\dfrac{2}{18} =\dfrac{5}{36}+\dfrac{4}{36} =\dfrac{9}{36} =\dfrac14$. \\ [2mm]
      On trouve bien une probabilité de $\dfrac14 =0,25$ donc, {\blue l'affirmation est vraie}.
