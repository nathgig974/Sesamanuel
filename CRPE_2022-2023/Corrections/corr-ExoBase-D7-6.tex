\ \\ [-5mm]
   \begin{enumerate}
      \item Les valeurs ordonnées sont : 6 - 6 - 6 - 6 - 8 - 10 - 10 - 11 - 11 - 12 - 12 - 14 - 14 - 16 - 16 - 17 - 20 - 25 - 25 - 30.
         \begin{itemize}
            \item $30-6 =24$ donc, {\blue l'étendue est de 24}.
            \item La médiane est une valeur entre la 10\up{e} (qui est 12) et la 11\up{e} (qui est 12 aussi), {\blue la médiane est donc égale à 12}.
            \item $(4\times6+8+2\times10+2\times11+2\times12+2\times14+2\times16+17+20+2\times25+30)\div20 =275\div20 =13,75$. \\
               {\blue La moyenne est de 13,75}.
         \end{itemize}
      \item Les boules sont indiscernables au toucher, il y a donc équiprobabilité. \\ [1mm]
         \begin{enumerate}
            \item {\blue La probabilité de \og tirer un nombre pair \fg{} est de} $\dfrac{15}{20} ={\blue \dfrac34}$.
            \item {\blue La probabilité de \og tirer une voyelle ou un nombre pair \fg{} est de} $\dfrac{16}{20} ={\blue \dfrac45}$. \smallskip
            \item {\blue La probabilité de \og tirer une voyelle et un nombre pair \fg{} est de} $\dfrac{8}{20} ={\blue \dfrac25}$. \smallskip
         \end{enumerate}
      \setcounter{enumi}{2}
      \item La probabilité d’obtenir grâce au cinquième tirage le mot \og M A T H S \fg{} correspond au cas ou le cinquième tirage est un \og S \fg{} alors qu'on a déjà tiré un \og M \fg, un \og A \fg, un \og T \fg{} et un \og H \fg. On a donc 2 possibilités pour le \og S \fg{} sur un total de 16 boules restantes, soit une probabilité de $\dfrac{2}{16} ={\blue \dfrac18}$.
   \end{enumerate}

