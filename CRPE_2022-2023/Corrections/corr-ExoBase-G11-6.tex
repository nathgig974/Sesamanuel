\ \\ [-5mm]
   \begin{enumerate}
      \item Volume du cylindre en mètre cube : $V_{\text{cylindre}} = \pi\times1,3^2\times2,4$ ; \\ [1mm]
         Volume du cône en mètre cube : $V_{\text{cône}} = \dfrac13\times\pi\times1,3^2\times1,6$ ; \\
         Volume du silo en mètre cube : $V =V_{\text{cylindre}}+V_{\text{cône}} =4,056\pi+\dfrac13\times2,704\pi \approx 15,5739$. \\
         {\blue Le silo a un volume d'environ \umc{15,57}}.
      \item Quantité de farine en litre pour 48 vaches pendant 90 jours : $C_{\text{vaches}} =90\times48\times\ul{3} =\ul{12\,960}$. \\
         Capacité en litre du silo : $C_{\text{silo}} \approx\dfrac67\times\umc{15,5739} \approx\dfrac67\times\udmc{15573,9} \approx\dfrac67\times\ul{15573,9} \approx \ul{13349}$. \\
         On a $C_{\text{silo}}>C_{\text{vaches}}$ donc, {\blue l'éleveur aura suffisamment de farine pour nourrir ses 48 vaches pendant 90 jours}.
      \item Remarque : A et D  sont les centres des bases des solides de révolution, donc (SA) est orthogonale à la base du cône et (AD) est orthogonale à la base du cylindre. On a alors (SA) perpendiculaire à (AB) et (AD) perpendiculaire à (DC) et à (AB). \\
         D'où ABHS et ABCD sont des rectangles et leurs côtés sont deux à deux égaux et parallèles. \\
         Les points H, M, N et H, B, C sont alignés dans cet ordre. \\ [1mm]
         On calcule les rapports $\dfrac{\text{HN}}{\text{HM}}$ et $\dfrac{\text{HC}}{\text{HB}}$ avec les valeurs suivantes : \\ [1mm]
         HN = SN $-$ SH = SN $-$ AB = \um{3,3} $-$ \um{1,3} = \um{2}. \\
         HM = SM $-$ SH = SM $-$ AB = \um{2,1} $-$ \um{1,3} = \um{0,8}. \\
         HC = HB + BC = SA + AD = \um{1,6} + \um{2,4} = \um{4}. \\
         HB = SA = \um{1,6}. \\ [1mm]
         D'où $\dfrac{\text{HN}}{\text{HM}} =\dfrac{\um{2}}{\um{0,8}} =2,5$ et $\dfrac{\text{HC}}{\text{HB}} =\dfrac{\um{4}}{\um{1,6}} =2,5$. \\ [1mm]
         Les rapports sont égaux, donc d'après la réciproque du théorème de Thalès, les droites (NC) et (MB) sont parallèles. {\blue Les deux échelles sont parallèles}.
   \end{enumerate}
