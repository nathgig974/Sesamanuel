\ \\ [-5mm]
\begin{enumerate}
   \item Analyse des productions d'élèves :
   \begin{itemize}
      \item {\bf Robin.} Il sait que multiplier par 5 revient à multiplier par 10, puis à diviser par 2. \\
      Il commence par calculer $10\times68$ car il sait effectuer une multiplication par 10 mentalement, il obtient 680. Puis il divise par 2 et trouve 340. \\
      Son résultat et sa procédure sont justes, il y a cependant une erreur d'écriture puisqu'il écrit en ligne ses calculs à la suite, le statut du signe \og = \fg{} n'est alors pas respecté puisqu'on peut lire \og $10\times68 =340$ \fg. Il aurait dû écrie $10\times68 =680$, puis $680\div2 =340$.
      \item {\bf Eléonore.} Elle passe par la décomposition de 68 comme $70-2$ puis elle utilise la distributivité de la multiplication par rapport à la soustraction. \\
      Elle effectue $70\times5 =350$ certainement en calculant mentalement que $7\times5 =35$ puis en \og ajoutant un zéro \fg{} afin d'effectuer la multiplication par 10. Ensuite, elle effectue le calcul $2\times 5 =10$ et enfin, elle soustrait ses deux résultats pour obtenir 340. \\
      Sa procédure et son résultat sont justes.
      \item {\bf Lucie.} Elle utilise la décomposition additive de 68 qui est $60+8$ puis elle utilise la distributivité de la multiplication sur l'addition. \\
      Elle effectue $5\times60 =300$ certainement en calculant mentalement que $5\times6 =30$ puis en \og ajoutant un zéro \fg{} afin d'effectuer la multiplication par 10. Ensuite, elle effectue le calcul $5\times8 =40$ et enfin, elle additionne ses deux résultats pour obtenir 340. \\
      Sa procédure et son résultat sont justes.
      \item {\bf Mathys.} Il utilise l'écriture de 68 en multipliant successivement 5 par 6 ce qui donne 30 et 5 par 8 ce qui donne 40 puis il additionne ces deux résultats. \\
      Sa procédure aurait pu être correcte s'il avait pensé que $5\times6$ correspondait au nombre de dizaines mais il ne semble pas avoir conscience de la signification de l'écriture positionnelle de notre système de numération puisqu'il considère le 6 des dizaines et le 8 des unités de la même manière. Son résultat est faux.
   \end{itemize}
   \item Voici quelques démarches possibles par un élève de cycle 3 : \\ [3mm]
\begin{tabular}{|p{4.8cm}|p{5.2cm}|p{4.6cm}|}
      \hline
      Calculs. & Démarche. & Connaissances. \\
      \hline
      $100\times28 =2\,800$ \newline
      $2\,800\div2 =1\,400$ \newline
      $1\,400\div2 =700$.
      &
      25 c'est 100 divisé par 4 ; diviser par 4, c'est diviser par 2 puis encore diviser par 2.
      &
      Multiplication par 100, division par 2 (prendre la moitié). \\
      \hline
      $100\times28 =2\,800$ \newline
      $2\,800\div4 =700$.
      &
      Multiplier par 25, c'est multiplier par 100 puis diviser par 4.
      &
      Multiplication par 100, \newline
      division par 4 (table des 4). \\
      \hline
      $28 =30-2$ \newline
      $25\times30 =750$ \newline
      $25\times2 =50$ \newline
      $750-50 =700$
      &
      Décomposition de 28 à partir de la dizaine la plus proche 30,  multiplication par 25 de 30 et 2, soustraction des résultats.
      &
      Décomposition d'un nombre, multiplication par 25 de nombre simples, distribution de la multiplication sur la soustraction. \\
      \hline
      $25\times28 =(20+5)\times(20+8)$ \newline
      $20\times20 =400$ \newline
      $20\times8 =160$ \newline
      $5\times20 =100$ \newline
      $5\times8 =40$ \newline
      $400+160+100+40 =700$
      &
      Décomposition additive des deux facteurs, calculs de tous les termes issus des multiplications, addition des résultas obtenus.
      &
      Décomposition additive, distributivité de la multiplication sur l'addition, multiplications par 2, 5 et 10. \\
      \hline
   \end{tabular}
\end{enumerate}
