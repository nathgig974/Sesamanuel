\ \\ [-5mm]
   \begin{enumerate}
      \item D'après le document 3, étant donnée la vitesse moyenne calculée supérieure à 100 km/h, il faut diminuer la vitesse de 5\,\%, soit appliquer un coefficient multiplicateur de $1-\dfrac{5}{100} =0,95$. \\ [1mm]
         Or, $123\times0,95 =116,85$. Donc, {\blue La vitesse retenue est de 116,85 km/h}.
      \item L'automobiliste parcourt \ukm{5,1} (document 1) en \umin{4}. \\ [1mm]
         On a $v =\dfrac{d}{t} =\dfrac{\ukm{5,1}}{\umin{4}}$ \\ [2mm]
         \hspace*{17mm} $=\dfrac{\ukm{5,1}}{\dfrac{4}{60}\uh{}}$ \\ [1mm]
         \hspace*{17mm} $=5,1\times\dfrac{60}{4}\,\ukm{}/\uh{}$ \\ [1mm]
         \hspace*{17mm} $=76,5\,\ukm{}/\uh{}$. \\ [1mm]
         D'après le document 3, étant donnée la vitesse moyenne calculée inférieure à 100 km/h, il faut diminuer la vitesse de 5 km/h, soit 76,5 km/h $-$ 5 km/h = 71,5 km/h. \\
         Conclusion : {\blue la vitesse retenue est de 71,5 km/h}.
      \item À une vitesse retenue de 114 km/h, la vitesse calculée était nécessairement supérieure, donc supérieure à 100 km/h, ce qui signifie qu'une réduction de 5\,\% a été appliquée à la vitesse calculée. \\ [1mm]
         Soit $v$ la vitesse calculée, on a alors $v\times0,95 =114\,\ukm{}/\uh{} \iff v =\dfrac{114\,\ukm{}/\uh{}}{0,95} =120\ukm{}/\uh{}$. \\
         {\blue La vitesse moyenne calculée était de 120 km/h}.
      \item Entre 9 h 17 min 56 s et 9 h 22 min 07, il s'est écoulé 4 min 11 s, soit \us{251}. \\
         Il a donc parcouru \ukm{5,1} en \us{251}. \\
         En \uh{1}, soit \us{3600}, il aurait donc parcouru $\dfrac{\ukm{5,1}\times\us{3600}}{\us{251}} \approx\ukm{73,15}$, d'où une vitesse de 73,15 km/h environ. \\
         La correction appliquée est un retrait de 5 km/h, soit 68,15 km/h. \\
         La vitesse étant limitée à 70 km/h, d'après le document 1, {\blue le conducteur ne sera pas verbalisé}.
   \end{enumerate}
