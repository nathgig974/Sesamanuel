\ \\ [-5mm]
\begin{enumerate}
   \item Il y a trois démarches principales :
   \begin{itemize}
      \item celle de Benjamin et d'Océane qui calculent séparément la partie entière et la partie décimale soit par des calculs en ligne pour Benjamin, soit par un calcul posé en colonnes pour Océane. Ensuite, ils associent leurs résultats en plaçant une virgule entre les deux.
      \item celle d'Isabelle qui passe par les fractions décimales : elle transforme ses nombres décimaux en fractions décimales de dénominateur 100 afin d'harmoniser les écritures pour pouvoir ensuite en faire une somme unique.
      \item celle de Pierre, une solution hybride qui consiste à décoder ses nombres décimaux en une somme d'un entier (unité) et d'une fraction décimale, puis il effectue des calculs séparés sur les fractions, puis sur les entiers, et enfin il ajoute ses résultats pour pouvoir obtenir un nombre en écriture décimale.
   \end{itemize}
   \item Pour Benjamin et Olivier, ils pensent les nombres entiers décimaux comme deux nombres séparés par une virgule, sans lien entre les deux.
   \item On peut utiliser un instrument de calcul comme des abaques (bouliers ou abaques à jetons) afin d'avoir une représentation plus visuelle du nombre, ce qui a également l'avantage de travailler sur la manipulation et sur l'aspect historique du calcul posé ; on peut également utiliser un tableau de numération et placer les nombres dans le tableau ; enfin, l'enseignant peur revenir à la manière dont ont été introduits les nombres décimaux par les fractions décimales et oraliser les nombres du type \og 3 et 12 centièmes \fg{} au lieu de \og 3 virgule 12 \fg{}.
\end{enumerate}
