\ \\ [-5mm]
\begin{enumerate}
   \item \textcolor{A1}{$\bullet$} mesurer le périmètre d'une figure (ici à l'aide d'une quadrillage) ;
   \begin{itemize}
      \item mesurer l'aire d'une surface (ici à partir d'un pavage simple) ;
      \item différencier aire et périmètre d'une figure.
   \end{itemize}
\end{enumerate}
\Coupe
\begin{enumerate}
\setcounter{enumi}{1}
   \item Sur papier pointé, les unités de longueur et d'aire ne sont pas clairement matérialisées et les élèves devront construire mentalement le nombre de ces unités pour chaque figure. De plus, à la vérification, les \og points \fg{} du réseau pointé situés sur les segment des figures ne se verront pas forcément après tracé.
   \item
   \begin{enumerate}
      \item \textcolor{A1}{$\bullet$} Deux figures d'Axel sont superposables, Axel a donc trouvé uniquement deux figures différentes. D'ailleurs, il faudra probablement repréciser aux élèves ce que sont deux figures {\it différentes}.
      \begin{itemize}
         \item {\bf Axel} a commencé par tracer une succession de segments de même longueur que la figure A, mais après 4 segments correspondant à un côté du quadrillage et deux segments diagonale, il ne lui reste que deux segments de type côté de quadrillage à placer, ce qui est impossible pour pouvoir fermer la figure.
      \end{itemize}
      \item {\bf Jean} n'a pas bien compris la consigne : il a tracé trois figures ayant même périmètre que la figure A, puis deux figures ayant même aire que la figure B. Il semble donc avoir acquis les notions de périmètre et d'aire, mais n'a pas compris le mot \og à la fois \fg{} et ne considère donc pas les deux critères simultanément.
      \item Les trois figures proposées par {\bf Timéo} ont la même aire que la figure B. Il semble les avoir construites à partir d'une base carrée de côté 2 carreaux, à laquelle il a ajouté deux demi-carrés. Cependant, seule la première figure possède le bon périmètre. L'objectif sur les aires et donc atteint, celui sur les périmètres non. Tout comme Jean, il n'a pas pris en compte les deux critères à la fois.
      \item Le maître peut proposer aux élèves de calculer le périmètre et l'aire de chacune de leurs figures afin de vérifier si les élèves ont acquis les compétences sur les aires et les périmètres. Ensuite, il pourra leur demander de relire à voie haute la consigne et vérifier que celle-ci est bien comprise.
   \end{enumerate}
\end{enumerate}
