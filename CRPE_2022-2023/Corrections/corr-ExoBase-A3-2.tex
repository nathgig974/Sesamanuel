\ \\ [-5mm]
\begin{enumerate}
      \item $1+2+1+3+3+5+6+4+2+1+2+2+1 =33$. Donc, \bm{il y a 33 élèves dans cette classe.} \\
      $18-3 =15$. Donc, \bm{l'étendue est de 15.}
      \smallskip
      \item $p =\dfrac{1+2+1+3+3}{33}\times100 =\dfrac{10}{33}\times100 \approx30,3$. \\ [1mm]
      Donc, le pourcentage d'élèves ayant obtenu une note inférieure ou égale à 8 est d'environ \bm{30,3\,\%.} \\
      \item
      \begin{itemize}
         \item $\overline{m} =\dfrac{1\times3+2\times5\dots\times2\times17+1\times18}{33} =\dfrac{330}{33} =10$.
         \item $33\times0,25 =8,25$ ; le premier quartile correspond à la 9\up{ième} valeur, c'est à dire 8.
         \smallskip
         \item La médiane correspond à la valeur centrale, c'est-à-dire la 17\up{ième}, c'est 10.
         \item $33\times0,75 =24,75$ ; le troisième quartile correspond à la 25\up{ième} valeur, c'est à dire 11.
         \smallskip
      \end{itemize}
      La calculatrice nous donne également \bm{$\overline{m} = 10$ ; $Q_1 =8$ ; Méd $= 10$ ; $Q_3 =11$.} \\
      \item Soit $n$ la note obtenue au quatrième devoir, on a  : \\
      $\dfrac{3\times9+n}{4} =9,5 \iff 27+n =9,5\times4 \iff n=38-27\iff n=11$. \\ [1mm]
     \bm{Bastien a obtenu 11 au dernier devoir.}
  \end{enumerate}
