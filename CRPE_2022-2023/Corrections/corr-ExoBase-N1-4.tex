\ \\ [-5mm]
   \begin{enumerate}
      \item
      \begin{enumerate}
         \item Le système Cincofile est un système de numération positionnel de base 5. \\
         Dans notre système positionnel de base 10, \, \ding{110} \ding{110} \ding{110} est représenté par le nombre $4\times5^2+4\times5^1+4\times5^0 =4\times25+4\times5+4 ={\blue 124}.$
      \end{enumerate}
   \textcolor{G1}{\bf b)} On effectue les divisions euclidiennes successives par 5 : \\
      $\opidiv[remainderstyle.2=\textcolor{blue}]{273}{5}$ \quad $\opidiv[remainderstyle.2=\textcolor{blue}]{54}{5}$ \quad $\opidiv[remainderstyle=\textcolor{blue}]{10}{5}$ \quad $\opidiv[remainderstyle=\textcolor{blue}]{2}{5}$. \\ [1mm]
      Donc, 273 s'écrit $\overline{2043}^5$ en base 5, ce qui est codé par \, {\blue\ding{54} \ding{108} \ding{110} \ding{116}} \\
      \setcounter{enumi}{1}
      \item
      \begin{enumerate}
          \item On doit enlever une unité au nombre \ding{116} \ding{110} \ding{108} \\
            Or, le chiffre du premier rang (rang des unités) est nul, on va donc \og prendre \fg{} une unité au rang 2 (que l'on pourrait appeler rang des quinaires) que l'on va ajouter au rang 1, auquel on peut cette fois-ci supprimer une unité, il restera donc \ding{110} alors qu'au deuxième rang, \ding{110} sera devenu \ding{116} \\
Le nombre recherché est donc {\blue \ding{116} \ding{116} \ding{110}}
            \item On ajoute une unité au premier rang \ding{110} qui nous donne \og 5 \fg, donc une unité de rang 1 et une unité des quinaires (la retenue) en plus. \\
            Le chiffre de ce rang devient alors \ding{108} avec une unité supplémentaire au rang 3 : \ding{54} devient \ding{116} \\
            Le nombre suivant \og \ding{54} \ding{110} \ding{110} \fg{} est donc {\blue \ding{116} \ding{108} \ding{108}}
      \end{enumerate}
   \end{enumerate}
