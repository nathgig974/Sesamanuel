\begin{itemize}
   \item {\bf Erreur 1 :} l'élève semble avoir compris que deux droites sont perpendiculaires si l'une des deux droites est verticale. \\
   L'horizontale et la verticale sont deux directions naturellement privilégiées dans la vie courante. D'autre part, pour beaucoup d'enfants, la perpendicularité est associée à la notion d'angle droit, droit étant lui même associé à vertical (\og tiens toi droit ! \fg).
   \item {\bf Erreur 2 :} on peut repérer la même erreur que précédemment, l'enfant a aussi pu mal utiliser son équerre. \\
   Dans ce dernier cas, il n'a peut-être pas compris l'utilisation de cet instrument : quel \og bord \fg{} je mets en correspondance avec mon dessin ? Quel \og bord \fg{} me permet de tracer la droite demandée ?
   \item {\bf Erreur 3 :} l'élève a sûrement bien placé son équerre et l'a probablement déplacée jusqu'au \og bout \fg{} de la droite. N'atteignant pas le point, il a estimé qu'il était impossible de tracer la droite. \\
   Cette démarche est directement liée à la conception que l'élève a de la représentation graphique d'une droite, il ne la perçoit pas comme étant \og infinie \fg.
   \item {\bf Erreur 4 :} la lettre représentant le point M semble à peu près correctement placée. En revanche, l'élève assimile le point à la lettre qui le nomme et n'indique pas le code qui le représente. \\
   Il peut aussi y avoir confusion du terme \og milieu \fg{} qui, dans le langage courant, signifie être entre les limites d'un objet.
   \item {\bf Erreur 5 :} l'élève a tracé {\it un} rectangle ABCD, puis {\it un} segment [AC], sans penser qu'il pourrait y avoir une relation entre les deux. \\
   Il a effectué les instructions une par une au lieu de les prendre dans leur globalité.
   \item {\bf Erreur 6 :} l'élève semble avoir construit un carré, puis, en tâtonnant (traces de la pointe du compas), a essayé de placer au mieux le cercle passant par les sommets du carré. \\
   La non-reconnaissance du lien entre les figures de base est peut-être due au fait que ce lien est à construire (les diagonales du carré).
   \item {\bf Erreur 7 :} \\
      {\bf Patrice} a identifié le cercle et un quadrilatère, commence par faire tracer le cercle puis le carré mais ses instructions ne sont pas très claires : pour le cercle, il parle simplement de compas et de la moitié de 8 cm sans préciser qu'il faut le tracer. Le carré est tracé côté par côté en donnant sa longueur et son orientation. Cet élève utilise beaucoup le langage spatial et un vocabulaire d'action en faisant référence aux instruments. Quand il utilise un vocabulaire mathématique, il y a des imprécisions et des faux sens (largeur et hauteur du cercle). \\
      {\bf Géraldine} a également identifié le carré et le cercle qu'elle caractérise convenablement, mais ne met pas en évidence les relations entre ces deux figures. L'absence de référence de relations entre le carré et le cercle a plusieurs significations : les relations ne sont pas visibles ; elle ne pense pas cela utile de le préciser, ou encore elle pense que le plus important est de décrire les figures qu'elle voit. \\
      {\bf Guillaume} a identifié le carré (par la longueur de son côté) et le cercle (par son centre), ainsi que la relation entre eux. Par conte, il ne donne pas le rayon du cercle et confond \og angle \fg{} et \og sommet de l'angle \fg{}. L'oubli du rayon est peut-être dû à une difficulté de se décentrer, et l'imprécision du vocabulaire fait penser à une trop grande influence du langage courant (angle d'une table).
\end{itemize}
