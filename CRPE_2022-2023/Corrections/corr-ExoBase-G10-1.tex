   On a la figure suivante : \\
   \begin{center}
   {\psset{unit=0.9}
      \begin{pspicture}(0,-0.5)(5,4.8)
         \pstGeonode[PosAngle={-135,-45,45,135},CurveType=polygon,PointSymbol=none](0,0){A}(6.2,0){B}(6.2,5.2){C}(0,5.2){D}
         \pstLineAB[linecolor=B1]{A}{C}
      \end{pspicture}}
   \end{center}
   \begin{enumerate}
      \item Le triangle ABC étant rectangle en B, on peut appliquer le théorème de Pythagore, avec des mesures en cm : \\
      $\text{AC}^2 =\text{AB}^2+\text{BC}^2 \iff \text{AC}^2 =6,2^2+5,2^2$ \\
      $\iff \text{AC}^2 =38,44+27,04 =65,48$ \\
      \; $\Longrightarrow \text{AC} =\sqrt{65,48} \approx8,09$. \\
      {\blue La diagonale de sa console mesure environ \ucm{8,1}.}
      \item 3 pouces correspondent à \ucm{8,1} ; donc, 1 pouce correspond à $\ucm{8,1}\div3 =\ucm{2,7}$. \\
      {\blue Un pouce vaut environ \ucm{2,7}.}
      \item $10,1\times2,7 = 27,27$ donc, {\blue la diagonale d'un netbook de 10,1 pouces est d'environ \ucm{27,3}.}
   \end{enumerate}
