\ \\ [-5mm]
   \begin{enumerate}
      \item D'après l'énoncé, les droites $(OH)$ et $(\Delta)$ sont perpendiculaires. \\
         Par ailleurs, la droite $(\Delta)$ est tangente en $T$ à $(C_3)$, donc les droites $(JT)$ et ($\Delta)$ sont perpendiculaires. \\
         Par conséquent, les droites $(OH)$ et $(JT)$ sont parallèles comme droites perpendiculaires à la même droite $(\Delta)$. \\
         De plus, les points $E, H$ et $T$ sont alignés dans cet ordre et les points $E, O $ et $J$ sont alignés dans cet ordre. \\
         D'après le théorème de Thalès : $\dfrac{EO}{EJ} =\dfrac{EH}{ET} =\dfrac{OH}{JT}$ \\ [1mm]
         \hspace*{4.2cm} $\Rightarrow \dfrac{3r}{5r} =\dfrac{a}{r}$ \\ [1mm]
         \hspace*{4.3cm} $\Rightarrow a =\dfrac{3r^2}{5r} =\dfrac{3r}{5}$. \quad D'où on a bien {\blue $a =\dfrac35r$.}
      \item $a$ s'écrit sous la forme $\dfrac{p}{q}$ avec $p =3r$ ($r$ entier) et $q =5$ (entier). \\ [1mm]
         Pas définition, {\blue $a$ est un nombre rationnel comme quotient de deux entiers}.
      \item $a =\dfrac35r =\dfrac{6r}{10}$ avec $6r$ un nombre entier, donc $a$ peut s'écrire sous la forme d'une fraction décimale. \\ [1mm]
         D'où : {\blue $a$ est toujours un nombre décimal}.
      \item $a =\dfrac{3r}{5}$ donc, $a$ est un nombre entier si, et seulement si, $3r$ est divisible par 5. \\ [1mm]
         Or, 3 et 5 sont premiers entre eux, donc $r$ doit être divisible par 5 ce qui signifie qu'il est multiple de 5. \\
         {\blue $a$ est un nombre entier si, et seulement si, $r$ est  multiple de 5}.
      \item Pour que $a$ soit un nombre premier, il faut que $a$ soit un nombre entier et qu'il soit divisible uniquement par 1 et lui-même. Or $a$ est entier si et seulement si $r$ est multiple de 5 d'après la question précédente, c'est-à-dire s'il s'écrit $r = 5k$, où $k$ est un entier. \\
         Dans ce cas, $a = \dfrac{3\times\cancel{5}k}{\cancel{5}} =3k$. \\ [1mm]
         $a$ est donc un multiple de $3$. Il ne peut être premier que lorsque $k$ est égal à 1, c'est-à-dire $r = 5\times1 =5$. \\
         {\blue $a$ ne peut être premier que s'il prend la valeur 3}. \footnote{La question demande seulement si le nombre $a$ peut être un nombre premier, donc une réponse du type \og Pour $r =5$ on obtient $a =3$ qui est bien un nombre premier \fg{} est tout à fait juste.}
      \item On applique le théorème de Pythagore dans le triangle rectangle $OHB$ rectangle en $H$ : \\
         $OB^2 = OH^2+ HB^2 \iff HB^2 = OB^2-OH^2$ \\
         \hspace*{4.4cm} $=r^2-\left(\dfrac35r\right)^2$ \\ [1mm]
         \hspace*{4.4cm} $= \dfrac{25}{25}r^2-\dfrac{9}{25}r^2$ \\ [1mm]
         \hspace*{4.4cm} $= \dfrac{16}{25} r^2$. \qquad D'où : {\blue $HB =\dfrac45r$.}
         \smallskip
      \item Dans le triangle $OAB$, on a $OA = OB$ donc, $O$ appartient à la médiatrice du segment $[AB]$. \\
         De plus, la droite $(OH)$ est perpendiculaire à la droite $(AB)$, donc la droite $(OH)$ est la médiatrice du segment $[AB]$ : elle coupe $[AB]$ en $H$, on milieu. \\
         $H$ étant le milieu du segment $[AB]$, on a $b =AB = 2HB =2\times\dfrac45r =\dfrac{8}{5}r$. \\
         D'où : {\blue Le point $H$ est le milieu de $[AB]$ et on a $b =\dfrac{8}{5}r$.}
         \smallskip
      \item Pour que $b$ soit un nombre premier, il faut que $b$ soit un nombre entier et qu'il soit divisible uniquement par 1 et lui-même. Or $b$ est entier si et seulement si $8r$ est est divisible par 5. Avec 8 et 5 premiers entre eux, c'est dont $r$ qui doit être divisible par 5, c'est-à-dire qu'il s'écrit $r = 5k$, où $k$ est un entier. \\
         Dans ce cas, $b =\dfrac{8\times\cancel{5}k}{\cancel{5}} =8k$. \\
         $b$ est alors un multiple de 8 et comme $8 =2^3$ est composé, $b$ ne peut jamais être un nombre premier. \\
         {\blue Il n'existe pas de nombre $r$ tel que $b$ soit premier.}
   \end{enumerate}
