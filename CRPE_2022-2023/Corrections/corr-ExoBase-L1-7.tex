\ \\ [-5mm]
   \begin{enumerate}
      \item 45 est composé de 4 dizaines et de 5 unités.
      \begin{itemize}
         \item Étape 1 : on calcule $4\times5 =20$ ce qui correspond au nombre de centaines.
         \item Étape 2 : on écrit 25 à droite de 20.
      \end{itemize}
      On a donc {\blue $45^2 =2\,025$.}
      \item $n^2 =(10d+5)^2$ \\
        \hspace*{0.8cm} $=(10d)^2+2\times10d\times5+5^2$ \\
         \hspace*{0.8cm} $=100d^2+100d+25$ \\
         \hspace*{0.4cm} {\blue $n^2 =100d(d+1)+25.$}
   \end{enumerate}

\Coupe

   \begin{enumerate}
   \setcounter{enumi}{2}
      \item Le terme \og $d(d+1)$ \fg{} correspond à la multiplication du nombre de dizaines $d$ par l'entier qui le suit $(d+1)$. Le résultat obtenu est multiplié par 100 ce qui signifie que $d(d+1)$ est le nombre de centaines (étape 1) ; à ce nombre, on ajoute 25 qui est un nombre inférieur à 100 c'est la raison pour laquelle il suffit d'écrire 25 à droite du nombre de centaines pour obtenir le résultat (étape 2).
      \item Un nombre décimal $n'$ de partie décimale 5 peut s'écrire sous la forme de la fraction décimale $n' =\dfrac{n}{10}$ où $n$ est un nombre entier qui se termine par 5. \\
         On a alors $n' =\dfrac{n^2}{100} =\dfrac{100d(d+1)+25}{100} =d(d+1)+0,25$. \\ [1mm]
         Par analogie avec ce qui a été vu pour les entiers, il suffit donc de multiplier la partie entière par son successeur pour obtenir la partie entière et de prendre 25 comme partie décimale. \\
         Si on applique ce procédé à 3,5, on effectue le produit de 3 par 4 qui donne 12 (partie entière) et 25 est la partie décimale donc : {\blue $3,5^2 =12,25$.}
   \end{enumerate}
