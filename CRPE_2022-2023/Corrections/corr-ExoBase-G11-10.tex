\ \\ [-5mm]
   \begin{enumerate}
      \item Calcul de la somme $S_1$ des aires des deux carrés gris : $S_1 =3^2+4^2 =25$. \\
      Calcul de l'aire $S_2$ du carré blanc  : $S_2 =5^2 =25$. \\
     {\blue La somme des aires des deux carrés gris est égale à l'aire du carré blanc}.
      \item Dans la question précédente, on a montré que $3^2+4^2 =5^2$ soit $AC^2+AB^2 =CB^2$. \\
         D'après la réciproque du théorème de Pythagore, le triangle $ABC$ est rectangle en A. {\blue Claude a raison}.
      \item $\bullet$ Dans le triangle $MNP$, rectangle en $P$, on utilise le théorème de Pythagore : \\
         $MN^2 =MP^2+PN^2 =(\ucm{12})^2+(\ucm{5})^2 =\ucmq{169}$. D'où $MN =\ucm{13}$. \\
         $13$ est un nombre entier naturel, c'est donc aussi un nombre décimal. \\
         $\bullet$ Mes points $M, I, N$ et $M, J, P$ sont alignés dans cet ordre, les droites $(IJ)$ et $(NP)$ sont parallèles puisqu'on a des carrés. D'après le théorème de Thalès, on a : $\dfrac{MJ}{MP} =\dfrac{MI}{MN} =\dfrac{JI}{PN} \iff \dfrac{\ucm{3}}{\ucm{12}} =\dfrac{MI}{MN} =\dfrac{JI}{\ucm{5}}$. \\
         Soit $IJ =\dfrac{\ucm{3}\times\ucm{5}}{\ucm{12}} =\dfrac{\ucmq{15}}{\ucm{12}} =\dfrac54\ucm{} =\dfrac{5}{2^2}\ucm{}$. \\ [1mm]
         Ce nombre, mis sous forme irréductible, ne comporte que des puissances de 2 au dénominateur, c'est donc un nombre décimal. {\blue Dominique a raison}.
      \item On considère la figure suivante qui est une coupe parallèle à la base de la figure 3 :
         \begin{minipage}{7cm}
            {\psset{unit=0.8}
            \small
               \begin{pspicture}(-1,-0.8)(7.5,2.75)
                  \psline(0,0)(3,0)(3,1)(7,1)(7,2)
                  \psline[linestyle=dashed](3,0)(7,0)(7,1)
                  \psline[linestyle=dashed](0,0)(3,1)(7,2)
                  \rput(0,-0.3){$R$}
                  \rput(3,-0.3){$U$}
                  \rput(3,1.3){$S$}
                  \rput(7,-0.3){$V$}
                  \rput(7,2.3){$T$}
               \end{pspicture}
            }
         \end{minipage}
         \begin{minipage}{9cm}
            On a $\dfrac{RU}{RV} =\dfrac{\ucm{3}}{\ucm{7}} =\dfrac37$ et $\dfrac{SU}{TV} =\dfrac{\ucm{1}}{\ucm{2}} =\dfrac12$ soit $\dfrac{RU}{RV}\neq\dfrac{SU}{TV}$.
         \end{minipage}
            Or, les point $R,U,V$ sont alignés et les droites $(SU)$ et $(TV)$ sont parallèles, puisque ce sont les supports des côté opposés du carré du milieu. Ainsi, si les points $R, S, T$ étaient alignés, d'après le théorème de Thalès les rapports seraient égaux ce qui n'est pas le cas. Donc, {\blue Camille a tort}.
         %\end{minipage}
   \end{enumerate}
