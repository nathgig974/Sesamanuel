\ \\ [-5mm]
\begin{enumerate}
   \item Dans le triangle AHB rectangle en H, d'après le théorème de Pythagore, on a : \\
   $\text{AB}^2 =\text{AH}^2+\text{HB}^2 \iff 5^2 =\text{AH}^2+3^2 \iff\text{AH}^2 =25-9 =16\Longrightarrow \text{AH} =\sqrt{16} =4$. \\
   \bm{La hauteur AH mesure 4 cm.}
   \item Dans le triangle AHC rectangle en H, d'après le théorème de Pythagore, on a : \\
   $\text{AC}^2 =\text{AH}^2+\text{HC}^2 \iff \text{AC}^2 =4^2+7^2\iff\text{AC}^2 =16+49 =65\Longrightarrow \text{AC} =\sqrt{65}$. \\
   \bm{Le côté AC mesure environ 8,1 cm.}
\end{enumerate}
