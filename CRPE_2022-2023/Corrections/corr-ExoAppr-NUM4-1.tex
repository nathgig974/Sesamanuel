{\bf A.}
\begin{enumerate}
   \item Cet élève semble maîtriser l'addition de deux nombres entiers, mais il n'a pas compris le sens de l'écriture à virgule d'un nombre décimal qu'il traite comme deux d'entiers séparés par une virgule : il additionne séparément les parties entières et décimales des deux nombres.
   \item
   \begin{enumerate}
      \item Il s'agit de la même erreur que dans la question précédente. Comme les entiers écrits à gauche de la virgule sont identiques, il compare ceux de droite : $3<6<100$, donc $5,03<5,6<5,100.$
      \item Il pourrait dire les nombres décimaux en insistant sur la fonction des différents chiffres : \og cinq et trois centièmes \fg, \og cinq et six dixièmes \fg{}, \og cinq et cent millièmes \fg.
   \end{enumerate}
\end{enumerate}

{\bf B.}
\begin{enumerate}
   \item Un nombre décimal est un nombre qui peut s'écrire sous la forme d'une fraction décimale, c'est à dire un nombre dont le dénominateur est 10, 100, 1\,000\dots
   \item L'enseignant peut, par exemple, faire remarquer qu'en ajoutant $\dfrac{5}{10}$ à $\dfrac{5}{10}$ (déjà placé sur la droite graduée), on trouve $\dfrac{10}{10}$ qui peut aussi s'écrire 1. Donc deux nombres décimaux non entiers peuvent avoir une somme qui est un entier.
    \item C'est vrai car la somme de deux fractions décimales est une fraction décimale. Si les dénominateurs sont les mêmes, c'est immédiat ; sinon, on utilise les représentations équivalentes : $\dfrac{3}{10}=\dfrac{30}{100}=\dots$ pour s'y ramener.
   \item Le principal intérêt touche à la propriété fondamentale de l'ensemble des nombres décimaux : on peut toujours intercaler un nombre décimal entre deux nombres décimaux. \\
   Un deuxième intérêt pourrait être de montrer l'effet récursif de la construction des nombres décimaux, ce que l'on peut appeler \og l'effet zoom \fg{} de la droite numérique puisqu'on peut diviser en 10 chaque unité de la droite : l'unité, $\dfrac1{10}, \dfrac1{100}$\dots
   \end{enumerate}
