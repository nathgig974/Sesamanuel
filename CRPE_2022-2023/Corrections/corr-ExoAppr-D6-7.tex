\ \\ [-5mm]
\begin{enumerate}
   \item
   \begin{enumerate}
      \item \bm{12 d'archers} ont gagné exactement six points lors de ce championnat.
      \item \bm{75 d'archers} ont gagné trois points ou plus lors de ce championnat.
      \item L'effectif est de 80, le score médian correspond donc à une valeur comprise entre le 40\up{ième} et le 41\up{ième} archer, classés par ordre croissant du nombre de points par exemple. \\
      Or, le 40\up{ième} et le 41\up{ième} archer ont tous deux marqué 7 points donc, \bm{le score médian est de 7 points.}
   \end{enumerate}
   \item
   \begin{enumerate}
      \item Calculons le score moyen des archers du club A : \\ [1mm]
      $m =\dfrac{5\times2+9\times3+8\times5+12\times6+14\times7+6\times8+8\times9+18\times10}{80} =\dfrac{547}{80} =6,8375$. \\ [1mm]
      Or, le score moyen du club B est de 7 points donc, \bm{c'est le club B qui a le score moyen le plus élevé}.
      \item Les 10 meilleurs archers du club A ont tous marqué 10 points alors que la moyenne des 10 meilleurs archers du club B est de 9,9 points donc, \\
      \bm{selon les scores de leurs dix meilleurs archers, c'est le club A qui l'emporte}.
   \end{enumerate}
\end{enumerate}
