\themaL
\vspace*{-1cm}
{\Huge\textsf{Sommaire des réjouissances du tome 1 !}}
  %\pagestyle{affichesommaire}
  %\thispagestyle{firstaffichesommaire}

%bandeau nombres et calculs
\begin{pspicture}(0,-0.5)(\linewidth,\dimexpr\SquareWidth*3+1)
    \psframe*[linewidth=0pt,linecolor=B1](0,0)(\linewidth,\dimexpr\SquareWidth*3)
    \rput(8.4,0.5){\textcolor{white}{\Large\textsf{NOMBRES ET CALCULS}}}
  \end{pspicture}

      N1 - Numérations et bases \dotfill \pageref{N1} \\
      N2 - Ensembles de nombres et calculs \dotfill \pageref{N2} \\
      N3 - Calcul littéral \dotfill \pageref{N3} \\
      N4 - Arithmétique \dotfill \pageref{N4}
   
%bandeau organisation et gestion de données
\begin{pspicture}(0,-0.5)(\linewidth,\dimexpr\SquareWidth*3+1)
    \psframe*[linewidth=0pt,linecolor=PartieStatistique](0,0)(\linewidth,\dimexpr\SquareWidth*3)
    \rput(8.4,0.5){\textcolor{white}{\Large\textsf{ORGANISATION ET GESTION DE DONNÉES}}}
\end{pspicture} 

      D5 - Fonctions et tableurs \dotfill \pageref{D5} \\
      D6 - Proportionnalité \dotfill \pageref{D6} \\
      D7 - Probabilités \dotfill \pageref{D7} \\
      D8 - Statistiques \dotfill \pageref{D8}

%bandeau géométrie
\begin{pspicture}(0,-0.5)(\linewidth,\dimexpr\SquareWidth*3+1)
    \psframe*[linewidth=0pt,linecolor=A1](0,0)(\linewidth,\dimexpr\SquareWidth*3)
    \rput(8.5,0.5){\textcolor{white}{\Large\textsf{GÉOMÉTRIE}}}
  \end{pspicture}
    
    G9 - Géométrie plane \dotfill \pageref{G9} \\
    G10 - Théorème de Pythagore et trigonométrie \dotfill \pageref{G10} \\
    G11 - Théorème de Thalès et transformations \dotfill \pageref{G11} \\
    G12 - Géométrie dans l'espace \dotfill \pageref{G12}
   
%bandeau grandeurs et mesures
\begin{pspicture}(0,-0.5)(\linewidth,\dimexpr\SquareWidth*3+1)
    \psframe*[linewidth=0pt,linecolor=G1](0,0)(\linewidth,\dimexpr\SquareWidth*3)
    \rput(8.5,0.5){\textcolor{white}{\Large\textsf{GRANDEURS ET MESURES}}}
  \end{pspicture}
    
    M13 - Périmètres et aires \dotfill \pageref{M13} \\
    M14 - Volumes et autres grandeurs \dotfill \pageref{M14}
    
%bandeau transveral
\begin{pspicture}(0,-0.5)(\linewidth,\dimexpr\SquareWidth*3+1)
    \psframe*[linewidth=0pt,linecolor=J1](0,0)(\linewidth,\dimexpr\SquareWidth*3)
    \rput(8.5,0.5){\textcolor{white}{\Large\textsf{NUMÉRIQUE}}}
  \end{pspicture}
    
   P15 - Algorithmes et programmation \dotfill \pageref{T15} \\

\bigskip

{\bf Liste des références utilisées dans les deux tomes} \dotfill \pageref{ref} \\
{\bf Solutions des exercices} \dotfill \pageref{sol}
   