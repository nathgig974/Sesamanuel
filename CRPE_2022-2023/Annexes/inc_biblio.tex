\annexe{Ressources\dots} \label{ref} 

\bigskip

{\setlength{\baselineskip}{1.2\baselineskip}

\textcolor{A1}{\Large Ressources institutionnelles}
\begin{description}
   \item[édu1] \href{https://eduscol.education.fr/83/j-enseigne-au-cycle-1}{J'enseigne au cycle 1.} Programme, recommandations pédagogiques, ressources d'accompagnement, évaluation et infothèque.
   \item[édu2] \href{https://eduscol.education.fr/84/j-enseigne-au-cycle-2}{J'enseigne au cycle 2.} Programme, à consulter, évaluation, ressources d'accompagnement et infothèque.
   \item[édu3] \href{https://eduscol.education.fr/87/j-enseigne-au-cycle-3}{J'enseigne au cycle 3.} Programme, à consulter, évaluation, ressources d'accompagnement et infothèque.
   \item[con22] \href{http://www.devenirenseignant.gouv.fr/cid98675/sujets-des-epreuves-ecrites-et-conseils-des-jurys-des-concours-de-recrutement-de-professeurs-des-ecoles.html}{Sujets de concours.}
   \item[crp22] \href{http://www.devenirenseignant.gouv.fr/pid33983/enseigner-maternelle-elementaire-crpe.html}{Le concours CRPE : s'informer, s'inscrire.}
\end{description} 

\bigskip

\textcolor{A1}{\Large Bibliographie et articles}
\begin{description}
   \item[abi13] Abiteboul (2013). \href{http://www.epi.asso.fr/revue/editic/itic-ecole-prog_2013-12.htm}{\it Proposition d'orientations générales pour un programme d'informatique à l'école primaire}. EPI
   
   \item[asp15] Aspinall B. (2015). \href{http://brianaspinall.com/10-reasons-to-teach-coding-sketchnote-by-sylviaduckworth/}{\it Page de Brian Aspinall}.
   
   \item[bou12] Boule F. (2012). {\it Le calcul mental au quotidien}. Cycles 2 et 3 (nouvelle édition mise à jour). Canopé, CRDP.

   \item[bri98] Brissiaud R. (1998). \href{http://page.perso.brissiaud.pagesperso-orange.fr/pages/Page2.html}{\it Les fractions et les décimaux au CM1}. Actes du {\small XXV}\up{e} colloque des formateurs de mathématiques, IREM Brest.
   \item[bri03] Brissiaud R. (2003). {\it Comment les enfants apprennent à calculer} (nouvelle édition). Retz.
   \item[bri07] Brissiaud R. (2007). {\it Premiers pas vers les maths}. Les chemins de la réussite à l’école maternelle. Retz.
   \item[bri13] Brissiaud R. (2013). {\it Apprendre à calculer à l’école} – Les pièges à éviter en contexte francophone. Retz.
   
   \item[bro87-1] Brousseau G. et N. (1987). {\it Rationnels et décimaux dans la scolarité obligatoire}. IREM Bordeaux.
   \item[bro87-2] Brousseau G. (1987). {\it Le puzzle de Brousseau}, Recherche en didactique, n\degre2.1.
   \item[bro00] Brousseau G. (2000). {\it Les propriétés didactiques de la géométrie élémentaire}. Séminaire de Didactique des Mathématiques, Rethymon.

   \item[cer10] Cerclé V. (2010). \href{http://www.apmep.fr/IMG/pdf/Puzzle_Carroll_.pdf}{\it Un puzzle de Lewis Caroll}. Plot n\degre29, APMEP.
      
   \item[cha13-1] Charnay R, Mante M. (2013). {\it Mathématiques tome 1}. Professeur des écoles, admissibilité. Hatier concours.
   \item[cha13-2] Charnay R, Mante M. (2013). {\it Mathématiques tome 2}. Professeur des écoles, admissibilité. Hatier concours.
   
   \item[cuc09] Cuchin D. (2009). {\it Construire des rituels à la maternelle}. Collection Découvertes Gallimard.
      
   \item[dav14] Daval N. (2014). \href{http://irem.univ-reunion.fr/spip.php?article753}{\it Les abaques, outils de numération et de calcul} - IREM Réunion.
   \item[dav16] \href{http://irem.univ-reunion.fr/spip.php?article886}{Daval N. (2016). {\it Codage et mathématiques : du langage aux algorithmes, des ressources pour débuter à l’école.}} IREM Réunion.
   \item[dav18] Daval N, Tournès D. (2018). {\it De l'abaque à jetons au calcul posé}. Passerelles : Enseigner les mathématiques par leur histoire au cycle 3, Moyon et Tournès. IREM.    
      
   \item[dow] Dowek G. \href{http://www.lsv.fr/~dowek/Philo/grenoble.pdf}{\it Sciences, langages et langues}.
   
   \item[gue96] Guedj D. (1996). {\it L'empire des nombres}. Retz.
   
   \item[hou17] Houdement C. (2017), {\it Résolution de problèmes arithmétiques à l’école}, Grand N, n°100, IREM de Grenoble.
  
   \item[mar15] Margolinas, C. (2015), {\it Des mathématiques à l’école maternelle}, actes du colloque \og Des mathématiques à l’école maternelle \fg, ENSC Ho Chi Minh.
   
   \item[mar16] Martin Y. (2016). \href{http://irem.univ-reunion.fr/spip.php?article802}{\it Curvica}, activités mathématiques ludiques. IREM Réunion.

   \item[mot10-1] Motteau D, Chernak S. (2010). {\it Mathématiques épreuve écrite}. Concours professeur des écoles. Nathan.
   \item[mot10-2] Motteau D, Chernak S. (2010). {\it Mathématiques épreuve orale}. Concours professeur des écoles. Nathan.

   \item[sav19] Savary A. (2019). \href{http://centre-alain-savary.ens-lyon.fr/CAS/mathematiques-en-education-prioritaire/premieres-annees-de-mathernelle-1/situations-de-classe-et-entretien/lappel-emilie-et-elisabeth}{\it L'appel, un rituel pour construire le nombre}. Centre Alain-Savary, Ifé.
   
   \item[ver86] Vergnaud G. (1986). {\it Psychologie
du développement cognitif et didactique des mathématiques : un exemple, les structures additives}. Grand N n\degre38.
   \item[ver88] Vergnaud G., Brousseau G., Hulin M. (1988). {\it Didactique et Acquisition des Connaissances Scientifiques}. Actes du Colloque de Sèvres, Mai 1987, Grenoble, La Pensée Sauvage.
   
   \item[ver08] Verschaffel L., De Corte E. (2008). {\it Enseignement et apprentissage des mathématiques. Que disent les recherches psychopédagogiques}. De Boeck Supérieur.
   
   \item[vil18] Vilani C., Torossian, C. (2018). {\it 21 mesures pour l'enseignement des mathématiques}. Ministère de l'éducation nationale.
\end{description}

\bigskip


\textcolor{A1}{\Large Sitographie et logiciels}

\begin{description}
   \item[ARPEME] \href{http://www.arpeme.fr/index.php?id_page=27}{Annales corrigées des concours depuis 1997 - COPIRELEM.}
   
   \item[Code studio] \href{https://studio.code.org}{Cours progressifs pour débuter en programmation.}
 
   \item[GeoGebra] \href{https://www.geogebra.org}{Logiciel de géométrie dynamique.}
   
   \item[La main à la pâte] \href{https://www.fondation-lamap.org/fr/123codez}{Fondation La main à la pâte : 1, 2, 3\dots{} codez !}
   
   \item[Les fondamentaux] \href{https://www.reseau-canope.fr/lesfondamentaux/discipline/mathematiques.html}{Des films agités pour bien cogiter - Canopé.}
   
   \item[Mathenpoche] \href{http://mathenpoche.sesamath.net}{Site de l'association Sésamath qui propose des ressources pour le collège.}
   
   \item[Mathématiques et CRPE] \href{http://dpernoux.free.fr/CRPE.htm}{Site de Dominique Pernoux, ex-formateur à l'IUFM d'Alsace.}
   
   \item[Nath et matiques] \href{http://mathematiques.daval.free.fr}{Site de Nathalie Daval, formatrice à la FDE de Montpellier.}
   
   \item[Parimaths] \href{http://www.parimaths.com}{Site de Catherine Marchetti-Jacques, retraitée passionnée de l'Éducation Nationale.}
   
   \item[Primaths] \href{http://primaths.fr}{Site d'Yves Thomas, formateur à l'ESPE des Pays de la Loire.}
   
   \item[Scratch] \href{https://scratch.mit.edu}{Logiciel de programmation, cycles 3 et 4.}  
   \item[ScratchJr] \href{http://www.scratchjr.org}{Application de programmation pour tablette, cycles 1, 2 et 3.}
   
   \item[TFM] \href{http://www.uvp5.univ-paris5.fr/TFM/}{Téléformation en mathématiques - Université Paris Descartes.}
\end{description}


\par}

\label{sol}