\themaL
\vspace*{-1.5cm}
{\Huge\textsf Sommaire}

%bandeau nombres et calculs
\begin{pspicture}(0,-0.5)(\linewidth,\dimexpr\SquareWidth*3+1)
    \psframe*[linewidth=0pt,linecolor=B1](0,0)(\linewidth,\dimexpr\SquareWidth*3)
    \rput(8.4,0.5){\textcolor{white}{\Large\textsf{NOMBRES ET CALCULS}}}
  \end{pspicture}

   \begin{multicols}{2}
      S01 Enchaînement d'opérations \pfb \pageref{S01} \\
      S04 Nombres relatifs \pfb \pageref{S04} \\
      S08 Expressions littérales \pfb \pageref{S08} \\
      S11 Multiples et diviseurs \pfb \pageref{S11} \\
      S14 Comparaison et égalité de fractions \pfb \pageref{S14} \\
      S17 Distributivité \pfb \pageref{S17} \\
      S20 Somme et différence de nombres relatifs \pfb \pageref{S20} \\
      S23 Nombres premiers \pfb \pageref{S23} \\
      S26 Somme et différence de fractions \pfb \pageref{S26} \\
      S29 Introduction aux équations \pfb \pageref{S29}
   \end{multicols}
   
%bandeau géométrie
\begin{pspicture}(0,-0.5)(\linewidth,\dimexpr\SquareWidth*3+1)
    \psframe*[linewidth=0pt,linecolor=A1](0,0)(\linewidth,\dimexpr\SquareWidth*3)
    \rput(8.5,0.5){\textcolor{white}{\Large\textsf{GÉOMÉTRIE}}}
  \end{pspicture}
    
   \begin{multicols}{2}
      S02 Les angles \pfb \pageref{S02} \\
      S05 Repérage et déplacements \pfb \pageref{S05} \\
      S09 Somme des angles d'un triangle \pfb \pageref{S09} \\
      S12 La symétrie centrale \pfb \pageref{S12} \\
      S15 L'inégalité triangulaire \pfb \pageref{S15} \\
      S18 Reconnaître des solides \pfb \pageref{S18} \\      
      S21 Le parallélogramme \pfb \pageref{S21} \\
      S24 Représenter les solides \pfb \pageref{S24} \\
      S27 Les droites du triangle \pfb \pageref{S27}   
   \end{multicols}
   
%bandeau organisation et gestion de données
\begin{pspicture}(0,-0.5)(\linewidth,\dimexpr\SquareWidth*3+1)
    \psframe*[linewidth=0pt,linecolor=Goldenrod](0,0)(\linewidth,\dimexpr\SquareWidth*3)
    \rput(8.4,0.5){\textcolor{white}{\Large\textsf{ORGANISATION ET GESTION DE DONNÉES}}}
\end{pspicture} 

   \begin{multicols}{2}
      S06 Interpréter et représenter des données \pfb \pageref{S06} \\
      S10 Probabilités \pfb \pageref{S10} \\
      S16 Proportionnalité \pfb \pageref{S16} \\
      S22 Traiter des données \pfb \pageref{S22} \\
      S28 Le ratio \pfb \pageref{S28}
   \end{multicols}

%bandeau grandeurs et mesures
\begin{pspicture}(0,-0.5)(\linewidth,\dimexpr\SquareWidth*3+1)
    \psframe*[linewidth=0pt,linecolor=G1](0,0)(\linewidth,\dimexpr\SquareWidth*3)
    \rput(8.5,0.5){\textcolor{white}{\Large\textsf{GRANDEURS ET MESURES}}}
  \end{pspicture}
    
   \begin{multicols}{2}
      S03 Aires et périmètres \pfb \pageref{S03} \\
      S07 Horaires et durées \pfb \pageref{S07} \\
      S13 Calcul d'aires \pfb \pageref{S13} \\
      S19 Volumes et capacités \pfb \pageref{S19} \\
      S25 L'aire du parallélogramme \pfb \pageref{S25} \\
      S30 Transformations et grandeurs \pfb \pageref{S30}
   \end{multicols}
    
%\bigskip

%{\bf Solutions des exercices} \pfb \pageref{solutions}
   