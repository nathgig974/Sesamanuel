\ \\ [-5mm]
   \begin{enumerate}
      \item
         \begin{enumerate}
            \item {\blue $16+17 =33$}.
            \item On peut conjecturer que  {\blue la somme de deux nombres consécutifs est un nombre impair}.
         \end{enumerate}
      \setcounter{enumi}{1}
      \item
         \begin{enumerate}
            \item Le nombre suivant $n$ s'écrit {\blue $n+1$}.
            \item $(n)+(n+1) =n+n+1 ={\blue 2n+1}$.
            \item $2n+1$ est un nombre pair ($2n$) auquel on ajoute 1, c'est donc un nombre impair. \\
               {\blue La conjecture est vraie}.
         \end{enumerate}
      \setcounter{enumi}{2}
      \item Soit $n$ un nombre quelconque, le suivant s'écrit $n+1$ et celui d'après $n+2$. \\
         La somme de trois nombres consécutifs s'écrit alors : \\
         $(n)+(n+1)+(n+2) =n+n+1+n+2 =3n+3$. \\
         $3n+3$ est la somme de deux nombres multiples de 3, {\blue c'est donc un multiple de 3}.
   \end{enumerate}
