   \textcolor{G1}{$\bullet$} Figure A : on a un rectangle de côtés \um{8} et \um{4,5}. \\
      $\mathcal{P} =2\times(L+\ell) =2\times(\um{8}+\um{4,5}) =\blue\um{25}$. \\
      $\mathcal{A} =L\times\ell =\um{8}\times\um{4,5} =\blue\umq{36}$. \\
   \textcolor{G1}{$\bullet$} Figure B : on a un carré de côté \umm{12,3}. \\
      $\mathcal{P} =4\times c =4\times\umm{12,3} =\blue\umm{49,2}$. \\
      $\mathcal{A} =c\times c =\umm{12,3}\times\umm{12,3} =\blue\ummq{151,29}$. \\
   \textcolor{G1}{$\bullet$} Figure C : on a un triangle de base \ucm{6,6} et de hauteur associée \ucm{4,4}. \\
      $\mathcal{P} =\ucm{6,6}+\ucm{6,9}+\ucm{4,6} =\blue \ucm{18,1}$. \\ [1mm]
      $\mathcal{A} =\dfrac{b\times h}{2} =\dfrac{\ucm{6,6}\times\ucm{4,4}}{2}=\blue \ucmq{14,52}$. \\ [1mm]
   \textcolor{G1}{$\bullet$} Figure D : on a un triangle de base \ukm{8,5} et de hauteur associée \ukm{8,5}. \\
      $\mathcal{P} =2\times\ukm{8,5}+\ukm{12} =\blue \ukm{29}$. \\ [1mm]
      $\mathcal{A} =\dfrac{b\times h}{2}  =\dfrac{\ukm{8,5}\times\ukm{8,5}}{2}=\blue \ukmq{36,125}$. \\ [1mm]
   \textcolor{G1}{$\bullet$}  Figure E : on a un disque de rayon \udm{1,5}. \\
      $\mathcal{P} =2\times\pi\times r =2\times\pi\times\udm{1,5} \approx\blue \udm{9,42}$. \\
      $\mathcal{A} =\pi\times r^2 =\pi\times(\udm{1,5})^2 \approx\blue \udmq{7,07}$. \\
   \textcolor{G1}{$\bullet$}  Figure F : on a un disque de rayon \um{2,8}. \\
      $\mathcal{P} =2\times\pi\times r =2\times\pi\times\um{2,8} \approx\blue \um{17,59}$. \\
      $\mathcal{A} =\pi\times r^2 =\pi\times(\um{2,8})^2 \approx\blue \umq{24,63}$. \\
