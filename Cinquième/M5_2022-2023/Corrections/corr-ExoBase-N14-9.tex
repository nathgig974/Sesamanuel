\ \\ [-5mm]
   \begin{enumerate}
      \item Une part de gâteau coûte \ueuro{0,50}. \\
         S'ils ont collecté \ueuro{20}, cela signifie qu'ils ont vendus au total 40 parts de gâteau. \\
         Modélisons la situation par un graphique en barre : \\ [1mm]
         \ModeleBarre{Turquoise 8 {"\ueuro{20} = 40 parts de gâteaux"}}{LightSkyBlue -2 "Jim $\frac14$" PaleTurquoise -3 "Paul $\frac38$" PowderBlue -3 "Jane"} \\
         On observe que 8 briques unité correspondent à 40 parts de gâteaux. \\
         Une brique correspond donc à 5 parts. \\
         {\blue Jim a vendu 10 parts de gâteaux, Paul et Jane en ont vendu 15 chacun}.
      \item Modélisons la situation par un graphique en barre : \\ [1mm]
         \ModeleBarre{Turquoise 7 {"Mes économies"}}{LightSkyBlue -4 "manteau" PaleTurquoise -1 "chaussettes" PowderBlue -2 "\ueuro{9,52}"} \\
         \ueuro{9,52} correspondent à 2 briques unité. \\
         Une brique est donc égale à \ueuro{4,76} ($9,52\div2$) ; \\
         7 briques équivalent à \ueuro{33,32} ($7\times4,76$). \\
         {\blue Au départ, j'avais \ueuro{33,32}.}
   \end{enumerate}
