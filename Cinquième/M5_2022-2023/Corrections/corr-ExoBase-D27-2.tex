   \ \\ [-5mm]
   \begin{enumerate}
      \item $1+2+1+3+3+5+6+4+2+1+2+2+1 =33$. Donc, {\blue il y a 33 élèves dans cette classe}.
      \item Tableau des fréquences en \%. \\ \smallskip
      {\small
      \hautab{1.3}
      \begin{lctableau}{\linewidth}{14}
         \hline
         N & 3 & 5 & 6 & 7 & 8 & 9 & \!10 & \!11 & \!12 & \!13 & \!14 & \!17 & \!18 \\
         \hline
         E & 1 & 2 & 1 & 3 & 3 & 5 & 6 & 4 & 2 & 1 & 2 & 2 & 1 \\
         \hline
         F & 3 & 6 & 3 & 9 & 9 & \!\!15 & \!\!18 & \!\!12 & 6 & 3 & 6 & 6 & 3 \\
         \hline
     \end{lctableau}}
      \item $p =\dfrac{1+2+1+3+3}{33}\times100 =\dfrac{10}{33}\times100 \approx30,3$. \\ [1.5mm]
      Environ {\blue 30,3\,\% des élèves} ont obtenu une note inférieure ou égale à 8. \smallskip
      \item $\overline{m} =(1\times3+2\times5+1\times6+3\times7+3\times8+5\times9+6\times10+4\times11+2\times12+1\times13+2\times14+2\times17+1\times18)\div33 =330\div33 =10$. \\
      {\blue La moyenne est de 10}.
      \item La moyenne après le 4\up{e} devoir est de 9,5 donc, la somme de ses notes est de $4\times9,5 =38$. \\
         La somme des trois premiers devoirs était de $3\times9 =27$.  Or, $38-27 =11$ donc  : \\
         {\blue Bastien a obtenu 11} au dernier devoir.
  \end{enumerate}
