   \ \\ [-5mm]
   \begin{enumerate}
      \item $A =a\times a =\blue a^2$
      \item $B =b\times b\times b =\blue b^3$
      \item $C =c\times c\times3 =\blue3c^2$
      \item $D =9+d\times d\times d =\blue9+d^3$
      \item $E =a\times a\times b\times3 =\blue3a^2b$
      \item $F =x\times x\times x-2\times y\times y =\blue x^3-2y^2$
      \item $G =(a+b)\times(a+b) =\blue(a+b)^2$
      \item $H =(x+y)(x+y)(x+y) =\blue(x+y)^3$ \medskip
   \end{enumerate}

\bigskip
\corec{Défis !!!}
\medskip

\hspace*{-7.5mm} \textcolor{orange}{\bf Défi 1} \\
   \begin{enumerate}
      \item
      \begin{enumerate}
         \item {\blue $3+4 =7$}.
         \item {\blue $16+17 =33$}.
         \item On peut conjecturer que  {\blue la somme de deux nombres consécutifs est un nombre impair}.
      \end{enumerate}
      \item
      \begin{enumerate}
         \item Le nombre suivant $n$ s'écrit {\blue $n+1$}.
         \item $(n)+(n+1) =n+n+1 ={\blue 2n+1}$.
         \item $2n+1$ est un nombre pair ($2n)$ auquel on ajoute 1, c'est donc un nombre impair. \\
            {\blue La conjecture est vraie}. \bigskip
      \end{enumerate}
   \end{enumerate}

\hspace*{-7.5mm} \textcolor{orange}{\bf Défi 2} \\
   \begin{enumerate}
      \item Étape 1 : {\blue 1 carré}, étape 2 : {\blue 3 carrés}, \\
         étape 3 : {\blue 5~carrés}, étape 4 : {\blue 7 carrés}.
      \item Étape 5 : {\blue 9 carrés}, étape 6 : {\blue 11 carrés}.
      \item À l'étape 2021, il y aura 2021 carrés horizontaux et 2021 carrés à la verticale, donc 4042 carrés auxquels il faut enlever le carré de l'angle que l'on a compté deux fois, soit {\blue 4041 carrés}.
      \item Le nombre de carrés à l'étape $n$ est égal à $n+n-1 ={\blue 2n-1}$.
      \item Le nombre de carrés à l'étape 2021 vaut $2\times2021-1 ={\blue 4041}$.
   \end{enumerate}

