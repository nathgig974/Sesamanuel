\ \\ [-5mm]
   \begin{enumerate}
      \item Le ratio 5 : 11 pour Kawtar et Bekir signifie que pour 5 parts pour Kawtar, on a 11 parts pour Bekir pour un total de 16 parts correspondant à 48 macarons. \\ [2mm]
            \qquad \Ratio[Figure,Longueur=6cm,TexteTotal=48 macarons,CouleurUn=LightSkyBlue,CouleurDeux=IndianRed]{5,11} \\
         $48\div16 =3$ donc, une part vaut 3 macarons. \\
         {\blue Kawtar aura 15 macarons et Bekir en aura 33.}
      \item Le ratio 4 : 3 : 2 pour Kawtar, Bekir et Talita signifie que pour 4 parts pour Kawtar, on a 3 parts pour Bekir et Talita en a 2 pour un total de 9 parts. \\ [2mm]
            \quad \Ratio[Figure,Longueur=7cm,TexteTotal=9 parts,CouleurUn=LightSkyBlue,CouleurDeux=IndianRed,CouleurTrois=Gold]{4,3,2} \\
         Si Bekir a 9 macarons, alors la valeur d'une part est de 3 macarons.
         {\blue Kawtar a donc 12 macarons et Talita en a 6.}
      \item Le ratio 5 : 8 pour Bekir et Talita signifie que pour 5 parts pour Bekir, on a 8 parts pour Talita, soir 3 parts de plus pour Talita.\\ [2mm]
            \qquad \Ratio[Figure,Longueur=6cm,TexteTotal=3 parts de plus pour Talita,CouleurUn=LightSkyBlue,CouleurDeux=IndianRed]{5,8} \\
         Talita a fait 66 macarons de plus que Bekir, donc une part vaut 22 macarons ($66\div3$). \\
         {\blue Talita a donc préparé 176 macarons.}
   \end{enumerate}
