   On peut, par exemple, calculer les aires une par une jusqu'à obtenir $\umq{1,5} = \ummq{1500000}$. \\
   Le tableau ci-dessous récapitule la largeur et la longueur obtenue par chaque rectangle en mm ainsi que son aire en mm$^2$. \medskip

   \begin{tabular}{|c|c|c|c|}
      \hline
      Jour & Largeur & Longueur & Aire \\
      \hline
      0 & 2 & 3 & 6 \\
      \hline
      1 & 3 & 5 & 15 \\
      \hline
      2 & 5 & 8 & 40 \\
      \hline
      3 & 8 & 13 & 104 \\
      \hline
      4 & 13 & 22 & 286 \\
      \hline
      5 & 22 & 35 & 770 \\
      \hline
      6 & 35 & 57 & 1 995 \\
      \hline
      7 & 57 & 94 & 5 358 \\
      \hline
      8 & 94 & 151 & 14 194 \\
      \hline
      9 & 151 & 245 & 36 995 \\
      \hline
      10 & 245 & 396 & 97 020 \\
      \hline
      11 & 396 & 641 & 253 836 \\
      \hline
      12 & 641 & 1 037 & 664 717 \\
      \hline
      13 & 1 037 & 1 678 & 1 740 086 \\
      \hline
   \end{tabular}
   \medskip

   {\blue Il faudra 13 jours pour que l'aire du rectangle dépasse \umq{1,5}.}
