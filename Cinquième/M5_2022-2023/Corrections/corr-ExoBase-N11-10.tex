   \ \\ [-5mm]
   \begin{enumerate}
      \item Tout nombre divisible par 3 est divisible par 9 : \\
         {\blue faux}. \\
         Par exemple, 6 est divisible par 3 mais pas par 9.
      \item Tout nombre divisible par 9 est divisible par 3 : \\
         {\blue vrai}. \\
         Un nombre divisible par 9 s'écrit $9k$ où $k$ est un nombre entier. Or, $9k =3\times(3k)$ donc il est aussi divisible par 3.
      \item Tout nombre divisible par 2 et 3 est divisible par 5 : {\blue faux}. \\
      Par exemple, 6 est divisible par 2 et par 3 mais il n'est pas divisible par 5.
      \item Tout nombre dont le chiffre des unités est 2 est divisible par 2 : {\blue vrai}. \\
      Un nombre qui se termine par 0, 2, 4, 6 ou 8 est divisible par 2.
      \item Tout nombre dont le chiffre des unités est 3 est divisible par 3 : {\blue faux}. \\
      Par exemple, 13 se termine par 3 mais n'est pas divisible par 3.
   \end{enumerate}
