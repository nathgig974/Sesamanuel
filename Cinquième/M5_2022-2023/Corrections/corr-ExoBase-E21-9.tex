   \begin{enumerate}
      \item Un trapèze est un quadrilatère possédant deux côtés parallèles. Ici, ce n'est pas le cas. \\
         D'ailleurs, Sam Loyd le définit ainsi : \og forme géométrique à quatre côtés, dont aucun pris deux à deux sont parallèles.\fg
      \item Le trapézium est composé :
         \begin{itemize}
            \item d'un {\blue carré} ;
            \item d'un {\blue trapèze rectangle} ;
            \item de deux {\blue triangles rectangles} ;
            \item d'un {\blue hexagone} (non convexe).
         \end{itemize}
      \item \dots
      \item Le carré :
   \end{enumerate}

   \begin{pspicture}(-0.77,7.5)(7,16.5)
   \psset{fillstyle=solid,fillcolor=yellow!30,linewidth=0.7mm}
      \pspolygon(2,9)(0,13)(4,15)(6,11)
      \psset{fillstyle=none}
      \psgrid[subgriddiv=0,gridlabels=0,gridcolor=gray](-1,8)(7,16)
      \psline(1,11)(6,11)
      \psline(2,11)(2,13)(5,13)
      \psline(4,11)(4,15)
   \end{pspicture}

   Le parallélogramme : \\
   \begin{pspicture}(5.22,10)(14,15.5)
   \psset{fillstyle=solid,fillcolor=yellow!30,linewidth=0.7mm}
      \pspolygon(6,15)(11,15)(13,11)(8,11)
      \psset{fillstyle=none}
      \psgrid[subgriddiv=0,gridlabels=0,gridcolor=gray](5,10)(14,15)
      \psline(9,11)(9,13)(12,13)
      \psline(11,11)(11,15)
      \psline(7,13)(11,15)
   \end{pspicture}

\Coupe

   Le triangle : \\
   \begin{pspicture}(5.17,-0.5)(13,5.5)
   \psset{fillstyle=solid,fillcolor=yellow!30,linewidth=0.7mm}
      \pspolygon(3,0)(13,0)(11,4)
      \psset{fillstyle=none}
      \psgrid[subgriddiv=0,gridlabels=0,gridcolor=gray](5,0)(13,5)
      \psline(8,0)(7,2)
      \psline(9,0)(9,2)(12,2)
      \psline(11,0)(11,4)
   \end{pspicture}

   La croix : \\
   \begin{pspicture}(-0.8,0.5)(7,9.4)
   \psset{fillstyle=solid,fillcolor=yellow!30,linewidth=0.7mm}
      \pspolygon(0,4)(0,6)(2,6)(2,8)(4,8)(4,6)(6,6)(6,4)(4,4)(4,2)(2,2)(2,4)
      \psset{fillstyle=none}
      \psgrid[subgriddiv=0,gridlabels=0,gridcolor=gray](-1,1)(7,9)
      \psline(3,2)(1,6)
      \psline(2,4)(6,6)
      \psline(2,6)(4,6)
   \end{pspicture}

   Le rectangle : \\
   \begin{pspicture}(6.2,7)(14,11.5)
   \psset{fillstyle=solid,fillcolor=yellow!30,linewidth=0.7mm}
     \psframe(8,6)(13,10)
      \psset{fillstyle=none}
      \psgrid[subgriddiv=0,gridlabels=0,gridcolor=gray](7,5)(14,11)
      \psline(10,6)(8,10)
      \psline(8,8)(9,8)(13,10)
      \psline(11,6)(11,8)(13,8)
   \end{pspicture}
