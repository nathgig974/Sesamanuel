   Un carreau représente un carré de \ucm{1} de côté. \\
   \begin{enumerate}
      \item Patron du pavé de mesures \ucm{5}, \ucm{4} et \ucm{2}. \\
         {\psset{unit=0.5}
         \begin{pspicture}(0,0)(14,11.5)
            \psset{linecolor=blue}
            \psgrid[subgriddiv=1,gridlabels=0pt,gridcolor=lightgray](0,0)(14,11)
            \psline(3,1)(3,3)(1,3)(1,8)(3,8)(3,10)(7,10)(7,8)(13,8)(13,3)(7,3)(7,1)(3,1)
            \psframe(3,3)(7,8)
            \psline(9,3)(9,8)
         \end{pspicture}}
   \end{enumerate}

\Coupe

   \begin{enumerate}
      \setcounter{enumi}{1}
      \item Développement du cylindre de hauteur \ucm{8} de diamètre \ucm{2} : le périmètre du disque vaut \\ [1mm]
         $2\times\pi\times\dfrac{\ucm{2}}{2} \approx\ucm{6,28}$. \\
         {\psset{unit=0.5}
         \begin{pspicture}(-1,-1.5)(13,7.5)
            \psgrid[subgriddiv=1,gridlabels=0pt,gridcolor=lightgray](-1,-1)(13,7)
            \psset{linecolor=blue}
            \psframe(2,0)(10,6.28)
            \pscircle(1,3){1}
            \pscircle(11,3){1}
            \psdots(1,3)(11,3)
         \end{pspicture}}
      \item Développement du cylindre de hauteur \ucm{2} de rayon \ucm{2,5} : le périmètre du disque vaut \\
         $2\times\pi\times\ucm{2,5} \approx\ucm{15,71}$. \\
        {\psset{unit=0.41}
        \begin{pspicture}(-1,-1)(16,14)
           \psgrid[subgriddiv=1,gridlabels=0pt,gridcolor=lightgray](-1,-1)(16,13)
           \psset{linecolor=blue}
           \psframe(0,5)(15.7,7)
           \pscircle(5,2.5){2.5}
           \pscircle(8,9.5){2.5}
           \psdots(5,2.5)(8,9.5)
         \end{pspicture}}
   \end{enumerate}
