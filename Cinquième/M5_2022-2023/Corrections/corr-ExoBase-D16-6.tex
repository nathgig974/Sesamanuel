\ \\ [-5mm]
   \begin{enumerate}
      \item 1\up{er} verre : $\dfrac{3,5}{100}\times\ucl{20} =\ucl{0,7}$. \\ [1mm]
         {\blue Le premier verre contient \ucl{0,7} de sirop}. \smallskip
      \item 2\up{e} verre : $\dfrac{5}{100}\times\ucl{10} =\ucl{0,5}$. \\ [1mm]
         {\blue Le deuxième verre contient \ucl{0,7} de sirop}.
      \item Dans les verres, il y a $\ucl{0,7}+\ucl{0,5} =\ucl{1,2}$ de sirop. \\
         Au départ, Selene avait $\uml{15} = \ucl{1,5}$, {\blue il lui restera donc \ucl{0,3} de sirop}.
   \end{enumerate}
