\themaL
\vspace*{-1cm}
{\Huge\textsf Sommaire}

%bandeau nombres et calculs
\begin{pspicture}(0,-0.25)(\linewidth,\dimexpr\SquareWidth*3+1)
    \psframe*[linewidth=0pt,linecolor=Crimson](0,0)(\linewidth,\dimexpr\SquareWidth*3)
    \rput(8.4,0.7){\textcolor{white}{\Large\textsf{NOMBRES ET CALCULS}}}
  \end{pspicture}

   \begin{multicols}{2}
      S01 Enchaînement d'opérations \pointilles \pageref{S01} \\
      S04 Nombres relatifs \pointilles \pageref{S04} \\
      S08 Expressions algébriques \pointilles \pageref{S08} \\
      S11 Multiples et diviseurs \pointilles \pageref{S11} \\
      S14 Comparaison et égalité de fractions \pointilles \pageref{S14} \\
      S17 La distributivité simple \pointilles \pageref{S17} \\
      S20 Somme et différence de nombres relatifs \pointilles \pageref{S20} \\
      S23 Nombres premiers \pointilles \pageref{S23} \\
      S26 Somme et différence de fractions \pointilles \pageref{S26} \\
   \end{multicols}
   
%bandeau géométrie
\begin{pspicture}(0,-0.25)(\linewidth,\dimexpr\SquareWidth*3+1)
    \psframe*[linewidth=0pt,linecolor=DodgerBlue](0,0)(\linewidth,\dimexpr\SquareWidth*3)
    \rput(8.5,0.7){\textcolor{white}{\Large\textsf{GÉOMÉTRIE}}}
  \end{pspicture}
    
   \begin{multicols}{2}
      S02 Angles particuliers \pointilles \pageref{S02} \\
      S05 Repérage dans le plan \pointilles \pageref{S05} \\
      S09 Somme des angles d'un triangle \pointilles \pageref{S09} \\
      S12 La symétrie centrale \pointilles \pageref{S12} \\
      S15 L'inégalité triangulaire \pointilles \pageref{S15} \\
      S18 Reconnaître des solides \pointilles \pageref{S18} \\      
      S21 Le parallélogramme \pointilles \pageref{S21} \\
      S24 Représenter des solides \pointilles \pageref{S24} \\
      S28 Les droites du triangle \pointilles \pageref{S27}   
   \end{multicols}
   
%bandeau organisation et gestion de données
\begin{pspicture}(0,-0.25)(\linewidth,\dimexpr\SquareWidth*3+1)
    \psframe*[linewidth=0pt,linecolor=violet](0,0)(\linewidth,\dimexpr\SquareWidth*3)
    \rput(8.4,0.7){\textcolor{white}{\Large\textsf{ORGANISATION ET GESTION DE DONNÉES}}}
\end{pspicture} 

   \begin{multicols}{2}
      S06 Interpréter et représenter des données \pointilles \pageref{S06} \\
      S10 Notions de probabilités \pointilles \pageref{S10} \\
      S16 Proportionnalité \pointilles \pageref{S16} \\
      S22 Le ratio \pointilles \pageref{S22} \\
      S27 Fréquence et moyenne \pointilles \pageref{S28}
   \end{multicols}

%bandeau grandeurs et mesures
\begin{pspicture}(0,-0.25)(\linewidth,\dimexpr\SquareWidth*3+1)
    \psframe*[linewidth=0pt,linecolor=Green](0,0)(\linewidth,\dimexpr\SquareWidth*3)
    \rput(8.5,0.7){\textcolor{white}{\Large\textsf{GRANDEURS ET MESURES}}}
  \end{pspicture}
    
   \begin{multicols}{2}
      S07 Horaires et durées \pointilles \pageref{S07} \\
      S13 Calcul d'aires \pointilles \pageref{S13} \\
      S19 Volume du prisme et du cylindre \pointilles \pageref{S19} \\
      S25 L'aire du parallélogramme \pointilles \pageref{S25} \\
      S29 Propriétés des symétries \pointilles \pageref{S30}
   \end{multicols}
   
%bandeau programmation
\begin{pspicture}(0,-0.25)(\linewidth,\dimexpr\SquareWidth*3+1)
    \psframe*[linewidth=0pt,linecolor=orange](0,0)(\linewidth,\dimexpr\SquareWidth*3)
    \rput(8.5,0.7){\textcolor{white}{\Large\textsf{ALGORITHMIQUE ET PROGRAMMATION}}}
  \end{pspicture}
    
   \begin{multicols}{2}
      S03 En route vers la programmation \pointilles \pageref{S03} \\
   \end{multicols}

   