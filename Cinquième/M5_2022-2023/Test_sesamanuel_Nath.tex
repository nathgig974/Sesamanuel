%\listfiles

\PassOptionsToPackage{table}{xcolor}
\PassOptionsToPackage{svgnames}{xcolor}

\documentclass[nocrop]{sesamanuel_college_5e_new}
\renewcommand\PrefixeCorrection{Corrections/}
\usepackage{etex}
\usepackage{ProfCollege}
%\usepackage{pstcol,pst-grad}%

\NewThema{N}{n}{nombres\\\& calculs}{Nombres\\\& calculs}{NOMBRES\\\& CALCULS}{red}{red!50}

%\NewThema{E}{e}{espace \&\\géométrie}{Espace \&\\géométrie}{ESPACE \&\\ GEOMETRIE}{DodgerBlue}{DodgerBlue!50}
%
%\NewThema{D}{d}{organisation et \\et gestion de données}{Organisation et \\et gestion de données}{ORGANISATION ET\\GESTION DE DONNÉES}{violet}{violet!50}
%
%\NewThema{M}{m}{grandeurs\\et mesures}{Grandeurs\\et mesures}{GRANDEURS\\ET MESURES}{Green}{Green!50}
%
%\NewThema{A}{a}{algorithmique \&\\programmation}{Algorithmique \&\\ programmation}{ALGORITHMIQUE \&\\PROGRAMMATION}{orange}{orange!50}

\renewcommand\ListeMethodesThemes{{n}{N},{d}{D},{g}{G},{m}{M},{a}{A}}
 
 
\begin{document}

\frontmatter
\pagenumbering{gobble}
%% Couleurs blue, brown, cyan, green, lime, magenta, olive, orange, pink, purple, red, teal, violet, white, black, gray, yellow
\tikzset{fondA/.style=DarkCyan}
\tikzset{fondB/.style=Lime}
\tikzset{fondPostit/.style={color= yellow!40}}
\tikzset{ombrePunaise/.style={color={blue!10!gray}}}
\tikzset{ombrePostit/.style={color={black},opacity=.5}}
\tikzset{punaise/.style={ball color=red}}

% \epingle{point}{angle}{échelle}
\newcommand{\epingle}[3]{
\coordinate[rotate=#2,yshift={#3*0.375cm}] (e) at #1;
\coordinate[shift={++(60:0.75)}] (g) at (e);
\begin{scope} [scale=1.5]
 \begin{scope}[rotate=-30]
   \coordinate[shift={++(30:0.75)}] (h) at (e);
   \draw[ombrePunaise,line cap=round,line width=4pt] (e) -- ++(60:0.75);
   \fill [ombrePunaise,rotate=-30,scale=0.5] (h) ellipse (.65 and .3) ;
   \fill [ombrePunaise,rotate=60,scale=0.5] (h) ++(0.4,0) ellipse (.4 and .3);
   \fill [ombrePunaise,rotate=60,scale=0.5] (h) ++(0.8,0) ellipse (.2 and .4);
 \end{scope}
 \draw[line cap=round,line width=4pt] (e) -- ++(60:0.75);
 \fill [punaise,rotate=-30,scale=0.5] (g) ellipse (.65cm and .3cm) ;
 \fill [punaise,rotate=60,scale=0.5] (g) ++(0.4,0) ellipse (.4 and .3);
 \fill [punaise,rotate=60,scale=0.5] (g) ++(0.8,0) ellipse (.2 and .4);
\end{scope}}

% \postIt{(point)}{angle}{échelle}{ligne 1}{ligne 2}{ligne 3}}
\newcommand{\postIt}[6]
{\begin{scope} [rotate=#2,scale=0.8]
\fill [red,ombrePostit] #1 ++ ($#3*(-1.45,0.72)$) -- ++ ($#3*(2.86,0)$) 
 .. controls+(0,0)and+($#3*(-0.25,0.05)$).. ++ ($#3*(0.25,-2.4)$)
 .. controls+($#3*(-0.1,-0.1)$)and+(0,0).. ++ ($#3*(-2.95,0.1)$)
 -- cycle;
\fill [ombrePostit] #1 ++ ($#3*(-1.45,0.72)$) -- ++ ($#3*(2.86,0)$) 
 .. controls+(0,0)and+($#3*(-0.25,0.05)$).. ++ ($#3*(0.2,-2.35)$)
 .. controls+($#3*(-0.1,-0.1)$)and+(0,0).. ++ ($#3*(-2.95,0.1)$)
 -- cycle;
\fill [ombrePostit] #1 ++ ($#3*(-1.45,0.72)$) -- ++ ($#3*(2.86,0)$) 
 .. controls+(0,0)and+($#3*(-0.25,0.05)$).. ++ ($#3*(0.15,-2.3)$)
 .. controls+($#3*(-0.1,-0.1)$)and+(0,0).. ++ ($#3*(-2.95,0.1)$)
 -- cycle;
\fill [fondPostit] #1 ++ ($#3*(-1.45,0.72)$) -- ++ ($#3*(2.86,0)$) 
 .. controls+(0,0)and+($#3*(-0.2,0.1)$).. ++($#3*(0.1,-2.25)$)
 .. controls+($#3*(-0.1,-0.1)$)and+(0,0).. ++ ($#3*(-2.95,0.1)$)
 -- cycle;
\end{scope}
\epingle{#1}{#2}{#3}
\draw #1 node [scale=#3,rotate=#2] {#4};
\draw #1 node [scale=#3,rotate=#2,below={#3*0.2cm}] {#5};
\draw #1 node [scale=#3,rotate=#2,below={#3*0.6cm}] {#6};}


%=======================================
\begin{tikzpicture}[remember picture,overlay]
% fond bicolore
\coordinate (cp) at (current page);
\coordinate (cpc) at (current page.center);
\coordinate (cpe) at ($ (current page.east) + (0.75cm,0cm) $);
\coordinate (cpne) at ($ (current page.north east) + (0.75cm,1.4cm) $);
\coordinate (cpn) at ($ (current page.north) + (1cm,1.4cm) $);
\coordinate (cpnw) at ($ (current page.north west) + (-0.75cm,1.4cm) $);
\coordinate (cpw) at ($ (current page.west) + (-0.75cm,0cm) $);
\coordinate (cpsw) at ($ (current page.south west) + (-0.75cm,-1.4cm) $);
\coordinate (cps) at ($ (current page.south) + (0.75cm,-1.4cm) $);
\coordinate (cpse) at ($ (current page.south east) + (0.75cm,-1.4cm) $);
\fill[fondA] (cps) .. controls (cpw) and (cpe) .. (cpn) -- (cpnw)  -- (cpsw) -- cycle;
\fill[fondB] (cps) .. controls (cpw) and (cpe) .. (cpn) -- (cpne)  -- (cpse) -- cycle;


% Titre
\draw (cp)  node [xshift=-2cm,yshift=9cm,scale=8] {\textcolor{Lime}{Manuel}};
\draw (cp)  node [xshift=-2cm,yshift=6.5cm,scale=3.5] {\textcolor{Lime}{de Mathématiques}};
\draw (cp)  node [xshift=-2cm,yshift=4cm,scale=8] {\textcolor{Lime}{5\up{e}}};
\draw (cp)  node [xshift=-6cm,yshift=-1.5cm]{\psHomothetie[linecolor=Lime](0,0){1.5}{\psBill}};

\draw (cp)  node [xshift=5.5cm,yshift=0cm]{\psKangaroo[linecolor=Lime,fillcolor=blue]{3}};
\draw (cp)  node [xshift=6.9cm,yshift=0cm]{\psKangaroo[linecolor=Lime,fillcolor=orange]{3}};
\draw (cp)  node [xshift=7.63cm,yshift=-1.5cm,xscale=-1]{\psKangaroo[linecolor=Lime,fillcolor=violet]{3}};
\draw (cp)  node [xshift=6.23cm,yshift=-1.5cm,xscale=-1]{\psKangaroo[linecolor=Lime,fillcolor=red]{3}};
\draw (cp)  node [xshift=4.82cm,yshift=-1.5cm,xscale=-1]
{\psKangaroo[linecolor=Lime,fillcolor=gray]{3}};
\draw (cp)  node [xshift=5cm,yshift=-5cm,scale=3] {\textcolor{DarkCyan}{\cursive Montpellier}};
\draw (cp)  node [xshift=5cm,yshift=-7cm,scale=3] {\textcolor{DarkCyan}{\cursive Collège Simone Veil}};

% post'it
\coordinate[xshift=-5.5cm,yshift=-9.5cm] (postit) at (cp);
\postIt{(postit)}{-17}{1.75} {\footnotesize\cursive Nathalie Daval}{\footnotesize année}{\footnotesize 2022-2023}
\end{tikzpicture}


\mainmatter

\themaN

\chapter{Enchaînement d'opérations}
\label{S01}

%%%%%%%%%%%%%%%%%%%%%%%%%%%%%%%%%%%%%%
\begin{autoeval}
   \small
   \begin{enumerate}
      \item Il utilise, dans le cas des nombres décimaux, les écritures décimales et fractionnaires et passe de l’une à l’autre, en particulier dans le cadre de la résolution de problèmes.
   \end{enumerate}
\end{autoeval}


\begin{prerequis}
   \small
   \begin{itemize}
      \item Connaissance
      \item[\com] Compétence
   \end{itemize}
\end{prerequis}

\vfill

\begin{debat}[Débat : un peu d'histoire]
   Texte\medskip
   \begin{center}
      Image
   \end{center} \medskip
   \begin{cadre}[B2][J4]
      \begin{center}
         Vidéo : \href{https://www.youtube.com/watch?v=bkGMa1EJkSA}{\bf Histoire de la virgule}, chaîne Youtube de {\it Maths 28}.
      \end{center}
   \end{cadre}
\end{debat}


%%%%%%%%%%%%%%%%%%%%%%%%%%%%%%
%%%%%%%%%%%%%%%%%%%%%%%%%%%%%%
\activites

\begin{activite}[Nom de l'activité]
   {\bf Objectifs :} \dots
   \begin{QCM}
      \partie[Partie 1 de l'activité]
      \vspace*{18cm}
   \end{QCM}
\end{activite}


%%%%%%%%%%%%%%%%%%%%%%%%%%%%%%%%%%%%
%%%%%%%%%%%%%%%%%%%%%%%%%%%%%%%%%%%%
\cours 


%%%%%%%%%%%%%%%%%
\section{Section 1}

\begin{definition}
   Définition
\end{definition}

\begin{propriete}
   Propriété
\end{propriete}

\begin{exemple*1}
   Exemple 1
\end{exemple*1}

\begin{methode}[Nom de la méthode]
   Dans un calcul, on effectue dans l'ordre :
   \begin{itemize}
      \item les calculs entre parenthèses, en commençant par les plus intérieures ;
      \item les multiplications et les divisions ;
      \item les additions et soustractions.
   \end{itemize}
   \exercice
      Exemple
   \correction
      Sa correction
\end{methode}

\begin{remarque}
   Une remarque
\end{remarque}


%%%%%%%%%%%%%%%%%%%%%%%%%%%%%%%%%%%%
%%%%%%%%%%%%%%%%%%%%%%%%%%%%%%%%%%%%
\exercicesbase

\begin{colonne*exercice}

\begin{exercice} %1
   Exercice 1
\end{exercice}

\begin{corrige}
   Corrigé de l'exo 1
\end{corrige}

\bigskip


\begin{exercice}[éventuel nom] %2
   Exercice 2
\end{exercice}

\begin{corrige}
   Corrigé de l'exo 2
\end{corrige}


\end{colonne*exercice}


%%%%%%%%%%%%%%%%%%%%%%%%%%%%%%%%%%
%%%%%%%%%%%%%%%%%%%%%%%%%%%%%%%%%%
\Recreation

\begin{enigme}[Nom de l'énigme]

   \partie[nom de la partie]

\end{enigme}


\begin{corrige}
   Corrigé de l'énigme
\end{corrige}  


%\annexe{Plans de travail}
\label{PDT}

%macros pour les bulles

\newcommand{\bulle}[3]{
   \psframe[linewidth=1mm,framearc=0.1,linecolor=#1](0,0)(5,3)
   \psline[linecolor=#1](0,2.3)(5,2.3)
   \rput(2.5,2.65){#2}
   \rput(2.5,1.15){\begin{minipage}{4cm}
                           {\small\darkgray #3}
                         \end{minipage}}}
                        
\newcommand{\bullelongue}[3]{
   \psframe[linewidth=1mm,framearc=0.1,linecolor=#1](0,0)(8.5,3)
   \psline[linecolor=#1](0,2.3)(8.5,2.3)
   \rput(4.25,2.65){#2}
   \rput(4.25,1.15){\begin{minipage}{8cm}
                           {\small\darkgray #3}
                         \end{minipage}}}
   
\newcommand{\bullecours}[3]{
   \psframe[linewidth=1mm,framearc=0.1,linecolor=#1](0,0)(12.5,3)
   \psline[linecolor=#1](0,2.3)(12.5,2.3)
   \rput(6.25,2.65){#2}
   \rput(6.25,1.15){\begin{minipage}{12cm}
                        {\small\darkgray #3}
                      \end{minipage}}}
   
\newcommand{\bulleQR}[3]{
   \psframe[linewidth=1mm,framearc=0.1,linecolor=#1](0,0)(4,7)
   \psline[linecolor=#1](0,6.3)(4,6.3)
   \rput(2,6.65){#2}
   \rput(2,3.15){\begin{minipage}{3.5cm}
                           \begin{center}
                              {\small\darkgray #3}
                           \end{center}
                         \end{minipage}}}
 
\begin{center}


%%%%%%%%%% Séquence 1 %%%%%%%%%%
%%%%%%%%%%%%%%%%%%%%%%%%%%%
\begin{pspicture}(0.5,0)(18,11)            
   {\color{red}
      \rput(9,5.75){\parbox{5cm}{\centering\large S1 \par  ENCHAÎNEMENT D'OPĖRATIONS}}} %bulle centrale  
   \rput[l](0,8){%bulle NNO : connaissances et compétences
      \pspolygon[fillstyle=solid,fillcolor=A1,linecolor=A1](6,0)(8,-1.5)(6.4,0)
      \bullecours
         {A1}
         {Je connais mon cours}
         {C1 : J'utilise et je range différentes représentations d'un même nombre décimal \hfill $\square$ \par
          C2 : Je traduis un enchaînement d’opérations à l’aide d’une expression et inversement \hfill $\square$ \par
          C3 : J'effectue un enchaînement d'opérations en respectant les priorités opératoires \hfill $\square$}}         
   \rput[l](14,4){%bulle ENE : Aide vidéo
      \pspolygon[fillstyle=solid,fillcolor=A1,linecolor=A1](0,3.2)(-2.5,2)(0,3.5)
      \bulleQR
         {A1}
         {Aide en vidéo}
         {\qrcode{https://www.youtube.com/watch?v=TJH-fiwAt5s} \par \medskip
          Calculer avec des priorités \par \bigskip
          \qrcode{https://www.youtube.com/watch?v=_yF5ItbcN28&feature=youtu.be} \par \medskip
          Traduire une expression}}    
      \rput[l](0,4){%bulle O : Questions flash
         \pspolygon[fillstyle=solid,fillcolor=Goldenrod,linecolor=Goldenrod](5,1.35)(6.5,1.5)(5,1.65)
         \bulle
            {Goldenrod}
            {Questions flash}
            {\psline[linecolor=darkgray](1.75,-0.5)(2.25,0.5)
             \rput(2.75,0){\darkgray\Huge 5}}}     
      \rput[l](0,0){%bulle SO : Compétence 1
         \pspolygon[fillstyle=solid,fillcolor=B1,linecolor=B1](5,2)(7.2,4.5)(5,2.35)
         \bulle
            {B1}
            {Compétence 1}
            {Activité d'approche \hfill $\star$ \hfill $\square$ \par
             Exercice 1 \hfill $\star$ \hfill $\square$ \par
             Exercice 2 \hfill $\star\star$ \hfill $\square$ \par
             Exercice 3 \hfill $\star\star\star$ \hfill $\square$}}
      \rput[l](6.5,0){%bulle S : compétence 2
         \pspolygon[fillstyle=solid,fillcolor=B1,linecolor=B1](2.35,3)(2.5,4.5)(2.65,3)
         \bulle
            {B1}
            {Compétence 2}
            {Exercice 4 \hfill $\star$ \hfill $\square$ \par
             Exercice 5 \hfill $\star\star$ \hfill $\square$ \par
             Exercice 6 \hfill $\star\star$ \hfill $\square$}}             
      \rput[l](13,0){%bulle SE : compétence 3
          \pspolygon[fillstyle=solid,fillcolor=B1,linecolor=B1](0,2)(-2.3,4.5)(0,2.35)
          \bulle
            {B1}
            {Compétence 3}
            {Exercice 7 \hfill $\star$ \hfill $\square$ \par
             Exercice 8 \hfill $\star$ \hfill $\square$ \par
             Exercice 9 \hfill $\star\star$ \hfill $\square$ \par
             Exercice 10 \hfill $\star\star\star$ \hfill $\square$ \par
             Récréation \hfill $\star\star\star$ \hfill $\square$}}                  
\end{pspicture}
   

%%%%%%%%%% Séquence 2 %%%%%%%%%%
%%%%%%%%%%%%%%%%%%%%%%%%%%%
\begin{pspicture}(0.5,0.5)(18,12.25)            
   {\color{DodgerBlue}
      \rput(9,5.75){\parbox{5cm}{\centering\large S2 \par ANGLES \par PARTICULIERS}}} %bulle centrale  
   \rput[l](0,8){%bulle NNO : connaissances et compétences
      \pspolygon[fillstyle=solid,fillcolor=A1,linecolor=A1](6,0)(8,-1.5)(6.4,0)
      \bullecours
         {A1}
         {Je connais mon cours}
         {C1 : Je reconnais des angles alternes-internes \hfill $\square$ \par
          C2 : Je reconnais des angles correspondants \hfill $\square$ \par
          C3 : J'utilise les propriétés des angles alternes-internes et correspondants pour montrer que des droites sont parallèles ou pour déterminer des angles \hfill $\square$}}         
   \rput[l](14,4){%bulle ENE : Aide vidéo
      \pspolygon[fillstyle=solid,fillcolor=A1,linecolor=A1](0,3.2)(-2.5,2)(0,3.5)
      \bulleQR
         {A1}
         {Aide en vidéo}
         {\qrcode{https://www.youtube.com/watch?v=c8CuPY-KaNM&list=PLVUDmbpupCaoTCiYBCUGfCyenktNbkIdt} \par \medskip
          Angles alternes-internes \par \bigskip
          \qrcode{https://www.youtube.com/watch?v=ErUq2wdA_PE&list=PLVUDmbpupCaoTCiYBCUGfCyenktNbkIdt&index=2} \par \medskip
          Angles correspondants}}    
      \rput[l](0,4){%bulle O : Questions flash
         \pspolygon[fillstyle=solid,fillcolor=Goldenrod,linecolor=Goldenrod](5,1.35)(6.5,1.5)(5,1.65)
         \bulle
            {Goldenrod}
            {Questions flash}
            {\psline[linecolor=darkgray](1.75,-0.5)(2.25,0.5)
             \rput(2.75,0){\darkgray\Huge 5}}}     
      \rput[l](0,0){%bulle SSO : Compétence 1
         \pspolygon[fillstyle=solid,fillcolor=B1,linecolor=B1](5,3)(8,4.5)(5.5,3)
         \bullelongue
            {B1}
            {Compétences 1 et 2}
            {Activité d'approche \hfill $\star$ \hfill $\square$ \par
             Exercice 1 \hfill $\star$ \hfill $\square$ \par
             Exercice 2 \hfill $\star$ \hfill $\square$ \par
             Exercice 3 \hfill $\star\star\star$ \hfill $\square$}}
      \rput[l](10,0){%bulle S : compétence 2
         \pspolygon[fillstyle=solid,fillcolor=B1,linecolor=B1](3,3)(0.5,4.5)(3.5,3)
         \bullelongue
            {B1}
            {Compétence 3}
            {Exercice 4 \hfill $\star$ \hfill $\square$ \par
             Exercice 5 \hfill $\star\star$ \hfill $\square$ \par
             Exercice 6 \hfill $\star\star$ \hfill $\square$ \par
             Exercice 7 \hfill $\star\star\star$ \hfill $\square$}}                             
\end{pspicture}


%%%%%%%%%% Séquence 3 %%%%%%%%%%
%%%%%%%%%%%%%%%%%%%%%%%%%%%
\begin{pspicture}(0.5,0)(18,10)            
   {\color{orange}
      \rput(9,5.75){\parbox{5cm}{\centering\large S3 \par EN ROUTE VERS LA PROGRAMMATION}}} %bulle centrale  
   \rput[l](0,8){%bulle NNO : connaissances et compétences
      \pspolygon[fillstyle=solid,fillcolor=A1,linecolor=A1](6,0)(8,-1.5)(6.4,0)
      \bullecours
         {A1}
         {Je connais mon cours}
         {C1 : Je réalise des activités d’algorithmique débranchée \hfill $\square$ \par
          C2 : Je traduis un script de déplacement ou de construction géométrique \hfill $\square$ \par
          C3 : J'écrit un script de déplacement ou de construction géométrique \hfill $\square$}}         
   \rput[l](14,4){%bulle ENE : Aide vidéo
      \pspolygon[fillstyle=solid,fillcolor=A1,linecolor=A1](0,3.2)(-2.5,2)(0,3.5)
      \bulleQR
         {A1}
         {Aide en vidéo}
         {\qrcode{https://www.youtube.com/watch?v=pdtMUgnmRa4&list=PLVUDmbpupCaqKLNci7_86rbIt61SMhJPd&index=1} \par \medskip
          Prise en main de Scratch \par \bigskip
          \qrcode{https://www.youtube.com/watch?v=8Sfarvw6jgg&list=PLVUDmbpupCaqKLNci7_86rbIt61SMhJPd&index=2} \par \medskip
          Utiliser une boucle}}    
      \rput[l](0,4){%bulle O : Questions flash
         \pspolygon[fillstyle=solid,fillcolor=Goldenrod,linecolor=Goldenrod](5,1.35)(6.5,1.5)(5,1.65)
         \bulle
            {Goldenrod}
            {Questions flash}
            {\psline[linecolor=darkgray](1.75,-0.5)(2.25,0.5)
             \rput(2.75,0){\darkgray\Huge 5}}}    
      \rput[l](0,0){%bulle SO : Compétence 1
         \pspolygon[fillstyle=solid,fillcolor=B1,linecolor=B1](5,2)(7.2,4.5)(5,2.35)
         \bulle
            {B1}
            {Compétence 1}
            {Activité d'approche \hfill $\star\star\star$ \hfill $\square$ \par
             Récréation \hfill $\star\star\star$ \hfill $\square$}}
      \rput[l](6.5,0){%bulle S : compétence 2
         \pspolygon[fillstyle=solid,fillcolor=B1,linecolor=B1](2.35,3)(2.5,4.5)(2.65,3)
         \bulle
            {B1}
            {Compétence 2}
            {Exercice 1 \hfill $\star$ \hfill $\square$ \par
             Exercice 2 \hfill $\star\star$ \hfill $\square$}}           
      \rput[l](13,0){%bulle SE : compétence 3
          \pspolygon[fillstyle=solid,fillcolor=B1,linecolor=B1](0,2)(-2.3,4.5)(0,2.35)
          \bulle
            {B1}
            {Compétence 3}
            {Exercice 1 \hfill $\star$ \hfill $\square$ \par
             Exercice 3 \hfill $\star$ \hfill $\square$ \par
             Exercice 4 \hfill $\star\star$ \hfill $\square$}}               
\end{pspicture}


%%%%%%%%%% Séquence 4 %%%%%%%%%%
%%%%%%%%%%%%%%%%%%%%%%%%%%%
\begin{pspicture}(0.5,0.5)(18,12.5)            
   {\color{red}
      \rput(9,5.75){\parbox{5cm}{\centering\large S4 \par NOMBRES \par RELATIFS}}} %bulle centrale  
   \rput[l](0,8){%bulle NNO : connaissances et compétences
      \pspolygon[fillstyle=solid,fillcolor=A1,linecolor=A1](6,0)(8,-1.5)(6.4,0)
      \bullecours
         {A1}
         {Je connais mon cours}
         {C1 : Je connais la notion de nombre relatif et d'opposé \hfill $\square$ \par
          C2 : Je repère et je place sur un axe gradué des nombres relatifs \hfill $\square$ \par
          C3 : Je compare, range, encadre des nombres relatifs \hfill $\square$}}         
   \rput[l](14,4){%bulle ENE : Aide vidéo
      \pspolygon[fillstyle=solid,fillcolor=A1,linecolor=A1](0,3.2)(-2.5,2)(0,3.5)
      \bulleQR
         {A1}
         {Aide en vidéo}
         {\qrcode{https://www.youtube.com/watch?v=SImiMoRB0vU} \par \medskip
          Droite graduée \par \bigskip
          \qrcode{https://www.youtube.com/watch?v=DYbRr4B42h8} \par \medskip
          Comparer des nombres}}    
      \rput[l](0,4){%bulle O : Questions flash
         \pspolygon[fillstyle=solid,fillcolor=Goldenrod,linecolor=Goldenrod](5,1.35)(6.5,1.5)(5,1.65)
         \bulle
            {Goldenrod}
            {Questions flash}
            {\psline[linecolor=darkgray](1.75,-0.5)(2.25,0.5)
             \rput(2.75,0){\darkgray\Huge 5}}}    
      \rput[l](0,0){%bulle SO : Compétence 1
         \pspolygon[fillstyle=solid,fillcolor=B1,linecolor=B1](5,2)(7.2,4.5)(5,2.35)
         \bulle
            {B1}
            {Compétence 1}
            {Activité d'approche \hfill $\star\star$ \hfill $\square$ \par
             Exercice 3 \hfill $\star$ \hfill $\square$ \par
             Exercice 4 \hfill $\star$ \hfill $\square$ \par
             Récréation \hfill $\star\star\star$ \hfill $\square$}}
      \rput[l](6.5,0){%bulle S : compétence 2
         \pspolygon[fillstyle=solid,fillcolor=B1,linecolor=B1](2.35,3)(2.5,4.5)(2.65,3)
         \bulle
            {B1}
            {Compétence 2}
            {Exercice 1 \hfill $\star\star$ \hfill $\square$ \par
             Exercice 2 \hfill $\star\star$ \hfill $\square$ \par
             Exercice 5 \hfill $\star$ \hfill $\square$ \par
             Exercice 6 \hfill $\star$ \hfill $\square$}}           
      \rput[l](13,0){%bulle SE : compétence 3
          \pspolygon[fillstyle=solid,fillcolor=B1,linecolor=B1](0,2)(-2.3,4.5)(0,2.35)
          \bulle
            {B1}
            {Compétence 3}
            {Exercice 7 \hfill $\star$ \hfill $\square$ \par
             Exercice 8 \hfill $\star\star$ \hfill $\square$ \par
             Exercice 9 \hfill $\star\star\star$ \hfill $\square$ \par
             Exercice 10 \hfill $\star\star$ \hfill $\square$ }}               
\end{pspicture}


%%%%%%%%%% Séquence 5 %%%%%%%%%%
%%%%%%%%%%%%%%%%%%%%%%%%%%%
\begin{pspicture}(0.5,0)(18,10)            
   {\color{DodgerBlue}
      \rput(9,5.75){\parbox{5cm}{\centering\large S5 \par Repérage \par dans le plan}}} %bulle centrale  
   \rput[l](0,8){%bulle NNO : connaissances et compétences
      \pspolygon[fillstyle=solid,fillcolor=A1,linecolor=A1](6,0)(8,-1.5)(6.4,0)
      \bullecours
         {A1}
         {Je connais mon cours}
         {C1 : Je sais lire les coordonnées dans un repère du plan \hfill $\square$ \par
          C2 : Je sais placer des points dans un repère du plan \hfill $\square$}}         
   \rput[l](14,4){%bulle ENE : Aide vidéo
      \pspolygon[fillstyle=solid,fillcolor=A1,linecolor=A1](0,3.2)(-2.5,2)(0,3.5)
      \bulleQR
         {A1}
         {Aide en vidéo}
         {\qrcode{https://www.youtube.com/watch?v=AHNYuKCoCvU&t=280s} \par \medskip
          Repère du plan \par \bigskip
          \qrcode{https://www.youtube.com/watch?v=kAtwKV3DqKI&list=PLVUDmbpupCap2d3CpwVNB_SrkYWEFsmBX&index=7} \par \medskip
          Lire et placer des points}}    
      \rput[l](0,4){%bulle O : Questions flash
         \pspolygon[fillstyle=solid,fillcolor=Goldenrod,linecolor=Goldenrod](5,1.35)(6.5,1.5)(5,1.65)
         \bulle
            {Goldenrod}
            {Questions flash}
            {\psline[linecolor=darkgray](1.75,-0.5)(2.25,0.5)
             \rput(2.75,0){\darkgray\Huge 5}}}    
      \rput[l](0,0){%bulle SSO : Compétence 1
         \pspolygon[fillstyle=solid,fillcolor=B1,linecolor=B1](5,3)(8,4.5)(5.5,3)
         \bullelongue
            {B1}
            {Compétence 1}
            {Exercice 1 \hfill $\star$ \hfill $\square$ \par
             Exercice 4 \hfill $\star\star$ \hfill $\square$}}
      \rput[l](10,0){%bulle S : compétence 2
         \pspolygon[fillstyle=solid,fillcolor=B1,linecolor=B1](3,3)(0.5,4.5)(3.5,3)
         \bullelongue
            {B1}
            {Compétence 2}
            {Activité d'approche \hfill $\star\star\star$ \hfill $\square$ \par
             Exercice 2 \hfill $\star\star$ \hfill $\square$ \par
             Exercice 3 \hfill $\star\star$ \hfill $\square$ \par
             Récréation \hfill $\star\star$ \hfill $\square$}}             
\end{pspicture}


%%%%%%%%%% Séquence 6 %%%%%%%%%%
%%%%%%%%%%%%%%%%%%%%%%%%%%%
\begin{pspicture}(0.5,0.5)(18,12.5)            
   {\color{red}
      \rput(9,5.75){\parbox{5cm}{\centering\large S4 \par NOMBRES \par RELATIFS}}} %bulle centrale  
   \rput[l](0,8){%bulle NNO : connaissances et compétences
      \pspolygon[fillstyle=solid,fillcolor=A1,linecolor=A1](6,0)(8,-1.5)(6.4,0)
      \bullecours
         {A1}
         {Je connais mon cours}
         {C1 : Je connais la notion de nombre relatif et d'opposé \hfill $\square$ \par
          C2 : Je repère et je place sur un axe gradué des nombres relatifs \hfill $\square$ \par
          C3 : Je compare, range, encadre des nombres relatifs \hfill $\square$}}         
   \rput[l](14,4){%bulle ENE : Aide vidéo
      \pspolygon[fillstyle=solid,fillcolor=A1,linecolor=A1](0,3.2)(-2.5,2)(0,3.5)
      \bulleQR
         {A1}
         {Aide en vidéo}
         {\qrcode{https://www.youtube.com/watch?v=SImiMoRB0vU} \par \medskip
          Droite graduée \par \bigskip
          \qrcode{https://www.youtube.com/watch?v=DYbRr4B42h8} \par \medskip
          Comparer des nombres}}    
      \rput[l](0,4){%bulle O : Questions flash
         \pspolygon[fillstyle=solid,fillcolor=Goldenrod,linecolor=Goldenrod](5,1.35)(6.5,1.5)(5,1.65)
         \bulle
            {Goldenrod}
            {Questions flash}
            {\psline[linecolor=darkgray](1.75,-0.5)(2.25,0.5)
             \rput(2.75,0){\darkgray\Huge 5}}}    
      \rput[l](0,0){%bulle SO : Compétence 1
         \pspolygon[fillstyle=solid,fillcolor=B1,linecolor=B1](5,2)(7.2,4.5)(5,2.35)
         \bulle
            {B1}
            {Compétence 1}
            {Activité d'approche \hfill $\star\star$ \hfill $\square$ \par
             Exercice 3 \hfill $\star$ \hfill $\square$ \par
             Exercice 4 \hfill $\star$ \hfill $\square$ \par
             Récréation \hfill $\star\star\star$ \hfill $\square$}}
      \rput[l](6.5,0){%bulle S : compétence 2
         \pspolygon[fillstyle=solid,fillcolor=B1,linecolor=B1](2.35,3)(2.5,4.5)(2.65,3)
         \bulle
            {B1}
            {Compétence 2}
            {Exercice 1 \hfill $\star\star$ \hfill $\square$ \par
             Exercice 2 \hfill $\star\star$ \hfill $\square$ \par
             Exercice 5 \hfill $\star$ \hfill $\square$ \par
             Exercice 6 \hfill $\star$ \hfill $\square$}}           
      \rput[l](13,0){%bulle SE : compétence 3
          \pspolygon[fillstyle=solid,fillcolor=B1,linecolor=B1](0,2)(-2.3,4.5)(0,2.35)
          \bulle
            {B1}
            {Compétence 3}
            {Exercice 7 \hfill $\star$ \hfill $\square$ \par
             Exercice 8 \hfill $\star\star$ \hfill $\square$ \par
             Exercice 9 \hfill $\star\star\star$ \hfill $\square$ \par
             Exercice 10 \hfill $\star\star$ \hfill $\square$ }}               
\end{pspicture}
\end{center}



\AfficheCorriges[2]

\backmatter
%\fancyfoot[R]{}
%\tikzset{fondA/.style=yellow}
\tikzset{fondB/.style=purple}

%=======================================
\begin{tikzpicture}[remember picture,overlay]
% fond bicolore
\coordinate (cp) at (current page);
\coordinate (cpc) at (current page.center);
\coordinate (cpe) at ($ (current page.east) + (0.75cm,0cm) $);
\coordinate (cpne) at ($ (current page.north east) + (0.75cm,1.4cm) $);
\coordinate (cpn) at ($ (current page.north) + (1cm,1.4cm) $);
\coordinate (cpnw) at ($ (current page.north west) + (-0.75cm,1.4cm) $);
\coordinate (cpw) at ($ (current page.west) + (-0.75cm,0cm) $);
\coordinate (cpsw) at ($ (current page.south west) + (-0.75cm,-1.4cm) $);
\coordinate (cps) at ($ (current page.south) + (0.75cm,-1.4cm) $);
\coordinate (cpse) at ($ (current page.south east) + (0.75cm,-1.4cm) $);
\fill[fondA] (cps) .. controls (cpw) and (cpe) .. (cpn) -- (cpnw)  -- (cpsw) -- cycle;
\fill[fondB] (cps) .. controls (cpw) and (cpe) .. (cpn) -- (cpne)  -- (cpse) -- cycle;

\end{tikzpicture}




\end{document}
