\annexe{Plans de travail}
\label{PDT}

%macros pour les bulles

\newcommand{\bulle}[3]{
   \psframe[linewidth=1mm,framearc=0.1,linecolor=#1](0,0)(5,3)
   \psline[linecolor=#1](0,2.3)(5,2.3)
   \rput(2.5,2.65){#2}
   \rput(2.5,1.15){\begin{minipage}{4cm}
                            {\small\darkgray #3}
                          \end{minipage}}}
                        
\newcommand{\bullelongue}[3]{
   \psframe[linewidth=1mm,framearc=0.1,linecolor=#1](0,0)(8.25,3)
   \psline[linecolor=#1](0,2.3)(8.25,2.3)
   \rput(4.125,2.65){#2}
   \rput(4.125,1.15){\begin{minipage}{7.75cm}
                                 {\small\darkgray #3}
                               \end{minipage}}}
   
\newcommand{\bullemegalongue}[3]{
   \psframe[linewidth=1mm,framearc=0.1,linecolor=#1](0,0)(18,3)
   \psline[linecolor=#1](0,2.3)(18,2.3)
   \rput(9,2.65){#2}
   \rput(9,1.15){\begin{minipage}{17.5cm}
                                {\small\darkgray #3}
                             \end{minipage}}}
                         
\newcommand{\bullecours}[3]{
   \psframe[linewidth=1mm,framearc=0.1,linecolor=#1](0,0)(12.5,3)
   \psline[linecolor=#1](0,2.3)(12.5,2.3)
   \rput(6.25,2.65){#2}
   \rput(6.25,1.15){\begin{minipage}{12cm}
                              {\small\darkgray #3}
                            \end{minipage}}}
   
\newcommand{\bulleQR}[4]{
   \psframe[linewidth=1mm,framearc=0.1,linecolor=#1](0,0)(4,7)
   \psline[linecolor=#1](0,6.3)(4,6.3)
   \rput(2,6.65){#2}
   \rput(2,3.5){\small\darkgray #3}
   \rput(2,0.5){\small\darkgray #4}}


\begin{center}

%%%%%%%%%% Séquence 1 %%%%%%%%%%
%%%%%%%%%%%%%%%%%%%%%%%%%%%
\begin{pspicture}(0.5,0)(18,10.5)            
   {\color{red}
      \rput(9,5.75){\parbox{5cm}{\centering\large S1 \par  ENCHAÎNEMENT D'OPÉRATIONS}}} %bulle centrale  
   \rput[l](0,8){%bulle NNO : connaissances et compétences
      \pspolygon[fillstyle=solid,fillcolor=A1,linecolor=A1](6,0)(8,-1.5)(6.4,0)
      \bullecours
         {A1}
         {Je connais mon cours}
         {C1 : J'utilise et je range différentes représentations d'un même nombre décimal \hfill $\square$ \par
          C2 : Je traduis un enchaînement d’opérations à l’aide d’une expression et inversement \hfill $\square$ \par
          C3 : J'effectue un enchaînement d'opérations en respectant les priorités opératoires \hfill $\square$}}        
   \rput[l](14,4){%bulle ENE : Aide vidéo
      \pspolygon[fillstyle=solid,fillcolor=A1,linecolor=A1](0,3.2)(-2.5,2)(0,3.5)
      \bulleQR
         {A1}
         {Aide en vidéo}
         {Calculer avec des priorités}
         {Traduire une expression}}
         \tikz[remember picture,overlay]{\node at (16,8.9) {\qrcode{https://www.yout-ube.com/watch?v=TJH-fiwAt5s}};}
         \tikz[remember picture,overlay]{\node at (15.9,5.9) {\qrcode{https://www.yout-ube.com/watch?v=_yF5ItbcN28&feature=youtu.be}};}
      \rput[l](0,4){%bulle O : Questions flash
         \pspolygon[fillstyle=solid,fillcolor=Goldenrod,linecolor=Goldenrod](5,1.35)(6.5,1.5)(5,1.65)
         \bulle
            {Goldenrod}
            {Questions flash}
            {\psline[linecolor=darkgray](1.75,-0.5)(2.25,0.5)
             \rput(2.75,0){\darkgray\Huge 5}}}
      \rput[l](0,0){%bulle SO : Compétence 1
         \pspolygon[fillstyle=solid,fillcolor=B1,linecolor=B1](5,2)(7.2,4.5)(5,2.35)
         \bulle
            {B1}
            {Compétence 1}
            {Activité A \hfill $\star$ \hfill $\square$ \par
             Exercice 1 \hfill $\star$ \hfill $\square$ \par
             Exercice 2 \hfill $\star\star$ \hfill $\square$ \par
             Exercice 3 \hfill $\star\star\star$ \hfill $\square$}}
      \rput[l](6.5,0){%bulle S : compétence 2
         \pspolygon[fillstyle=solid,fillcolor=B1,linecolor=B1](2.35,3)(2.5,4.5)(2.65,3)
         \bulle
            {B1}
            {Compétence 2}
            {Exercice 4 \hfill $\star$ \hfill $\square$ \par
             Exercice 5 \hfill $\star\star$ \hfill $\square$ \par
             Exercice 6 \hfill $\star\star$ \hfill $\square$}}             
      \rput[l](13,0){%bulle SE : compétence 3
          \pspolygon[fillstyle=solid,fillcolor=B1,linecolor=B1](0,2)(-2.3,4.5)(0,2.35)
          \bulle
            {B1}
            {Compétence 3}
            {Exercice 7 \hfill $\star$ \hfill $\square$ \par
             Exercice 8 \hfill $\star$ \hfill $\square$ \par
             Exercice 9 \hfill $\star\star$ \hfill $\square$ \par
             Exercice 10 \hfill $\star\star\star$ \hfill $\square$ \par
             Récréation \hfill $\star\star\star$ \hfill $\square$}}                  
\end{pspicture}
   

%%%%%%%%%% Séquence 2 %%%%%%%%%%
%%%%%%%%%%%%%%%%%%%%%%%%%%%
\begin{pspicture}(0.5,0.5)(18,12.5)            
   {\color{DodgerBlue}
      \rput(9,5.75){\parbox{5cm}{\centering\large S2 \par ANGLES \par PARTICULIERS}}} %bulle centrale  
   \rput[l](0,8){%bulle NNO : connaissances et compétences
      \pspolygon[fillstyle=solid,fillcolor=A1,linecolor=A1](6,0)(8,-1.5)(6.4,0)
      \bullecours
         {A1}
         {Je connais mon cours}
         {C1 : Je reconnais des angles alternes-internes \hfill $\square$ \par
          C2 : Je reconnais des angles correspondants \hfill $\square$ \par
          C3 : J'utilise les propriétés des angles alternes-internes et correspondants pour \par \hspace*{6mm} montrer que des droites sont parallèles ou pour déterminer des angles \hfill $\square$}}         
   \rput[l](14,4){%bulle ENE : Aide vidéo
      \pspolygon[fillstyle=solid,fillcolor=A1,linecolor=A1](0,3.2)(-2.5,2)(0,3.5)
      \bulleQR
         {A1}
         {Aide en vidéo}
         {Angles alternes-internes}
         {Angles correspondants}}
         \tikz[remember picture,overlay]{\node at (16,8.9) {\qrcode{https://www.yout-ube.com/watch?v=c8CuPY-KaNM&list=PLVUDmbpupCaoTCiYBCUGfCyenktNbkIdt}};}
         \tikz[remember picture,overlay]{\node at (15.9,5.9) {\qrcode{https://www.yout-ube.com/watch?v=ErUq2wdA_PE&list=PLVUDmbpupCaoTCiYBCUGfCyenktNbkIdt&index=2}};}
      \rput[l](0,4){%bulle O : Questions flash
         \pspolygon[fillstyle=solid,fillcolor=Goldenrod,linecolor=Goldenrod](5,1.35)(6.5,1.5)(5,1.65)
         \bulle
            {Goldenrod}
            {Questions flash}
            {\psline[linecolor=darkgray](1.75,-0.5)(2.25,0.5)
             \rput(2.75,0){\darkgray\Huge 5}}}     
      \rput[l](0,0){%bulle SSO : Compétence 1
         \pspolygon[fillstyle=solid,fillcolor=B1,linecolor=B1](5,3)(8,4.5)(5.5,3)
         \bullelongue
            {B1}
            {Compétences 1 et 2}
            {Activité A \hfill $\star$ \hfill $\square$ \par
             Exercice 1 \hfill $\star$ \hfill $\square$ \par
             Exercice 2 \hfill $\star$ \hfill $\square$ \par
             Exercice 3 \hfill $\star\star\star$ \hfill $\square$}}
      \rput[l](9.75,0){%bulle S : compétence 2
         \pspolygon[fillstyle=solid,fillcolor=B1,linecolor=B1](3,3)(0.5,4.5)(3.5,3)
         \bullelongue
            {B1}
            {Compétence 3}
            {Exercice 4 \hfill $\star$ \hfill $\square$ \par
             Exercice 5 \hfill $\star\star$ \hfill $\square$ \par
             Exercice 6 \hfill $\star\star$ \hfill $\square$ \par
             Exercice 7 \hfill $\star\star\star$ \hfill $\square$}}                             
\end{pspicture}


%%%%%%%%%% Séquence 3 %%%%%%%%%%
%%%%%%%%%%%%%%%%%%%%%%%%%%%
\begin{pspicture}(0.5,0)(18,10)            
   {\color{orange}
      \rput(9,5.75){\parbox{5cm}{\centering\large S3 \par EN ROUTE VERS LA PROGRAMMATION}}} %bulle centrale  
   \rput[l](0,8){%bulle NNO : connaissances et compétences
      \pspolygon[fillstyle=solid,fillcolor=A1,linecolor=A1](6,0)(8,-1.5)(6.4,0)
      \bullecours
         {A1}
         {Je connais mon cours}
         {C1 : Je réalise des activités d’algorithmique débranchée \hfill $\square$ \par
          C2 : Je traduis un script de déplacement ou de construction géométrique \hfill $\square$ \par
          C3 : J'écris un script de déplacement ou de construction géométrique \hfill $\square$}}         
   \rput[l](14,4){%bulle ENE : Aide vidéo
      \pspolygon[fillstyle=solid,fillcolor=A1,linecolor=A1](0,3.2)(-2.5,2)(0,3.5)
      \bulleQR
         {A1}
         {Aide en vidéo}
         {Prise en main de Scratch}
         {Utiliser une boucle}}  
         \tikz[remember picture,overlay]{\node at (16,8.9) {\qrcode{https://www.yout-ube.com/watch?v=pdtMUgnmRa4&list=PLVUDmbpupCaqKLNci7_86rbIt61SMhJPd&index=1}};}
         \tikz[remember picture,overlay]{\node at (15.9,5.9) {\qrcode{https://www.yout-ube.com/watch?v=8Sfarvw6jgg&list=PLVUDmbpupCaqKLNci7_86rbIt61SMhJPd&index=2}};}
      \rput[l](0,4){%bulle O : Questions flash
         \pspolygon[fillstyle=solid,fillcolor=Goldenrod,linecolor=Goldenrod](5,1.35)(6.5,1.5)(5,1.65)
         \bulle
            {Goldenrod}
            {Questions flash}
            {\psline[linecolor=darkgray](1.75,-0.5)(2.25,0.5)
             \rput(2.75,0){\darkgray\Huge 5}}}    
      \rput[l](0,0){%bulle SO : Compétence 1
         \pspolygon[fillstyle=solid,fillcolor=B1,linecolor=B1](5,2)(7.2,4.5)(5,2.35)
         \bulle
            {B1}
            {Compétence 1}
            {Activité A \hfill $\star\star\star$ \hfill $\square$ \par
             Récréation \hfill $\star\star\star$ \hfill $\square$}}
      \rput[l](6.5,0){%bulle S : compétence 2
         \pspolygon[fillstyle=solid,fillcolor=B1,linecolor=B1](2.35,3)(2.5,4.5)(2.65,3)
         \bulle
            {B1}
            {Compétence 2}
            {Exercice 1 \hfill $\star$ \hfill $\square$ \par
             Exercice 2 \hfill $\star\star$ \hfill $\square$}}           
      \rput[l](13,0){%bulle SE : compétence 3
          \pspolygon[fillstyle=solid,fillcolor=B1,linecolor=B1](0,2)(-2.3,4.5)(0,2.35)
          \bulle
            {B1}
            {Compétence 3}
            {Exercice 1 \hfill $\star$ \hfill $\square$ \par
             Exercice 3 \hfill $\star$ \hfill $\square$ \par
             Exercice 4 \hfill $\star\star$ \hfill $\square$}}               
\end{pspicture}


%%%%%%%%%% Séquence 4 %%%%%%%%%%
%%%%%%%%%%%%%%%%%%%%%%%%%%%
\begin{pspicture}(0.5,0.5)(18,12.5)            
   {\color{red}
      \rput(9,5.75){\parbox{5cm}{\centering\large S4 \par NOMBRES \par RELATIFS}}} %bulle centrale  
   \rput[l](0,8){%bulle NNO : connaissances et compétences
      \pspolygon[fillstyle=solid,fillcolor=A1,linecolor=A1](6,0)(8,-1.5)(6.4,0)
      \bullecours
         {A1}
         {Je connais mon cours}
         {C1 : Je connais la notion de nombre relatif et d'opposé \hfill $\square$ \par
          C2 : Je repère et je place sur un axe gradué des nombres relatifs \hfill $\square$ \par
          C3 : Je compare, range, encadre des nombres relatifs \hfill $\square$}}         
   \rput[l](14,4){%bulle ENE : Aide vidéo
      \pspolygon[fillstyle=solid,fillcolor=A1,linecolor=A1](0,3.2)(-2.5,2)(0,3.5)
      \bulleQR
         {A1}
         {Aide en vidéo}
         {Droite graduée}
         {Comparer des nombres}}
         \tikz[remember picture,overlay]{\node at (16,8.9) {\qrcode{https://www.yout-ube.com/watch?v=SImiMoRB0vU}};}
          \tikz[remember picture,overlay]{\node at (15.9,5.9) {\qrcode{https://www.yout-ube.com/watch?v=DYbRr4B42h8}};}   
      \rput[l](0,4){%bulle O : Questions flash
         \pspolygon[fillstyle=solid,fillcolor=Goldenrod,linecolor=Goldenrod](5,1.35)(6.5,1.5)(5,1.65)
         \bulle
            {Goldenrod}
            {Questions flash}
            {\psline[linecolor=darkgray](1.75,-0.5)(2.25,0.5)
             \rput(2.75,0){\darkgray\Huge 5}}}    
      \rput[l](0,0){%bulle SO : Compétence 1
         \pspolygon[fillstyle=solid,fillcolor=B1,linecolor=B1](5,2)(7.2,4.5)(5,2.35)
         \bulle
            {B1}
            {Compétence 1}
            {Activité A \hfill $\star\star$ \hfill $\square$ \par
             Exercice 3 \hfill $\star$ \hfill $\square$ \par
             Exercice 4 \hfill $\star$ \hfill $\square$ \par
             Récréation \hfill $\star\star\star$ \hfill $\square$}}
      \rput[l](6.5,0){%bulle S : compétence 2
         \pspolygon[fillstyle=solid,fillcolor=B1,linecolor=B1](2.35,3)(2.5,4.5)(2.65,3)
         \bulle
            {B1}
            {Compétence 2}
            {Exercice 1 \hfill $\star\star$ \hfill $\square$ \par
             Exercice 2 \hfill $\star\star$ \hfill $\square$ \par
             Exercice 5 \hfill $\star$ \hfill $\square$ \par
             Exercice 6 \hfill $\star$ \hfill $\square$}}           
      \rput[l](13,0){%bulle SE : compétence 3
          \pspolygon[fillstyle=solid,fillcolor=B1,linecolor=B1](0,2)(-2.3,4.5)(0,2.35)
          \bulle
            {B1}
            {Compétence 3}
            {Exercice 7 \hfill $\star$ \hfill $\square$ \par
             Exercice 8 \hfill $\star\star$ \hfill $\square$ \par
             Exercice 9 \hfill $\star\star\star$ \hfill $\square$ \par
             Exercice 10 \hfill $\star\star$ \hfill $\square$ }}               
\end{pspicture}


%%%%%%%%%% Séquence 5 %%%%%%%%%%
%%%%%%%%%%%%%%%%%%%%%%%%%%%
\begin{pspicture}(0.5,0)(18,10)            
   {\color{DodgerBlue}
      \rput(9,5.75){\parbox{5cm}{\centering\large S5 \par REPÉRAGE \par DANS LE PLAN}}} %bulle centrale  
   \rput[l](0,8){%bulle NNO : connaissances et compétences
      \pspolygon[fillstyle=solid,fillcolor=A1,linecolor=A1](6,0)(8,-1.5)(6.4,0)
      \bullecours
         {A1}
         {Je connais mon cours}
         {C1 : Je sais lire les coordonnées dans un repère du plan \hfill $\square$ \par
          C2 : Je sais placer des points dans un repère du plan \hfill $\square$}}         
   \rput[l](14,4){%bulle ENE : Aide vidéo
      \pspolygon[fillstyle=solid,fillcolor=A1,linecolor=A1](0,3.2)(-2.5,2)(0,3.5)
      \bulleQR
         {A1}
         {Aide en vidéo}
         {Repère du plan}
         {Lire et placer des points}}
         \tikz[remember picture,overlay]{\node at (16,8.9) {\qrcode{https://www.yout-ube.com/watch?v=AHNYuKCoCvU&t=280s}};}
          \tikz[remember picture,overlay]{\node at (15.9,5.9) {\qrcode{https://www.yout-ube.com/watch?v=kAtwKV3DqKI&list=PLVUDmbpupCap2d3CpwVNB_SrkYWEFsmBX&index=7}};}
      \rput[l](0,4){%bulle O : Questions flash
         \pspolygon[fillstyle=solid,fillcolor=Goldenrod,linecolor=Goldenrod](5,1.35)(6.5,1.5)(5,1.65)
         \bulle
            {Goldenrod}
            {Questions flash}
            {\psline[linecolor=darkgray](1.75,-0.5)(2.25,0.5)
             \rput(2.75,0){\darkgray\Huge 5}}}    
      \rput[l](0,0){%bulle SSO : Compétence 1
         \pspolygon[fillstyle=solid,fillcolor=B1,linecolor=B1](5,3)(8,4.5)(5.5,3)
         \bullelongue
            {B1}
            {Compétence 1}
            {Exercice 1 \hfill $\star$ \hfill $\square$ \par
             Exercice 4 \hfill $\star\star$ \hfill $\square$}}
      \rput[l](9.75,0){%bulle S : compétence 2
         \pspolygon[fillstyle=solid,fillcolor=B1,linecolor=B1](3,3)(0.5,4.5)(3.5,3)
         \bullelongue
            {B1}
            {Compétence 2}
            {Activité A \hfill $\star\star\star$ \hfill $\square$ \par
             Exercice 2 \hfill $\star\star$ \hfill $\square$ \par
             Exercice 3 \hfill $\star\star$ \hfill $\square$ \par
             Récréation \hfill $\star\star$ \hfill $\square$}}             
\end{pspicture}


%%%%%%%%%% Séquence 6 %%%%%%%%%%
%%%%%%%%%%%%%%%%%%%%%%%%%%%
\begin{pspicture}(0.5,0.5)(18,12.5)            
   {\color{violet}
      \rput(9,5.75){\parbox{5cm}{\centering\large S6 \par INTERPRÉTER \par REPRÉSENTER \par DES DONNÉES}}} %bulle centrale  
   \rput[l](0,8){%bulle NNO : connaissances et compétences
      \pspolygon[fillstyle=solid,fillcolor=A1,linecolor=A1](6,0)(8,-1.5)(6.4,0)
      \bullecours
         {A1}
         {Je connais mon cours}
         {C1 : Je recueille et j'organise des données \hfill $\square$ \par
          C2 : Je lis et j'interprète des données d'un tableau ou d'un diagramme \hfill $\square$ \par
          C3 : Je représente des données sous forme d'un tableau ou d'un diagramme \hfill $\square$}}        
   \rput[l](14,4){%bulle ENE : Aide vidéo
      \pspolygon[fillstyle=solid,fillcolor=A1,linecolor=A1](0,3.2)(-2.5,2)(0,3.5)
      \bulleQR
         {A1}
         {Aide en vidéo}
         {Diagramme en bâtons}
         {Diagramme circulaire}}
         \tikz[remember picture,overlay]{\node at (16,8.9) {\qrcode{https://www.yout-ube.com/watch?v=NZnhF5VDy04&feature=youtu.be}};}
          \tikz[remember picture,overlay]{\node at (15.9,5.9) {\qrcode{https://www.yout-ube.com/watch?v=gpCY_3zq3bk&feature=youtu.be}};}
      \rput[l](0,4){%bulle O : Questions flash
         \pspolygon[fillstyle=solid,fillcolor=Goldenrod,linecolor=Goldenrod](5,1.35)(6.5,1.5)(5,1.65)
         \bulle
            {Goldenrod}
            {Questions flash}
            {\psline[linecolor=darkgray](1.75,-0.5)(2.25,0.5)
             \rput(2.75,0){\darkgray\Huge 5}}}    
      \rput[l](0,0){%bulle SO : Compétence 1
         \pspolygon[fillstyle=solid,fillcolor=B1,linecolor=B1](5,2)(7.2,4.5)(5,2.35)
         \bulle
            {B1}
            {Compétence 1}
            {Activité A \hfill $\star\star$ \hfill $\square$}}
      \rput[l](6.5,0){%bulle S : compétence 2
         \pspolygon[fillstyle=solid,fillcolor=B1,linecolor=B1](2.35,3)(2.5,4.5)(2.65,3)
         \bulle
            {B1}
            {Compétence 2}
            {Exercice 1 \hfill $\star$ \hfill $\square$ \par
             Exercice 2 \hfill $\star$ \hfill $\square$ \par
             Exercice 3 \hfill $\star$ \hfill $\square$}}          
      \rput[l](13,0){%bulle SE : compétence 3
          \pspolygon[fillstyle=solid,fillcolor=B1,linecolor=B1](0,2)(-2.3,4.5)(0,2.35)
          \bulle
            {B1}
            {Compétence 3}
            {Exercice 2 \hfill $\star\star$ \hfill $\square$ \par
             Exercice 3 \hfill $\star\star$ \hfill $\square$ \par
             Récréation \hfill $\star\star$ \hfill $\square$}}              
\end{pspicture}


%%%%%%%%%% Séquence 7 %%%%%%%%%%
%%%%%%%%%%%%%%%%%%%%%%%%%%%
\begin{pspicture}(0.5,0)(18,10)            
   {\color{Green}
      \rput(9,5.75){\parbox{5cm}{\centering\large S7 \par DURÉES ET \par HORAIRES}}} %bulle centrale  
   \rput[l](0,8){%bulle NNO : connaissances et compétences
      \pspolygon[fillstyle=solid,fillcolor=A1,linecolor=A1](6,0)(8,-1.5)(6.4,0)
      \bullecours
         {A1}
         {Je connais mon cours}
         {C1 : J'effectue des calculs de durées et d’horaires \hfill $\square$ \par
          C2 : J'effectue des conversions d’unités de durée \hfill $\square$ \par
          C3 : Je résous des problèmes de durées et d'horaires \hfill $\square$}}         
   \rput[l](14,4){%bulle ENE : Aide vidéo
      \pspolygon[fillstyle=solid,fillcolor=A1,linecolor=A1](0,3.2)(-2.5,2)(0,3.5)
      \bulleQR
         {A1}
         {Aide en vidéo}
         {Convertir des durées}
         {Calculer des durées}}
         \tikz[remember picture,overlay]{\node at (16,8.9) {\qrcode{https://www.lumni.fr/video/comment-convertir-les-heures-en-minutes}};}
          \tikz[remember picture,overlay]{\node at (15.9,5.9) {\qrcode{https://www.yout-ube.com/watch?v=ZV7VG7NzDwE&list=PLVUDmbpupCaqtcNOvmjURARFwa5Vj5xER&index=6}};}
      \rput[l](0,4){%bulle O : Questions flash
         \pspolygon[fillstyle=solid,fillcolor=Goldenrod,linecolor=Goldenrod](5,1.35)(6.5,1.5)(5,1.65)
         \bulle
            {Goldenrod}
            {Questions flash}
            {\psline[linecolor=darkgray](1.75,-0.5)(2.25,0.5)
             \rput(2.75,0){\darkgray\Huge 5}}}     
      \rput[l](0,0){%bulle SO : Compétence 1
         \pspolygon[fillstyle=solid,fillcolor=B1,linecolor=B1](5,2)(7.2,4.5)(5,2.35)
         \bulle
            {B1}
            {Compétence 1}
            {Exercice 5 \hfill $\star\star$ \hfill $\square$ \par
             Exercice 6 \hfill $\star$ \hfill $\square$ \par
             Exercice 7 \hfill $\star$ \hfill $\square$ \par
             Exercice 8 \hfill $\star$ \hfill $\square$}}
      \rput[l](6.5,0){%bulle S : compétence 2
         \pspolygon[fillstyle=solid,fillcolor=B1,linecolor=B1](2.35,3)(2.5,4.5)(2.65,3)
         \bulle
            {B1}
            {Compétence 2}
            {Exercice 1 \hfill $\star\star$ \hfill $\square$ \par
             Exercice 2 \hfill $\star\star$ \hfill $\square$ \par
             Exercice 3 \hfill $\star\star$ \hfill $\square$ \par
             Exercice 4 \hfill $\star\star$ \hfill $\square$}}             
      \rput[l](13,0){%bulle SE : compétence 3
          \pspolygon[fillstyle=solid,fillcolor=B1,linecolor=B1](0,2)(-2.3,4.5)(0,2.35)
          \bulle
            {B1}
            {Compétence 3}
            {Activité A \hfill $\star\star$ \hfill $\square$ \par
             Exercice 9 \hfill $\star\star$ \hfill $\square$ \par
             Exercice 10 \hfill $\star\star\star$ \hfill $\square$ \par
             Récréation \hfill $\star\star\star$ \hfill $\square$}}                  
\end{pspicture}
   

%%%%%%%%%% Séquence 8 %%%%%%%%%%
%%%%%%%%%%%%%%%%%%%%%%%%%%%
\begin{pspicture}(0.5,0.5)(18,12.5)            
   {\color{Red}
      \rput(9,5.75){\parbox{5cm}{\centering\large S8 \par EXPRESSIONS \par ALGÉBRIQUES}}} %bulle centrale  
   \rput[l](0,8){%bulle NNO : connaissances et compétences
      \pspolygon[fillstyle=solid,fillcolor=A1,linecolor=A1](6,0)(8,-1.5)(6.4,0)
      \bullecours
         {A1}
         {Je connais mon cours}
         {C1 : Je simplifie une expression littérale \hfill $\square$ \par
          C2 : Je produis une expression littérale \hfill $\square$ \par
          C3 : J'utilise une lettre pour traduire et démontrer une propriété générale \hfill $\square$}}         
   \rput[l](14,4){%bulle ENE : Aide vidéo
      \pspolygon[fillstyle=solid,fillcolor=A1,linecolor=A1](0,3.2)(-2.5,2)(0,3.5)
      \bulleQR
         {A1}
         {Aide en vidéo}
         {Traduire une expression}
         {Simplifier une expression}}
         \tikz[remember picture,overlay]{\node at (16,8.9) {\qrcode{https://www.yout-ube.com/watch?v=_yF5ItbcN28}};}
          \tikz[remember picture,overlay]{\node at (15.9,5.9) {\qrcode{https://www.yout-ube.com/watch?v=eBPOd0bTBro&list=PLVUDmbpupCapuHDg65cf4sNTNkYXVVGP-&index=3}};}
      \rput[l](0,4){%bulle O : Questions flash
         \pspolygon[fillstyle=solid,fillcolor=Goldenrod,linecolor=Goldenrod](5,1.35)(6.5,1.5)(5,1.65)
         \bulle
            {Goldenrod}
            {Questions flash}
            {\psline[linecolor=darkgray](1.75,-0.5)(2.25,0.5)
             \rput(2.75,0){\darkgray\Huge 5}}}     
      \rput[l](0,0){%bulle SO : Compétence 1
         \pspolygon[fillstyle=solid,fillcolor=B1,linecolor=B1](5,2)(7.2,4.5)(5,2.35)
         \bulle
            {B1}
            {Compétence 1}
            {Exercice 4 \hfill $\star$ \hfill $\square$ \par
             Exercice 5 \hfill $\star$ \hfill $\square$ \par
             Exercice 6 \hfill $\star\star$ \hfill $\square$}}
      \rput[l](6.5,0){%bulle S : compétence 2
         \pspolygon[fillstyle=solid,fillcolor=B1,linecolor=B1](2.35,3)(2.5,4.5)(2.65,3)
         \bulle
            {B1}
            {Compétence 2}
            {Exercice 1 \hfill $\star\star$ \hfill $\square$ \par
             Exercice 2 \hfill $\star$ \hfill $\square$ \par
             Exercice 3 \hfill $\star\star$ \hfill $\square$}}             
      \rput[l](13,0){%bulle SE : compétence 3
          \pspolygon[fillstyle=solid,fillcolor=B1,linecolor=B1](0,2)(-2.3,4.5)(0,2.35)
          \bulle
            {B1}
            {Compétence 3}
            {Activité A \hfill $\star\star$ \hfill $\square$ \par
             Exercice 7 \hfill $\star\star$ \hfill $\square$ \par
             Exercice 8 \hfill $\star\star\star$ \hfill $\square$ \par
             Récréation \hfill $\star\star\star$ \hfill $\square$}}                  
\end{pspicture}


%%%%%%%%%% Séquence 9 %%%%%%%%%%
%%%%%%%%%%%%%%%%%%%%%%%%%%%
\begin{pspicture}(0.5,0)(18,10)            
   {\color{DodgerBlue}
      \rput(9,5.75){\parbox{5cm}{\centering\large S9 \par SOMME DES \par ANGLES D'UN \par TRIANGLE}}} %bulle centrale  
   \rput[l](0,8){%bulle NNO : connaissances et compétences
      \pspolygon[fillstyle=solid,fillcolor=A1,linecolor=A1](6,0)(8,-1.5)(6.4,0)
      \bullecours
         {A1}
         {Je connais mon cours}
         {C1 : J'applique directement la propriété des angles d'un triangles pour savoir si un  \par \hspace*{6mm} triangle est constructible ou pour déterminer un angle \hfill $\square$ \par
          C2 : J'effectue des démonstrations utilisant la propriété de la somme des angles \hfill $\square$}}         
   \rput[l](14,4){%bulle ENE : Aide vidéo
      \pspolygon[fillstyle=solid,fillcolor=A1,linecolor=A1](0,3.2)(-2.5,2)(0,3.5)
      \bulleQR
         {A1}
         {Aide en vidéo}
         {La règle des \udeg{180}}
         {Démontrer}}
         \tikz[remember picture,overlay]{\node at (16,8.9) {\qrcode{https://www.yout-ube.com/watch?v=S1vCp-O7fbw&list=PLVUDmbpupCaqW33IMWG2n_73O4Jy7GEse&index=11}};}
          \tikz[remember picture,overlay]{\node at (15.9,5.9) {\qrcode{https://www.yout-ube.com/watch?v=x0UA6kbiDcM&list=PLVUDmbpupCaqW33IMWG2n_73O4Jy7GEse&index=12}};}
      \rput[l](0,4){%bulle O : Questions flash
         \pspolygon[fillstyle=solid,fillcolor=Goldenrod,linecolor=Goldenrod](5,1.35)(6.5,1.5)(5,1.65)
         \bulle
            {Goldenrod}
            {Questions flash}
            {\psline[linecolor=darkgray](1.75,-0.5)(2.25,0.5)
             \rput(2.75,0){\darkgray\Huge 5}}}    
      \rput[l](0,0){%bulle SSO : Compétence 1
         \pspolygon[fillstyle=solid,fillcolor=B1,linecolor=B1](5,3)(8,4.5)(5.5,3)
         \bullelongue
            {B1}
            {Compétence 1}
            {Activité A \hfill $\star$ \hfill $\square$ \par
             Exercice 1 \hfill $\star$ \hfill $\square$ \par
             Exercice 2 \hfill $\star$ \hfill $\square$ \par
             Exercice 3 \hfill $\star\star$ \hfill $\square$}}
      \rput[l](9.75,0){%bulle S : compétence 2
         \pspolygon[fillstyle=solid,fillcolor=B1,linecolor=B1](3,3)(0.5,4.5)(3.5,3)
         \bullelongue
            {B1}
            {Compétence 2}
            {Exercice 4 \hfill $\star\star$ \hfill $\square$ \par
             Exercice 5 \hfill $\star\star$ \hfill $\square$ \par
             Exercice 6 \hfill $\star\star\star$ \hfill $\square$ \par
             Récréation \hfill $\star\star$ \hfill $\square$}}
\end{pspicture}

             
%%%%%%%%%% Séquence 10 %%%%%%%%%%
%%%%%%%%%%%%%%%%%%%%%%%%%%%
\begin{pspicture}(0.5,0.5)(18,12.5)           
   {\color{violet}
      \rput(9,5.75){\parbox{5cm}{\centering\large S10 \par PROBABILITÉS}}} %bulle centrale  
   \rput[l](0,8){%bulle NNO : connaissances et compétences
      \pspolygon[fillstyle=solid,fillcolor=A1,linecolor=A1](6,0)(8,-1.5)(6.4,0)
      \bullecours
         {A1}
         {Je connais mon cours}
         {C1 : Je place un événement sur une échelle de probabilités \hfill $\square$ \par
          C2 : Je calcule des probabilités dans des cas simples \hfill $\square$}}         
   \rput[l](14,4){%bulle ENE : Aide vidéo
      \pspolygon[fillstyle=solid,fillcolor=A1,linecolor=A1](0,3.2)(-2.5,2)(0,3.5)
      \bulleQR
         {A1}
         {Aide en vidéo}
         {Problème de hasard}
         {Calculer un probabilité}}
         \tikz[remember picture,overlay]{\node at (16,8.9) {\qrcode{https://www.yout-ube.com/watch?v=6EtRH4udcKY&feature=youtu.be}};}
          \tikz[remember picture,overlay]{\node at (15.9,5.9) {\qrcode{https://www.yout-ube.com/watch?v=a9Mb5v7Z4Mw}};}
      \rput[l](0,4){%bulle O : Questions flash
         \pspolygon[fillstyle=solid,fillcolor=Goldenrod,linecolor=Goldenrod](5,1.35)(6.5,1.5)(5,1.65)
         \bulle
            {Goldenrod}
            {Questions flash}
            {\psline[linecolor=darkgray](1.75,-0.5)(2.25,0.5)
             \rput(2.75,0){\darkgray\Huge 5}}}    
      \rput[l](0,0){%bulle SSO : Compétence 1
         \pspolygon[fillstyle=solid,fillcolor=B1,linecolor=B1](5,3)(8,4.5)(5.5,3)
         \bullelongue
            {B1}
            {Compétence 1}
            {Activité A \hfill $\star\star$ \hfill $\square$ \par
             Exercice 1 \hfill $\star$ \hfill $\square$ \par
             Exercice 2 \hfill $\star$ \hfill $\square$}}
      \rput[l](9.75,0){%bulle S : compétence 2
         \pspolygon[fillstyle=solid,fillcolor=B1,linecolor=B1](3,3)(0.5,4.5)(3.5,3)
         \bullelongue
            {B1}
            {Compétence 2}
            {\begin{multicols}{2}
               Exercice 3 \hfill $\star$ \hfill $\square$ \par
               Exercice 4 \hfill $\star\star$ \hfill $\square$ \par
               Exercice 5 \hfill $\star\star$ \hfill $\square$ \par
               Exercice 6 \hfill $\star\star$ \hfill $\square$ \par
               Exercice 7 \hfill $\star\star\star$ \hfill $\square$ \par
               Récréation \hfill $\star$ \hfill $\square$
             \end{multicols}}}                    
\end{pspicture}


%%%%%%%%%% Séquence 11 %%%%%%%%%%
%%%%%%%%%%%%%%%%%%%%%%%%%%%
\begin{pspicture}(0.5,0)(18,10)            
   {\color{red}
      \rput(9,5.75){\parbox{5cm}{\centering\large S11 \par  MULTIPLES ET DIVISEURS}}} %bulle centrale  
   \rput[l](0,8){%bulle NNO : connaissances et compétences
      \pspolygon[fillstyle=solid,fillcolor=A1,linecolor=A1](6,0)(8,-1.5)(6.4,0)
      \bullecours
         {A1}
         {Je connais mon cours}
         {C1 : Je calcule le quotient et le reste d'une division euclidienne \hfill $\square$ \par
          C2 : Je détermine si un nombre entier est ou n’est pas multiple ou diviseur d’un \par \hspace*{6mm} autre nombre entier \hfill $\square$ \par
          C3 : J'utilise les critères de divisibilité (par 2, 3, 5, 9, 10) \hfill $\square$}}         
   \rput[l](14,4){%bulle ENE : Aide vidéo
      \pspolygon[fillstyle=solid,fillcolor=A1,linecolor=A1](0,3.2)(-2.5,2)(0,3.5)
      \bulleQR
         {A1}
         {Aide en vidéo}
         {Multiples et diviseurs}
         {Critères de divisibilité}}
         \tikz[remember picture,overlay]{\node at (16,8.9) {\qrcode{https://www.yout-ube.com/watch?v=-PLZFlAG99Q&feature=youtu.be}};}
          \tikz[remember picture,overlay]{\node at (15.9,5.9) {\qrcode{https://www.yout-ube.com/watch?v=BJDE6uOrmYQ}};} 
      \rput[l](0,4){%bulle O : Questions flash
         \pspolygon[fillstyle=solid,fillcolor=Goldenrod,linecolor=Goldenrod](5,1.35)(6.5,1.5)(5,1.65)
         \bulle
            {Goldenrod}
            {Questions flash}
            {\psline[linecolor=darkgray](1.75,-0.5)(2.25,0.5)
             \rput(2.75,0){\darkgray\Huge 5}}}     
      \rput[l](0,0){%bulle SO : Compétence 1
         \pspolygon[fillstyle=solid,fillcolor=B1,linecolor=B1](5,2)(7.2,4.5)(5,2.35)
         \bulle
            {B1}
            {Compétence 1}
            {Exercice 1 \hfill $\star$ \hfill $\square$ \par
             Exercice 2 \hfill $\star$ \hfill $\square$ \par
             Exercice 3 \hfill $\star\star$ \hfill $\square$}}
      \rput[l](6.5,0){%bulle S : compétence 2
         \pspolygon[fillstyle=solid,fillcolor=B1,linecolor=B1](2.35,3)(2.5,4.5)(2.65,3)
         \bulle
            {B1}
            {Compétence 2}
            {Activité A \hfill $\star\star$ \hfill $\square$ \par
             Exercice 4 \hfill $\star$ \hfill $\square$ \par
             Exercice 5 \hfill $\star\star$ \hfill $\square$ \par
             Récréation \hfill $\star\star$ \hfill $\square$}}             
      \rput[l](13,0){%bulle SE : compétence 3
          \pspolygon[fillstyle=solid,fillcolor=B1,linecolor=B1](0,2)(-2.3,4.5)(0,2.35)
          \bulle
            {B1}
            {Compétence 3}
            {Exercice 6 \hfill $\star\star$ \hfill $\square$ \par
             Exercice 7 \hfill $\star$ \hfill $\square$ \par
             Exercice 8 \hfill $\star$ \hfill $\square$ \par
             Exercice 9 \hfill $\star\star$ \hfill $\square$ \par
             Exercice 10 \hfill $\star\star\star$ \hfill $\square$}}                  
\end{pspicture}


%%%%%%%%%% Séquence 12 %%%%%%%%%%
%%%%%%%%%%%%%%%%%%%%%%%%%%%
\begin{pspicture}(0.5,0.5)(18,12.5)            
   {\color{DodgerBlue}
      \rput(9,5.75){\parbox{5cm}{\centering\large 12 \par LA SYMÉTRIE \par CENTRALE}}} %bulle centrale  
   \rput[l](0,8){%bulle NNO : connaissances et compétences
      \pspolygon[fillstyle=solid,fillcolor=A1,linecolor=A1](6,0)(8,-1.5)(6.4,0)
      \bullecours
         {A1}
         {Je connais mon cours}
         {C1 : Je fais la distinction entre symétrie centrale et symétrie axiale \hfill $\square$ \par
          C2 : Je transforme une figure par symétrie centrale \hfill $\square$ \par
          C3 : J'identifie des symétries dans des frises, des pavages, des rosaces \hfill $\square$}}         
   \rput[l](14,4){%bulle ENE : Aide vidéo
      \pspolygon[fillstyle=solid,fillcolor=A1,linecolor=A1](0,3.2)(-2.5,2)(0,3.5)
      \bulleQR
         {A1}
         {Aide en vidéo}
         {Symétrie sur feuille unie}
         {Symétrie sur quadrillage}}
         \tikz[remember picture,overlay]{\node at (16,8.9) {\qrcode{https://www.yout-ube.com/watch?v=gQZIWxzOfaE}};}
          \tikz[remember picture,overlay]{\node at (15.9,5.9) {\qrcode{https://www.yout-ube.com/watch?v=NIrHncshQLA}};}
      \rput[l](0,4){%bulle O : Questions flash
         \pspolygon[fillstyle=solid,fillcolor=Goldenrod,linecolor=Goldenrod](5,1.35)(6.5,1.5)(5,1.65)
         \bulle
            {Goldenrod}
            {Questions flash}
            {\psline[linecolor=darkgray](1.75,-0.5)(2.25,0.5)
             \rput(2.75,0){\darkgray\Huge 5}}}    
      \rput[l](0,0){%bulle SO : Compétence 1
         \pspolygon[fillstyle=solid,fillcolor=B1,linecolor=B1](5,2)(7.2,4.5)(5,2.35)
         \bulle
            {B1}
            {Compétence 1}
            {Activité A \hfill $\star\star$ \hfill $\square$ \par
             Exercice 2 \hfill $\star$ \hfill $\square$ \par
             Exercice 6 \hfill $\star$ \hfill $\square$}}
      \rput[l](6.5,0){%bulle S : compétence 2
         \pspolygon[fillstyle=solid,fillcolor=B1,linecolor=B1](2.35,3)(2.5,4.5)(2.65,3)
         \bulle
            {B1}
            {Compétence 2}
            {Exercice 1 \hfill $\star$ \hfill $\square$ \par
             Exercice 3 \hfill $\star$ \hfill $\square$ \par
             Exercice 4 \hfill $\star\star$ \hfill $\square$ \par
             Exercice 5 \hfill $\star\star$ \hfill $\square$}}           
      \rput[l](13,0){%bulle SE : compétence 3
          \pspolygon[fillstyle=solid,fillcolor=B1,linecolor=B1](0,2)(-2.3,4.5)(0,2.35)
          \bulle
            {B1}
            {Compétence 3}
            {Récréation \hfill $\star\star$ \hfill $\square$}}               
\end{pspicture}


%%%%%%%%%% Séquence 13 %%%%%%%%%%
%%%%%%%%%%%%%%%%%%%%%%%%%%%
\begin{pspicture}(0.5,0)(18,10)             
   {\color{Green}
      \rput(9,5.75){\parbox{5cm}{\centering\large S13 \par PÉRIMÈTRES \par ET AIRES}}} %bulle centrale  
   \rput[l](0,8){%bulle NNO : connaissances et compétences
      \pspolygon[fillstyle=solid,fillcolor=A1,linecolor=A1](6,0)(8,-1.5)(6.4,0)
      \bullecours
         {A1}
         {Je connais mon cours}
         {C1 : Je compare et détermine des périmètres et aires sans recours aux formules \hfill $\square$ \par
          C2 : Je calcule le périmètre et l'aire d'une figure avec des formules adaptées \hfill $\square$ \par
          C3 : J'effectue des conversions d'unités de longueurs et d'aires et j'utilise des \par
          \hspace*{6mm} unités adaptées \hfill $\square$}}         
   \rput[l](14,4){%bulle ENE : Aide vidéo
      \pspolygon[fillstyle=solid,fillcolor=A1,linecolor=A1](0,3.2)(-2.5,2)(0,3.5)
      \bulleQR
         {A1}
         {Aide en vidéo}
         {Aire d'une figure}
         {Le cas du disque}}
         \tikz[remember picture,overlay]{\node at (16,8.9) {\qrcode{https://www.yout-ube.com/watch?v=bMSrZjOBwcA&list=PLVUDmbpupCarppgOGwJQh5OeCwS6R6f-Y&index=3}};}
          \tikz[remember picture,overlay]{\node at (15.9,5.9) {\qrcode{https://www.yout-ube.com/watch?v=y-PV5LNmqsM&list=PLVUDmbpupCarppgOGwJQh5OeCwS6R6f-Y&index=4}};}
      \rput[l](0,4){%bulle O : Questions flash
         \pspolygon[fillstyle=solid,fillcolor=Goldenrod,linecolor=Goldenrod](5,1.35)(6.5,1.5)(5,1.65)
         \bulle
            {Goldenrod}
            {Questions flash}
            {\psline[linecolor=darkgray](1.75,-0.5)(2.25,0.5)
             \rput(2.75,0){\darkgray\Huge 5}}}     
      \rput[l](0,0){%bulle SO : Compétence 1
         \pspolygon[fillstyle=solid,fillcolor=B1,linecolor=B1](5,2)(7.2,4.5)(5,2.35)
         \bulle
            {B1}
            {Compétence 1}
            {Activité A \hfill $\star\star$ \hfill $\square$ \par
             Exercice 4 \hfill $\star$ \hfill $\square$ \par
             Exercice 7 \hfill $\star$ \hfill $\square$}}
      \rput[l](6.5,0){%bulle S : compétence 2
         \pspolygon[fillstyle=solid,fillcolor=B1,linecolor=B1](2.35,3)(2.5,4.5)(2.65,3)
         \bulle
            {B1}
            {Compétence 2}
            {Exercice 5 \hfill $\star\star\star$ \hfill $\square$ \par
             Exercice 6 \hfill $\star$ \hfill $\square$ \par
             Exercice 7 \hfill $\star\star$ \hfill $\square$ \par
             Exercice 8 \hfill $\star\star$ \hfill $\square$ \par
             Récréation \hfill $\star\star\star$ \hfill $\square$}}             
      \rput[l](13,0){%bulle SE : compétence 3
          \pspolygon[fillstyle=solid,fillcolor=B1,linecolor=B1](0,2)(-2.3,4.5)(0,2.35)
          \bulle
            {B1}
            {Compétence 3}
            {Exercice 1 \hfill $\star$ \hfill $\square$ \par
             Exercice 2 \hfill $\star\star$ \hfill $\square$ \par
             Exercice 3 \hfill $\star$ \hfill $\square$}}                  
\end{pspicture}


%%%%%%%%%% Séquence 14 %%%%%%%%%%
%%%%%%%%%%%%%%%%%%%%%%%%%%%
\begin{pspicture}(0.5,0.5)(18,12.5)            
   {\color{Red}
      \rput(9,5.75){\parbox{5cm}{\centering\large S14 \par COMPARAISON \par ET ÉGALITÉ \par DE FRACTIONS}}} %bulle centrale  
   \rput[l](0,8){%bulle NNO : connaissances et compétences
      \pspolygon[fillstyle=solid,fillcolor=A1,linecolor=A1](6,0)(8,-1.5)(6.4,0)
      \bullecours
         {A1}
         {Je connais mon cours}
         {C1 : Je reconnais et produis des fractions égales \hfill $\square$ \par
          C2 : Je compare, range, encadre des fractions dont les dénominateurs sont \par
          \hspace*{6mm} égaux ou multiples l’un de l’autre \hfill $\square$ \par
          C3 : Je résous des problèmes faisant intervenir des fractions \hfill $\square$}}         
   \rput[l](14,4){%bulle ENE : Aide vidéo
      \pspolygon[fillstyle=solid,fillcolor=A1,linecolor=A1](0,3.2)(-2.5,2)(0,3.5)
      \bulleQR
         {A1}
         {Aide en vidéo}
         {Fractions égales}
         {Simplifier une fraction}}
         \tikz[remember picture,overlay]{\node at (16,8.9) {\qrcode{https://www.yout-ube.com/watch?v=Ate81v_xUiY&list=PLVUDmbpupCaorU_NqVot4wsX1uyOIEKeG&index=3}};}
          \tikz[remember picture,overlay]{\node at (15.9,5.9) {\qrcode{https://www.yout-ube.com/watch?v=g5oV2wC6RfU&list=PLVUDmbpupCaorU_NqVot4wsX1uyOIEKeG&index=8}};}
      \rput[l](0,4){%bulle O : Questions flash
         \pspolygon[fillstyle=solid,fillcolor=Goldenrod,linecolor=Goldenrod](5,1.35)(6.5,1.5)(5,1.65)
         \bulle
            {Goldenrod}
            {Questions flash}
            {\psline[linecolor=darkgray](1.75,-0.5)(2.25,0.5)
             \rput(2.75,0){\darkgray\Huge 5}}}     
      \rput[l](0,0){%bulle SO : Compétence 1
         \pspolygon[fillstyle=solid,fillcolor=B1,linecolor=B1](5,2)(7.2,4.5)(5,2.35)
         \bulle
            {B1}
            {Compétence 1}
            {Exercice 4 \hfill $\star$ \hfill $\square$ \par
             Exercice 5 \hfill $\star\star$ \hfill $\square$ \par
             Exercice 6 \hfill $\star\star$ \hfill $\square$}}
      \rput[l](6.5,0){%bulle S : compétence 2
         \pspolygon[fillstyle=solid,fillcolor=B1,linecolor=B1](2.35,3)(2.5,4.5)(2.65,3)
         \bulle
            {B1}
            {Compétence 2}
            {Exercice 3 \hfill $\star\star$ \hfill $\square$ \par
             Exercice 7 \hfill $\star$ \hfill $\square$ \par
             Exercice 8 \hfill $\star\star$ \hfill $\square$}}             
      \rput[l](13,0){%bulle SE : compétence 3
          \pspolygon[fillstyle=solid,fillcolor=B1,linecolor=B1](0,2)(-2.3,4.5)(0,2.35)
          \bulle
            {B1}
            {Compétence 3}
            {Activité A \hfill $\star\star$ \hfill $\square$ \par
             Exercice 1 \hfill $\star$ \hfill $\square$ \par
             Exercice 2 \hfill $\star$ \hfill $\square$ \par
             Exercice 9 \hfill $\star\star\star$ \hfill $\square$ \par
             Récréation  \hfill $\star\star$ \hfill $\square$}}                  
\end{pspicture}


%%%%%%%%%% Séquence 15 %%%%%%%%%%
%%%%%%%%%%%%%%%%%%%%%%%%%%%
\begin{pspicture}(0.5,0)(18,10)             
   {\color{DodgerBlue}
      \rput(9,5.75){\parbox{5cm}{\centering\large S15 \par L'INÉGALITÉ \par TRIANGULAIRE}}} %bulle centrale  
   \rput[l](0,8){%bulle NNO : connaissances et compétences
      \pspolygon[fillstyle=solid,fillcolor=A1,linecolor=A1](6,0)(8,-1.5)(6.4,0)
      \bullecours
         {A1}
         {Je connais mon cours}
         {C1 : Je connais et j'utilise l'inégalité triangulaire \hfill $\square$ \par
          C2 : Je construis des triangles avec contraintes \hfill $\square$ \par
          C3 : Je mène des raisonnements en utilisant l'inégalité triangulaire ou des \par
          \hspace*{6mm} propriétés des figures \hfill $\square$}}         
   \rput[l](14,4){%bulle ENE : Aide vidéo
      \pspolygon[fillstyle=solid,fillcolor=A1,linecolor=A1](0,3.2)(-2.5,2)(0,3.5)
      \bulleQR
         {A1}
         {Aide en vidéo}
         {Triangle constructible}
         {Triangle non constructible}}
         \tikz[remember picture,overlay]{\node at (16,8.9) {\qrcode{https://www.yout-ube.com/watch?v=hwCjjX6R2XM&list=PLVUDmbpupCaqW33IMWG2n_73O4Jy7GEse&index=7}};}
          \tikz[remember picture,overlay]{\node at (15.9,5.9) {\qrcode{https://www.yout-ube.com/watch?v=JPinXSVQGWE&list=PLVUDmbpupCaqW33IMWG2n_73O4Jy7GEse&index=5} };}
      \rput[l](0,4){%bulle O : Questions flash
         \pspolygon[fillstyle=solid,fillcolor=Goldenrod,linecolor=Goldenrod](5,1.35)(6.5,1.5)(5,1.65)
         \bulle
            {Goldenrod}
            {Questions flash}
            {\psline[linecolor=darkgray](1.75,-0.5)(2.25,0.5)
             \rput(2.75,0){\darkgray\Huge 5}}}     
      \rput[l](0,0){%bulle SO : Compétence 1
         \pspolygon[fillstyle=solid,fillcolor=B1,linecolor=B1](5,2)(7.2,4.5)(5,2.35)
         \bulle
            {B1}
            {Compétence 1}
            {Activité A \hfill $\star\star$ \hfill $\square$ \par
             Exercice 1 \hfill $\star$ \hfill $\square$ \par
             Exercice 2 \hfill $\star\star$ \hfill $\square$ \par
             Exercice 3 \hfill $\star\star$ \hfill $\square$}}
      \rput[l](6.5,0){%bulle S : compétence 2
         \pspolygon[fillstyle=solid,fillcolor=B1,linecolor=B1](2.35,3)(2.5,4.5)(2.65,3)
         \bulle
            {B1}
            {Compétence 2}
            {Exercice 4 \hfill $\star$ \hfill $\square$ \par
             Exercice 5 \hfill $\star\star$ \hfill $\square$}}             
      \rput[l](13,0){%bulle SE : compétence 3
          \pspolygon[fillstyle=solid,fillcolor=B1,linecolor=B1](0,2)(-2.3,4.5)(0,2.35)
          \bulle
            {B1}
            {Compétence 3}
            {Exercice 6 \hfill $\star\star$ \hfill $\square$ \par
             Exercice 7 \hfill $\star\star\star$ \hfill $\square$ \par
             Récréation \hfill $\star\star$ \hfill $\square$}}                  
\end{pspicture}


%%%%%%%%%% Séquence 16 %%%%%%%%%%
%%%%%%%%%%%%%%%%%%%%%%%%%%%
\begin{pspicture}(0.5,0.5)(18,12.5)            
   {\color{violet}
      \rput(9,5.75){\parbox{5cm}{\centering\large S16 \par PROPORTIONNALITÉ}}} %bulle centrale  
   \rput[l](0,8){%bulle NNO : connaissances et compétences
      \pspolygon[fillstyle=solid,fillcolor=A1,linecolor=A1](6,0)(8,-1.5)(6.4,0)
      \bullecours
         {A1}
         {Je connais mon cours}
         {C1 : Je reconnais et je mets en \oe uvre des procédures de proportionnalité \hfill $\square$ \par
          C2 : Je résous des problèmes de pourcentages \hfill $\square$ \par
          C3 : Je résous des problèmes d'échelle, d'agrandissement-réduction et de proportion \hfill $\square$}}         
   \rput[l](14,4){%bulle ENE : Aide vidéo
      \pspolygon[fillstyle=solid,fillcolor=A1,linecolor=A1](0,3.2)(-2.5,2)(0,3.5)
      \bulleQR
         {A1}
         {Aide en vidéo}
         {Utiliser une échelle}
         {Appliquer un pourcentage}}
         \tikz[remember picture,overlay]{\node at (16,8.9) {\qrcode{https://www.yout-ube.com/watch?v=-nKF5P_xxyQ&feature=youtu.be}};}
          \tikz[remember picture,overlay]{\node at (15.9,5.9) {\qrcode{https://www.yout-ube.com/watch?v=2UVaPRdSMl0}};}
      \rput[l](0,4){%bulle O : Questions flash
         \pspolygon[fillstyle=solid,fillcolor=Goldenrod,linecolor=Goldenrod](5,1.35)(6.5,1.5)(5,1.65)
         \bulle
            {Goldenrod}
            {Questions flash}
            {\psline[linecolor=darkgray](1.75,-0.5)(2.25,0.5)
             \rput(2.75,0){\darkgray\Huge 5}}}     
      \rput[l](0,0){%bulle SO : Compétence 1
         \pspolygon[fillstyle=solid,fillcolor=B1,linecolor=B1](5,2)(7.2,4.5)(5,2.35)
         \bulle
            {B1}
            {Compétence 1}
            {Exercice 1 \hfill $\star$ \hfill $\square$ \par
             Exercice 2 \hfill $\star$ \hfill $\square$ \par
             Exercice 3 \hfill $\star$ \hfill $\square$ \par
             Exercice 4 \hfill $\star\star$ \hfill $\square$ \par
             Exercice 9 \hfill $\star\star\star$ \hfill $\square$}}
      \rput[l](6.5,0){%bulle S : compétence 2
         \pspolygon[fillstyle=solid,fillcolor=B1,linecolor=B1](2.35,3)(2.5,4.5)(2.65,3)
         \bulle
            {B1}
            {Compétence 2}
            {Exercice 5 \hfill $\star\star$ \hfill $\square$ \par
             Exercice 6 \hfill $\star\star$ \hfill $\square$}}             
      \rput[l](13,0){%bulle SE : compétence 3
          \pspolygon[fillstyle=solid,fillcolor=B1,linecolor=B1](0,2)(-2.3,4.5)(0,2.35)
          \bulle
            {B1}
            {Compétence 3}
            {Activité A \hfill $\star\star$ \hfill $\square$ \par
             Exercice 7 \hfill $\star$ \hfill $\square$ \par
             Exercice 8 \hfill $\star\star$ \hfill $\square$ \par
             Récréation  \hfill $\star\star\star$ \hfill $\square$}}                  
\end{pspicture}


%%%%%%%%%% Séquence 17 %%%%%%%%%%
%%%%%%%%%%%%%%%%%%%%%%%%%%%
\begin{pspicture}(0.5,0)(18,10)             
   {\color{Red}
      \rput(9,5.75){\parbox{5cm}{\centering\large S17 \par DISTRIBUTIVITÉ  \par SIMPLE ET \par ÉGALITÉS}}} %bulle centrale  
   \rput[l](0,8){%bulle NNO : connaissances et compétences
      \pspolygon[fillstyle=solid,fillcolor=A1,linecolor=A1](6,0)(8,-1.5)(6.4,0)
      \bullecours
         {A1}
         {Je connais mon cours}
         {C1 : J'utilise la distributivité simple pour réduire une expression littérale de \par
          \hspace*{6mm} la forme $ax+bx$ où $a$ et $b$ sont des nombres décimaux \hfill $\square$ \par
          C2 : J'utilise la distributivité simple pour effecteur des calculs numériques \hfill $\square$ \par
          C3 : Je teste si une égalité est vraie quand on lui attribue des valeurs numériques \hfill $\square$}}         
   \rput[l](14,4){%bulle ENE : Aide vidéo
      \pspolygon[fillstyle=solid,fillcolor=A1,linecolor=A1](0,3.2)(-2.5,2)(0,3.5)
      \bulleQR
         {A1}
         {Aide en vidéo}
         {Distributivité et calculs}
         {Tester une égalité}}
         \tikz[remember picture,overlay]{\node at (16,8.9) {\qrcode{https://www.yout-ube.com/watch?v=Jdvi2WbIkjo}};}
          \tikz[remember picture,overlay]{\node at (15.9,5.9) {\qrcode{https://www.yout-ube.com/watch?v=xZCXVgGT_Bk}};}  
      \rput[l](0,4){%bulle O : Questions flash
         \pspolygon[fillstyle=solid,fillcolor=Goldenrod,linecolor=Goldenrod](5,1.35)(6.5,1.5)(5,1.65)
         \bulle
            {Goldenrod}
            {Questions flash}
            {\psline[linecolor=darkgray](1.75,-0.5)(2.25,0.5)
             \rput(2.75,0){\darkgray\Huge 5}}}     
      \rput[l](0,0){%bulle SO : Compétence 1
         \pspolygon[fillstyle=solid,fillcolor=B1,linecolor=B1](5,2)(7.2,4.5)(5,2.35)
         \bulle
            {B1}
            {Compétence 1}
            {Activité A \hfill $\star\star$ \hfill $\square$ \par
             Exercice 1 \hfill $\star$ \hfill $\square$ \par
             Exercice 10 \hfill $\star\star\star$ \hfill $\square$ \par
             Récréation \hfill $\star\star$ \hfill $\square$}}
      \rput[l](6.5,0){%bulle S : compétence 2
         \pspolygon[fillstyle=solid,fillcolor=B1,linecolor=B1](2.35,3)(2.5,4.5)(2.65,3)
         \bulle
            {B1}
            {Compétence 2}
            {Exercice 2 \hfill $\star$ \hfill $\square$ \par
             Exercice 3 \hfill $\star\star$ \hfill $\square$ \par
             Exercice 4 \hfill $\star$ \hfill $\square$ \par
             Exercice 5 \hfill $\star\star$ \hfill $\square$ \par
             Exercice 6 \hfill $\star\star$ \hfill $\square$}}   
      \rput[l](13,0){%bulle SE : compétence 3
          \pspolygon[fillstyle=solid,fillcolor=B1,linecolor=B1](0,2)(-2.3,4.5)(0,2.35)
          \bulle
            {B1}
            {Compétence 3}
            {Exercice 7 \hfill $\star$ \hfill $\square$ \par
             Exercice 8 \hfill $\star\star$ \hfill $\square$ \par
             Exercice 9 \hfill $\star$ \hfill $\square$}}                  
\end{pspicture}


%%%%%%%%%% Séquence 18 %%%%%%%%%%
%%%%%%%%%%%%%%%%%%%%%%%%%%%
\begin{pspicture}(0.5,0.5)(18,12.5)           
   {\color{DodgerBlue}
      \rput(9,5.75){\parbox{5cm}{\centering\large S18 \par RECONNAÎTRE \par DES SOLIDES}}} %bulle centrale  
   \rput[l](0,8){%bulle NNO : connaissances et compétences
      \pspolygon[fillstyle=solid,fillcolor=A1,linecolor=A1](6,0)(8,-1.5)(6.4,0)
      \bullecours
         {A1}
         {Je connais mon cours}
         {C1 : Je reconnais un pavé droit, un cube, un cylindre, un prisme droit, une \par
         \hspace*{6mm} pyramide, un cône, une boule \hfill $\square$ \par
         C2 : Je connais les différentes caractéristiques des solides : faces, arêtes, sommets, vues \hfill $\square$}}         
   \rput[l](14,4){%bulle ENE : Aide vidéo
      \pspolygon[fillstyle=solid,fillcolor=A1,linecolor=A1](0,3.2)(-2.5,2)(0,3.5)
      \bulleQR
         {A1}
         {Aide en vidéo}
         {Prismes et pyramides}
         {Décrire le pavé droit}}
         \tikz[remember picture,overlay]{\node at (16,8.9) {\qrcode{https://lesfondamentaux.reseau-canope.fr/video/mathematiques/solides/tri-prismes-pyramides/distinguer-prisme-et-pyramide}};}
          \tikz[remember picture,overlay]{\node at (15.9,5.9) {\qrcode{https://lesfondamentaux.reseau-canope.fr/video/mathematiques/solides/paves-droits/decrire-le-pave-droit}};}
      \rput[l](0,4){%bulle O : Questions flash
         \pspolygon[fillstyle=solid,fillcolor=Goldenrod,linecolor=Goldenrod](5,1.35)(6.5,1.5)(5,1.65)
         \bulle
            {Goldenrod}
            {Questions flash}
            {\psline[linecolor=darkgray](1.75,-0.5)(2.25,0.5)
             \rput(2.75,0){\darkgray\Huge 5}}}    
      \rput[l](0,0){%bulle SSO : Compétence 1
         \pspolygon[fillstyle=solid,fillcolor=B1,linecolor=B1](5,3)(8,4.5)(5.5,3)
         \bullelongue
            {B1}
            {Compétence 1}
            {Activité A \hfill $\star\star$ \hfill $\square$ \par
             Exercice 1 \hfill $\star$ \hfill $\square$}}
      \rput[l](9.75,0){%bulle S : compétence 2
         \pspolygon[fillstyle=solid,fillcolor=B1,linecolor=B1](3,3)(0.5,4.5)(3.5,3)
         \bullelongue
            {B1}
            {Compétence 2}
            {Exercice 2 \hfill $\star$ \hfill $\square$ \par
            Récréation \hfill $\star$ \hfill $\square$}}                    
\end{pspicture}


%%%%%%%%%% Séquence 19 %%%%%%%%%%
%%%%%%%%%%%%%%%%%%%%%%%%%%%
\begin{pspicture}(0.5,0)(18,10)             
   {\color{Green}
      \rput(9,5.75){\parbox{5cm}{\centering\large S19 \par VOLUME DU \par PRISME ET DU \par CYLINDRE}}} %bulle centrale  
   \rput[l](0,8){%bulle NNO : connaissances et compétences
      \pspolygon[fillstyle=solid,fillcolor=A1,linecolor=A1](6,0)(8,-1.5)(6.4,0)
      \bullecours
         {A1}
         {Je connais mon cours}
         {C1 : Je calcule un volume par dénombrement et j'effectue des conversions \hfill $\square$ \par
          C2 : Je calcule le volume d’un pavé droit, d’un prisme droit, d’un cylindre \hfill $\square$ \par
          C3 : Je résous des problèmes impliquant des volumes \hfill $\square$}}         
   \rput[l](14,4){%bulle ENE : Aide vidéo
      \pspolygon[fillstyle=solid,fillcolor=A1,linecolor=A1](0,3.2)(-2.5,2)(0,3.5)
      \bulleQR
         {A1}
         {Aide en vidéo}
         {Volume du prisme}
         {Volume du cylindre}}
         \tikz[remember picture,overlay]{\node at (16,8.9) {\qrcode{https://www.yout-ube.com/watch?v=lsAWODx566E&list=PLVUDmbpupCaqaUVS62rEOogUuKuQLPNJT&index=4}};}
          \tikz[remember picture,overlay]{\node at (15.9,5.9) {\qrcode{https://www.yout-ube.com/watch?v=eJ8BSaTIpYU&list=PLVUDmbpupCaqaUVS62rEOogUuKuQLPNJT&index=7}};}
      \rput[l](0,4){%bulle O : Questions flash
         \pspolygon[fillstyle=solid,fillcolor=Goldenrod,linecolor=Goldenrod](5,1.35)(6.5,1.5)(5,1.65)
         \bulle
            {Goldenrod}
            {Questions flash}
            {\psline[linecolor=darkgray](1.75,-0.5)(2.25,0.5)
             \rput(2.75,0){\darkgray\Huge 5}}}     
      \rput[l](0,0){%bulle SO : Compétence 1
         \pspolygon[fillstyle=solid,fillcolor=B1,linecolor=B1](5,2)(7.2,4.5)(5,2.35)
         \bulle
            {B1}
            {Compétence 1}
            {Exercice 1 \hfill $\star$ \hfill $\square$ \par
             Exercice 3 \hfill $\star\star$ \hfill $\square$ \par
             Exercice 4 \hfill $\star\star$ \hfill $\square$}}
      \rput[l](6.5,0){%bulle S : compétence 2
         \pspolygon[fillstyle=solid,fillcolor=B1,linecolor=B1](2.35,3)(2.5,4.5)(2.65,3)
         \bulle
            {B1}
            {Compétence 2}
            {Exercice 2 \hfill $\star$ \hfill $\square$ \par
             Exercice 5 \hfill $\star\star$ \hfill $\square$ \par
             Exercice 6 \hfill $\star\star$ \hfill $\square$}}   
      \rput[l](13,0){%bulle SE : compétence 3
          \pspolygon[fillstyle=solid,fillcolor=B1,linecolor=B1](0,2)(-2.3,4.5)(0,2.35)
          \bulle
            {B1}
            {Compétence 3}
            {Activité A \hfill $\star\star\star$ \hfill $\square$ \par
             Exercice 7 \hfill $\star\star$ \hfill $\square$ \par
             Récréation \hfill $\star\star$ \hfill $\square$}}                  
\end{pspicture}


%%%%%%%%%% Séquence 20 %%%%%%%%%%
%%%%%%%%%%%%%%%%%%%%%%%%%%%
\begin{pspicture}(0.5,0.5)(18,12.5)           
   {\color{Red}
      \rput(9,5.75){\parbox{5cm}{\centering\large S20 \par SOMME ET \par DIFFÉRENCE DE \par NOMBRES RELATIFS}}} %bulle centrale  
   \rput[l](0,8){%bulle NNO : connaissances et compétences
      \pspolygon[fillstyle=solid,fillcolor=A1,linecolor=A1](6,0)(8,-1.5)(6.4,0)
      \bullecours
         {A1}
         {Je connais mon cours}
         {C1 : J'additionne et je soustrais des nombres relatifs \hfill $\square$ \par
         C2 : Je résous des problèmes avec des nombres relatifs \hfill $\square$}}         
   \rput[l](14,4){%bulle ENE : Aide vidéo
      \pspolygon[fillstyle=solid,fillcolor=A1,linecolor=A1](0,3.2)(-2.5,2)(0,3.5)
      \bulleQR
         {A1}
         {Aide en vidéo}
         {Sommes et différences}
         {Calculs avec parenthèses}}
         \tikz[remember picture,overlay]{\node at (16,8.9) {\qrcode{https://www.yout-ube.com/watch?v=9L4lz1NMPoY&list=PLVUDmbpupCaq9IBuon2YB9q3yChTQ20b3&index=2}};}
          \tikz[remember picture,overlay]{\node at (15.9,5.9) {\qrcode{https://www.yout-ube.com/watch?v=ZjrmsHRKajg&list=PLVUDmbpupCaq9IBuon2YB9q3yChTQ20b3&index=5}};}
      \rput[l](0,4){%bulle O : Questions flash
         \pspolygon[fillstyle=solid,fillcolor=Goldenrod,linecolor=Goldenrod](5,1.35)(6.5,1.5)(5,1.65)
         \bulle
            {Goldenrod}
            {Questions flash}
            {\psline[linecolor=darkgray](1.75,-0.5)(2.25,0.5)
             \rput(2.75,0){\darkgray\Huge 5}}}    
      \rput[l](0,0){%bulle SSO : Compétence 1
         \pspolygon[fillstyle=solid,fillcolor=B1,linecolor=B1](5,3)(8,4.5)(5.5,3)
         \bullelongue
            {B1}
            {Compétence 1}
            {Activité A \hfill $\star$ \hfill $\square$ \par
             Exercice 1 \hfill $\star$ \hfill $\square$ \par
             Exercice 2 \hfill $\star$ \hfill $\square$ \par
             Exercice 3 \hfill $\star\star$ \hfill $\square$ \par
             Exercice 4 \hfill $\star\star$ \hfill $\square$}}
      \rput[l](9.75,0){%bulle S : compétence 2
         \pspolygon[fillstyle=solid,fillcolor=B1,linecolor=B1](3,3)(0.5,4.5)(3.5,3)
         \bullelongue
            {B1}
            {Compétence 2}
            {Exercice 5 \hfill $\star\star$ \hfill $\square$ \par
             Exercice 6 \hfill $\star$ \hfill $\square$ \par
             Exercice 7 \hfill $\star\star$ \hfill $\square$ \par
             Exercice 8 \hfill $\star\star\star$ \hfill $\square$ \par
             Récréation \hfill $\star\star\star$ \hfill $\square$}}                    
\end{pspicture}


%%%%%%%%%% Séquence 21 %%%%%%%%%%
%%%%%%%%%%%%%%%%%%%%%%%%%%%
\begin{pspicture}(0.5,0)(18,10)             
   {\color{DodgerBlue}
      \rput(9,5.75){\parbox{5cm}{\centering\large S21 \par LE \par PARALLÉLOGRAMME}}} %bulle centrale  
   \rput[l](0,8){%bulle NNO : connaissances et compétences
      \pspolygon[fillstyle=solid,fillcolor=A1,linecolor=A1](6,0)(8,-1.5)(6.4,0)
      \bullecours
         {A1}
         {Je connais mon cours}
         {C1 : J'utilise les propriétés des quadrilatères \hfill $\square$ \par
          C2 : Je construis un parallélogramme \hfill $\square$ \par
          C3 : J'effectue un démonstration en utilisant les propriétés du parallélogramme  \hfill $\square$}}         
   \rput[l](14,4){%bulle ENE : Aide vidéo
      \pspolygon[fillstyle=solid,fillcolor=A1,linecolor=A1](0,3.2)(-2.5,2)(0,3.5)
      \bulleQR
         {A1}
         {Aide en vidéo}
         {Parallélogramme et côtés}
         {Parallélogramme et angles}}
         \tikz[remember picture,overlay]{\node at (16,8.9) {\qrcode{https://www.yout-ube.com/watch?v=BMEBEpdIVAw&list=PLVUDmbpupCareSDhV1nIRYEI3MBvQMIIY&index=3}};}
          \tikz[remember picture,overlay]{\node at (15.9,5.9) {\qrcode{https://www.yout-ube.com/watch?v=ornl3k7VbNk&list=PLVUDmbpupCareSDhV1nIRYEI3MBvQMIIY&index=5}};}
      \rput[l](0,4){%bulle O : Questions flash
         \pspolygon[fillstyle=solid,fillcolor=Goldenrod,linecolor=Goldenrod](5,1.35)(6.5,1.5)(5,1.65)
         \bulle
            {Goldenrod}
            {Questions flash}
            {\psline[linecolor=darkgray](1.75,-0.5)(2.25,0.5)
             \rput(2.75,0){\darkgray\Huge 5}}}     
      \rput[l](0,0){%bulle SO : Compétence 1
         \pspolygon[fillstyle=solid,fillcolor=B1,linecolor=B1](5,2)(7.2,4.5)(5,2.35)
         \bulle
            {B1}
            {Compétence 1}
            {Activité A \hfill $\star\star$ \hfill $\square$ \par
             Exercice 1 \hfill $\star$ \hfill $\square$ \par
             Exercice 2 \hfill $\star$ \hfill $\square$ \par
             Récréation  \hfill $\star\star$ \hfill $\square$}}
      \rput[l](6.5,0){%bulle S : compétence 2
         \pspolygon[fillstyle=solid,fillcolor=B1,linecolor=B1](2.35,3)(2.5,4.5)(2.65,3)
         \bulle
            {B1}
            {Compétence 2}
            {Exercice 3 \hfill $\star$ \hfill $\square$ \par
             Exercice 4 \hfill $\star\star$ \hfill $\square$ \par
             Exercice 5 \hfill $\star\star$ \hfill $\square$ \par
             Exercice 6 \hfill $\star\star$ \hfill $\square$}}             
      \rput[l](13,0){%bulle SE : compétence 3
          \pspolygon[fillstyle=solid,fillcolor=B1,linecolor=B1](0,2)(-2.3,4.5)(0,2.35)
          \bulle
            {B1}
            {Compétence 3}
            {Exercice 7 \hfill $\star\star$ \hfill $\square$ \par
             Exercice 8 \hfill $\star\star\star$ \hfill $\square$}}                  
\end{pspicture}


%%%%%%%%%% Séquence 22 %%%%%%%%%%
%%%%%%%%%%%%%%%%%%%%%%%%%%%
\begin{pspicture}(0.5,0.5)(18,12.5)           
   {\color{violet}
      \rput(9,5.75){\parbox{5cm}{\centering\large S22 \par LE RATIO}}} %bulle centrale  
   \rput[l](0,8){%bulle NNO : connaissances et compétences
      \pspolygon[fillstyle=solid,fillcolor=A1,linecolor=A1](6,0)(8,-1.5)(6.4,0)
      \bullecours
         {A1}
         {Je connais mon cours}
         {C1 : Je produis des ratios \hfill $\square$ \par
          C2 : J'utilise des ratios \hfill $\square$}}         
   \rput[l](14,4){%bulle ENE : Aide vidéo
      \pspolygon[fillstyle=solid,fillcolor=A1,linecolor=A1](0,3.2)(-2.5,2)(0,3.5)
      \bulleQR
         {A1}
         {Aide en vidéo}
         {Partager suivant un ratio}
         {Utiliser les ratios}}
         \tikz[remember picture,overlay]{\node at (16,8.9) {\qrcode{https://www.yout-ube.com/watch?v=LMwUa5oV1fw&list=PLVUDmbpupCao94NJIhVzR96gbJ1Srzxk-&index=4}};}
          \tikz[remember picture,overlay]{\node at (15.9,5.9) {\qrcode{https://www.yout-ube.com/watch?v=xEl2BCqVhN4&list=PLVUDmbpupCao94NJIhVzR96gbJ1Srzxk-&index=5}};} 
      \rput[l](0,4){%bulle O : Questions flash
         \pspolygon[fillstyle=solid,fillcolor=Goldenrod,linecolor=Goldenrod](5,1.35)(6.5,1.5)(5,1.65)
         \bulle
            {Goldenrod}
            {Questions flash}
            {\psline[linecolor=darkgray](1.75,-0.5)(2.25,0.5)
             \rput(2.75,0){\darkgray\Huge 5}}}    
      \rput[l](0,0){%bulle SSO : Compétence 1
         \pspolygon[fillstyle=solid,fillcolor=B1,linecolor=B1](5,3)(8,4.5)(5.5,3)
         \bullelongue
            {B1}
            {Compétence 1}
            {\begin{multicols}{2}
                Activité A \hfill $\star$ \hfill $\square$ \par
                Exercice 1 \hfill $\star$ \hfill $\square$ \par
                Exercice 2 \hfill $\star$ \hfill $\square$ \par
                Exercice 3 \hfill $\star$ \hfill $\square$ \par
                Exercice 4 \hfill $\star\star$ \hfill $\square$ \par
                Exercice 9 \hfill $\star\star$ \hfill $\square$ \par
                Récréation \hfill $\star\star$ \hfill $\square$
             \end{multicols}}}
      \rput[l](9.75,0){%bulle S : compétence 2
         \pspolygon[fillstyle=solid,fillcolor=B1,linecolor=B1](3,3)(0.5,4.5)(3.5,3)
         \bullelongue
            {B1}
            {Compétence 2}
            {\begin{multicols}{2}
                Activité A \hfill $\star$ \hfill $\square$ \par
                Exercice 5 \hfill $\star\star$ \hfill $\square$ \par
                Exercice 6 \hfill $\star\star$ \hfill $\square$ \par
                Exercice 7 \hfill $\star\star$ \hfill $\square$ \par
                Exercice 8 \hfill $\star\star\star$ \hfill $\square$ \par
                Exercice 9 \hfill $\star\star$ \hfill $\square$ \par
                Récréation \hfill $\star\star$ \hfill $\square$
             \end{multicols}}}                    
\end{pspicture}


%%%%%%%%%% Séquence 23 %%%%%%%%%%
%%%%%%%%%%%%%%%%%%%%%%%%%%%
\begin{pspicture}(0.5,0)(18,10)             
   {\color{Red}
      \rput(9,5.75){\parbox{5cm}{\centering\large S23 \par NOMBRES \par PREMIERS}}} %bulle centrale  
   \rput[l](0,8){%bulle NNO : connaissances et compétences
      \pspolygon[fillstyle=solid,fillcolor=A1,linecolor=A1](6,0)(8,-1.5)(6.4,0)
      \bullecours
         {A1}
         {Je connais mon cours}
         {C1 : Je détermine si un nombre est premier ou non et je décompose un nombre premier \hfill $\square$ \par
          C2 : J'utilise la décomposition en facteurs premiers pour produire des fractions égales \hfill $\square$ \par
          C3 : Je résous des problèmes en relation avec les nombres premiers  \hfill $\square$}}         
   \rput[l](14,4){%bulle ENE : Aide vidéo
      \pspolygon[fillstyle=solid,fillcolor=A1,linecolor=A1](0,3.2)(-2.5,2)(0,3.5)
      \bulleQR
         {A1}
         {Aide en vidéo}
         {Nombre premier}
         {Simplifier une fraction}}
         \tikz[remember picture,overlay]{\node at (16,8.9) {\qrcode{https://www.yout-ube.com/watch?v=g9PLLhnCv88&list=PLVUDmbpupCappZMuDL7e9MdUfjuOZMwdR&index=5}};}
          \tikz[remember picture,overlay]{\node at (15.9,5.9) {\qrcode{https://www.yout-ube.com/watch?v=HkqUaPYgwQM&list=PLVUDmbpupCappZMuDL7e9MdUfjuOZMwdR&index=9}};}
      \rput[l](0,4){%bulle O : Questions flash
         \pspolygon[fillstyle=solid,fillcolor=Goldenrod,linecolor=Goldenrod](5,1.35)(6.5,1.5)(5,1.65)
         \bulle
            {Goldenrod}
            {Questions flash}
            {\psline[linecolor=darkgray](1.75,-0.5)(2.25,0.5)
             \rput(2.75,0){\darkgray\Huge 5}}}     
      \rput[l](0,0){%bulle SO : Compétence 1
         \pspolygon[fillstyle=solid,fillcolor=B1,linecolor=B1](5,2)(7.2,4.5)(5,2.35)
         \bulle
            {B1}
            {Compétence 1}
            {Activité A \hfill $\star$ \hfill $\square$ \par
             Exercice 1 \hfill $\star$ \hfill $\square$ \par
             Exercice 2 \hfill $\star\star$ \hfill $\square$ \par
             Récréation  \hfill $\star\star$ \hfill $\square$}}
      \rput[l](6.5,0){%bulle S : compétence 2
         \pspolygon[fillstyle=solid,fillcolor=B1,linecolor=B1](2.35,3)(2.5,4.5)(2.65,3)
         \bulle
            {B1}
            {Compétence 2}
            {Exercice 3 \hfill $\star\star$ \hfill $\square$ \par
             Exercice 4 \hfill $\star\star$ \hfill $\square$ \par
             Exercice 5 \hfill $\star\star$ \hfill $\square$}}             
      \rput[l](13,0){%bulle SE : compétence 3
          \pspolygon[fillstyle=solid,fillcolor=B1,linecolor=B1](0,2)(-2.3,4.5)(0,2.35)
          \bulle
            {B1}
            {Compétence 3}
            {Exercice 6 \hfill $\star$ \hfill $\square$ \par
             Exercice 7 \hfill $\star\star$ \hfill $\square$ \par
             Exercice 8 \hfill $\star\star$ \hfill $\square$}}                  
\end{pspicture}


%%%%%%%%%% Séquence 24 %%%%%%%%%%
%%%%%%%%%%%%%%%%%%%%%%%%%%%
\begin{pspicture}(0.5,0.5)(18,12.5)           
   {\color{DodgerBlue}
      \rput(9,5.75){\parbox{5cm}{\centering\large S24 \par REPRÉSENTER  \par LE PAVÉ ET \par LE CYLINDRE}}} %bulle centrale  
   \rput[l](0,8){%bulle NNO : connaissances et compétences
      \pspolygon[fillstyle=solid,fillcolor=A1,linecolor=A1](6,0)(8,-1.5)(6.4,0)
      \bullecours
         {A1}
         {Je connais mon cours}
         {C1 : Je reconnais et je trace un pavé droit et un cylindre en perspective cavalière \hfill $\square$ \par
          C2 : Je construis le patron d'un pavé droit et d'un cylindre \hfill $\square$}}         
   \rput[l](14,4){%bulle ENE : Aide vidéo
      \pspolygon[fillstyle=solid,fillcolor=A1,linecolor=A1](0,3.2)(-2.5,2)(0,3.5)
      \bulleQR
         {A1}
         {Aide en vidéo}
         {Patron du pavé droit}
         {Patron du cylindre}}
         \tikz[remember picture,overlay]{\node at (16,8.9) {\qrcode{https://www.yout-ube.com/watch?v=WhwYCIcA220&list=PLVUDmbpupCaqEPQHrY1G0IgVgDBXI4oP2&index=2}};}
          \tikz[remember picture,overlay]{\node at (15.9,5.9) {\qrcode{https://www.yout-ube.com/watch?v=oRIISSBmdoI&list=PLVUDmbpupCaqaUVS62rEOogUuKuQLPNJT&index=5}};}
      \rput[l](0,4){%bulle O : Questions flash
         \pspolygon[fillstyle=solid,fillcolor=Goldenrod,linecolor=Goldenrod](5,1.35)(6.5,1.5)(5,1.65)
         \bulle
            {Goldenrod}
            {Questions flash}
            {\psline[linecolor=darkgray](1.75,-0.5)(2.25,0.5)
             \rput(2.75,0){\darkgray\Huge 5}}}    
      \rput[l](0,0){%bulle SSO : Compétence 1
         \pspolygon[fillstyle=solid,fillcolor=B1,linecolor=B1](5,3)(8,4.5)(5.5,3)
         \bullelongue
            {B1}
            {Compétence 1}
            {Activité A \hfill $\star\star\star$ \hfill $\square$ \par
             Exercice 1 \hfill $\star\star$ \hfill $\square$ \par
             Récréation \hfill $\star\star$ \hfill $\square$}}
      \rput[l](9.75,0){%bulle S : compétence 2
         \pspolygon[fillstyle=solid,fillcolor=B1,linecolor=B1](3,3)(0.5,4.5)(3.5,3)
         \bullelongue
            {B1}
            {Compétence 2}
            {Activité A \hfill $\star\star\star$ \hfill $\square$ \par
             Exercice 2 \hfill $\star\star$ \hfill $\square$ \par
             Exercice 3 \hfill $\star\star$ \hfill $\square$ \par
             Exercice 4 \hfill $\star$ \hfill $\square$ \par
             Exercice 5 \hfill $\star$ \hfill $\square$}}                    
\end{pspicture}


%%%%%%%%%% Séquence 25 %%%%%%%%%%
%%%%%%%%%%%%%%%%%%%%%%%%%%%
\begin{pspicture}(0.5,0)(18,10)             
   {\color{Green}
      \rput(9,5.75){\parbox{5cm}{\centering\large S25 \par L'AIRE DU \par PARALLÉLOGRAMME}}} %bulle centrale  
   \rput[l](0,8){%bulle NNO : connaissances et compétences
      \pspolygon[fillstyle=solid,fillcolor=A1,linecolor=A1](6,0)(8,-1.5)(6.4,0)
      \bullecours
         {A1}
         {Je connais mon cours}
         {C1 : Je calcule l'aire ou le périmètre d'un parallélogramme \hfill $\square$}}         
   \rput[l](14,4){%bulle ENE : Aide vidéo
      \pspolygon[fillstyle=solid,fillcolor=A1,linecolor=A1](0,3.2)(-2.5,2)(0,3.5)
      \bulleQR
         {A1}
         {Aide en vidéo}
         {Aire du parallélogramme}
         {Aire composée}}
         \tikz[remember picture,overlay]{\node at (16,8.9) {\qrcode{https://www.yout-ube.com/watch?v=BTLoR9iZXnM&list=PLVUDmbpupCareSDhV1nIRYEI3MBvQMIIY&index=7}};}
         \tikz[remember picture,overlay]{\node at (15.9,5.9) {\qrcode{https://www.youtube-nocookie.com/embed/vof06TmPcQk?list=PLVUDmbpupCarppgOGwJQh5OeCwS6R6f-Y&autoplay=1&iv_load_policy=3&loop=1&modestbranding=1&start=}};}
      \rput[l](0,4){%bulle O : Questions flash
         \pspolygon[fillstyle=solid,fillcolor=Goldenrod,linecolor=Goldenrod](5,1.35)(6.5,1.5)(5,1.65)
         \bulle
            {Goldenrod}
            {Questions flash}
            {\psline[linecolor=darkgray](1.75,-0.5)(2.25,0.5)
             \rput(2.75,0){\darkgray\Huge 5}}}     
      \rput[l](0,0){%bulle SO : Compétence 1
         \pspolygon[fillstyle=solid,fillcolor=B1,linecolor=B1](8.8,3)(9,4.5)(9.2,3)
         \bullemegalongue
            {B1}
            {Compétence 1}
            {\begin{multicols}{2}
                Activité A \hfill $\star\star$ \hfill $\square$ \par
                Exercice 1 \hfill $\star$ \hfill $\square$ \par
                Exercice 2 \hfill $\star\star$ \hfill $\square$ \par
                Exercice 3 \hfill $\star$ \hfill $\square$ \par
                Exercice 4 \hfill $\star\star$ \hfill $\square$ \par
                Exercice 5 \hfill $\star\star$ \hfill $\square$ \par
                Exercice 6 \hfill $\star\star$ \hfill $\square$ \par
                Récréation \hfill $\star\star$ \hfill $\square$
             \end{multicols}}}       
\end{pspicture}


%%%%%%%%%% Séquence 26 %%%%%%%%%%
%%%%%%%%%%%%%%%%%%%%%%%%%%%
\begin{pspicture}(0.5,0.5)(18,12.5)            
   {\color{Red}
      \rput(9,5.75){\parbox{5cm}{\centering\large S26 \par SOMME ET \par DIFFÉRENCE \par DE FRACTIONS}}} %bulle centrale  
   \rput[l](0,8){%bulle NNO : connaissances et compétences
      \pspolygon[fillstyle=solid,fillcolor=A1,linecolor=A1](6,0)(8,-1.5)(6.4,0)
      \bullecours
         {A1}
         {Je connais mon cours}
         {C1 : Je relie fractions, proportions et pourcentages \hfill $\square$ \par
          C2 : Je décompose une fraction sous la forme d’une somme ou d’une différence \par
             \hspace*{6mm} d’un entier et d’une fraction \hfill $\square$ \par
          C3 :  J'additionne ou je soustrais des fractions dont les dénominateurs sont \par
             \hspace*{6mm} égaux ou multiples l’un de l’autre \hfill $\square$}}         
   \rput[l](14,4){%bulle ENE : Aide vidéo
      \pspolygon[fillstyle=solid,fillcolor=A1,linecolor=A1](0,3.2)(-2.5,2)(0,3.5)
      \bulleQR
         {A1}
         {Aide en vidéo}
         {Additionner et soustraire}
         {des fractions}}
         \tikz[remember picture,overlay]{\node at (16,8.9) {\qrcode{https://www.youtube.com/watch?v=lGShZVQlXMQ&list=PLVUDmbpupCaorU_NqVot4wsX1uyOIEKeG&index=16}};}
          \tikz[remember picture,overlay]{\node at (15.9,5.9) {\qrcode{https://www.yout-ube.com/watch?v=9dxCWIdbXXU&list=PLVUDmbpupCaorU_NqVot4wsX1uyOIEKeG&index=17}};} 
      \rput[l](0,4){%bulle O : Questions flash
         \pspolygon[fillstyle=solid,fillcolor=Goldenrod,linecolor=Goldenrod](5,1.35)(6.5,1.5)(5,1.65)
         \bulle
            {Goldenrod}
            {Questions flash}
            {\psline[linecolor=darkgray](1.75,-0.5)(2.25,0.5)
             \rput(2.75,0){\darkgray\Huge 5}}}     
      \rput[l](0,0){%bulle SO : Compétence 1
         \pspolygon[fillstyle=solid,fillcolor=B1,linecolor=B1](5,2)(7.2,4.5)(5,2.35)
         \bulle
            {B1}
            {Compétence 1}
            {Exercice 7 \hfill $\star\star$ \hfill $\square$ \par
             Exercice 8 \hfill $\star\star\star$ \hfill $\square$}}
      \rput[l](6.5,0){%bulle S : compétence 2
         \pspolygon[fillstyle=solid,fillcolor=B1,linecolor=B1](2.35,3)(2.5,4.5)(2.65,3)
         \bulle
            {B1}
            {Compétence 2}
            {Activité \hfill $\star\star$ \hfill $\square$ \par
             Exercice 1 \hfill $\star$ \hfill $\square$ \par
             Exercice 2 \hfill $\star\star$ \hfill $\square$}}             
      \rput[l](13,0){%bulle SE : compétence 3
          \pspolygon[fillstyle=solid,fillcolor=B1,linecolor=B1](0,2)(-2.3,4.5)(0,2.35)
          \bulle
            {B1}
            {Compétence 3}
            {Exercice 3 \hfill $\star\star$ \hfill $\square$ \par
             Exercice 4 \hfill $\star$ \hfill $\square$ \par
             Exercice 5 \hfill $\star$ \hfill $\square$ \par
             Exercice 6 \hfill $\star\star$ \hfill $\square$ \par
             Récréation \hfill $\star\star\star$ \hfill $\square$}}
\end{pspicture}  
          

%%%%%%%%%% Séquence 27 %%%%%%%%%%
%%%%%%%%%%%%%%%%%%%%%%%%%%%
\begin{pspicture}(0.5,0)(18,10)           
   {\color{violet}
      \rput(9,5.75){\parbox{5cm}{\centering\large S27 \par FRÉQUENCE \par ET MOYENNE}}} %bulle centrale  
   \rput[l](0,8){%bulle NNO : connaissances et compétences
      \pspolygon[fillstyle=solid,fillcolor=A1,linecolor=A1](6,0)(8,-1.5)(6.4,0)
      \bullecours
         {A1}
         {Je connais mon cours}
         {C1 : Je calcule des effectifs, des fréquences, une moyenne \hfill $\square$ \par
          C2 : J'utilise et j'interprète des données statistiques \hfill $\square$}}         
   \rput[l](14,4){%bulle ENE : Aide vidéo
      \pspolygon[fillstyle=solid,fillcolor=A1,linecolor=A1](0,3.2)(-2.5,2)(0,3.5)
      \bulleQR
         {A1}
         {Aide en vidéo}
         {Calculer des fréquences}
         {Calculer une moyenne}}
         \tikz[remember picture,overlay]{\node at (16,8.9) {\qrcode{https://www.yout-ube.com/watch?v=MwNV5eCBFrI&list=PLVUDmbpupCaq-oSlU99muFSPnsApY1mHr&index=1}};}
          \tikz[remember picture,overlay]{\node at (15.9,5.9) {\qrcode{https://www.yout-ube.com/watch?v=U1NamiLxBaI&list=PLVUDmbpupCaq-oSlU99muFSPnsApY1mHr&index=4}};}
      \rput[l](0,4){%bulle O : Questions flash
         \pspolygon[fillstyle=solid,fillcolor=Goldenrod,linecolor=Goldenrod](5,1.35)(6.5,1.5)(5,1.65)
         \bulle
            {Goldenrod}
            {Questions flash}
            {\psline[linecolor=darkgray](1.75,-0.5)(2.25,0.5)
             \rput(2.75,0){\darkgray\Huge 5}}}    
      \rput[l](0,0){%bulle SSO : Compétence 1
         \pspolygon[fillstyle=solid,fillcolor=B1,linecolor=B1](5,3)(8,4.5)(5.5,3)
         \bullelongue
            {B1}
            {Compétence 1}
            {Exercice 1 \hfill $\star$ \hfill $\square$ \par
             Exercice 2 \hfill $\star\star$ \hfill $\square$ \par
             Exercice 3 \hfill $\star$ \hfill $\square$ \par
             Exercice 4 \hfill $\star\star$ \hfill $\square$}}
      \rput[l](9.75,0){%bulle S : compétence 2
         \pspolygon[fillstyle=solid,fillcolor=B1,linecolor=B1](3,3)(0.5,4.5)(3.5,3)
         \bullelongue
            {B1}
            {Compétence 2}
            {Activité A \hfill $\star$ \hfill $\square$ \par
             Récréation \hfill $\star\star$ \hfill $\square$}}                    
\end{pspicture}


%%%%%%%%%% Séquence 28 %%%%%%%%%%
%%%%%%%%%%%%%%%%%%%%%%%%%%%
\begin{pspicture}(0.5,0.5)(18,12.5)            
   {\color{DodgerBlue}
      \rput(9,5.75){\parbox{5cm}{\centering\large S28 \par LES DROITES \par DU TRIANGLE}}} %bulle centrale  
   \rput[l](0,8){%bulle NNO : connaissances et compétences
      \pspolygon[fillstyle=solid,fillcolor=A1,linecolor=A1](6,0)(8,-1.5)(6.4,0)
      \bullecours
         {A1}
         {Je connais mon cours}
         {C1 : Je construis les médiatrices d'un triangle \hfill $\square$ \par
          C2 : Je construis les hauteurs d'un triangle \hfill $\square$ \par
          C3 :  Je mène des raisonnements en utilisant les propriétés des médiatrices \par
             \hspace*{6mm} et des hauteurs \hfill $\square$}}         
   \rput[l](14,4){%bulle ENE : Aide vidéo
      \pspolygon[fillstyle=solid,fillcolor=A1,linecolor=A1](0,3.2)(-2.5,2)(0,3.5)
      \bulleQR
         {A1}
         {Aide en vidéo}
         {Médiatrice et hauteur}
         {Cercle circonscrit}}
         \tikz[remember picture,overlay]{\node at (16,8.9) {\qrcode{https://www.yout-ube.com/watch?v=NYKW2MHECnQ&list=PLVUDmbpupCaqW33IMWG2n_73O4Jy7GEse&index=8}};}
          \tikz[remember picture,overlay]{\node at (15.9,5.9) {\qrcode{https://www.youtube.com/watch?v=0h9bZZoQfJM&list=PLVUDmbpupCaqW33IMWG2n_73O4Jy7GEse&index=9}};}  
      \rput[l](0,4){%bulle O : Questions flash
         \pspolygon[fillstyle=solid,fillcolor=Goldenrod,linecolor=Goldenrod](5,1.35)(6.5,1.5)(5,1.65)
         \bulle
            {Goldenrod}
            {Questions flash}
            {\psline[linecolor=darkgray](1.75,-0.5)(2.25,0.5)
             \rput(2.75,0){\darkgray\Huge 5}}}     
      \rput[l](0,0){%bulle SO : Compétence 1
         \pspolygon[fillstyle=solid,fillcolor=B1,linecolor=B1](5,2)(7.2,4.5)(5,2.35)
         \bulle
            {B1}
            {Compétence 1}
            {Activité \hfill $\star\star$ \hfill $\square$ \par
             Exercice 1 \hfill $\star$ \hfill $\square$ \par
             Exercice 2 \hfill $\star\star$ \hfill $\square$}}
      \rput[l](6.5,0){%bulle S : compétence 2
         \pspolygon[fillstyle=solid,fillcolor=B1,linecolor=B1](2.35,3)(2.5,4.5)(2.65,3)
         \bulle
            {B1}
            {Compétence 2}
            {Exercice 3 \hfill $\star$ \hfill $\square$ \par
             Exercice 4 \hfill $\star\star$ \hfill $\square$}}             
      \rput[l](13,0){%bulle SE : compétence 3
          \pspolygon[fillstyle=solid,fillcolor=B1,linecolor=B1](0,2)(-2.3,4.5)(0,2.35)
          \bulle
            {B1}
            {Compétence 3}
            {Exercice 5 \hfill $\star\star$ \hfill $\square$ \par
             Exercice 6 \hfill $\star$ \hfill $\square$ \par
             Exercice 7 \hfill $\star\star\star$ \hfill $\square$ \par
             Récréation \hfill $\star\star$ \hfill $\square$}}
\end{pspicture}  

 
%%%%%%%%%% Séquence 29 %%%%%%%%%%
%%%%%%%%%%%%%%%%%%%%%%%%%%%
\begin{pspicture}(0.5,0)(18,10)           
   {\color{violet}
      \rput(9,5.75){\parbox{5cm}{\centering\large S29 \par PROPRIÉTÉS \par DES SYMÉTRIES}}} %bulle centrale  
   \rput[l](0,8){%bulle NNO : connaissances et compétences
      \pspolygon[fillstyle=solid,fillcolor=A1,linecolor=A1](6,0)(8,-1.5)(6.4,0)
      \bullecours
         {A1}
         {Je connais mon cours}
         {C1 : Je comprends l'effet des symétries : conservation du parallélisme, des longueurs \par
            \hspace*{6mm} et des angles \hfill $\square$ \par
          C2 : Je mène des raisonnements en utilisant des propriétés des symétries \hfill $\square$}}         
   \rput[l](14,4){%bulle ENE : Aide vidéo
      \pspolygon[fillstyle=solid,fillcolor=A1,linecolor=A1](0,3.2)(-2.5,2)(0,3.5)
      \bulleQR
         {A1}
         {Aide en vidéo}
         {Conservation des aires}
         {Centre et axe de symétrie}}
         \tikz[remember picture,overlay]{\node at (16,8.9) {\qrcode{https://www.yout-ube.com/watch?v=zEWQwYUMXZc&list=PLVUDmbpupCaq2_WKgsP0xJM0gOI1ZY6xK&index=7}};}
          \tikz[remember picture,overlay]{\node at (15.9,5.9) {\qrcode{https://www.yout-ube.com/watch?v=x2MqdM1t5Y4}};}
      \rput[l](0,4){%bulle O : Questions flash
         \pspolygon[fillstyle=solid,fillcolor=Goldenrod,linecolor=Goldenrod](5,1.35)(6.5,1.5)(5,1.65)
         \bulle
            {Goldenrod}
            {Questions flash}
            {\psline[linecolor=darkgray](1.75,-0.5)(2.25,0.5)
             \rput(2.75,0){\darkgray\Huge 5}}}    
      \rput[l](0,0){%bulle SSO : Compétence 1
         \pspolygon[fillstyle=solid,fillcolor=B1,linecolor=B1](5,3)(8,4.5)(5.5,3)
         \bullelongue
            {B1}
            {Compétence 1}
            {Activité A \hfill $\star\star$ \hfill $\square$ \par
             Exercice 1 \hfill $\star$ \hfill $\square$ \par
             Exercice 4 \hfill $\star$ \hfill $\square$ \par
             Récréation \hfill $\star\star$ \hfill $\square$}}
      \rput[l](9.75,0){%bulle S : compétence 2
         \pspolygon[fillstyle=solid,fillcolor=B1,linecolor=B1](3,3)(0.5,4.5)(3.5,3)
         \bullelongue
            {B1}
            {Compétence 2}
            {Exercice 2 \hfill $\star\star\star$ \hfill $\square$ \par
             Exercice 3 \hfill $\star\star$ \hfill $\square$ \par
             Exercice 5 \hfill $\star\star$ \hfill $\square$}}                    
\end{pspicture}
            
\end{center}

