  \ \\ [-5mm]
   \begin{enumerate}
      \item Les angles $\widehat{OSA} =\widehat{OSE}$ et $\widehat{AEN} =\widehat{SEN}$ sont alternes-internes et de même mesure, \udeg{38} donc : \\
         {\blue les droites $(OS)$ et $(EN)$ sont parallèles entre elles}.
      \item Le triangle $ANE$ est isocèle en $E$, les angles à sa base principale sont donc égaux d'où : {\blue $\widehat{ENA} =\widehat{EAN}$}. \\
         On calcule leur mesure : \\
         $\udeg{180}-\udeg{38} =\udeg{142}$ et $\udeg{142}\div2 =\udeg{71}$. \\
         D'où {\blue $\widehat{ENA} =\widehat{EAN} =\udeg{71}$}.
      \item Les angles $\widehat{NAE}$ et $\widehat{OAS}$ sont opposés par le sommet $A$, donc ils sont égaux.\\
         On a alors $\widehat{OAS} =\udeg{71}$. \\
         Les angles $\widehat{SOA} =\widehat{SON}$ et $\widehat{ENA}+\widehat{ENO}$ sont alternes-internes et les droites $(OS)$ et $(EN)$ sont parallèles entre elles. Les mesures des angles sont donc égales soit : $\widehat{SOA} =\widehat{ANE} =\udeg{71}$. \\
        Conclusion : {\blue le triangle $AOS$ est isocèle en $S$}.
   \end{enumerate}
