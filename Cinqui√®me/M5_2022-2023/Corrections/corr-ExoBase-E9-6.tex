   \begin{itemize}
      \item L'angle $\widehat{AMI}$ est le supplémentaire de l'angle $\widehat{xMA}$ de mesure $\udeg{127}$, donc il mesure $\udeg{180}-\udeg{127} =\udeg{53}$. Soit : {\blue $\widehat{AMI} =\udeg{53}$}.
      \item Les angles $\widehat{xMA}$ et $\widehat{MAD}$ sont alternes-internes et les droites $(DA)$ et $(MI)$ sont parallèles donc, ces deux angles sont égaux. Soit {\blue $\widehat{MAD} =\udeg{127}$}.
      \item Les angles $\widehat{AMI}$ et $\widehat{DAH}$ sont correspondants et les droites $(DA)$ et $(MI)$ sont parallèles, ils sont donc égaux et $\widehat{DAH} =\udeg{53}$. \\
         Dans le triangle $ADH$ : \\
         $\widehat{DAH}+\widehat{DHA}+\widehat{ADH} =\udeg{180}$. \\
         Donc, $\udeg{53}+\udeg{64}+\widehat{ADH} =\udeg{180}$ soit $\widehat{ADH} =\udeg{180}-\udeg{53}-\udeg{64} =\udeg{63}$. \\
          Les angles $\widehat{ADH}$ et $\widehat{ADI}$ sont supplémentaires donc, $\widehat{ADI} =\udeg{180}-\udeg{63} =\udeg{117}$. {\blue $\widehat{ADI} =\udeg{117}$}.
      \item Les angles $\widehat{ADH}$ et $\widehat{MID}$ sont correspondants et les droites $(DA)$ et $(MI)$ sont parallèles donc, ces deux angles sont égaux. Soit : {\blue $\widehat{MID} =\udeg{63}$}.
   \end{itemize}
