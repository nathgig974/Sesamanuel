\ \\ [-5mm]
   \begin{enumerate}
      \item {\blue vrai}, par lecture de la deuxième barre.
      \item {\blue faux}. Le diagramme ne nous permet pas de connaître le nombre d'élèves de ce collège puisqu'il s'agit des données de trois classes seulement.
      \item {\blue faux}. 15 élèves ont deux animaux de compagnie et  4 en ont trois. or, 15 n'est pas le triple de 4.
      \item {\blue faux}. $32+21+15 =68$. 68 élèves ont moins de trois animaux de compagnie.
      \item {\blue vrai}. 32 élèves n'ont pas d'animal de compagnie et 43 en ont au moins un.
      \item {\blue vrai}. Parmi les 43 élèves qui ont au moins un animal de compagnie, 22 en ont plusieurs, donc plus de la moitié.
   \end{enumerate}
