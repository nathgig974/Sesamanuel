   Montpellier est à la même UT que Paris. \\
   \begin{enumerate}
      \item \uh{20}\,\umin{00} \quad $\xrightarrow{+\umin{45}}$ \quad \uh{20}\,\umin{45}. \\
         Donc, Lucie termine son appel à \uh{20}\,45. \\ [1mm]
         \uh{20}\,\umin{45} \quad $\xrightarrow{-\uh{6}}$ \quad \uh{14}\,\umin{45}. \\
         À cette heure, il est {\blue \uh{14}\,45} à New-York.
      \item Pour New Delhi, il faut ajouter \uh{4}\,30 à l'heure de Paris, on peut décomposer ainsi : \\ [1mm]
         \uh{20}\,\umin{45} \quad $\xrightarrow{+\uh{4}}$ \quad \uh{24}\,\umin{45} = \uh{0}\,\umin{45}\\ [1mm]
         \uh{0}\,\umin{45} \quad $\xrightarrow{+\umin{30}}$ \quad \uh{1}\,\umin{15} \\ [1mm]
         Il est \uh{1}\,15 du matin donc, {\blue il n'est pas raisonnable d'appeler en Inde !}
   \end{enumerate}
