   \ \\ [-5mm]
   \begin{enumerate}
      \item On obtient le classement suivant :
      \begin{itemize}
         \item \blue les polyèdres : A, B, C ;
         \item \blue les non polyèdres : D, E, F, G.
      \end{itemize}
      \item On pourrait proposer un classement du type \og sucré, salé \fg{}, ou \og ça se mange - ça ne se mange pas \fg{} ou encore proposer un classement par couleur, par taille\dots{} ce qui n'est pas l'objectif attendu.
      \item Le rouleau de papier d'aluminium est un {\blue solide non fermé}.
      \item
      \begin{itemize}
         \item Pour les 5\up{e}1, il semble que les élèves ont \og repéré \fg{} les polyèdres A, B et C. \\
           Les boules forment une deuxième catégorie. \\
           Les autres emballages une troisième. \\
           On remarque que les cylindres et les cônes n'apparaissent pas dans le classement, sûrement en raison de leur forme mélangeant des faces planes comme pour les polyèdres A, B, C et des surfaces non planes comme pour les boules F.
         \item Pour les 5\up{e}2, les élèves ont classé ensemble les prismes et les cylindres, c'est-à-dire les solides ayant des faces opposées parallèles. \\
         Puis ils ont mis ensemble les solides \og pointus \fg{} comme les pyramides et les cônes. \\
         Enfin, les boules et autres emballages, constituent la catégorie des formes \og arrondies \fg{}, les cylindres et les cônes ayant déjà été classés.
      \end{itemize}
      \item Cette définition n'est pas pertinente : par exemple un cône peut rouler ou pas selon si on le pose sur sa base ou sur le côté. \\
      Un polyèdre régulier avec de multiples faces donne l'impression qu'il roule lorsqu'on le lance.
   \end{enumerate}
