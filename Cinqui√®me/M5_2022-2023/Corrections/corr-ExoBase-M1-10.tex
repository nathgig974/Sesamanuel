   Pour un volume à remplir, le robinet R1 met deux fois moins de temps que le robinet R2. Donc, pendant le temps total, R1 remplit deux \og unités de volume \fg{} pendant que R2 n'en remplit qu’une : \\ [1mm]
   \ModeleBarre[Largeur=2.5cm]{PowderBlue 3 {"\ul{1080}"}}{DeepSkyBlue 1 "R1" DeepSkyBlue 1 "R1" LightSkyBlue 1 "R2"} \\
   En partageant le volume total en trois parties égales, on trouve $\ul{1080}\div3 =\ul{360}$.
   R2 remplit un tiers du volume total du jacuzzi : \ul{360}. \\
   Or, $360\div12 =30$. Sachant que R2 remplit \ul{12} en \umin{1}, il remplit \ul{360} en \ul{30}. \\
   {\blue La réponse est 30 minutes (d)}. \bigskip
