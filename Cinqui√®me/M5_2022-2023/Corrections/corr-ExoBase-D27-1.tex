   \ \\ [-5mm]
   \begin{enumerate}
      \item
      \begin{enumerate}
         \item Tableau des effectifs et des fréquences en \% \\ \smallskip
         {\footnotesize
         \hautab{1.3}
         \begin{ltableau}{\linewidth}{12}
            \hline
            5 & 10 & 15 & 20 & 25 & 30 & 35 & 40 & 45 & 50 & 55 & 60 \\
            \hline
            1 & 3 & 2 & 2 & 4 & 7 & 5 & 4 & 3 & 2 & 0 & 3 \\
            \hline
            \!2,8 & \!8,3 & \!5,6 & \!5,6 & \!\!11,1 & \!\!19,4 & \!\!13,9 & \!\!11,1 & \!8,3 & \!5,6 & 0 & \!8,3 \\
            \hline
         \end{ltableau}}
         \item $\overline{m} =(1\times5+3\times10+2\times15+2\times20+4\times25+7\times30+5\times35+4\times40+3\times45+2\times50+0\times55+3\times60)\div36 =1\,165\div36 \approx32,4$. \\
         {\blue La moyenne est de \umin{32,4}}.
      \end{enumerate}
      \setcounter{enumi}{1}
      \item
      \begin{enumerate}
      \item Tableau des effectifs par classes. \\ \smallskip
      {\small
      \hautab{1.3}
      \begin{lctableau}{0.9\linewidth}{5}
         \hline
         Durée & $]\,0\,;\,15\,]$ & $]\,15\,;\,30\,]$ & $]\,30\,;\,45\,]$ & $]\,45\,;\,60\,]$ \\
         \hline
         Eff.& \quad 6 & \quad 13 & \quad 12 & \quad 5 \\
         \hline
      \end{lctableau}}
      \item $\overline{m} =(6\times7,5+13\times22,5+12\times37,5+5\times52,5)\div36 \approx 29,2$. {\blue La moyenne est de \umin{29,2}}. \\
         Ce résultat est différent de celui trouvé dans la question précédente car on perd en précision, puisqu'on ne s'occupe plus de la valeur exacte, mais de l'appartenance à un intervalle plus grand.
      \end{enumerate}
   \end{enumerate}
