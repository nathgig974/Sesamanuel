   Questions 1 à 4. Figure à l'échelle trois quarts : \\
   {\psset{unit=0.75}
   \begin{pspicture}(-4.5,-5)(5,5.3)
      \pscircle(0,0){3.5}
      \pscircle(0,0){5}
      \psline(3.5;240)(3.5;60)
      \psline(5;190)(5;10)
      \pspolygon[linecolor=blue](3.5;240)(5;190)(3.5;60)(5;10)
      \rput(3.8;240){N}
      \rput(5.3;190){Q}
      \rput(3.8;60){P}
      \rput(5.3;10){R}
      \rput(3.9;110){$(C_1)$}
      \rput(5.5;150){$(C_2)$}
      \rput{60}(0.5,1.5){\blue\ucm{3,5}}
      \rput{12}(2,0.7){\blue\ucm{5}}
      \rput(-0.2,0.4){$O$}
   \end{pspicture}}
   \begin{enumerate}
   \setcounter{enumi}{4}
      \item $O$ est le milieu du segment $[QR]$ et également le milieu du segment $[NP]$. Or, ces deux segments sont les diagonales du quadrilatère $NQPR$ donc, ses diagonales se coupent en leur milieu et par conséquent : {\blue $NQPR$ est un parallélogramme}.
      \item $NP =2OP =2\times\ucm{3,5} =\blue\ucm{7}$. \\
         $QR =2OR =2\times\ucm{5} =\blue\ucm{10}$.
   \end{enumerate}
