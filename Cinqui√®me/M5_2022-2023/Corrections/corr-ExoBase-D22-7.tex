\ \\ [-5mm]
   \begin{enumerate}
      \item Le ratio 10 : 6 : 5 pour la farine, le beurre et le sucre signifie que pour 10 parts de farine, on a 6 parts de beurre et 5 parts de sucre pour un total de 21 parts correspondant à une masse de \ug{630}. \\ [2mm]
           \quad  \Ratio[Figure,Longueur=7cm,TexteTotal=\ug{630},CouleurUn=LightSkyBlue,CouleurDeux=IndianRed,CouleurTrois=Gold]{10,6,5} \\
         $630\div21 =30$ donc, une part vaut \ug{30}. \\
         Il faut {\blue \ug{300} de farine, \ug{180} de beurre et \ug{150} de sucre.}
      \item Le ratio 3 : 4 : 5 pour Clémentine, Myrtille et Prune signifie que pour 3 parts pour Clémentine, on a 4 parts pour Myrtille et 5 parts pour Prune pour un total de 12 parts correspondant à \ueuro{120}. \\ [2mm]
           \quad  \Ratio[Figure,Longueur=7cm,TexteTotal=\ueuro{120},CouleurUn=LightSkyBlue,CouleurDeux=IndianRed,CouleurTrois=Gold]{3,4,5} \\
         $120\div12 =10$ donc, une part vaut \ueuro{10}. \\
         {\blue Clémentine recoit \ueuro{30}, Myrtille \ueuro{40} et Prune \ueuro{50}.}
   \end{enumerate}
