   \ \\ [-5mm]
   \begin{enumerate}
      \item Une vitesse de \ukmh{300 000} signifie que {\blue la lumière parcourt \ukm{300000} en une heure}.
      \item Le son parcourt \um{340} en une seconde, soit $6\times\um{340} =\um{2040}$ en 6 secondes. \\
         {\blue L'orage se situe à environ \ukm{2}}.
      \item $\ukm{8,5} =\um{8500}$. or, $\um{8500}\div\um{340} =25$ donc,  {\blue Erdogan entendra le tonnerre 25 secondes après l'avoir vu}.
      \item Le tonnerre met une seconde pour parcourir la distance de \um{340}, soit 3 secondes pour parcourir $3\times\um{340} =\um{1020} =\ukm{1,02}$. Cela signifie qu'à chaque fois que l'on compte 3 secondes, le son a parcouru \ukm{1}. C'est pourquoi {\blue il suffit de diviser le nombre de secondes par 3 pour obtenir la distance de l'orage}.
   \end{enumerate}

\Coupe

\corec{Quel degré est-il ?}

\begin{enumerate}
   \item Lorsqu'une aiguille parcourt la totalité du cadran, elle parcourt 360\degre. Donc, l'écart entre deux graduations des heures vaut 30\degre{} ($360\div12 =30$) et l'écart entre deux graduations des minutes vaut 6\degre{} ($360\div60 =6$).
   \begin{itemize}
      \item Il est 8 h\dots \\
         {\psset{unit=0.7}
          \begin{pspicture}(-5,-2.2)(2.5,2.2)
             \pscircle[linewidth=1mm](0,0){2}
             \multido{\i=0+6}{60}{\psline[linecolor=A1](1.8;\i)(2;\i)} %minutes
             \multido{\i=0+30}{12}{\psline[linecolor=B2](1.7;\i)(2;\i)} %heures
             \multido{\i=60+-30,\n=1+1}{12}{\rput(1.5;\i){\scriptsize\n}} %écritures
             \psline[linewidth=1mm,linecolor=B2]{->}(0,0)(1;-150)
             \psline[linewidth=1mm,linecolor=A1]{->}(0,0)(1.3;90)
             \psdot(0,0)
             \psarc[linecolor=J1]{<->} (0,0){1}{90}{-150}
            \rput(-0.5,0.25){\textcolor{J1}{\scriptsize 120\degre}}
         \end{pspicture}} \\
         À 8 h 00, la grande aiguille est sur le 12 et la petite aiguille est sur le 8. \\
         Il y a donc 4 graduations des heures entre les deux. \\
         Or, $4\times 30^\circ =120^\circ$. Donc, {\blue l'angle formé par les deux aiguilles lorsqu'il est 8 h 00 est de 120\degre}.
         \item Il est 10 h 30\dots \\
         {\psset{unit=0.7}
         \begin{pspicture}(-5,-2.2)(2.5,2.2)
            \pscircle[linewidth=1mm](0,0){2}
            \multido{\i=0+6}{60}{\psline[linecolor=A1](1.8;\i)(2;\i)} %minutes
            \multido{\i=0+30}{12}{\psline[linecolor=B2](1.7;\i)(2;\i)} %heures
            \multido{\i=60+-30,\n=1+1}{12}{\rput(1.5;\i){\scriptsize\n}} %écritures
            \psline[linewidth=1mm,linecolor=B2]{->}(0,0)(1;135)
            \psline[linewidth=1mm,linecolor=A1]{->}(0,0)(1.3;-90)
            \psdot(0,0)
            \psarc[linecolor=J1]{<->} (0,0){1}{135}{-90}
            \rput(-0.5,-0.25){\textcolor{J1}{\scriptsize 135\degre}}
          \end{pspicture}} \\
            À 10 h 30, la grande aiguille est sur le 6 et la petite aiguille est juste entre le 10 et le 11, il y a donc 4,5 graduations des heures entre les deux. \\
            Or, $4,5\times30^\circ =135^\circ$. Donc, {\blue l'angle formé par les deux aiguilles lorsqu'il est 10 h 30 est de 135\degre}.
         \item Il est 6h 20\dots \\
         {\psset{unit=0.7}
         \begin{pspicture}(-5,-2.2)(2.5,2.2)
            \pscircle[linewidth=1mm](0,0){2}
            \multido{\i=0+6}{60}{\psline[linecolor=A1](1.8;\i)(2;\i)} %minutes
            \multido{\i=0+30}{12}{\psline[linecolor=B2](1.7;\i)(2;\i)} %heures
            \multido{\i=60+-30,\n=1+1}{12}{\rput(1.5;\i){\scriptsize\n}} %écritures
            \psline[linewidth=1mm,linecolor=B2]{->}(0,0)(1;-100)
            \psline[linewidth=1mm,linecolor=A1]{->}(0,0)(1.3;-30)
            \psdot(0,0)
            \psarc[linecolor=J1]{<->} (0,0){1}{-100}{-30}
            \rput(0.35,-0.55){\textcolor{J1}{\scriptsize 70\degre}}
         \end{pspicture}} \\
         À 6 h 20, la grande aiguille est sur le 4 et la petite aiguille est entre le 6 et le 7, à un tiers du 6, il y a donc 2 graduations des heures plus un tiers entre les deux. \\
            Or, $2\times30^\circ+1\div3\times 30^\circ =60^\circ+10^\circ =70^\circ$. \\
            Donc, {\blue l'angle formé par les deux aiguilles lorsqu'il est 6 h 20 est de 70\degre}.
      \end{itemize}
   \end{enumerate}

\Coupe

   \begin{enumerate}
   \setcounter{enumi}{1}
      \item À chaque fois que l'aiguille des minutes parcourt 5 minutes (30\degre), l'aiguille des heures parcourt 2,5\degre ($30\div12 =2,5$). \\
      On choisit alors l'angle le plus petit entre les deux aiguilles et dans le tableau suivant, on indique l'angle le plus petit des aiguilles avec le \og 12 \fg. \\ [2mm]
      {\hautab{1.5}
      \begin{Ltableau}{0.9\linewidth}{4}{c}
         \hline
         heure & aiguille des minutes & aiguille des heures & angle entre les deux \\
         \hline
         0 h 00 & 0\degre & 0\degre & 0\degre \\
         0 h 05 & 30\degre & 2,5\degre & 27,5\degre \\
         0 h 10 & 60\degre & 5\degre & 55\degre \\
         0 h 15 & 90\degre & 7,5\degre & 82,5\degre \\
         0 h 20 & 120\degre & 10\degre & 110\degre \\
         0 h 25 & 150\degre & 12,5\degre & 137,5\degre \\
         0 h 30 & 180\degre & 15\degre & 165\degre \\
         0 h 35 & 150\degre & 17,5\degre & 167,5\degre \\
         0 h 40 & 120\degre & 20\degre & \textcolor{blue}{140\degre} \\
         \hline
      \end{Ltableau}}
   {\blue Il est minuit et 40 minutes}.
   \end{enumerate}

