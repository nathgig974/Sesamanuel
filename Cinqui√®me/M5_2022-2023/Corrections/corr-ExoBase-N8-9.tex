\hspace*{-7.5mm} \textcolor{orange}{\bf Défi 1} \\
   On peut modéliser ce défi par un schéma en barres. Soit $x$, le nombre choisi. \\
   \ModeleBarre[Separation=2]{DeepSkyBlue 1 "$x$" 1 "$x$" 1 "$x$" 1 "$x$" 1 "$x$" PowderBlue 2 "35"}{DeepSkyBlue 1 "$x$" 1 "$x$" SkyBlue 5 "146"}
   \ModeleBarre[Separation=3]{DeepSkyBlue 1 "$x$" 1 "$x$" 1 "$x$" PowderBlue 2 "35"}{ SkyBlue 3 "111" PowderBlue 2 "35"}
   \ModeleBarre{DeepSkyBlue 1 "$x$" 1 "$x$" 1 "$x$"}{SkyBlue 3 "111"} donc, $x =111\div3 ={\blue 37}$.

\hspace*{-7.5mm} \textcolor{orange}{\bf Défi 2} \\
   \begin{enumerate}
      \item Pour le rang 4, il suffit {\blue d'ajouter 1 carré à chacune des extrémités}.
      \item Au rang 1, on a 5 carrés. Au rang 2, $5+4 =9$ carrés. Au rang 3, $9+4 =13$ carrés.  Au rang 4, $13+4 =17$ carrés. {\blue Au rang 5, $17+4 =21$ carrés}. \\
         En continuant ainsi, on aura {\blue 41 carrés au rang 10 et 69 carrés au rang 17}.
         \item Soit $n$ le rang demandé, {\blue le nombre de carrés au rang $n$ est de $1+4\times n$}. \\
            Au rang 100, cela fait donc $1+4\times100 =401$.
         \item En enlevant 1 carré, on doit obtenir un multiple de 4. Or, $532-1 =531$ n'est pas un multiple de 4. \\
            $813-1 =812 =4\times203$ est un multiple de 4. \\
            {\blue Il y a donc 813 carrés au rang 203}.
   \end{enumerate}
