      \begin{itemize}
      \item Le triangle $REP$ est équilatéral donc, tous ses angles ont la même mesure. La somme faisant \udeg{180}, un angle mesure $\udeg{180}\div3 =\udeg{60}$. \\
         Conclusion : {\blue $\widehat{REP} =\udeg{60}$}.
      \item Le triangle $RAP$ est isocèle en $A$ dont les angles à sa base principale mesurent \udeg{38}. \\
      $\udeg{38}+\udeg{38} =\udeg{76}$ d'où $\widehat{RAP} =\udeg{180}-\udeg{76} =\udeg{104}$. \\
      Conclusion : {\blue $\widehat{RAP} =\udeg{104}$}.
      \item Le triangle $YES$ est rectangle en $E$ et $\udeg{90}+\udeg{50,36} =\udeg{140,36}$ d'où, $\widehat{ESY} =\udeg{180}-\udeg{140,36} =\udeg{39,64}$. \\
         Conclusion : {\blue $\widehat{ESY} =\udeg{39,64}$}.
      \item Le triangle $WHY$ est isocèle en $W$ dont l'angle à son sommet principal vaut \udeg{42,6}. \\
         $\udeg{180}-\udeg{42,6} =\udeg{137,4}$ ; $\widehat{WHY} =\udeg{137,4}\div2 =\udeg{68,7}$. \\
         Conclusion : {\blue $\widehat{WHY} =\udeg{68,7}$}.
   \end{itemize}
