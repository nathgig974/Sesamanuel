   \ \\ [-5mm]
   \begin{enumerate}
      \item La balle est située à quatre espaces vers la droite et trois espaces vers le haut, soit $4\times40$ unités = 160 unités et $3\times40$ unités = 120 unités. \\
      Ses coordonnées sont donc {\blue (160\,;\,120)}.
      \item \\
      \begin{enumerate}
         \item La touche $\to$ ajoute 80 à l'abscisse $x$ ; la touche $\gets$ ajoute $-40$ à l'abscisse $x$ ; donc, la succession $\to \, \gets$ ajoute $80+(-40) =40$ à l'abscisse $x$. \\
            Le chat a donc \og avancé \fg{} de 40 unités vers la droite,  {\blue il ne revient pas à sa position de départ}. \vspace*{11.5cm}
         \item On résume dans un tableau les déplacements : \\ \smallskip
            {\hautab{1.3}
            \begin{tabular}{|*{7}{c|}}
               \hline
               & départ & $\to$ & $\to$ & $\uparrow$ & $\gets$ & $\downarrow$ \\
               \hline
               $x$ & $-120$ & $-40$ & 40 & 40 & 0 & 0 \\
               \hline
               $y$ & $-80$ & $-80$ & $-80$ & 0 & 0 & $-40$ \\
               \hline
            \end{tabular}} \\ \smallskip
         Les coordonnées du chat après ces cinq déplacements sont {\blue $(0\,;\,-40)$}.
         \item Seul le {\blue déplacement 2} permet au chat d'attraper la balle.
      \end{enumerate}
      \setcounter{enumi}{2}
      \item Quand le chat atteint la balle, {\blue il dit \og Je t'ai attrapée \fg{} pendant 2 sec.} puis retourne au départ.
   \end{enumerate}
