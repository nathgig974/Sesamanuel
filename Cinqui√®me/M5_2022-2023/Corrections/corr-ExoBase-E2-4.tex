   Oui, Anita a raison : \\
   \begin{itemize}
      \item Première figure : l'angle supplémentaire à \udeg{119} de l'autre côté de $(d_1)$ vaut $\udeg{180}-\udeg{119} =\udeg{61}$. \\
      On a deux angles correspondants de même mesure donc, {\blue les droites $(d_1)$ et $(d_2)$ sont parallèles}.
      \item Deuxième figure : l'angle supplémentaire à \udeg{111} de l'autre côté de $(d_2)$ vaut $\udeg{180}-\udeg{111} =\udeg{69}$. \\
      On a deux angles alternes-internes de mesures différentes donc, {\blue $(d_1)$ et $(d_2)$ ne sont pas parallèles}.
   \end{itemize}
