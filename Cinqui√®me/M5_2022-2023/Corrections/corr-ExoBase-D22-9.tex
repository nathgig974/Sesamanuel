\ \\ [-5mm]
   \begin{enumerate}
      \item Un triangle équilatéral a trois angles de \udeg{60}. Les trois angles sont donc dans un ratio de 60 : 60 : 60, ou encore {\blue 1 : 1 : 1}.
      \item Un triangle rectangle isocèle a trois angles de \udeg{90}, \udeg{45} et \udeg{45}. Les trois angles sont donc dans un ratio de 90 : 45 : 45, ou encore {\blue 2 : 1 : 1}.
      \item Le ratio 1 : 2 : 3 pour les angles du triangle signifie que pour 1 part pour le premier angle, on a 2 parts pour le second et 3 parts pour le troisième pour un total de 6 parts, correspondant à \udeg{180}. \\ [2mm]
            \quad \Ratio[Figure,Longueur=7cm,TexteTotal=\udeg{180},CouleurUn=LightSkyBlue,CouleurDeux=IndianRed,CouleurTrois=Gold]{1,2,3} \\
         $180\div6 =30$ donc, la valeur d'une part est de \udeg{30}. Le triangle a donc trois angles de \udeg{30}, \udeg{60} et \udeg{90}.
         {\blue Le triangle est rectangle.}
   \end{enumerate}
