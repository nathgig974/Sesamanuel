   \ \\ [-5mm]
   \begin{enumerate}
      \item Pour 5 pièces : \\
         $5\times\umin{5} =\umin{25} ; 5\times\us{26} =\us{130} =\umin{2}\,\us{10}$. \\
         $\umin{25}+\umin{2}\,\us{10} =\blue \umin{27}\,\us{10}$.
      \item Pour 10 pièces, c'est-à-dire 2 fois 5 pièces : \\
         $2\times(\umin{27}\,\us{10}) =\blue \umin{54}\,\us{20}$.
      \item Pour 20 pièces, c'est à dire 2 fois 10 pièces : \\         $2\times(\umin{54}\,\us{20}) =\umin{108}\,\us{40} =\blue \uh{1}\,\umin{48}\,\us{40}$.
      \item Pour 100 pièces, c'est à dire 10 fois 10 pièces : \\
         $10\times(\umin{54}\,\us{20}) =\umin{540}\,\us{200} =\blue \uh{9}\,\umin{3}\,\us{20}$.
      \item $\uh{8} =8\times\us{3600} =\us{28800}$ et \\
         $\umin{5}\,\us{26} =5\times\us{60}+\us{26} =\us{326}$. \\
      Or, $28\,800\div326 \approx88,34$ donc, {\blue la machine aura fabriqué 88 pièces en \uh{8}}.
      \item La moitié de \umin{5} vaut \umin{2}\,\us{30} et la moitié de \us{26} vaut \us{13} donc, {\blue la machine met \umin{2}\,\us{43}}.
   \end{enumerate}
