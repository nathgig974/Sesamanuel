\begin{enumerate}
   \setcounter{enumi}{4}
      \item Format A4 : {\blue $\ell =\ucm{21}; L=\ucm{29,7}$}. \\
         Format A5 : {\blue $\ell =\ucm{29,7}\div2 = \ucm{14,85}; L=\ucm{21}$}.
      \item
      \begin{enumerate}
         \item La hauteur vaut {\blue $h_1 =\ucm{14,85}$}.
         \item Le périmètre du disque vaut {\blue $P_1 =\ucm{21}$}. \\
         Or, le périmètre d'un disque se calcule grâce à la formule $2\pi\times R$ où $R$ est le rayon du disque. \\
            Donc,  ${\blue R_1} =\ucm{21}\div(2\pi) {\blue \approx\ucm{3,34}}$.
         \item $\mathcal{V}_1 =\pi\times R_1^2\times h_1 \approx\pi\times(\ucm{3,34})^2\times\ucm{14,85}$ \\
            {\blue $\mathcal{V}_1 \approx\ucmc{520,44}$}.
      \end{enumerate}
      \setcounter{enumi}{6}
      \item
      \begin{enumerate}
         \item La hauteur vaut {\blue $h_2 =\ucm{21}$}.
         \item Le périmètre du disque vaut {\blue $P_2 =\ucm{14,85}$}. \\
            Donc,  ${\blue R_2} =\ucm{14,85}\div(2\pi) {\blue \approx\ucm{2,36}}$.
         \item $\mathcal{V}_2 =\pi\times R_2^2\times h_2 \approx\pi\times(\ucm{2,36})^2\times\ucm{21}$ \\
            {\blue $\mathcal{V}_2 \approx\ucmc{367,45}$}.
      \end{enumerate}
      \setcounter{enumi}{7}
      \item {\blue Le premier cylindre a le volume le plus grand}, malgré la feuille identique.
   \end{enumerate}
