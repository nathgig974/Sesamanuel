   \ \\ [-5mm]\begin{enumerate}
      \item On obtient le {\blue dessin d'un 9} : \\
         {\psset{unit=0.5cm,linecolor=white,fillstyle=solid,fillcolor=white}
         \begin{pspicture}(-4,1)(4,7.5)
            \psframe[fillcolor=lightgray](1,1)(4,6)
            \psframe(1,2)(3,3)
            \psframe(2,4)(3,5)
            \psgrid[gridlabels=0,subgriddiv=1,gridcolor=gray](5,7)
            \rput(1.5,1.5){\textbf{d}}
         \end{pspicture}}
      \item \\
      \begin{enumerate}
         \item Le motif peut être programmé grâce à la suite : \\
            {\blue 1S 2E 1N 1S 2E 1N 1S 2E 1N}
         \item On peut introduire une boucle de répétition, par exemple : {\blue 3$\times$(1S 2E 1N)}
      \end{enumerate}
   \end{enumerate}

\bigskip
\corec{Le jeu des dominogrammes}
\medskip

Les dominos s'enchaînent dans l'ordre de gauche à droite et de haut en bas soit : \\
\ding{40} -- \ding{110} -- \ding{57} -- \ding{52} -- \ding{70} -- \ding{74} -- \ding{87} -- \ding{115}
