\ \\ [-5mm]
   \begin{enumerate}
      \item Schéma du terrain de Nora : \\
      {\psset{unit=0.7}
   \begin{pspicture}(0,-0.5)(10,6)
      \pstGeonode[PointSymbol=none,PointName=none](0,1){a}(10,1){b}(0,4){c}(10,4){d}(0.33,0){f}(4.5,5.5){g}(6.33,0){h}(9,5){j}
      \pstLineAB{a}{b}
      \pstLineAB{c}{d}
      \pstLineAB{f}{g}
      \pstLineAB{h}{j}
      \pstInterLL[PosAngle=-45,PointSymbol=none,PointName=none]{a}{b}{f}{g}{S}
      \pstInterLL[PosAngle=-45,PointSymbol=none,PointName=none]{a}{b}{h}{j}{E}
      \pstInterLL[PosAngle=-45,PointSymbol=none,PointName=none]{c}{d}{f}{g}{I}
      \pstInterLL[PosAngle=-45,PointSymbol=none,PointName=none]{c}{d}{h}{j}{N}
      \pstMarkAngle{j}{N}{I}{\udeg{121}}
      \pstMarkAngle{E}{S}{I}{\udeg{49}}
      \pstMarkAngle{I}{N}{E}{\blue\udeg{59}}
      \pstMarkAngle{E}{N}{d}{\blue\udeg{121}}
      \pstMarkAngle{d}{N}{j}{\blue\udeg{59}}
      \pstMarkAngle{b}{E}{N}{\blue\udeg{59}}
      \pstMarkAngle{h}{E}{b}{\blue\udeg{121}}
      \pstMarkAngle{N}{E}{S}{\blue\udeg{121}}
      \pstMarkAngle{S}{E}{h}{\blue\udeg{59}}
      \pstMarkAngle{f}{S}{E}{\blue\udeg{131}}
      \pstMarkAngle{I}{S}{a}{\blue\udeg{131}}
      \pstMarkAngle{a}{S}{f}{\blue\udeg{49}}
      \pstMarkAngle{f}{S}{E}{\blue\udeg{131}}
      \pstMarkAngle{I}{S}{a}{\blue\udeg{131}}
      \pstMarkAngle{g}{I}{e}{\blue\udeg{131}}
      \pstMarkAngle{e}{I}{f}{\blue\udeg{49}}
      \pstMarkAngle{d}{I}{g}{\blue\udeg{49}}
      \pstMarkAngle{f}{I}{d}{\blue\udeg{131}}
   \end{pspicture}}
      \item Si les droites $(NA)$ ET $(OR)$ étaient parallèles, les angles correspondants en $N$ et $O$ par exemple seraient égaux, ce qui n'est pas le cas ici (\udeg{131}$\neq$\udeg{121}) donc, {\blue ces droites ne sont pas parallèles}.
      \item Dans le quadrilatère $NORA$, les droites $(NO)$ et $(RA)$ sont parallèles, mais les droites $(NA)$ et $(OR)$ ne le sont pas donc, {\blue le quadrilatère $NORA$ est un trapèze}.
   \end{enumerate}
