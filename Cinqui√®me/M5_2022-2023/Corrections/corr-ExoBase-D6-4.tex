
\setcounter{partie}{0}
\partie
   Dans la deuxième écriture, on note la présence du {\blue signe \og = \fg{}} avant le calcul. \\
   Cette écriture est interprétée comme un {\blue calcul que le tableur effectue} alors que la première écriture est interprétée comme un {\blue texte à afficher.}

\medskip

\partie
   \begin{enumerate}
      \item On commence par écrire 0 dans la cellule A2, puis 1 dans la cellule A3. \\
      Puis on sélectionne les cellules A2 et A3. \\
      Enfin, on \og tire \fg{} ces cellules vers le bas en plaçant la souris au coin à droite des cellules et en glissant vers le bas jusqu'à la cellule A10. \\
      {\it Le tableur interprète cette action comme une répétition de l'addition permettant de passant de 0 à 1, c'est-à-dire par l'ajout de 1 à chaque cellule}.
      \item Cette formule permet de trouver le {\blue nombre de tartelettes achetées} si on a acheté 0 muffin pour la ligne 2. \\
         En sélectionnant la cellule et en la tirant vers le bas, le tableur répète la formule pour 1 muffin, puis 2, puis 3\dots
      \item Cette formule permet de calculer le {\blue prix payé pour les muffins} si on en a acheté 0. \\
         En sélectionnant la cellule et en la tirant vers le bas, le tableur répète la formule pour 1 muffin, puis 2, puis 3\dots
      \item En D2, on peut écrire : {\blue \fbox{=2,5*B2}} pour calculer le prix des tartelettes. \\
         On tire la cellule vers le bas pour compléter la colonne.
      \item En E2, on peut écrire : {\blue \fbox{=C2+D2}} pour calculer le prix payé au total, somme du prix payé pour les muffins et de celui payé pour les tartelettes. \\
         On tire la cellule vers le bas pour compléter la colonne.
      \item On cherche, dans la colonne E, le prix de 11,50~\euro{}, on lit ce prix en E7, ce qui donne le nombre de muffins en A7 et le nombre de tartelettes en B7. \\
         Conclusion : {\blue Adrien a acheté 5 muffins et 3 tartelettes}.
   \end{enumerate}

\Coupe

On obtient le tableau suivant : \\ [2mm]
   \textsf{
      \hautab{1.5}
         \begin{tabular}{|>{\columncolor{lightgray!30}}c|C{0.9}|C{0.9}|C{0.9}|C{0.9}|C{0.9}|C{0.9}|}
            \hline
            \rowcolor{lightgray!30} & A & B & C & D & E \\
            \hline
            1 & \cellcolor{FondTableaux}{Muffins} & \cellcolor{FondTableaux}{\!\!\!Tartelettes} & \cellcolor{FondTableaux}{Prix muff.} & \cellcolor{FondTableaux}{Prix tarte.} & \cellcolor{FondTableaux}{Prix payé} \\
            \hline
            2 & 0 & 8 & 0 & 20 & 20 \\
            \hline
            3 & 1 & 7 & 0,8 & 17,5 & 18,3 \\
            \hline
            4 & 2 & 6 & 1,6 & 15 & 16,6 \\
            \hline
            5 & 3 & 5 & 2,4 & 12,5 & 14,9 \\
            \hline
            6 & 4 & 4 & 3,2 & 10 & 13,2 \\
            \hline
            7 & 5 & 3 & 4 & 7,5 & 11,5 \\
            \hline
            8 & 6 & 2 & 4,8 & 5 &  9,8 \\
            \hline
            9 & 7 & 1 & 5,6 & 2,5 & 8,1 \\
            \hline
            10 & 8 & 0 & 6,4 & 0 & 6,4 \\
            \hline
        \end{tabular}}
