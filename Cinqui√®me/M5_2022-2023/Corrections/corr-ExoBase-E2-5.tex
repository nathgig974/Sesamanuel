\ \\ [-5mm]
   \begin{enumerate}
       \item $\bullet$ Les angles $\widehat{IJK}$ et $\widehat{IJt}$ sont supplémentaires donc, ${\blue \widehat{IJK}} =180\up{\circ}-120\up{\circ}° ={\blue 60\up{\circ}}$. \\
          $\bullet$ Les angles $\widehat{JKI}$ et $\widehat{yIS}$ sont correspondants et $\widehat{yIS} =60\up{\circ}$ donc, {\blue $\widehat{JKI} =60\up{\circ}$}. \\
          $\bullet$ Les angles $\widehat{rIy}$ et $\widehat{IJt}$ sont correspondants et $\widehat{IJt} =120\up{\circ}$ donc, {\blue $\widehat{rIy} =120\up{\circ}$}. \\
          $\bullet$ Les angles $\widehat{yIJ}$ et $\widehat{rIy}$ sont supplémentaires donc, ${\blue \widehat{yIJ}} =180\up{\circ}-120\up{\circ}° ={\blue 60\up{\circ}}$. \\
          $\bullet$ Les angles $\widehat{xIK}$ et $\widehat{sIy}$ sont opposés par le sommet donc, ${\blue \widehat{xIK}} =\widehat{sIy} ={\blue 60\up{\circ}}$. Et enfin : \\
          ${\blue \widehat{KIJ}} =180\up{\circ} -\widehat{xIK}-\widehat{yIJ} =180\up{\circ}-60\up{\circ}-60\up{\circ} ={\blue 60\up{\circ}}$.
       \item Les trois angles du triangle sont égaux, donc {\blue le triangle $IJK$ est équilatéral}.
   \end{enumerate}
