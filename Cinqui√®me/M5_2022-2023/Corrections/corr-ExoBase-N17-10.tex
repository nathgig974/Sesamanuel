   {\bf Tour 1}
   {\small
      \ProgCalcul[Application,SansCalcul]{Ajouter 1, Doubler le résultat,Retrancher 2
      §
      n,+1 *2 -2 ,n+1 2\times(n+1)=2n+2 2n+2-2=2n}}
   Le résultat donné est le double du nombre choisi. {\blue Il suffit donc de prendre la moitié du résultat donné}. \\ [1mm]
   {\bf Tour 2}
   {\small
      \ProgCalcul[Application,SansCalcul]{Ajouter 1, Doubler le résultat,Ajouter 1,Retrancher le nombre pensé
      §
      n,+1 *2 +1 -n,n+1 2\times(n+1)=2n+2 2n+2+1=2n+3 2n+3-n=n+3}}
   Le résultat donné est le nombre choisi, augmenté de 3. {\blue Il suffit donc de soustraire 3 au résultat donné}. \\ [1mm]
   {\bf Tour 3}
   {\small
     \ProgCalcul[Application,Largeur=8cm,SansCalcul]{Enlever 1,Doubler le résultat,Enlever 1,Ajouter au résultat le nombre pensé,
      §
      n,-1 *2 -1 +n,n-1 2\times(n-1)=2n-2 2n-2-1=2n-3 2n-3+n=3n-3}}
   Le résultat donné est le triple du nombre choisi, diminué de 3. {\blue Il suffit donc d'ajouter 3 au résultat donné, puis de le diviser par 3}.
