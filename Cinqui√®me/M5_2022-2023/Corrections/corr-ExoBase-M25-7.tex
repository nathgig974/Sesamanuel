   \begin{enumerate}
      \item $\mathcal{A}_{PICK} =6\,u.\ell.\times4\,u\ell. ={\blue 24\,u.a.}$
      \item $\mathcal{A}(MOFE) =5\,u.\ell.\times5\,u.\ell. =25\,u.a.$ \\ [1mm]
         $\mathcal{A}(MOR) =\dfrac{5\,u.\ell.\times4\,u.\ell.}{2} =10\,u.a.$ \\ [1mm]
         $\mathcal{A}(MULE) =4\,u.\ell.\times3\,u.\ell.+\dfrac{4\,u.\ell.\times2\,u.\ell.}{2} =16\,u.a.$ \\ [1mm]
         En sommant, on obtient : $\mathcal{A}(FORMULE)$ \\
         $=25\,u.a.+10\,u.a.+16\,u.a. ={\blue 51\,u.a.}$ \smallskip
      \item {\blue $i =18$} et {\blue $b =14$} donc, $\mathcal{A} =18+\dfrac{14}{2}-1 ={\blue 24}$. \smallskip
      \item {\blue $i =41$} et {\blue $b =22$} donc, $\mathcal{A} =41+\dfrac{22}{2}-1 ={\blue 51}$. \smallskip
      \item On décompose comme à la question 2)
         \begin{itemize}
            \item MOFE : $i =16$, $b =20$, $\mathcal{A} =16+\dfrac{20}{2}-1 =25$. \smallskip
            \item MOR : $i =7$ et $b =8$, $\mathcal{A} =7+\dfrac{8}{2}-1 =10$. \smallskip
            \item MULE : $i =10$ et $b =14$, $\mathcal{A} =10+\dfrac{14}{2}-1 =16$. \smallskip
            \item FORMULE : $\mathcal{A} =25+10+16 =51$. \\
         {\blue La somme des résultats obtenus est égale au résultat trouvé à la question 2).}
      \end{itemize}
   \end{enumerate}
