\themaG
\graphicspath{{../../S27_Les_droites_du_triangle/Images/}}

\chapter{Les droites\\du triangle}
\label{S27}


%%%%%%%%%%%%%%%%%%%%%%%%%%%%%%%%%%%%%%%%%%
\begin{prerequis}
   \begin{itemize}
      \item Triangle : hauteurs et médiatrices.
   \end{itemize}
\end{prerequis}

\vfill

\begin{debat}[Débat : mot valise]
      Un {\bf mot-valise} est un mot formé par l'accolement du début d'un mot et la fin d'un autre mot. À l'heure actuelle, on invente régulièrement des mots-valises : {\it Brexit} pour Britain et exit, {\it Twictée} pour Twitter et dictée, {\it pourriel} pour poubelle et courriel\dots \\
      Les maths n'échappent par à la règle et le mot {\it médiatrice} est un mot-valise qui vient de médiane (dans un triangle, droite joignant un sommet au milieu du côté opposé) et bissectrice (droite coupant un angle en deux angles égaux). Il a été formé en 1923, donc très récemment.
   \begin{center}
       \begin{pspicture}(0,0)(3,3)
          \psline[linearc=0.2,linewidth=2mm,linecolor=gray](1,2)(1,2.4)(2,2.4)(2,2)
          \psframe[fillstyle=solid,fillcolor=B1,framearc=0.3](0,0)(3,2)
          \psframe[fillstyle=solid,fillcolor=A1](0.55,0)(0.85,2)
          \psarc(0.7,0.8){0.15}{180}{360}
          \psdot(0.7,0.85)
          \psframe[fillstyle=solid,fillcolor=A1](2.45,0)(2.15,2)
          \psarc(2.3,0.9){0.15}{180}{360}
          \psdot(2.3,0.95)
          \psset{linecolor=gray,linewidth=1.5mm}
          \psarc(0,0){0.3}{2}{88}
          \psarc(3,0){0.3}{92}{178}
          \psarc(3,2){0.3}{182}{268}
          \psarc(0,2){0.3}{272}{358}
       \end{pspicture}
   \end{center}
   \bigskip
   \begin{cadre}[B2][F4]
      \begin{center}
         Vidéo : \href{https://www.youtube.com/watch?v=M7npDrRJm6E}{\bf Les mots-valises}, chaîne YouTube {\it Image et communication}, épisode d'{\it Au pied de la lettre}.
      \end{center}
   \end{cadre}  
\end{debat}

\vfill

\textcolor{PartieGeometrie}{\sffamily\bfseries Cahier de compétences} : chapitre 8, exercices 18 à 26 ; 29.


%%%%%%%%%%%%%%%%%%%%%%%%%%%%%%%%%%%%%%%%%%
%%%%%%%%%%%%%%%%%%%%%%%%%%%%%%%%%%%%%%%%%%
\activites

\begin{activite}[Des droites concourantes]
   {\bf Objectifs :} tracer les médiatrices d'un triangle ; démontrer que les médiatrices d'un triangle sont concourantes ; faire une démonstration.
   \begin{QCM}
      \partie[construction de médiatrices]
      \ \\ [-10mm]
         \begin{enumerate}
            \item Expliquer à l'oral la construction de la médiatrice d'un segment d'après les schémas suivants : 
            \begin{center}
               {\psset{unit=0.75}
               \begin{pspicture}(0,-0.25)(5,3.5)
                  \pstGeonode[PosAngle=180,linecolor=A1](1,3){M}
                  \pstGeonode[PointSymbol=none,PointName=none](0,0){E}(4,3){F}
                  \pstOrtSym[linecolor=A1]{E}{F}{M}[N] 
                  \pstLineAB{B}{D} 
               \end{pspicture}
               \begin{pspicture}(0,-0.25)(5,3.5)
                  \pstGeonode[PosAngle=180,linecolor=A1](1,3){M}
                  \pstGeonode[PointSymbol=none,PointName=none](0,0){E}(4,3){F}
                  \psarc[linecolor=A1,linestyle=dashed](1,3){2}{260}{300}
                  \psarc[linecolor=A1,linestyle=dashed](1,3){2}{320}{360}
                 \pstOrtSym[linecolor=A1]{E}{F}{M}[N]  
                  \pstLineAB{B}{D}
                  \psarc[linecolor=A1,linestyle=dashed](3.18,0.16){2}{80}{120}
                  \psarc[linecolor=A1,linestyle=dashed](3.18,0.16){2}{140}{175}
               \end{pspicture} 
               \begin{pspicture}(0,-0.25)(4,3.5)
                  \pstGeonode[PosAngle=180,linecolor=A1](1,3){M}
                  \pstGeonode[PointSymbol=none,PointName=none](0,0){E}(4,3){F}
                  \psarc[linecolor=A1,linestyle=dashed](1,3){2}{260}{300}
                  \psarc[linecolor=A1,linestyle=dashed](1,3){2}{320}{360}
                  \pstOrtSym[linecolor=A1]{E}{F}{M}[N]  
                  \pstLineAB{B}{D}
                  \psarc[linecolor=A1,linestyle=dashed](3.18,0.16){2}{80}{120}
                  \psarc[linecolor=A1,linestyle=dashed](3.18,0.16){2}{140}{175} 
                  \pstMediatorAB[CodeFig=true,PointName=none,PointSymbol=none]{M}{N}{I}{F}  
                  \pstMediatorAB[PointName=none,PointSymbol=none]{N}{M}{I}{E}     
                  \pstLineAB[linecolor=B2,linewidth=0.05]{E}{F} 
               \end{pspicture}}
            \end{center}
            \item Donner une définition de la médiatrice d'un segment : \\ [2mm]
               \pf \smallskip
            \item Donner une propriété de la médiatrice d'un segment : \\ [2mm]
               \pf \smallskip
            \item Tracer la médiatrice de tous les côtés de ces deux polygones.
            \begin{center}
               {\psset{unit=0.8}
               \begin{pspicture}(0.5,-1)(20,5.5)
                  \pspolygon(0,2)(6,0)(9,3)
                  \pspolygon(12,0)(20,1)(16,5)(12,3)
               \end{pspicture}}
            \end{center}
            \item Pour quel polygone les médiatrices sont-elles concourantes ? \pfb \\
         \end{enumerate}

      \partie[démonstration] %%%
      \ \\ [-10mm]
         \begin{enumerate}
         \setcounter{enumi}{5}
            \item Sur une feuille, tracer un triangle $ABC$ puis tracer la médiatrice de $[AB]$ et la médiatrice de $[BC]$. \\
               Placer $O$, point d'intersection de ces deux médiatrices.
            \item $O$ se situe sur la médiatrice de $[AB]$. Comparer les longueurs $OA$ et $OB$ : \pfb \medskip
            \item $O$ se situe sur la médiatrice de $[BC]$. Comparer les longueurs $OB$ et $OC$ : \pfb \medskip
            \item En déduire une relation entre $OA$ et $OC$ : \pfb \medskip
            \item \, Que peut-on dire du point $O$ par rapport à $[CA]$ ? \pfb \medskip
            \item \, Tracer le cercle de centre $O$ passant par $A$. Que remarque-t-on ? \pfh \medskip
            \item \, Conclure : \pfb \\
         \end{enumerate}
   \end{QCM}
\end{activite}


%%%%%%%%%%%%%%%%%%%%%%%%%%%%%%%%%%%
%%%%%%%%%%%%%%%%%%%%%%%%%%%%%%%%%%%
\cours 

\section{Médiatrices d'un triangle} %%%%%%%%

\begin{definition}
   Les \textbf{médiatrices} d'un triangle sont les médiatrices des côtés du triangle, c'est-à-dire les trois droites perpendiculaires aux côtés qui passent par leur milieu.
\end{definition}

\bigskip

Pour tracer la médiatrice du segment $[AB]$ au compas, on choisit un écartement au compas et on trace deux arcs de cercle à partir de $A$ et de $B$ de part et d'autre du segment [$AB]$. Puis on trace la droite passant par les deux points formés par l'intersection des arcs de cercle.

\begin{tableau}[pr]{\linewidth}
   \hline %%%%%%%%%%%%%%%%%%%%%%
   \begin{tikzpicture}[scale=0.65]
      \draw (0.5,2.825)--(3.5,2.125);
      \draw[shift={(2,2.475)}, rotate=-13.7] (0,0) rectangle (0.3,0.3);
      \draw[very thick, color=B2] (1.8,1.7)--(2.5,4.5);
      \foreach \x/\y/\N/\pos in {0.5/2.825/A/left, 3.5/2.125/B/below, 2.3/3.7/M/right} {\draw (\x,\y) node{$\times$};\draw (\x,\y) node[\pos]{$\N$}; } 
      \draw (2,2.5) node[below left] {$O$};
      \draw (1.25,2.65) node[color=A1, rotate=76] {$\approx$};
      \draw (2.75,2.3) node[color=A1, rotate=76] {$\approx$};
      \draw[dashed] (0.5,2.825)--(2.3,3.7)--(3.5,2.125);
      \draw (0,1) node {};
   \end{tikzpicture}
   &
   \propriete{} Si un point appartient à la médiatrice d'un segment, alors il est équidistant des extrémités de ce segment.
   &
   Ici, $M$ appartient à la médiatrice de $[AB]$. \newline
   Donc $MA=MB$. \\
   \hline
   {\psset{unit=0.5,linewidth=0.025}
   \begin{pspicture}(-1.4,-2.7)(5,3.6)
      \psset{CodeFig=true,PointSymbol=none,linecolor=B2}
      \pstTriangle{A}(5,0){B}(1.5,3){C}
      \pstCircleABC[RightAngleSize=0.2,CodeFigColor=black]{A}{B}{C}{O}
   \end{pspicture}}
   &
   \propriete{} Dans un triangle, les trois médiatrices sont concourantes en un point qui est le centre du cercle circonscrit au triangle.
   &
   Ici, les médiatrices à $[AB]$, $[BC]$ et $[CA]$ se coupent en $O$ qui est le centre du cercle passant par les trois sommets $A$, $B$ et $C$. \\
   \hline
\end{tableau}
  

%%%%%%%%%%%%%%%%%%%%%%%%%%%%%%%%%%%%%%
\section{Hauteurs d'un triangle}

\begin{definition}
   Les \textbf{hauteurs} d'un triangle sont les hauteurs relatives aux sommets du triangle, c'est-à-dire les trois droites perpendiculaires aux côtés qui passent par le sommet opposé.
\end{definition}

\bigskip

Pour tracer la hauteur dans un triangle issue d'un sommet, on trace la droite passant par ce sommet et perpendiculaire au côté opposé.

\begin{tableau}[pr]{\linewidth}
   \hline %%%%%%%%%%%%%%%%%%%%%%
   {\psset{unit=0.7,linewidth=0.025}
   \begin{pspicture}(-0.8,-1.2)(5,3)
      \psset{CodeFig=true, PointSymbol=none,RightAngleSize=0.2}
      \pstTriangle{A}(4,0){B}(1.5,2.5){C}
      \pstProjection[PointName=none,CodeFigColor=B2]{B}{A}{C}
      \pstProjection[PointName=none,CodeFigColor=B2]{A}{C}{B}
      \pstProjection[PointName=none,CodeFigColor=black]{C}{B}{A}
      \rput(1.8,0.9){$H$}
   \end{pspicture}}
   &
   \propriete{} Dans un triangle, les trois hauteurs sont concourantes en un point appelé orthocentre du triangle.
   &
   Ici, $H$ est le point de concours de la hauteur issue de $C$ et de celle issue de $B$. Donc, $[AH]$ est la hauteur issue de $A$. \\
   \hline
\end{tableau}


%%%%%%%%%%%%%%%%%%%%%%%%%%%%%%%%%%%
\exercicesbase

\begin{colonne*exercice}

\serie{Tracés de médiatrices et de hauteurs} %%%%%

\bigskip

\begin{exercice} %1
   Suivre le programme suivant en codant les éléments construits : \smallskip
   \begin{enumerate}
      \item Construire un triangle $CJR$ quelconque. \smallskip
      \item Tracer en rouge la médiatrice du segment $[JR]$ à l'aide du compas et d'une règle graduée. \smallskip
      \item Tracer en noir la médiatrice du segment $[CJ]$ à la règle graduée et à l'équerre. \smallskip
      \item Construire la médiatrice $(d)$ du segment $[CR]$ avec seulement une équerre (non graduée).
   \end{enumerate}
\end{exercice}    

\begin{corrige}
   \ \\ [-5mm]
   \begin{pspicture}(-0.5,1)(6,4.25)
      \pstGeonode[CurveType=polygon,PosAngle={200,0,90}](1,1){C}(5.5,1){J}(2.5,4){R}
      \psset{CodeFig=true,CodeFigColor=black,PointName=none,linestyle=dashed}
      \pstMediatorAB[linecolor=red,SegmentSymbol=pstslash]{J}{R}{I}{K}
      \pstMediatorAB[linecolor=black,SegmentSymbol=pstslashh]{C}{J}{L}{M}
      \pstMediatorAB[linecolor=blue,SegmentSymbol=pstslashhh]{R}{C}{N}{d}
      
      \rput(2.2,2.3){\blue$(d)$}
   \end{pspicture}
\end{corrige}    

\bigskip


\begin{exercice} %2
   Dans chaque cas, construire le triangle puis son cercle circonscrit de centre $O$. \smallskip
   \begin{enumerate}
      \item Triangle $SKI$ tel que : \\ [1mm]
         $SI = \ucm{8} \; ; \widehat{KSI} =65$\degre et $\widehat{KIS} =45$\degre. \smallskip
      \item Triangle $GYM$ tel que : \\ [1mm]
         $GM = \ucm{4} \; ; GY = \ucm{5}$ et $\widehat{YGM} = 103$\degre. \smallskip
      \item Triangle $TIR$ tel que : \\ [1mm]
         $TIR$ est isocèle en $T$ ; $TI =\ucm{8}$ et $IR =\ucm{5,5}$. \smallskip
      \item Triangle $VTC$ tel que : \\
         $VTC$ est un triangle équilatéral de côté \ucm{4}.
   \end{enumerate}
\end{exercice}

\begin{corrige}
   \ \\ 
   \begin{pspicture}(0.65,-3)(8,5.75)
      \pstTriangle(0,0){S}(8,0){I}(2.34,5.66){K}
      \pstCircleABC[CodeFig=true,CodeFigColor=blue,SegmentSymbolA=pstslash,SegmentSymbolB=pstslashh,SegmentSymbolC=pstslashhh]{S}{K}{I}{O}
      \pstLabelAB[offset=-3mm]{S}{I}{\ucm{8}}
      \pstMarkAngle{I}{S}{K}{65 \degre}
      \pstMarkAngle[MarkAngleType=double]{K}{I}{S}{45 \degre}
   \end{pspicture}
 
   \begin{pspicture}(-1.5,-0.25)(6,6.5)
      \pstTriangle(0,0){G}(4,0){M}(-1.12,4.87){Y}
      \pstCircleABC[CodeFig=true,CodeFigColor=blue,SegmentSymbolA=pstslash,SegmentSymbolB=pstslashh,SegmentSymbolC=pstslashhh]{G}{M}{Y}{O}
      \pstLineAB{G}{M}
      \pstLineAB{G}{Y}
      \pstLineAB{Y}{M}
      \pstLabelAB[offset=-3mm]{G}{M}{\ucm{4}}
      \pstLabelAB[offset=-3mm]{Y}{G}{\ucm{5}}
      \pstMarkAngle{Y}{M}{G}{45 \degre}
      \pstMarkAngle{M}{G}{Y}{103 \degre}
   \end{pspicture}

\Coupe

   \begin{pspicture}(0,-3)(8,6.25)
      \pstTriangle(0,0){T}(8,0){I}(6.11,5.16){R}
      \pstCircleABC[CodeFig=true,CodeFigColor=blue,SegmentSymbolA=pstslash,SegmentSymbolB=pstslashh,SegmentSymbolC=pstslashh]{T}{R}{I}{O}
      \pstLineAB{T}{R}
      \pstLineAB{T}{I}
      \pstLineAB{I}{R}
      \pstLabelAB[offset=-3mm]{T}{I}{\ucm{8}}
      \pstLabelAB{R}{I}{\ucm{5,5}}
   \end{pspicture}
   
   \begin{pspicture}(-2,-1)(5,4.5)
      \pstTriangle(0,0){V}(4,0){T}(2,3.46){C}
      \pstCircleABC[CodeFig=true,CodeFigColor=blue,SegmentSymbolA=pstslashh,SegmentSymbolB=pstslashh,SegmentSymbolC=pstslashh]{V}{C}{T}{O}
      \pstLineAB{V}{C}
      \pstLineAB{V}{T}
      \pstLineAB{T}{C}
      \pstLabelAB[offset=-3mm]{V}{T}{\ucm{4}}
   \end{pspicture}
\end{corrige}

\bigskip


\begin{exercice} %3
   Construction de hauteurs.
   \begin{enumerate}
      \item Construire un triangle $BLE$ puis tracer :
      \begin{itemize}
         \item en bleu, la hauteur issue du sommet $E$ ;
         \item en noir, la hauteur issue du sommet $B$ ;
         \item en rouge, la hauteur relative à $[BE]$.
      \end{itemize}
       \item Quelle remarque peut-on faire ?
   \end{enumerate} 
\end{exercice}

\begin{corrige}
   \ \\ [-5mm]
   \begin{pspicture}(0,0.5)(6,5.5)
      \pstGeonode[CurveType=polygon,PosAngle={200,0,90}](0,1){B}(6,2){L}(4,5){E}
      \pstProjection[CodeFig=true,CodeFigColor=blue,PointName=none]{B}{L}{E}
      \pstProjection[CodeFig=true,CodeFigColor=black,PointName=none]{L}{E}{B}
      \pstProjection[CodeFig=true,CodeFigColor=red,PointName=none]{E}{B}{L}
   \end{pspicture} \\
   On remarque que {\blue les hauteurs sont concourantes} en un point appelé l'orthocentre du triangle.
\end{corrige}

\bigskip


\begin{exercice} %4
   Tracer les hauteurs dans les cas suivants :
   \begin{enumerate}
      \item Un triangle $ONE$ ayant trois angles aigus, quelle remarque peut-on faire sur l'orthocentre ? \smallskip
      \item Un triangle $TWO$ tel que l'angle $\widehat{TWO}$ soit obtus, quelle remarque peut-on faire sur l'orthocentre ? \smallskip
      \item Un triangle $TRE$ rectangle en $R$, quelle remarque peut-on faire sur l'orthocentre ?
   \end{enumerate}
\end{exercice}

\begin{corrige}
   \ \\ [-5mm]
   \begin{pspicture}(-0.5,0.5)(6,4.5)
      \pstGeonode[CurveType=polygon,PosAngle={200,0,90}](0,1){O}(5,0.5){N}(2,4){E}
      \psset{CodeFig=true,CodeFigColor=red,PointName=none}
      \pstProjection{N}{O}{E}
      \pstProjection{O}{E}{N}
      \pstProjection{E}{N}{O}
      \rput(5,3.5){\parbox{2.5cm}{L'orthocentre se situe {\blue à l'intérieur du triangle}}}
   \end{pspicture}
   
   \begin{pspicture}(0.8,-2.2)(8,4)
      \pstGeonode[CurveType=polygon,PosAngle={200,0,90}](1,3){T}(5.5,1){W}(8,3.5){O}
      \psset{CodeFig=true,CodeFigColor=red,PointName=none,linecolor=red,linestyle=dashed}
      \pstProjection{T}{W}{O}[A]
      \pstProjection{O}{T}{W}[B]
      \pstProjection{W}{O}{T}[C]
      \pstLineAB[nodesepB=-2.5]{O}{A}
      \pstLineAB[nodesepB=-3]{B}{W}
      \pstLineAB[nodesepB=-2.5]{T}{C}
      \psset{linecolor=black}
      \pstLineAB{W}{C}
      \pstLineAB{W}{A}
      \rput(2,-0.5){\parbox{2.5cm}{L'orthocentre se situe {\blue à l'extérieur du triangle}}}
   \end{pspicture}
   
   \begin{pspicture}(-0.4,-0.5)(6,3.75)
      \pstGeonode[CurveType=polygon,PosAngle={200,0,90}]{T}(6,0){R}(6,3.5){E}
      \psset{CodeFig=true,CodeFigColor=red,PointName=none}
      \psframe(6,0)(5.6,0.4)
      \pstProjection{E}{T}{R}
      \pstLineAB[linestyle=dashed,linecolor=red]{T}{E}
      \rput(1,2.5){\parbox{3cm}{L'orthocentre est {\blue au point $R$} où le triangle est rectangle}}
   \end{pspicture}
\end{corrige}

\bigskip


\serie{Conjectures et démonstrations} %%%%%

\bigskip

\begin{exercice} %5
   \begin{enumerate}
   \item Tracer un triangle $YES$ quelconque.
      \item Placer :
      \begin{itemize}
         \item le milieu $O$ du côté $[ES]$ ;
         \item le milieu $U$ du côté $[YS]$ ;
         \item le milieu $I$ du côté $[YE]$.
      \end{itemize}
      \item Tracer le triangle $OUI$ puis ses hauteurs.
      \item Placer le point $T$ orthocentre du triangle $OUI$.
      \item Trace le cercle de centre $T$ et de rayon
$[TY]$.
      \item Quelle conjecture peut-on écrire ?
   \end{enumerate}
\end{exercice}

\begin{corrige}
   \ \\ [-5mm]
   \begin{pspicture}(-1,-2.25)(6,4.25)
      \psset{CodeFig=true}
      \pstGeonode[CurveType=polygon,PosAngle={200,0,90}](0,0){Y}(6,0.5){E}(3.8,4){S}
      \pstMiddleAB[PosAngle=-90,SegmentSymbol=pstslashhh]{Y}{E}{I}
      \pstMiddleAB[SegmentSymbol=pstslash]{Y}{S}{U}
      \pstMiddleAB[PosAngle=30,SegmentSymbol=pstslashh]{E}{S}{O}
      \pstLineAB{O}{U}
      \pstLineAB{O}{I}
      \pstLineAB{I}{U}
      \psset{CodeFigColor=red,PointName=none}
      \pstProjection{O}{U}{I}[J]
      \pstProjection{I}{O}{U}[K]
      \pstProjection{U}{I}{O}
      \pstInterLL[PointName=T,PosAngle=65]{I}{J}{K}{U}{T}
      \pstCircleOA{T}{Y}
   \end{pspicture} \\
   On peut conjecturer que {\blue le centre du cercle circonscrit au triangle $YES$ est l'orthocentre du triangle $OUI$}.
\end{corrige}

\bigskip


\begin{exercice} %6
   \begin{enumerate}
      \item Tracer un triangle $BAC$ rectangle en $A$.
      \item Placer un point $M$ à l'extérieur du triangle $ABC$.
      \item La droite perpendiculaire à $(AB)$ passant par $M$ coupe $[AB]$ en $I$ et la droite perpendiculaire à $[AC]$ passant par $M$ coupe $[AC]$ en $J$.
      \item Placer le point $P$ sur la demi-droite $[MI)$ tel que $I$ soit le milieu de $[MP]$ et le point $Q$ sur la demi-droite $[MJ)$ tel que $J$ soit le milieu de $[MQ]$.
      \item Que représente le point $A$ pour le triangle $MQP$ ? Justifier.
   \end{enumerate}
\end{exercice}

\begin{corrige}
   \ \\ [-5mm]
   \begin{pspicture}(1,-3.6)(8,3)
      \pstGeonode[CurveType=polygon,PosAngle={200,0,90}](1,0){B}(5,0){A}(5,3){C}
      \pstRightAngle{B}{A}{C}
      \pstGeonode[PosAngle={135,-135,45}](2,1.6){M}(2,-1.6){P}(8,1.6){Q}
      \psset{CodeFig=true,CodeFigColor=blue}
      \pstProjection[PosAngle=45]{A}{B}{M}[I]
      \pstProjection[PosAngle=45]{A}{C}{M}[J]
      \pstCircleOA{A}{M}
      \psset{linecolor=blue}
      \pstSegmentMark[SegmentSymbol=pstslash]{M}{I}
      \pstSegmentMark[SegmentSymbol=pstslash]{P}{I}
      \pstSegmentMark[SegmentSymbol=pstslashh]{M}{J}
      \pstSegmentMark[SegmentSymbol=pstslashh]{Q}{J}
      \pstLineAB{P}{Q}
      \rput(5.5,-1.5){\parbox{4cm}{$A$ est le {\blue centre du cercle circonscrit au triangle $MPQ$}.}}     
   \end{pspicture}

   \begin{itemize}
      \item la droite $(CJ)$ est perpendiculaire à la droite $(MQ)$ et coupe le segment $[MQ]$ en son milieu $J$ , il s'agit donc de la médiatrice du segment $[MQ]$ ;
      \item la droite $(BA)$ est perpendiculaire à la droite $(MP)$ et coupe le segment $[MP]$ en son milieu $I$, il s'agit donc de la médiatrice du segment $[MP]$ ;
      \item or, les médiatrices du triangle $MPQ$ sont concourantes en un point qui est le centre de son cercle circonscrit ;
      \item $(CJ)$ et $(BI)$ se coupent en $A$, qui est bien le centre du triangle circonscrit au triangle $MPQ$.
   \end{itemize}
\end{corrige}

\bigskip


\begin{exercice} %7
   Amira avait tracé un triangle $AVU$ au crayon et les médiatrices de deux des côtés au stylo. Son voisin Arthur a effacé le triangle mais a laissé le point $A$ et les deux médiatrices. Reconstruire le triangle d'Amira.
   \begin{center}
      \begin{pspicture}(0,0)(6,5.5)
         \rput(1,3.3){$A$}
         \rput(1,3){$\times$}
         \psline(0,0.5)(6,4)
         \psline(4,0)(2.5,4)
      \end{pspicture}
   \end{center}
\end{exercice}

\begin{corrige}
   Il suffit de construire les points $U$ et $V$ symétriques du point $A$ par rapport aux deux droites déjà tracées.
   \begin{center}
      \begin{pspicture}(0,0)(6,4)
         \pstGeonode[PosAngle=135](1,3){A}
         \pstGeonode[PointName=none,PointSymbol=none](0,0.5){B}(6,4){C}(4,0){D}(2.5,4){E}
         \rput(1,3){$\times$}
         \pstLineAB{B}{C}
         \pstLineAB{E}{D}
         \psset{CodeFig=true,linecolor=blue,CodeFigColor=blue}
         \pstOrtSym[SegmentSymbol=pstslash]{B}{C}{A}[V]
         \pstOrtSym[SegmentSymbol=pstslashh]{D}{E}{A}[U]
         \pstLineAB{A}{V}
         \pstLineAB{A}{U}
         \pstLineAB{U}{V}
      \end{pspicture}
   \end{center}

\smallskip 
\corec{La droite d'Euler}
\smallskip

\begin{enumerate}
\setcounter{enumi}{4}
   \item $H, O$ et $G$ sont confondus lorsque {\blue le triangle $ABC$ est équilatéral}.
   \item On peut conjecturer que {\blue les points $H, O$ et $G$ sont alignés}.
   \item On peut conjecturer que {\blue $OH =3\,OG$}.
\end{enumerate}

\end{corrige}


\end{colonne*exercice}

\vfill\hfill {\it\footnotesize Source : Sesamath, le manuel 5\up{e}. Génération 5 - 2013}

%%%%%%%%%%%%%%%%%%%%%%%%%%%%%%%%%%%%%%%%%%
\Recreation

\enigme[La droite d'Euler]
   Ouvrir Geogebra et choisir l'onglet \textbf{Géométrie}.
   \partie[construction de la figure]
      \begin{tabular}{|cp{5.5cm}|p{4.5cm}|p{5cm}|}
         \hline
         & Instructions & Outil GeoGebra & Action \\
         \hline
         \textcolor{B1}{\bf1)} & \multicolumn{3}{l|}{Construction du {\bf triangle} $ABC$} \\
         & Tracer un triangle $ABC$ & polygone & cliquer en trois points quelconques du plan \\
         \hline
         \textcolor{B1}{\bf2)} & \multicolumn{3}{l|}{Construction des trois {\bf hauteurs} et de l'orthocentre $H$} \\
         & Tracer les hauteurs du triangle & droites perpendiculaires & sélectionner pour chaque hauteur le sommet et son côté opposé \\
         & Placer l'orthocentre & intersection entre deux objets & sélectionner deux hauteurs parmi les trois \\ 
         & Renommer l'orthocentre en $H$ & clic droit propriétés & nom du point : $H$ \\
         & Effacer les hauteurs & clic droit & décocher \og afficher l'objet \fg \\
         \hline
         \textcolor{B1}{\bf3)} & \multicolumn{3}{l|}{Construction des trois {\bf médiatrices} et du centre du cercle circonscrit $O$} \\
         & Tracer les médiatrices du triangle & médiatrices & choisir pour chaque médiatrice deux sommets du triangle \\
         & Placer le centre du cercle circonscrit & \dots & \dots \\ 
         & Renommer le centre en $O$ & \dots & \dots \\
         & Tracer le cercle circonscrit & cercle (centre-point) & choisir le centre $O$ et le sommet $A$ \\
         & Effacer les médiatrices & \dots & \dots \\
         \hline
         \textcolor{B1}{\bf4)} & \multicolumn{3}{p{15cm}|}{Construction des trois {\bf médianes} et du centre de gravité $G$. \newline
         {\it La médiane d'un côté du triangle est la droite passant par le milieu du côté et le sommet opposé. \newline
         Le point de concours des médianes s'appelle le centre de gravité.}} \\
         & Tracer les médianes du triangle  &  milieu ou centre & pour chaque médiane, sélectionner deux sommets du triangle \\
         & & droite passant par deux points & sélectionner un sommet et le milieu du côté opposé \\
         & Placer le centre de gravité & \dots & \dots \\
         & Renommer le centre en G & \dots & \dots \\
         & Effacer les médianes et les milieux & \dots & \dots \\
         \hline
      \end{tabular}
   \bigskip
   
   \partie[constatations]
      \begin{enumerate}
      \setcounter{enumi}{4}
         \item $H$, $O$ et $G$ peuvent-ils être confondus ? Dans quels cas ? \\ [3mm] 
            \pf \smallskip
         \item Dans le cas où aucun point n'est confondu, que peut-on conjecturer sur l'alignement des points $H$, $O$ et $G$ ? \\ [3mm]
            \pf \smallskip 
          \item Peut-on conjecturer l'existence d'une relation de longueur entre $OH$ et $OG$ ? \\ [3mm]
            \pf
      \end{enumerate} 

