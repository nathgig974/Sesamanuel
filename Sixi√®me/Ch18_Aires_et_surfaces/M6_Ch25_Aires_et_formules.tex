\themaM
\graphicspath{{../Ch18_Aires_et_surfaces/Images/}}

\chapter{Aires et formules}
\label{C25}


%%%%%%%%%%%%%%%%%%%%%%%%%%%%%%%%%%%%%
%%%%%%%%%%%%%%%%%%%%%%%%%%%%%%%%%%%%%
\begin{prerequis}[Connaissances et compétences associées]
   \begin{itemize}
      \item Déterminer la mesure de l’aire d’une surface en utilisant une formule : formules de l’aire d’un carré, d’un rectangle, d’un triangle, d’un disque.
   \end{itemize}
\end{prerequis}

\vfill

\begin{debat}[Débat : des aires en images]
   Dans un triangle, on peut mettre deux triangles de même surface, son {\bf aire} correspond donc au double de l'aire de l'un de ces triangles.
   \begin{center}
      \begin{pspicture}(0,-0.5)(8.5,2.5)
         \psframe[fillstyle=solid,fillcolor=A3,linecolor=A3](0.5,0)(3.5,2)
         \psframe[fillstyle=solid,fillcolor=A3,linecolor=A3](5,0)(8.5,2)
         \pspolygon[linecolor=B1](5,0)(8.5,0)(6,2)
         \psline[linecolor=B1,linestyle=dashed](6,0)(6,2)
         \psarc{<-}(5.5,1){0.3}{-45}{180}
         \psarc{->}(7.25,1){0.3}{25}{250}
      \end{pspicture}
   \end{center}
   \begin{cadre}[B2][F4]
      \begin{center}
         Vidéo : \href{https://www.youtube.com/watch?v=5_PiZrfLghQ}{\bf Aire de figures simples}, chaîne YouTube de {\it Science silencieuse}.
      \end{center}
   \end{cadre}
\end{debat}

\vfill

\textcolor{PartieGeometrie}{\sffamily\bfseries Cahier de compétences} : chapitre 11, exercices 12 à 22 ; 29 ; 35 ; 40 à 43.


%%%%%%%%%%%%%%%%%%%%%%%%%%%%%%%%%%%%
%%%%%%%%%%%%%%%%%%%%%%%%%%%%%%%%%%%%
\activites 

\begin{activite}[Aire d'un triangle]
   {\bf Objectif :} calculer l'aire d'un rectangle, d'un triangle ; déterminer la formule de l'aire d'un rectangle, d'un triangle.
   \begin{QCM}
      \partie[aire du rectangle]
         \begin{enumerate}
            \item Découper le rectangle 1 ci-dessous, puis le paver de carrés de côté \ucm{1}.
               \begin{center}
                  {\psset{unit=0.5}
                  \begin{pspicture}(0,-0.3)(5,3.5)
                     \psframe(0,0)(5,3)
                     \psgrid[subgriddiv=0,gridlabels=0pt](0,0)(1,3)
                     \psline(3,0)(3,1)(1,1)
                     \psline(2,0)(2,2)(1,2)
                  \end{pspicture}}
                \end{center}
             \item Combien de carrés de \ucm{1} de côté y a-t-il dans ce rectangle ? \pf \medskip
             \item Quelle est l'aire du rectangle ? \pf \medskip
             \item Rappeler la formule de l'aire d'un rectangle de longueur $L$ et de largeur $\ell$ : \pf
         \end{enumerate}
         
      \partie[aire du triangle rectangle]
         \begin{enumerate}
            \setcounter{enumi}{4}
            \item Découper le rectangle 2 puis tracer l'une de ses diagonales. \medskip
            \item Découper le rectangle suivant la diagonale, que peut-on dire des deux triangles obtenus ? \\ [1mm]
               \pf \medskip
            \item Comment peut-on obtenir l'aire d'un triangle en fonction de l'aire du rectangle ? \\ [1mm]
               \pf \medskip
            \item En déduire la formule de l'aire d'un triangle rectangle de base $b$ et de hauteur $h$ : \pf
         \end{enumerate}
         
      \partie[aire du triangle quelconque]
         \begin{enumerate}
            \setcounter{enumi}{8}
            \item Découper le rectangle 3 puis placer une point A sur l'un de ses côtés. \medskip
            \item \, Tracer le triangle de sommet A et de base le côté opposé à A, le découper. \medskip
            \item \, Est-il possible de coller les deux morceaux restants sur le triangle à la manière d'un puzzle ? \pf \medskip
            \item \, En déduire la formule de l'aire d'un triangle de base $b$ et de hauteur $h$. \pf
         \end{enumerate}
   \end{QCM}
   \begin{center}
      \begin{pspicture}(0,-0.5)(5.5,3)
         \psframe(0,0)(5,3)
         \rput(2.5,-0.5){rectangle 1}
      \end{pspicture}
      \begin{pspicture}(0,-0.5)(5.5,4)
         \psframe(0,0)(5,3)
         \rput(2.5,-0.5){rectangle 2}
      \end{pspicture}
      \begin{pspicture}(0,-0.5)(5,4)
         \psframe(0,0)(5,3)
         \rput(2.5,-0.5){rectangle 3}
      \end{pspicture}
   \end{center}

\end{activite}


%%%%%%%%%%%%%%%%%%%%%%%%%%%%%%%%%%%%
%%%%%%%%%%%%%%%%%%%%%%%%%%%%%%%%%%%%
\cours 

%%%%%%%%%%%%%%%%%%%%%%%%%%%%%%%%%%%%%%
\section{Hauteurs d'un triangle}

\begin{definition}
   Les \textbf{hauteurs} d'un triangle sont les hauteurs relatives aux sommets du triangle, c'est-à-dire les trois droites perpendiculaires aux côtés qui passent par le sommet opposé.
\end{definition}

\begin{pspicture}(-0.5,-0.5)(4,4)
   \psset{CodeFig=true, PointSymbol=none,RightAngleSize=0.2}
   \pstTriangle{A}(4.5,0){B}(1.5,3){C}
   \pstProjection[PointName=none,CodeFigColor=B1]{B}{C}{A}
   \pstLabelAB[offset=-0.3,npos=0.56]{A}{A'}{\footnotesize \textcolor{B1}{hauteur issue de A}}
   \pstLineAB[linecolor=A1,linewidth=0.5mm]{C}{B}
   \pstLabelAB{C}{B}{\footnotesize \textcolor{A1}{base relative à la hauteur issue de A}}
\end{pspicture}
\begin{pspicture}(-1.5,-0.5)(4,4)
   \psset{CodeFig=true, PointSymbol=none,RightAngleSize=0.2}
   \pstTriangle{A}(4.5,0){B}(1.5,3){C}
   \pstProjection[PointName=none,CodeFigColor=B1]{C}{A}{B}
   \pstLabelAB[offset=-0.3,npos=0.4]{B'}{B}{\footnotesize \textcolor{B1}{hauteur issue de B}}
   \pstLineAB[linecolor=A1,linewidth=0.5mm]{A}{C}
   \pstLabelAB[offset=0.5]{A}{C}{\footnotesize \textcolor{A1}{base relative à la hauteur issue de B}}
\end{pspicture}
\begin{pspicture}(-1.5,-0.5)(4,4)
   \psset{CodeFig=true, PointSymbol=none,RightAngleSize=0.2}
   \pstTriangle{A}(4.5,0){B}(1.5,3){C}
   \pstProjection[PointName=none,CodeFigColor=B1]{B}{A}{C}
   \pstLabelAB[offset=-0.3,npos=0.45]{C'}{C}{\footnotesize \textcolor{B1}{hauteur issue de C}}
   \pstLineAB[linecolor=A1,linewidth=0.5mm]{A}{B}
   \pstLabelAB[offset=-0.3]{A}{B}{\footnotesize \textcolor{A1}{base relative à la hauteur issue de C}}
\end{pspicture}


%%%%%%%%%%%%%%%%%%%%%%%%
\section{Aires usuelles}

\medskip

\begin{Ltableau}{\linewidth}{4}{C{4}|p{2.6cm}|C{3.5}|p{5cm}}
   \hline 
   Figure plane & Mesures & Exemple & Calcul \\
   \hline
   \begin{pspicture}(0,0)(4,4) % rectangle
      {\footnotesize
      \pstGeonode[PointName=none,linecolor=B2,PointSymbol=none](0.5,1){A}(3.5,1){B}(3.5,3){C}(0.5,3){D}
      \pstSegmentMark[linecolor=B1]{A}{B}
      \pstSegmentMark[SegmentSymbol=MarkCros,linecolor=A1]{B}{C}
      \pstSegmentMark[linecolor=B1]{C}{D}
      \pstSegmentMark[SegmentSymbol=MarkCros,linecolor=A1]{D}{A}
      \rput(2,2){\small rectangle}
      \rput(2,0.6){\textcolor{B1}{$L$}}
      \rput(2,3.5){\textcolor{B1}{$L$}}
      \rput(0.1,2){\textcolor{A1}{$\ell$}}
      \rput(3.8,2){\textcolor{A1}{$\ell$}}}
   \end{pspicture}
   &
   \begin{minipage}[b]{3cm}
      $\mathcal{A} =L\times \ell$ \\ [4mm]
      Pour un carré \\
      de côté $c$, on a : \\
      $\mathcal{A} =c\times c =c^2$ \\ [3mm]
   \end{minipage}
   &
   \begin{pspicture}(0,-0.2)(4,3)
      {\footnotesize
      \psframe[fillstyle=solid,fillcolor=lightgray!50](1,0.5)(3,3.5)
      \psline[linestyle=dashed]{<->}(1,0.2)(3,0.2)
      \rput(2,-0.1){\udm{0,2}}
      \psline[linestyle=dashed]{<->}(0.7,0.5)(0.7,3.5)
      \rput{90}(0.4,2){\udm{0,3}}}
   \end{pspicture}
   &
   \begin{minipage}[b]{5cm}
      $\mathcal{A}=\udm{0,2}\times\udm{0,3} =\udmq{0,06}$ \\ [12mm]
   \end{minipage} \\
   \hline
      {\footnotesize
      \begin{pspicture}(0,0)(4,4) % triangles
      \pstGeonode[PointName=none,PointSymbol=none](0.5,0.5){A}(3.5,0.5){B}(1,3.5){C}(1,0.5){H}
      \pstLineAB[linecolor=A1]{A}{B}
      \pstLineAB{A}{C}
      \pstLineAB{C}{B}
      \pstLineAB[linecolor=B1]{C}{H}
      \pstRightAngle[linecolor=B1]{B}{H}{C}
      \rput(1.9,1.2){\small triangle}
      \rput(2,0.2){\textcolor{A1}{$b$}}
      \rput(1.2,2){\textcolor{B1}{$h$}}
   \end{pspicture}}
   &
   \begin{minipage}[b]{3cm}
      $\mathcal{A} =\dfrac{b\times h}{2}$ \\ [10mm]
   \end{minipage}
   &
   \begin{pspicture}(0,0)(4,4)
      {\footnotesize
      \pspolygon[fillstyle=solid,fillcolor=lightgray!50](0.5,3)(3.2,3)(2.5,0.4)
      \psline[linestyle=dashed]{<->}(0.5,3.3)(3.2,3.3)
      \psline(2.2,3)(2.2,2.7)(2.5,2.7)
      \rput(2,3.6){\umm{27}}
      \psline[linestyle=dashed]{<->}(2.5,0.4)(2.5,3)
      \rput{90}(2.15,2){\umm{26}}}
   \end{pspicture}
   &
   \begin{minipage}[b]{5cm}
      $\mathcal{A} =\dfrac{\umm{27}\times\umm{26}}{2} =\ummq{351}$ \\ [12mm]
   \end{minipage} \\ 
   \hline
   \begin{pspicture}(0,0.4)(4,3.7) %disque
      {\footnotesize
      \pscircle(2,2){1.3}
      \psdots(2,2)
      \psline[linecolor=B1,arrowsize=0.2]{<->}(2,2)(3.3,2)
      \rput(2.75,2.2){\textcolor{B1}{$r$}}
      \rput(2,1.6){\small disque}}
   \end{pspicture}
   &
   \begin{minipage}[b]{3cm}
      $\mathcal{A} =\pi\times r\times r$ \\
      $\mathcal{A} =\pi\times r^2$ \\ [7mm]
   \end{minipage}
   &
   \begin{pspicture}(0.2,0.4)(4,3.7)
      \footnotesize
      \pscircle[fillstyle=solid,fillcolor=lightgray](2,2){1.3}
      \psline[linestyle=dashed,arrowsize=0.2]{<->}(2,2)(3.3,2)
      \rput(2.65,2.35){\ucm{1,2}}
   \end{pspicture}
   &
   \begin{minipage}[b]{5cm}
      $\mathcal{A} =\pi\times(\ucm{1,2})^2\approx\ucmq{4,52}$ \\ [8mm]
   \end{minipage} \\
   \hline
\end{Ltableau}


%%%%%%%%%%%%%%%%%%%%%%%%%%%%%%%%%%%
%%%%%%%%%%%%%%%%%%%%%%%%%%%%%%%%%%%
\exercicesbase

\begin{colonne*exercice}

\serie{Aire de polygones} %%%

\begin{exercice} %1
   Calculer l'aire des figures suivantes. \\ [1mm]
   {\psset{unit=0.6}
   \small
   \begin{pspicture}(-1.5,-0.5)(5,4)
      \psframe(0,0)(4,3)
      \psframe(0,0)(0.3,0.3)
      \rput(2,0){x}
      \rput(2,3){x}
      \rput(0,1.5){=}
      \rput(4,1.5){=}
      \rput(2,-0.5){\um{6}}
      \rput{90}(-0.5,1.5){\um{4}}
      \rput(2,1.5){\ding{182}}
   \end{pspicture}
   \begin{pspicture}(-1,0)(4,4.5)
      \psframe(0,0)(4,4)
      \psframe(0,0)(0.3,0.3)
      \rput(2,0){x}
      \rput(2,4){x}
      \rput(0,2){x}
      \rput(4,2){x}
      \rput(2,-0.5){\ucm{27}}
      \rput(2,2){\ding{183}}
   \end{pspicture}
   
    \begin{pspicture}(7.5,0)(14,5)
      \pspolygon(8,2)(14,2)(8,5)
      \rput(11,1.5){\umm{83}}
      \rput{90}(7.5,3.5){\umm{25}}
      \rput{-30}(11,4){\umm{86,68}}
      \psframe(8,2)(8.3,2.3)
      \rput(9.5,3){\ding{184}}
   \end{pspicture}
   \begin{pspicture}(1,0)(4,7)
      \pspolygon(1,5)(7,5)(6,1)
      \psline(6,1)(6,5)
      \psframe(6,5)(6.3,4.7)
      \rput{-40}(3.2,2.5){\ucm{6,9}}
      \rput{78}(7,3){\ucm{4,6}}
      \rput(4.5,5.5){\ucm{6,6}}
      \rput{90}(5.5,2.9){\ucm{4,4}}
      \rput(4,4){\ding{185}}
   \end{pspicture}}
\end{exercice}

\begin{exercice} %2
   Compléter le tableau où $c$ est la longueur du côté du carré et $\mathcal{A}$ son aire.
   \begin{center}
      {\hautab{1.3}
      \begin{Ctableau}{0.9\linewidth}{5}{c}
         \hline
         $c$ & \ucm{3} & \udm{7} & & \\
         \hline
         $\mathcal{A}$ & & & \ummq{64} & \ukmq{36} \\
         \hline  
      \end{Ctableau}} \medskip
   \end{center}
\end{exercice}

\begin{exercice} %3
   Compléter le tableau où $L$ est la longueur du côté du rectangle, $\ell$ sa largeur et $\mathcal{A}$ son aire.
   \begin{center}
      {\hautab{1.3}
      \begin{Ctableau}{0.9\linewidth}{5}{c}
         \hline
         $L$ &\udm{ 3,5} & \umm{7,4} & & \um{7,2} \\
         \hline
         $\ell$ & \udm{2,8} & \umm{21} & \ukm{3,75} & \\
         \hline
         $\mathcal{A}$ & & & \ukmq{23} & \umq{25,6} \\
         \hline  
      \end{Ctableau}}
   \end{center}
\end{exercice}

\begin{exercice} %4
   Compléter le tableau où $b$ est la base du triangle, $h$ sa hauteur et $\mathcal{A}$ son aire.
   \begin{center}
      {\hautab{1.3}
      \begin{Ctableau}{0.9\linewidth}{5}{c}
         \hline
         $b$ &\ucm{ 3} & \udm{7,4} & & \um{5,4} \\
         \hline
         $h$ & \ucm{8} & \udm{2,1} & \ukm{4} & \\
         \hline
         $\mathcal{A}$ & & & \ukmq{12} & \umq{13,5} \\
         \hline  
      \end{Ctableau}}
   \end{center}
\end{exercice}

\medskip

\serie{Aire de disques} %%%%%%%%%%%%%%%%

\begin{exercice} %2
   Compléter le tableau où $r$ est le rayon du disque, $d$ son diamètre et $\mathcal{A}$ son aire.
   \begin{center}
      {\hautab{1.3}
      \begin{Ctableau}{0.9\linewidth}{5}{c}
         \hline
         $r$ & \ucm{5} & \udm{7,5} & & \\
         \hline
         $d$ & & & \umm{8} & \ukm{61} \\
         \hline
         $\mathcal{A}$ & & & & \\
         \hline  
      \end{Ctableau}} \medskip
   \end{center}
\end{exercice}

\begin{exercice} %4
   Calculer l'aire des figures suivantes. \\
   {\psset{unit=0.5}
   \small
    \begin{pspicture}(-1.5,0)(6,6)
      \pscircle(3,3){2}
      \psline(3,3)(5,3)
      \rput(4,2.5){15 km}
      \rput(2.5,3){\ding{182}}
   \end{pspicture}
   \begin{pspicture}(9.5,0)(14,6)
      \pscircle(12,3){2}
      \psline(10,3)(14,3)
      \rput(12,3.5){5,6 m}
      \rput(12,2.5){\ding{183}}
   \end{pspicture}
   
   \begin{pspicture}(7,1)(14,5)
      \psarc(11,2){3}{0}{180}
      \psline(8,2)(14,2)
      \rput(11,1.5){\umm{8,3}}
      \rput(11,3){\ding{184}}
   \end{pspicture}
   \begin{pspicture}(-2.5,0)(4,4)
      \psframe(0,0)(4,4)
      \psarc(4,2){2}{-90}{90}
      \rput(2,0.8){\ucm{20}}
      \psdots(0,2)(2,4)(2,0)(4,2)
      \rput(3,2){\ding{185}}
   \end{pspicture}
   }
\end{exercice}

\bigskip

%%%%%%%
\serie{Problèmes}

\begin{exercice}
   Le rectangle suivant a pour longueur $AC =\ucm{30}$ et pour largeur $AG =\ucm{20}$.
   \begin{center}
   {\footnotesize
      \begin{pspicture}(0,0)(3,2.5)
         \pstGeonode[CurveType=polygon,PointSymbol=none,PosAngle={-135,-90,-45,0,45,90,135,180}]{G}(1.9,0){F}(3,0){E}(3,0.9){D}(3,2){C}(0.9,2){B}(0,2){A}(0,0.7){H}
         \pspolygon(1.9,0)(3,0.9)(0.9,2)(0,0.7)
         \pstLabelAB{A}{B}{\ucm{9}}
         \pstLabelAB{C}{D}{\ucm{11}}
         \pstLabelAB[offset=-0.3]{F}{E}{\ucm{11}}
         \pstLabelAB{H}{A}{\ucm{13}}
      \end{pspicture}}
   \end{center}
   \begin{enumerate}
      \item En utilisant les données de la figure, calculer la mesure de $BC, DE, GF$ et $GH$.
      \item Déterminer l'aire des triangles $HAB, BCD, DEF$ et $FGH$.
      \item En déduire l'aire du quadrilatère $BDFH$.
   \end{enumerate}
\end{exercice}

\begin{exercice}
   On considère la figure suivante : \\
   {\psset{unit=0.7,linewidth=0.5mm}
   \small
    \begin{pspicture}(-4,-1)(8,4.5)
      \psgrid[subgriddiv=0,gridlabels=0pt,gridcolor=lightgray](-1,-1)(8,4)
      \psarc(3,0){3}{90}{180}
      \psarc(3,1){2}{0}{90}
      \psarc(4,1){1}{-90}{0}
      \psline(0,0)(4,0)
      \psline{<->}(6,2)(7,2)
      \rput(6.5,1.5){\ucm{4}}
      \psset{linewidth=0.4mm}
      \psline(3,0)(3,3)
      \psline(3,1)(5,1)
      \psline(4,0)(4,1)
   \end{pspicture}}
   \begin{enumerate}
      \item De quoi est composée cette figure ?
      \item Déterminer alors son aire.
   \end{enumerate}
\end{exercice}

\end{colonne*exercice}


%%%%%%%%%%%%%%%%%%%%%%%%%%%%%%%%%%%%%
%%%%%%%%%%%%%%%%%%%%%%%%%%%%%%%%%%%%%
\Recreation

\enigme[La formule de Pick]
   On travaille dans un réseau pointé à maille carrée. On note $u.\ell.$ l'unité de longueur et $u.a.$ l'unité d'aire. \\
   On appelle polygone de Pick, un polygone non aplati construit sur un tel réseau et dont chacun des sommets est un point du réseau. On considère la figure $FORMULES$ suivante :
   \begin{center}
      \small
      {\psset{unit=0.5}
      \begin{pspicture}(6,0)(22,11.5)
         \pstGeonode[fillstyle=solid,fillcolor=lightgray!30,CurveType=polygon,PosAngle={90,135,-135,-45,-45,45,45}](9,10){R}(7,6){O}(7,1){F}(12,1){S}(12,3){E}(16,3){L}(16,6){U}(12,6){M}
         \psframe[fillstyle=solid,fillcolor=lightgray!30](20,5)(21,6)
         \psgrid[griddots=1,gridlabels=0,subgriddiv=1,gridwidth=0.5mm](6,0)(22,11)
         \rput(20.5,4.5){1 $u.a.$}
         \psline{<->}(20,9)(21,9)
         \rput(20.5,8.5){1 $u.\ell.$}      
      \end{pspicture}}
   \end{center}

   \partie[avec les formules classiques]
      \begin{enumerate}
         \item Décomposer la figure $FORMULES$ en figures connues dont on connait les formules de calcul d'aire ? \\ [2mm]
            \pf \\
         \item Calculer alors l'aire du polygone $FORMULES$, en unités d'aire, en détaillant les étapes du raisonnement. \\ [3mm]
            \pf \\ [3mm]
            \pf \\ [3mm]
            \pf
      \end{enumerate}

   \partie[avec la formule de Pick]
      La formule de Pick permet de calculer l'aire $\mathcal{A}$ d'un polygone de Pick, à partir du nombre $i$ de points du réseau strictement intérieurs à ce polygone et du nombre $b$ de points du réseau sur le bord du polygone : \fbox{$\mathcal{A} =i+\dfrac{b}{2}-1$}. 
      \begin{enumerate}
      \setcounter{enumi}{2}
          \item Appliquer cette formule au polygone $FORMULES$. \\ [3mm]
            $i =\pfb \quad b =\pfb$ \quad donc, $\mathcal{A} = \pfb$ \\
          \item Appliquer la formule de Pick aux trois polygones de Pick $MOFS$, $MOR$ et $MULE$. \\
         Vérifier que la somme des résultats obtenus est égale à l'aire totale de la figure. \\ [3mm]
         $MOFS : \;\; i =\pfb \quad b =\pfb$ \quad donc, $\mathcal{A}_1 = \pfb$ \\ [3mm]
         $MOR : \quad i =\pfb \quad b =\pfb$ \quad donc, $\mathcal{A}_2 = \pfb$ \\ [3mm]
         $MULE : \; i =\pfb \quad b =\pfb$ \quad donc, $\mathcal{A}_3 = \pfb$ \\ [3mm]
         Aire totale : \pf
      \end{enumerate}


