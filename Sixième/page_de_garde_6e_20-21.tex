\usetikzlibrary{calc}

% Couleurs blue, brown, cyan, green, lime, magenta, olive, orange, pink, purple, red, teal, violet, white, black, gray, yellow
\tikzset{fondA/.style=blue}
\tikzset{fondB/.style=Cyan}
\tikzset{fondPostit/.style={color= yellow!40}}
\tikzset{ombrePunaise/.style={color={blue!10!gray}}}
\tikzset{ombrePostit/.style={color={black},opacity=.5}}
\tikzset{punaise/.style={ball color=red}}

% \epingle{point}{angle}{échelle}
\newcommand{\epingle}[3]{
\coordinate[rotate=#2,yshift={#3*0.375cm}] (e) at #1;
\coordinate[shift={++(60:0.75)}] (g) at (e);
\begin{scope} [scale=1.5]
 \begin{scope}[rotate=-30]
   \coordinate[shift={++(30:0.75)}] (h) at (e);
   \draw[ombrePunaise,line cap=round,line width=4pt] (e) -- ++(60:0.75);
   \fill [ombrePunaise,rotate=-30,scale=0.5] (h) ellipse (.65 and .3) ;
   \fill [ombrePunaise,rotate=60,scale=0.5] (h) ++(0.4,0) ellipse (.4 and .3);
   \fill [ombrePunaise,rotate=60,scale=0.5] (h) ++(0.8,0) ellipse (.2 and .4);
 \end{scope}
 \draw[line cap=round,line width=4pt] (e) -- ++(60:0.75);
 \fill [punaise,rotate=-30,scale=0.5] (g) ellipse (.65cm and .3cm) ;
 \fill [punaise,rotate=60,scale=0.5] (g) ++(0.4,0) ellipse (.4 and .3);
 \fill [punaise,rotate=60,scale=0.5] (g) ++(0.8,0) ellipse (.2 and .4);
\end{scope}}

% \postIt{(point)}{angle}{échelle}{ligne 1}{ligne 2}{ligne 3}}
\newcommand{\postIt}[6]
{\begin{scope} [rotate=#2,scale=0.8]
\fill [red,ombrePostit] #1 ++ ($#3*(-1.45,0.72)$) -- ++ ($#3*(2.86,0)$) 
 .. controls+(0,0)and+($#3*(-0.25,0.05)$).. ++ ($#3*(0.25,-2.4)$)
 .. controls+($#3*(-0.1,-0.1)$)and+(0,0).. ++ ($#3*(-2.95,0.1)$)
 -- cycle;
\fill [ombrePostit] #1 ++ ($#3*(-1.45,0.72)$) -- ++ ($#3*(2.86,0)$) 
 .. controls+(0,0)and+($#3*(-0.25,0.05)$).. ++ ($#3*(0.2,-2.35)$)
 .. controls+($#3*(-0.1,-0.1)$)and+(0,0).. ++ ($#3*(-2.95,0.1)$)
 -- cycle;
\fill [ombrePostit] #1 ++ ($#3*(-1.45,0.72)$) -- ++ ($#3*(2.86,0)$) 
 .. controls+(0,0)and+($#3*(-0.25,0.05)$).. ++ ($#3*(0.15,-2.3)$)
 .. controls+($#3*(-0.1,-0.1)$)and+(0,0).. ++ ($#3*(-2.95,0.1)$)
 -- cycle;
\fill [fondPostit] #1 ++ ($#3*(-1.45,0.72)$) -- ++ ($#3*(2.86,0)$) 
 .. controls+(0,0)and+($#3*(-0.2,0.1)$).. ++($#3*(0.1,-2.25)$)
 .. controls+($#3*(-0.1,-0.1)$)and+(0,0).. ++ ($#3*(-2.95,0.1)$)
 -- cycle;
\end{scope}
\epingle{#1}{#2}{#3}
\draw #1 node [scale=#3,rotate=#2] {#4};
\draw #1 node [scale=#3,rotate=#2,below={#3*0.2cm}] {#5};
\draw #1 node [scale=#3,rotate=#2,below={#3*0.4cm}] {#6};}


%=======================================
\begin{tikzpicture}[remember picture,overlay]
% fond bicolore
\coordinate (cp) at (current page);
\coordinate (cpc) at (current page.center);
\coordinate (cpe) at ($ (current page.east) + (1.7cm,0cm) $);
\coordinate (cpne) at ($ (current page.north east) + (1.7cm,2.9cm) $);
\coordinate (cpn) at ($ (current page.north) + (1.5cm,2.9cm) $);
\coordinate (cpnw) at ($ (current page.north west) + (-0.2cm,2.9cm) $);
\coordinate (cpw) at ($ (current page.west) + (-0.2cm,0cm) $);
\coordinate (cpsw) at ($ (current page.south west) + (-0.2cm,-0.2cm) $);
\coordinate (cps) at ($ (current page.south) + (1.5cm,-0.2cm) $);
\coordinate (cpse) at ($ (current page.south east) + (1.7cm,-0.2cm) $);
\fill[fondA] (cps) .. controls (cpw) and (cpe) .. (cpn) -- (cpnw)  -- (cpsw) -- cycle;
\fill[fondB] (cps) .. controls (cpw) and (cpe) .. (cpn) -- (cpne)  -- (cpse) -- cycle;


% Titre
\draw (cp)  node [xshift=-2cm,yshift=9cm,scale=8] {\textcolor{Cyan}{Manuel}};
\draw (cp)  node [xshift=-2cm,yshift=6.5cm,scale=3.5] {\textcolor{Cyan}{de Mathématiques}};
\draw (cp)  node [xshift=-2cm,yshift=4cm,scale=8] {\textcolor{Cyan}{6\up{e}}};
\draw (cp)  node [xshift=-6cm,yshift=-1.5cm]{\psHomothetie[linecolor=Cyan](0,0){1.5}{\psBill}};

\draw (cp)  node [xshift=5.5cm,yshift=0cm]{\psKangaroo[linecolor=blue,fillcolor=green]{3}};
\draw (cp)  node [xshift=6.9cm,yshift=0cm]{\psKangaroo[linecolor=blue,fillcolor=orange]{3}};
\draw (cp)  node [xshift=7.63cm,yshift=-1.5cm,xscale=-1]{\psKangaroo[linecolor=blue,fillcolor=violet]{3}};
\draw (cp)  node [xshift=6.23cm,yshift=-1.5cm,xscale=-1]{\psKangaroo[linecolor=blue,fillcolor=red]{3}};
\draw (cp)  node [xshift=4.82cm,yshift=-1.5cm,xscale=-1]
{\psKangaroo[linecolor=blue,fillcolor=teal]{3}};
\draw (cp)  node [xshift=5cm,yshift=-5cm,scale=3] {\textcolor{blue}{\cursive Montpellier}};
\draw (cp)  node [xshift=5cm,yshift=-7cm,scale=3] {\textcolor{blue}{\cursive Collège Simone Veil}};

% post'it
\coordinate[xshift=-5.5cm,yshift=-9.5cm] (postit) at (cp);
\postIt{(postit)}{-17}{1.75} {\footnotesize\cursive Nathalie Daval}{\footnotesize année}{\footnotesize 2020-2021}


\end{tikzpicture}

\pagebreak

\thispagestyle{empty}

Ce manuel est composé de l'ensemble des activités, cours, exercices pour la classe de 6\up{e} du collège Simone Veil de Montpellier pendant l'année 2020-2021. \\ [1mm]
Il a été écrit en \LaTeX{} avec la classe \href{https://www.ctan.org/pkg/sesamanuel}{\blue sesamanuel} distribuée librement par l'association \href{http://www.sesamath.net}{\blue sesamath}. Si vous y voyez des erreurs ou des coquilles, même minimes, vous pouvez me les signaler à cette adresse : \href{mailto:nathalie.daval@ac-montpellier.fr}{nathalie.daval@ac-montpellier.fr} \\
Je remercie à ce propos Jean-Félix Navarro qui a effectué une relecture attentive de ce livret. [10mm]


La progression est dite spiralée, c'est-à-dire que chaque notion est découpée en plusieurs courts chapitres conçus pour durer une semaine, ce qui permet de revoir les notions plusieurs fois dans l'année.

Chaque chapitre du présent manuel est composé de la manière suivante : \\
\begin{itemize}
   \item \textcolor{B2}{\sffamily\bfseries Connaissances et compétences associées} : les connaissances et compétences associées au cycle 3 définies par le \href{https://cache.media.eduscol.education.fr/file/A-Scolarite_obligatoire/37/5/Programme2020_cycle_3_comparatif_1313375.pdf}{programme en vigueur à compter de la rentrée de l'année scolaire 2018-2019}. \\
   \item \textcolor{C1}{\sffamily\bfseries Débat} : un petit texte culturel illustré permettant d'échanger sur un thème en rapport au chapitre. Un morceau d'histoire, de l'étymologie, du vocabulaire, une curiosité mathématique\dots{} le tout agrémenté d'une courte vidéo de vulgarisation scientifique. \\
   \item \textcolor{PartieGeometrie}{\sffamily\bfseries Cahier de compétences} : les pages correspondant au chapitre dans le cahier MYRIADE. Chaque élève pourra, s'il le souhaite, travailler en autonomie en classe ou à la maison sur ce cahier. \\
   \item \colorbox{G2}{\textcolor{white}{\sffamily\bfseries Activité d'approche}} : une activité à faire en classe permettant de découvrir une notion du chapitre. \\
   \item \colorbox{A2}{\textcolor{white}{\sffamily\bfseries Trace écrite}} : l'essentiel du cours à connaître. \\
   \item \colorbox{C2}{\textcolor{white}{\sffamily\bfseries Entraînement}} : les exercices à faire en priorité.  \\   
   \item \colorbox{PartieStatistique}{\textcolor{white}{\sffamily\bfseries Récréation, énigmes}} : une activité ludique liée au chapitre.
   \item
\end{itemize}

\ \\ [10mm]

En complément de ce manuel, nous travaillerons sur le {\it Cahier de compétences - maths, collection Myriade} de chez Bordas de manière plus individuelle comme outil d'entraînement, de réinvestissement, d'approfondissement ou d'évaluation. Mais surtout, en cas de période de confinement total ou partiel, il permettra une meilleure continuité pédagogique en plus des heures de classes virtuelles.



