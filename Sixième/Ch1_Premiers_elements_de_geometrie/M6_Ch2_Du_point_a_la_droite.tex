\themaG
\graphicspath{{../Ch1_Premiers_elements_de_geometrie/Images/}}

\chapter{Du point à la droite}
\label{C02}

%%%%%%%%%%%%%%%%%%%%%%%%%%%%%%%%%%%%%%%%%%
\begin{prerequis}[Connaissances et compétences abordées]
   \begin{itemize}
      \item Alignement, appartenance.
      \item Segment de droite.
      \item Vocabulaire associé au milieu et à ses propriétés.
   \end{itemize}
\end{prerequis}

\vfill

\begin{debat}[Débat : les définition d'Euclide]
   La mathématicien grec {\it Euclide}, considéré comme le père de la géométrie, définit les objets géométriques au III\up{e} siècle av. J.-C. Son 1\up{er} livre comprend notamment les définitions suivantes :
   \begin{itemize}
      \item le {\bf point} est ce dont la partie est nulle ;
      \item une {\bf ligne} est une longueur sans largeur ;
      \item la {\bf ligne droite} est celle qui est également placée entre ses points.
   \end{itemize}
   \begin{pspicture}(-4,-0.25)(9,2.75)
      \psset{linecolor=B1}
      \psdot(0,1)
      \psbezier(2,0)(3,2)(4,0.5)(5,2)
      \psline(6,0)(9,2)
   \end{pspicture}
   \bigskip
   \begin{cadre}[B2][F4]
      \begin{center}
         Vidéo : \href{https://www.youtube.com/watch?v=enZpq8jvFEs}{\bf L'axiomatique. Les éléments d'Euclide}, chaîne YouTube de {\it Monsieur Phi}, de 2 min 30 s à 8 min.
      \end{center}
   \end{cadre}
\end{debat}

\vfill

\textcolor{PartieGeometrie}{\large\sffamily\bfseries Cahier de compétences} : $\varnothing$


%%%%%%%%%%%%%%%%%%%%%%%%%%%%%%%%%%%%
%%%%%%%%%%%%%%%%%%%%%%%%%%%%%%%%%%%%%
\activites

\begin{activite}[Faisceaux de traits]
   {\bf Objectifs :} employer des relations d'incidence comme l'appartenance d'un point
à un ou deux segments ou droites, intersection, alignement ; mise en place d'une chronologie de tracé. 
   \begin{QCM}
      \partie[première construction]
         {\psset{unit=0.7}
         \begin{pspicture}(-5,-1)(15,11)
            \pstGeonode[PointName=none,linewidth=1mm,PointSymbol=+](2,9){Z}(3.5,3){Y}(6,3.5){C}(8,6){D}(4,1){E}(5.42,4.27){F}(11,4.5){G}
            \pstLineAB[nodesepA=-4,nodesepB=-5]{C}{D}
            \pstLineAB[nodesepA=-1,nodesepB=-3]{Z}{Y}
            \pstLineAB[nodesepA=-4,nodesepB=-7]{Y}{D}
            \pstLineAB[nodesepA=-4,nodesepB=-8]{Y}{C}
            \pstLineAB[nodesepA=-1,nodesepB=-4]{Z}{C}
            \pstLineAB[nodesepA=-1,nodesepB=-7]{Z}{D}
         \end{pspicture}} \\
         Reproduire la figure ci-dessus en s'aidant des points d'intersection déjà placés. \\
         \begin{pspicture}(-4,0)(15,10)
            \pstGeonode[PointName=none,linewidth=1mm,PointSymbol=+](2,9){Z}(3.5,3){Y}(6,3.5){C}(8,6){D}
         \end{pspicture}
   \end{QCM}

   \begin{QCM}
      \partie[seconde construction]
       Reproduire la figure projetée au tableau en s'aidant des points d'intersection déjà placés. \\
         \begin{pspicture}(0.5,0)(15,19)
            \pstGeonode[PointName=none,linewidth=1mm,PointSymbol=+](9.5,5){Z}(10,6.5){Y}(10.5,13){C}(12.5,14){D}(15.5,2.5){E}
         \end{pspicture}
   \end{QCM}

\pagebreak

   Figure à projeter : \\
   \begin{pspicture}(0,0)(16,19) %PointName=none,
      \pstGeonode[linewidth=1mm,PointSymbol=+,PointName=none](9.5,5){A}(10,6.5){B}(10.5,13){C}(12.5,14){D}(15.5,2.5){E}
      \pstInterLL[linewidth=1mm,PointSymbol=+,PointName=none]{A}{E}{C}{D}{F}
      \pstInterLL[linewidth=1mm,PointSymbol=+,PointName=none]{B}{E}{C}{D}{G}
      \pstInterLL[linewidth=1mm,PointSymbol=+,PointName=none]{A}{G}{B}{F}{H}
      \pstInterLL[linewidth=1mm,PointSymbol=+,PointName=none]{B}{D}{C}{E}{I}
      \pstInterLL[linewidth=1mm,PointSymbol=+,PointName=none]{H}{D}{B}{G}{J}
      \pstInterLL[linewidth=1mm,PointSymbol=+,PointName=none]{B}{C}{G}{I}{K}
      \pstInterLL[linewidth=1mm,PointSymbol=+,PointName=none]{E}{C}{D}{H}{L}
      \pstLineAB[nodesepA=-10,nodesepB=-2]{A}{E}
      \pstLineAB[nodesepA=-10,nodesepB=-2]{B}{E}
      \pstLineAB[nodesepA=-3,nodesepB=-2]{A}{D}
      \pstLineAB[nodesepA=-12,nodesepB=-2]{C}{D}
      \pstLineAB[nodesepA=-3,nodesepB=-2]{C}{E}
      \pstLineAB{A}{G}
      \pstLineAB{B}{F}
      \pstLineAB{B}{F}
      \pstLineAB{H}{D}
      \pstLineAB{G}{I}
   \end{pspicture}
\end{activite}

 
%%%%%%%%%%%%%%%%%%%%%%%%%%%%%%%%%
%%%%%%%%%%%%%%%%%%%%%%%%%%%%%%%%%
\cours 

\section{Droites, demi-droites, segments} %%%%%%%%%

\begin{definition}
   \begin{itemize}
      \item Une {\bf droite} est une ligne rectiligne infinie. On peut noter une droite de différentes façons : \\
      \begin{pspicture}(-0.5,-0.7)(13,1)
         \pstGeonode[PosAngle=-90,PointSymbol=+](0,0){A}(3,0.5){B}
         \pstLineAB[nodesep=-0.5,linecolor=A1]{A}{B}
         \pstLabelAB{A}{B}{\small droite $(AB)$}
         \psline[linecolor=A1](4,0.25)(8,0)
         \rput(7.25,-0.35){$(d)$}
         \rput{-4}(6,0.5){\small droite $(d)$}
         \psline[linecolor=A1](8.5,-0.25)(12.5,0.5)
         \rput(9,-0.15){+}
         \rput(9,-0.5){$T$}
         \rput(12,0.2){$u$}
         \rput{9}(10.5,0.5){\small droite $(Tu)$}
      \end{pspicture}
      \item Une {\bf demi-droite} est une portion de droite limitée d'un seul côté par un point appelé origine. La demi-droite d'origine $A$ passant par $B$ se note $[AB)$. \\
      \begin{pspicture}(-5,-0.2)(5,1)
         \pstGeonode[PosAngle=180,PointSymbol=+](0,0){A}
         \pstGeonode[PosAngle=-90,PointSymbol=+](4,0.5){B}
         \pstLineAB[nodesepB=-1,linecolor=A1]{A}{B}
         \pstLabelAB{A}{B}{\small demi-droite $[AB)$}
      \end{pspicture}
      \item Un \textbf{segment} est une portion de droite limitée par deux points appelés extrémités. Le segment d'extrémités $A$ et $B$ se note $[AB]$ ou $[BA]$. \\
      \begin{pspicture}(-5,0.4)(5,1.2)
         \pstGeonode[PosAngle={180,-90},PointSymbol=+](0,0.2){A}(4,0.7){B}
         \pstLineAB[linecolor=A1]{A}{B}
         \pstLabelAB{A}{B}{\small segment $[AB]$}
      \end{pspicture}
   \end{itemize}
\end{definition}

\begin{exemple*1}
   \begin{minipage}{5cm}
      \begin{pspicture}(-1,-0.25)(4,1.8)
         \pstGeonode[PosAngle=-90,PointSymbol=+](0,0){A}(3,0){B}(2,1.25){C}
         \pstLineAB[nodesep=-0.5]{A}{B}
         \pstLineAB[nodesepB=-0.75]{A}{C}
         \pstLineAB{C}{B}
      \end{pspicture}
   \end{minipage}
   \begin{minipage}{6.5cm}
   \begin{itemize}
      \item $(AB)$ est une droite ;
      \item $[AC)$ est la demi-droite d'origine $A$ passant par $C$ ;
      \item $[CB]$ est le segment d'extrémités $C$ et $B$.
   \end{itemize}
   \end{minipage}
\end{exemple*1}


\section{Points particuliers} %%%%%%%%%

\begin{definition}
   Trois points sont {\bf alignés} s'ils appartiennent à une même droite.
   \begin{center}
   \begin{pspicture}(-2,-0.3)(11,-0.1)
      \pstGeonode[PosAngle=-90,PointSymbol=+](0,0){A}(4,0){B}(1.5,0){C}
      \pstLineAB[nodesep=-1]{A}{B}
      \rput(8,0){\small les points $A, B$ et $C$ sont alignés.}
   \end{pspicture}
   \end{center}
\end{definition}

\begin{notation}
   le symbole $\in$ signifie \og {\bf appartient} à \fg{} et $\not\in$ signifie \og n'appartient pas à \fg.
\end{notation}

\begin{exemple*1}
   On a par exemple $C\in(AB)$ et $B\in(AC)$ mais $A\notin[CB)$ et $B\notin[AC]$.
\end{exemple*1}

\begin{definition}
   Le {\bf milieu} $I$ du segment $[AB]$ est le point de ce segment qui est équidistant de $A$ et de $B$.
\end{definition}

\begin{remarque}
   la longueur d'un segment $[AB]$ se note $AB$, et pour indiquer que l'on a des mesures égales, c'est à dire $AI =IB$, on effectue un codage de la figure. \\
   Exemples de codes que l'on peut utiliser : \textcolor{B1}{\textsf x, +, o, /\!\!/} \dots
\end{remarque}

\begin{exemple*1}
   \begin{pspicture}(-1,1.5)(5,2) 
      \psline[linecolor=A1]{|-|}(0,1.5)(5,1.5)
      \psline[linecolor=B1](2.5,1.4)(2.5,1.6)
      \rput(0,1.9){$F$}
      \rput(2.5,1.9){$I$}
      \rput(5,1.9){$L$}
      \rput(1.25,1.5){\textcolor{B1}{\Large$\times$}}
      \rput(3.75,1.5){\textcolor{B1}{\Large$\times$}}
   \end{pspicture}
\end{exemple*1}

\medskip

\begin{definition}
   Deux droites sont {\bf sécantes} lorsqu'elles se coupent en un point appelé {\bf point d'intersection}.
\end{definition}


%%%%%%%%%%%%%%%%%%%%%%%%%%%%%%%%%%%%%%%%%%
\exercicesbase

\begin{colonne*exercice}

\serie{Droites, demi-droites, segments} %%%%%%%%%%%%%%%%%%%%%
 
\begin{exercice} %1
   Repasser en rouge la partie de la droite correspondant aux écritures mathématiques.
   \begin{enumerate}
      \item \begin{pspicture}(0,-0.1)(7,0.4)
                  \small
                  \psline(0,0)(6,0)
                  \pstGeonode[PointSymbol=+,PosAngle=-90](1,0){A}(2,0){B}(3,0){C}(4,0){D}(5,0){E}
                  \rput(7,0){$[AC)$}
               \end{pspicture}               
      \item \begin{pspicture}(0,-0.1)(7,0.7)
                  \small
                  \psline(0,0)(6,0)
                  \pstGeonode[PointSymbol=+,PosAngle=-90](1,0){D}(2.3,0){R}(3.1,0){O}(4.5,0){I}
                  \rput(5.7,-0.2){t}
                  \rput(7,0){$[IR]$}
               \end{pspicture}
      \item \begin{pspicture}(0,-0.1)(7,0.7)
                  \small
                  \psline(0,0)(6,0)
                  \pstGeonode[PointSymbol=+,PosAngle=-90](1.2,0){A}(2.2,0){U}(3.2,0){C}(4.2,0){H}(5.2,0){E}
                  \rput(0.2,-0.2){$g$}
                  \rput(7,0){$(gH]$}
               \end{pspicture} \\
   \end{enumerate}
\end{exercice}
 
\bigskip
 
\begin{exercice} %2
   Nommer la partie de la droite qui a été repassée en gras de deux manières différentes.
   \begin{enumerate}
      \item \begin{pspicture}(0,-0.1)(7,0.4)
                  \small
                  \psline(0,0)(6,0)
                  \pstGeonode[PointSymbol=+,PosAngle=-90](1,0){A}(2,0){B}(3,0){C}(4,0){D}(5,0){E}
                  \pstLineAB[linewidth=0.5mm]{C}{B}
               \end{pspicture}               
      \item \begin{pspicture}(0,-0.1)(7,0.7)
                  \small
                  \psline(0,0)(6,0)
                  \pstGeonode[PointSymbol=+,PosAngle=-90](1,0){D}(2.3,0){R}(3.1,0){O}(4.5,0){I}
                  \rput(5.7,-0.2){t}
                  \psline[linewidth=0.5mm](3.1,0)(6,0)
               \end{pspicture}
      \item \begin{pspicture}(0,-0.1)(7,0.7)
                  \small
                  \psline(0,0)(6,0)
                  \pstGeonode[PointSymbol=+,PosAngle=-90](1.2,0){A}(2.2,0){U}(3.2,0){C}(4.2,0){H}(5.2,0){E}
                  \rput(0.2,-0.2){$g$}
                  \psline[linewidth=0.5mm](0,0)(2.2,0)
               \end{pspicture} \\ [-1mm]
   \end{enumerate}
\end{exercice}

\bigskip

\begin{exercice} %3
   On considère la droite suivante : \\
   \begin{pspicture}(-1,-0.4)(7,0.4)
      \small
      \psline(0,0)(6,0)
      \pstGeonode[PointSymbol=+,PosAngle=-90](1,0){A}(2.25,0){B}(3.5,0){C}(5,0){D}
   \end{pspicture}           
   \begin{enumerate}
      \item Écrire tous les noms possibles pour cette droite.
      \item Combien y aurait-il de noms en plus si on avait placé cinq points sur la droite ?
      \item Combien faut-il de points pour que la droite ait six noms possibles ?
   \end{enumerate}
\end{exercice}

\bigskip

\serie{Points particuliers} %%%%%%%%%%%%%%%%%%%%%

\begin{exercice} %4
   Compléter les phrases à l'aide de la figure. \\
   {\psset{yunit=0.6}
   \small
   \begin{pspicture}(-0.3,-0.3)(8,5.3)
      \pstGeonode[PointSymbol=none,PosAngle={90,110,90,100,120,90}](1,4){A}(3.5,4){B}(7,4){C}(3,2.8){D}(2.5,1.3){E}(4,2.2){F}
      \pstLineAB[nodesepA=-1,nodesepB=-1]{A}{C}
      \pstLineAB[nodesepA=-0.8,nodesepB=-1.3]{B}{E}
      \pstLineAB[nodesepA=-1,nodesepB=-3]{A}{F}
      \pstLineAB[nodesepA=-1,nodesepB=-2]{C}{E}
      \rput(6.5,1.1){$(d_1)$}
      \rput(0.4,3.6){$(d_2)$}
      \rput(1,0.8){$(d_3)$}
      \rput(2.6,0.3){$(d_4)$}
   \end{pspicture}}
   \begin{enumerate}
      \item Les droites $(d_1)$ et $(d_2)$ se coupent en \pfb
      \item Le point d'intersection de $(d_1)$ et $(d_3)$ est \pfb
      \item $C$ est le point d'intersection de \pfb et \pfb
      \item Le point $B$ est à l'intersection de \pfb et \pfb
   \end{enumerate}
\end{exercice}

\bigskip

\begin{exercice} %5
   Compléter avec $\in$ ou $\notin$. \\
   \begin{pspicture}(-0.3,-0.1)(7,0.6)
      \small
      \psline(0,0)(7.5,0)
      \pstGeonode[PointSymbol=+,PosAngle=90](1,0){O}(3,0){U}(6,0){F}
   \end{pspicture}  
   \begin{colenumerate}{3}
      \item $O \pfb [UF]$
      \item $O \pfb [UF)$
      \item $O \pfb (UF)$
      \item $U \pfb [FO)$
      \item $U \pfb [OF)$
      \item $F \pfb (OU)$
   \end{colenumerate}
\end{exercice}

\bigskip

\begin{exercice} %6
   Compléter avec $\in$ ou $\notin$. \\
   {\psset{yunit=0.6}
   \begin{pspicture}(-0.5,0.6)(8,5.2)
      \pstGeonode[PointSymbol=+,PosAngle={90,110,90,100,120,90}](1,4){Q}(5,4){X}(7,4){M}(3.63,2.52){O}(2.5,1.3){L}(6,1.2){Z}(1,2){V}
      \pstLineAB[nodesepA=-1,nodesepB=-0.5]{Q}{M}
      \pstLineAB[nodesepA=-0.8,nodesepB=-1]{Q}{Z}
      \pstLineAB[nodesepB=-1]{L}{X}
   \end{pspicture}}
   \begin{colenumerate}{3}
      \item $X \pfb (QM)$
      \item $X \pfb [QM]$
      \item $Q \pfb [XM]$
      \item $X \pfb [QM)$
      \item $Q \pfb (OZ)$
      \item $O \pfb [LX]$
      \item $L \pfb [XO)$
      \item $V \pfb (OM)$
      \item $L \pfb [OX)$
   \end{colenumerate}
\end{exercice}

\bigskip

\begin{exercice} %7
   Vrai (V) ou faux (F) ?
   \begin{enumerate}
      \item Si $C \in (AB)$ alors $A \in (BC)$.
      \item Si $E \in [DF]$ alors $D \in [EF]$.
      \item Si $C \in [AB)$ mais $C \notin [AB]$ alors $A \in [CB)$.
      \item Si $C \in [BA)$ mais $C \notin [AB]$ alors $B \in [AC)$.
   \end{enumerate}
\end{exercice}

\bigskip

\begin{exercice} %8
   En s'aidant des points déjà marqués, placer les points $H, I, L$ et $M$. \\
   {\psset{yunit=0.5}
   \begin{pspicture}(-0.5,0)(8,5.5)
      \pstGeonode[PointSymbol=+,PosAngle=90](1.5,4){A}(3.5,2){B}(5.5,4){C}(7,1){D}(0.5,0.5){E}
   \end{pspicture}}
   \begin{colenumerate}{2}
      \item $H \in [AB)$ et $H \in [ED]$
      \item $I \in [CB)$ et $I \in [ED]$
      \item $L \in [BD]$ et $L \in [CH]$
      \item $M \in [AI)$ et $M \in [DH)$
   \end{colenumerate}
\end{exercice}

\bigskip

\begin{exercice}
   Construire le milieu de chaque segment sans utiliser d'instrument de géométrie. Coder la figure.
   \begin{center}
   \psset{xunit=0.5,yunit=0.4}
      \begin{pspicture}(0,0)(13,9)
         \psgrid[gridlabels=0,subgriddiv=0,gridcolor=lightgray](0,0)(13,9)
         \pstGeonode[PointSymbol=+,dotangle=45,PosAngle=-80](1,2){A}(7,2){B}(1,4){C}(2,8){D}(8,8){E}(12,2){F}(4,5){G}(8,4){H}
         \pstLineAB{A}{B}
         \pstLineAB{C}{D}
         \pstLineAB{E}{F}
         \pstLineAB{G}{H}
      \end{pspicture}
   \end{center}
\end{exercice}

\flushright{\it\footnotesize Source : Les cahiers Sesamath 6\up{e}. Magnard-Sésamath 2017.}
\end{colonne*exercice}


%%%%%%%%%%%%%%%%%%%%%%%%%%%%%%%%%%%%%%%%%%
\Recreation

\enigme[La pipopipette]
   \partie[présentation du jeu]
      La pipopipette ou \og jeu des petits carrés \fg{} est un jeu se pratiquant à deux joueurs en tour par tour dont l'idée serait attribuée à des élèves de l'École polytechnique à la fin du {\small XIX}\up{e} siècle : pipo désignait à l'époque cette école en argot. \\
      {\bf But du jeu :} former des carrés. Le gagnant est celui qui a fermé le plus de carrés. \\
      {\bf Règles du jeu :}
         \begin{itemize}
            \item À chaque tour, chaque joueur trace un petit segment suivant le quadrillage de la feuille.
            \item Chaque fois qu'un joueur peut fermer un carré, il marque le carré de son signe rejoue.
            \item Quand la grille est remplie, on compte le nombre de carrés fermés pour chaque joueur.
         \medskip
         \end{itemize}
      {\bf Exemple avec un terrain de 2 $\times$ 2:}
         \begin{center}
         {\psset{unit=0.7}
            \begin{tabular}{*{6}{C{2.4}}}
               \begin{pspicture}(0,0)(2,1.8)
                  \psgrid[subgriddiv=0,gridlabels=0,gridcolor=lightgray](0,0)(2,2)
                  \psset{linewidth=0.5mm}
                  \psline[linecolor=A1](0,2)(1,2)
               \end{pspicture}
               &
               \begin{pspicture}(0,0)(2,1.8)
                  \psgrid[subgriddiv=0,gridlabels=0,gridcolor=lightgray](0,0)(2,2)
                  \psset{linewidth=0.5mm,linecolor=A1}
                  \psline(0,2)(1,2)
                  \psset{linecolor=B1}
                  \psline(1,0)(2,0)
               \end{pspicture}
               &
               \begin{pspicture}(0,0)(2,1.8)
                  \psgrid[subgriddiv=0,gridlabels=0,gridcolor=lightgray](0,0)(2,2)
                  \psset{linewidth=0.5mm,linecolor=A1}
                  \psline(0,2)(2,2)
                  \psset{linecolor=B1}
                  \psline(1,0)(2,0)
               \end{pspicture}
               &
               \begin{pspicture}(0,0)(2,1.8)
                  \psgrid[subgriddiv=0,gridlabels=0,gridcolor=lightgray](0,0)(2,2)
                  \psset{linewidth=0.5mm,linecolor=A1}
                  \psline(0,2)(2,2)
                  \psset{linecolor=B1}
                  \psline(0,0)(2,0)
               \end{pspicture}
               &
               \begin{pspicture}(0,0)(2,1.8)
                  \psgrid[subgriddiv=0,gridlabels=0,gridcolor=lightgray](0,0)(2,2)
                  \psset{linewidth=0.5mm,linecolor=A1}
                  \psline(0,1)(0,2)(2,2)
                  \psset{linecolor=B1}
                  \psline(0,0)(2,0)
               \end{pspicture}
               &
               \begin{pspicture}(0,0)(2,1.8)
                  \psgrid[subgriddiv=0,gridlabels=0,gridcolor=lightgray](0,0)(2,2)
                  \psset{linewidth=0.5mm,linecolor=A1}
                  \psline(0,1)(0,2)(2,2)
                  \psset{linecolor=B1}
                  \psline(0,0)(2,0)(2,1)
               \end{pspicture} \\
               \textcolor{A1}{joueur 1} & \textcolor{B1}{joueur 2} & \textcolor{A1}{joueur 1} & \textcolor{B1}{joueur 2} & \textcolor{A1}{joueur 1} & \textcolor{B1}{joueur 2} \\ [5mm]
               \begin{pspicture}(0,0)(2,2)
                  \psgrid[subgriddiv=0,gridlabels=0,gridcolor=lightgray](0,0)(2,2)
                  \psset{linewidth=0.5mm,linecolor=A1}
                  \psline(1,1)(0,1)(0,2)(2,2)
                  \psset{linecolor=B1}
                  \psline(0,0)(2,0)(2,1)
               \end{pspicture}
               &
               \begin{pspicture}(0,0)(2,2)
                  \psgrid[subgriddiv=0,gridlabels=0,gridcolor=lightgray](0,0)(2,2)
                  \psset{linewidth=0.5mm,linecolor=A1}
                  \psline(1,1)(0,1)(0,2)(2,2)
                  \psset{linecolor=B1}
                  \psline(0,0)(2,0)(2,1)
                  \psline(1,1)(1,2)
                  \psdot[dotstyle=+](0.5,1.5)
               \end{pspicture}
               &
               \begin{pspicture}(0,0)(2,2)
                  \psgrid[subgriddiv=0,gridlabels=0,gridcolor=lightgray](0,0)(2,2)
                  \psset{linewidth=0.5mm,linecolor=A1}
                  \psline(1,1)(0,1)(0,2)(2,2)
                  \psset{linecolor=B1}
                  \psline(0,1)(0,0)(2,0)(2,1)
                  \psline(1,1)(1,2)
                  \psdot[dotstyle=+](0.5,1.5)
               \end{pspicture}
               &
               \begin{pspicture}(0,0)(2,2)
                  \psgrid[subgriddiv=0,gridlabels=0,gridcolor=lightgray](0,0)(2,2)
                  \psset{linewidth=0.5mm,linecolor=A1}
                  \psline(1,0)(1,1)(0,1)(0,2)(2,2)
                  \psdot[dotstyle=*](0.5,0.5)
                  \psset{linecolor=B1}
                  \psline(0,1)(0,0)(2,0)(2,1)
                  \psline(1,1)(1,2)
                  \psdot[dotstyle=+](0.5,1.5)
               \end{pspicture}
               &
               \begin{pspicture}(0,0)(2,2)
                  \psgrid[subgriddiv=0,gridlabels=0,gridcolor=lightgray](0,0)(2,2)
                  \psset{linewidth=0.5mm,linecolor=A1}
                  \psline(1,0)(1,1)(0,1)(0,2)(2,2)
                  \psline(1,1)(2,1)
                  \psdots[dotstyle=*](0.5,0.5)(1.5,0.5)
                  \psset{linecolor=B1}
                  \psline(0,1)(0,0)(2,0)(2,1)
                  \psline(1,1)(1,2)
                  \psdot[dotstyle=+](0.5,1.5)
               \end{pspicture}
               &
                \begin{pspicture}(0,0)(2,2)
                 \psgrid[subgriddiv=0,gridlabels=0,gridcolor=lightgray](0,0)(2,2)
                  \psset{linewidth=0.5mm,linecolor=A1}
                  \psline(1,0)(1,1)(0,1)(0,2)(2,2)
                  \psline(1,1)(2,1)(2,2)
                  \psdots[dotstyle=*](0.5,0.5)(1.5,0.5)(1.5,1.5)
                  \psset{linecolor=B1}
                  \psline(0,1)(0,0)(2,0)(2,1)
                  \psline(1,1)(1,2)
                  \psdot[dotstyle=+](0.5,1.5)
               \end{pspicture}
               \\
               \textcolor{A1}{joueur 1} & \textcolor{B1}{joueur 2} & \textcolor{B1}{joueur 2} & \textcolor{A1}{joueur 1} & \textcolor{A1}{joueur 1} & \textcolor{A1}{joueur 1} \\
            \end{tabular}}
         \end{center}
      Le \textcolor{A1}{joueur 1} a pris possession de trois carrés alors que le \textcolor{B1}{joueur 2} en a un seul, c'est donc le \textcolor{A1}{joueur 1} qui gagne. \\

   \partie[à vous de jouer !!!]
      En binôme, compléter  ces grilles.
      \begin{center}
         \begin{tabular}{C{2}C{3.1}C{4.2}C{5}}
            \begin{pspicture}(0,0)(2,2)
               \psgrid[subgriddiv=0,gridlabels=0,gridcolor=lightgray](0,0)(2,2)
            \end{pspicture}
            &
            \begin{pspicture}(0,0)(3,3)
               \psgrid[subgriddiv=0,gridlabels=0,gridcolor=lightgray](0,0)(3,3)
            \end{pspicture}
            &
            \begin{pspicture}(0,0)(4,4)
               \psgrid[subgriddiv=0,gridlabels=0,gridcolor=lightgray](0,0)(4,4)
            \end{pspicture}
           &
           \begin{pspicture}(0,0)(5,4.5)
               \psgrid[subgriddiv=0,gridlabels=0,gridcolor=lightgray](0,0)(5,5)
            \end{pspicture} \\ [5mm]
         \end{tabular} \\
         \begin{tabular}{C{6.2}C{9}}
            \begin{pspicture}(0,0)(6,3)
               \psgrid[subgriddiv=0,gridlabels=0,gridcolor=lightgray](0,0)(6,3)
            \end{pspicture}
            &
            \begin{pspicture}(0,0)(9,3.5)
               \psgrid[subgriddiv=0,gridlabels=0,gridcolor=lightgray](0,0)(9,4)
            \end{pspicture} \\
         \end{tabular}
      \end{center}

