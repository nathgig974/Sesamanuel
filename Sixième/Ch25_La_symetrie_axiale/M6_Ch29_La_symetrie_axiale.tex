\themaG
\graphicspath{{../Ch25_La_symetrie_axiale/Images}}

\chapter{La symétrie axiale}
\label{C29}


%%%%%%%%%%%%%%%%%%%%%%%%%%%%%%%%%%%%%%%%%%
\begin{prerequis}[Connaissances et compétences abordées]
   \begin{itemize}
      \item Compléter une figure par symétrie axiale.
      \item Construire le symétrique d’un point, d’un segment, d’une droite par rapport à un axe donné : figure symétrique.
   \end{itemize}
\end{prerequis}

\vfill

\begin{debat}[Débat : c'est WOW]
   Dans la nature, on trouve de nombreux objets ayant une {\bf symétrie axiale} : animaux, paysages, monuments\dots{} regardez et observez autour de vous !
   \begin{center} 
      {\psset{unit=0.8}
      \begin{pspicture}(-3,-0.5)(9,4)
         %\psgrid[subgriddiv=0, gridlabels=0,gridcolor=lightgray](-1,-1)(9,4)
         \psframe(-1,0)(3,2)
         \psline(-1,2)(0,3)(3,2)
         \psframe(1,0)(2,1)
         \psline[linecolor=A1,linewidth=0.5mm](4,-1)(4,4)
         \psframe[linecolor=B2](5,0)(9,2)
         \psline[linecolor=B2](5,2)(8,3)(9,2)
         \psframe[linecolor=B2](6,0)(7,1)
      \end{pspicture}}
   \end{center}
   \bigskip
   \begin{cadre}[B2][F4]
      \begin{center}
         Vidéo : \href{https://www.tfo.org/fr/univers/cest-wow/100378804/la-symetrie}{\bf La symétrie}, chaîne {\it TFO}, série {\it c'est WOW}, vidéo de 1\!:00 à 5\!:00.
      \end{center}
   \end{cadre}
\end{debat}

\vfill

\textcolor{PartieGeometrie}{\sffamily\bfseries Cahier de compétences} : chapitre 29, exercices 11 à 26. 


%%%%%%%%%%%%%%%%%%%%%%%%%%%%%%%%%%%%
%%%%%%%%%%%%%%%%%%%%%%%%%%%%%%%%%%%%
\activites

\begin{activite}[Les napperons]
   {\bf Objectif :} construire une figure complexe par découpage à l'aide de symétries. 
   \begin{QCM}
      \partie[pliage rosace]
         Comme dans le chapitre 21, effectuer le pliage ci-dessous dans une feuille carrée de \ucm{10} de côté.
         \begin{center}
            \begin{pspicture}(-0.5,-0.5)(3.5,3.3)
               \rput(-0.3,1.5){1)}
               \psframe(0,0)(3,3)
               \psline[linestyle=dashed](1.5,0)(1.5,3)
               \psarc{<-}(1.5,1.5){0.75}{0}{180}
               \rput(3.3,1.5){$\Rightarrow$}
            \end{pspicture}
            \begin{pspicture}(0,-0.5)(2,3.3)
               \psframe(0,0)(1.5,3)
            \end{pspicture}
            \begin{pspicture}(-0.5,-0.5)(2,3.3)
               \rput(-0.3,1.5){2)}
               \psframe(0,0)(1.5,3)
               \psline[linestyle=dashed](0,1.5)(1.5,1.5)
               \psarc{->}(0.75,1.5){0.5}{90}{-90}
               \rput(1.8,0.75){$\Rightarrow$}
            \end{pspicture}
            \begin{pspicture}(0,-0.5)(2,3.3)
               \psframe(0,0)(1.5,1.5)
            \end{pspicture}
            \begin{pspicture}(-0.5,-0.5)(2,3.3)
               \rput(-0.3,1.5){3)}
               \psframe(0,0)(1.5,1.5)
               \psline[linestyle=dashed](0,1.5)(1.5,0)
               \psarc{->}(0.75,0.75){0.4}{45}{-135}
               \rput(1.8,0.75){$\Rightarrow$}
            \end{pspicture}
            \begin{pspicture}(0,-0.5)(2,3.3)
               \pspolygon(0,0)(0,1.5)(1.5,0)
            \end{pspicture}
         \end{center}
    
      \partie[un joli napperon]
         Utiliser maintenant le pliage rosace pour réaliser le napperon ci-dessous.
         \begin{center}
            \begin{pspicture}(-5,-5.5)(5,5.5)
            \pspolygon(-5,-5)(-0.5,-5)(0,-4)(0.5,-5)(5,-5)(5,-0.5)(4,0)(5,0.5)(5,5)(0.5,5)(0,4)(-0.5,5)(-5,5)(-5,0.5)(-4,0)(-5,-0.5)
            \psset{PointSymbol=none,PointName=none}
            \pstGeonode{O}(5,0){Z}(0,5){Y}(5,-5){X}
            \pstTriangle(-4,-3){M}(-3,-4){N}(-3,-3){C}
            \pstTriangle(-3,-2){D}(-2,-3){E}(-2,-2){F}
            \psset{CurveType=polygon}
            \pstSymO{O}{M,N,C}
            \pstSymO{O}{D,E,F}
            \pstOrtSym{O}{Z}{M,N,C}
            \pstOrtSym{O}{Z}{D,E,F}
            \pstOrtSym{O}{Y}{M,N,C}
            \pstOrtSym{O}{Y}{D,E,F}
            \pstGeonode(-3.25,0){G}(-2.75,-0.5){H}(-2.25,0){I}(-2.75,0.5){J}
            \pstSymO{O}{G,H,I,J}
            \pstOrtSym{O}{X}{G,H,I,J}
            \pstSymO{O}{G',H',I',J'}
            \pspolygon(-1,-0.5)(-0.5,-0.5)(-0.5,-1)(0,-0.5)(0.5,-1)(0.5,-0.5)(1,-0.5)(0.5,0)(1,0.5)(0.5,0.5)(0.5,1)(0,0.5)(-0.5,1)(-0.5,0.5)(-1,0.5)(-0.5,0)
         \end{pspicture}
      \end{center}
   \end{QCM}
   \vfill\hfill{\it\footnotesize Source : inspiré de l'article \href{https://irem.univ-grenoble-alpes.fr/medias/fichier/68n3_1555658318837-pdf}{Le napperon, un problème pour travailler la symétrie axiale}, Grand N n\degre68, Marie-Lise Peltier, 2001}.
\end{activite}


%%%%%%%%%%%%%%%%%%%%%%%%%%%%%
%%%%%%%%%%%%%%%%%%%%%%%%%%%%%
\cours 

%%%%%%%%%%%%%%%
\section{Symétrie axiale}

\begin{definition}
   Deux figures sont {\bf symétriques} par rapport à un axe de symétrie si elles se superposent par pliage de long de cet axe.
\end{definition}
 
\medskip
  
Pour construire la figure symétrique d'une figure, on construit le symétrique de chaque point qui compose la figure.


%%%%%%%%%%%%%%%%%%%%%%%%%%%%%%%%
\section{Tracer le symétrique d'une figure par rapport à un axe}

\begin{methode*1}
   Pour tracer le symétrique d'un point par rapport à un axe, on peut :
   \begin{itemize}
      \item utiliser l'équerre et la règle ou le compas ;
      \item utiliser uniquement le compas ;
      \item utiliser un quadrillage.
   \end{itemize}
   \exercice
      {\psset{unit=0.63}
      \small
         {\bf Équerre et règle graduée ou compas.} \\
         \begin{tabular}{C{3.5}C{3.5}C{3.5}C{3.5}}   
            \begin{pspicture}(0,-0.5)(4,4)
               \pstGeonode[PosAngle=180](1,3){A}
               \pstGeonode[PointSymbol=none,PointName=none](0,0){B}(4,3){C}
               \pstLineAB[linecolor=B1]{B}{C}
               \rput(3.85,2.4){\textcolor{B1}{$(d)$}}
            \end{pspicture}
            &
            \begin{pspicture}(0,-0.5)(4,4)
               \pstGeonode[PosAngle=180](1,3){A}
               \pstGeonode[PointSymbol=none,PointName=none](0,0){B}(4,3){C}
               \pstLineAB[linecolor=B1]{B}{C}
               \equerre{1.73}{1.32}{37}{1.2}
               \rput(3.85,2.4){\textcolor{B1}{$(d)$}}
            \end{pspicture} 
            &
            \begin{pspicture}(0,-0.5)(4,4)
               \pstGeonode[PosAngle=180](1,3){A}
               \pstGeonode[PointSymbol=none,PointName=none](0,0){B}(4,3){C}
               \psline[linecolor=H1](0.64,3.5)(3.6,-0.5)
               \pstLineAB[linecolor=B1]{B}{C}
               \rput(3.85,2.4){\textcolor{B1}{$(d)$}}     
            \end{pspicture}
            &
            \begin{pspicture}(0,-0.5)(4,4)
               \pstGeonode[PosAngle=180](1,3){A}
               \pstGeonode[PointSymbol=none,PointName=none](0,0){B}(4,3){C}
               \pstLineAB[linecolor=B1]{B}{C}
               \pstGeonode[PointSymbol=none,PointName=none](0,0){B}(4,3){C}
               \pstOrtSym[CodeFig=true,CodeFigColor=H1]{B}{C}{A}[A']
               \rput(3.85,2.4){\textcolor{B1}{$(d)$}}
            \end{pspicture}
            \\
            symétrique de $A$ par rapport à $(d)$ :
            &
            construire la perpendiculaire à $(d)$ passant par $A$,
            &
            prolonger la perpendiculaire de l'autre côté de $(d)$,
            &
            reporter la distance de $A$ à $(d)$ de l'autre côté de la droite.
            \\
         \end{tabular} \\
         {\bf Compas uniquement.} \\
            \begin{tabular}{C{5}C{5}C{5}}   
               \begin{pspicture}(0,-0.5)(4,3.8)
                  \pstGeonode[PosAngle=180](1,3){A}
                  \pstGeonode[PointSymbol=none,PointName=none](0,0){B}(4,3){C}
                  \pstLineAB[linecolor=B1]{B}{C}
                  \rput(3.85,2.4){\textcolor{B1}{$(d)$}}
               \end{pspicture}
               &
               \begin{pspicture}(0,-0.5)(4,3.8)
                  \pstGeonode[PosAngle=180](1,3){A}
                  \pstGeonode[PointSymbol=none,PointName=none](0,0){B}(4,3){C}
                  \pstLineAB[linecolor=B1]{B}{C}
                  \psarc[linecolor=cyan,linestyle=dashed](1,3){2}{260}{350}
                  \pstGeonode[PointName=none](1.34,1.02){M}(2.76,2.07){N}
                  \rput(3.85,2.4){\textcolor{B1}{$(d)$}}
               \end{pspicture} 
               &
               \begin{pspicture}(0,-0.5)(4,3.8)
                  \pstGeonode[PosAngle=180](1,3){A}
                  \pstGeonode[PointSymbol=none,PointName=none](0,0){B}(4,3){C}
                  \pstLineAB[linecolor=B1]{B}{C}
                  \psarc[linecolor=cyan,linestyle=dashed](1.34,1.02){2}{310}{350}
                  \psarc[linecolor=cyan,linestyle=dashed](2.76,2.07){2}{260}{300}
                  \pstGeonode[PointName=none](1.34,1.02){M}(2.76,2.07){N}
                  \pstOrtSym{B}{C}{A}[A']
                  \rput(3.85,2.4){\textcolor{B1}{$(d)$}}     
               \end{pspicture}
               \\
               symétrique de $A$ par rapport à $(d)$ :
               &
               tracer un arc de cercle qui coupe $(d)$ en deux points,
               &
               depuis chaque point, tracer un arc de cercle avec le même écart.
               \\
               \end{tabular} \\ [3mm]
               {\bf Dans un quadrillage :} si l'axe est horizontal ou vertical, il suffit de reporter le nombre de carreaux séparant le point de l'axe de l'autre côté de cet axe. Si le quadrillage est en diagonale : \\
               \psset{unit=0.8}
               \begin{tabular}{C{5}C{5}C{5}}   
                  \begin{pspicture}(0,0)(6,5.5)
                     \psgrid[subgriddiv=0,gridlabels=0pt,gridcolor=lightgray](0,0)(6,5)
                     \psline[linecolor=B1](1,0)(6,5)
                     \psdot(1,4)
                     \rput(0.7,3.7){$A$}
                     \rput(5,4.6){\textcolor{B1}{$(d)$}}
                  \end{pspicture}
                  &
                  \begin{pspicture}(0,0)(6,5.5)
                     \psgrid[griddots=8,subgriddiv=0,gridlabels=0pt,gridcolor=gray](0,0)(6,5)
                     \psline[linecolor=B1](1,0)(6,5)
                     \psdot(1,4)
                     \rput(0.7,3.7){$A$}
                     \rput(5,4.6){\textcolor{B1}{$(d)$}}
                     \psset{linecolor=cyan}
                     \psline(1,4)(3,2)
                     \psarc{<-}(1.5,3.5){0.65}{-45}{135}
                     \psarc{<-}(2.5,2.5){0.65}{-45}{135}
                     \rput(1.65,3.75){\textcolor{cyan}{1}}
                     \rput(2.65,2.75){\textcolor{cyan}{2}}
                  \end{pspicture}
                  &
                  \begin{pspicture}(0,0)(6,5.5)
                     \psgrid[griddots=8,subgriddiv=0,gridlabels=0pt,gridcolor=gray](0,0)(6,5)
                     \psline[linecolor=B1](1,0)(6,5)
                     \psdots(1,4)(5,0)
                     \rput(0.7,3.7){$A$}
                      \rput(5,4.6){\textcolor{B1}{$(d)$}}
                     \psset{linecolor=cyan}
                     \psline(1,4)(3,2)
                     \psarc{<-}(1.5,3.5){0.65}{-45}{135}
                     \psarc{<-}(2.5,2.5){0.65}{-45}{135}
                     \rput(1.65,3.75){\textcolor{cyan}{1}}
                     \rput(2.65,2.75){\textcolor{cyan}{2}}
                     \psset{linecolor=H1}
                     \psline(5,0)(3,2)
                     \psarc{<-}(3.5,1.5){0.65}{-45}{135}
                     \psarc{<-}(4.5,0.5){0.65}{-45}{135}
                     \rput(3.65,1.75){\textcolor{H1}{1}}
                     \rput(4.65,0.75){\textcolor{H1}{2}}
                      \rput(5.6,0){$A'$}
                  \end{pspicture} \\
                  symétrique du point $A$ par rapport à $(d)$ :
                  &
                  compter le nombre de diagonales entre le point et la droite,
                  &
                  reporter ce nombre de l'autre côté en diagonal. \\
               \end{tabular}}
               \vspace*{-5mm}
\end{methode*1}


%%%%%%%%%%%%%%%%%%%%%%%%%%%%%%%%
%%%%%%%%%%%%%%%%%%%%%%%%%%%%%%%%
\exercicesbase

\begin{colonne*exercice}

\serie{Figures symétriques} %%%%%%%%%%%

\begin{exercice} %1
   Pour chacune des figures suivantes, dire s'il s'agit ou pas d'une symétrie axiale. Si oui, tracer l'axe de symétrie.
   \begin{center}
      \begin{pspicture}(8,7.5)
         \def\cocottea{\pspolygon(0,0)(0.5,0)(0.75,0.25)(1,0)(1,0.5)(0.75,0.75)(1,1)(0.5,1)(0.5,0.5)}
         \def\cocotteb{\pspolygon[fillstyle=solid,fillcolor=lightgray](0,0)(0.5,0)(0.75,0.25)(1,0)(1,0.5)(0.75,0.75)(1,1)(0.5,1)(0.5,0.5)}
         \def\cocottec{\pspolygon[fillstyle=solid,fillcolor=lightgray](0,0)(-0.5,0)(-0.75,0.25)(-1,0)(-1,0.5)(-0.75,.75)(-1,1)(-0.5,1)(-0.5,0.5)}
         \psframe(0,0)(8,7.5)
         \psline(4,0)(4,7.5)
         \psline(8,0)(8,7.5)
         \psline(0,2.5)(8,2.5)
         \psline(0,5.5)(8,5.5)
         \rput(0.5,6){\cocottea} \rput(3.5,6){\cocottec}
         \rput{90}(6,6){\cocotteb} \rput(6,6){\cocottea} 	
         \rput(1.5,4){\cocottea} \rput(1.5,2.75){\cocotteb}
         \rput(5.75,4.25){\cocotteb} \rput{180}(6,4){\cocottea}                         
         \rput(0.5,1){\cocottea} \rput(3.25,0.5){\cocottec}     
         \rput{-30}(4.25,0.75){\cocottea} \rput{80}(7.5,2.25){\cocottec}	               
      \end{pspicture}
   \end{center}
\end{exercice}

\begin{exercice} %2
   Sur la figure ci-dessous, $DEFG$ est un rectangle de centre $L$. Les points $R, C, N$ et $S$ sont les milieux respectifs des côtés $[DE], [EF], [FG]$ et $[GD]$.
   \begin{center}
      {\psset{xunit=0.5}
      \small
         \begin{pspicture}(-1,0)(8.5,4.5)
            \psframe(0,0)(8,4)
            \pspolygon(4,0)(8,2)(4,4)(0,2)
            \psline(0,4)(8,0)
            \psline(0,0)(8,4)
            \psline(0,2)(8,2)
            \psline(4,0)(4,4)
   \pstGeonode[PosAngle={-135,-90,-45,-90,-90,180,70,0,90,90,135,90,45},PointSymbol=none](0,0){E}(4,0){C}(8,0){F}(2,1){Y}(6,1){P}(0,2){R}(4,2){L}(8,2){N}(2,3){H}(6,3){K}(0,4){D}(4,4){S}(8,4){G}
         \end{pspicture}}
   \end{center}
   \begin{enumerate}
      \item Colorier en rouge le symétrique du triangle $DHS$ par rapport à la droite $(RN)$.
      \item Colorier en orange le symétrique du triangle $DHS$ par rapport à la droite $(SC)$.
      \item Colorier en bleu le symétrique du triangle $DHS$ par rapport à la droite $(LK)$.
      \item Colorier en vert le symétrique du triangle $DHS$ par rapport à la droite $(DF)$.
      \item Colorier en noir le symétrique du triangle $DHS$ par rapport à la droite $(RS)$.
   \end{enumerate}
\end{exercice}

\begin{exercice} %3
   Tracer la figure symétrique de toutes les figures par rapport à la droite.
   {\psset{unit=0.45}
   \begin{center}
      \begin{pspicture}(18,18)
         \psgrid[gridlabels=0,gridcolor=lightgray,subgriddiv=0](18,18)
         \psline(0,0)(18,18)
          \psset{linewidth=1.5pt}
         \pspolygon(1,8)(2,9)(3,8)(5,8)(6,9)(5,9)(6,10)(5,11)(3,11)(2,10)(1,11)
         \pspolygon(1,13)(3,17)(7,17)(9,13)
         \pscircle(5,15){1}
         \psdot(5,10)
         \psline(7,13)(7,17)
         \psline(3,13)(3,17)
         \psarc(2,2){1.41}{45}{225}
         \psarc(3,3){1.41}{45}{225}
         \pscircle(3,6){1}
         \pscircle(3,5){1.41}
         \pspolygon(10,14)(13,14)(14,15)(10,15)
         \psline(12,15)(12,18)(11,16)(12,16)
      \end{pspicture}
   \end{center}}
\end{exercice}

\begin{exercice} %4
   Reproduire chaque lettre et la droite sur le cahier puis construire le symétrique par rapport à la droite.
   \begin{center}
      {\psset{unit=0.45}
      \begin{pspicture}(3,3)
        \psgrid[subgriddiv=0,gridlabels=0,gridcolor=lightgray](-1,0)(2,3)
        \psset{linewidth=0.6mm}
        \psline(0,1)(0,3)(1,3)(1,1)
        \psline(0,2)(1,2)
        \psline[linecolor=B1](-1,0)(2,0)  
     \end{pspicture}
     \begin{pspicture}(-1,0)(2,3)
        \psgrid[subgriddiv=0,gridlabels=0,gridcolor=lightgray](-1,0)(1,3)
        \psset{linewidth=0.6mm}
        \psline(1,1)(0,1)(0,3)(1,3)
        \psline(0,2)(1,2)
        \psline[linecolor=B1](-1,0)(-1,3)
     \end{pspicture}
     \begin{pspicture}(4,3)
        \psgrid[subgriddiv=0,gridlabels=0,gridcolor=lightgray](0,0)(3,3)
        \psset{linewidth=0.6mm}
        \psline(0,1)(0,3)
        \psline(0,2)(1,2)
        \psline(1,1)(1,3)
        \psline[linecolor=B1](1,0)(3,2)
     \end{pspicture}
     \begin{pspicture}(4,3)
        \psgrid[subgriddiv=0,gridlabels=0,gridcolor=lightgray](0,0)(3,3)
        \psset{linewidth=0.6mm}
        \psline(1,2)(1,0)(2,0)
        \psline[linecolor=B1](0,3)(3,0)
     \end{pspicture}
     \begin{pspicture}(2,3)
        \psgrid[subgriddiv=0,gridlabels=0,gridcolor=lightgray](0,0)(2,3)
        \psset{linewidth=0.6mm}
        \pscircle(1,1){1}
        \psline[linecolor=B1](0,2)(2,2)
     \end{pspicture}}
   \end{center}
\end{exercice}  

\begin{exercice} %5
   Trouver les erreurs dans les tracés de symétriques suivants (la figure d'origine est dans la partie grise).
   \begin{center}
   {\psset{unit=0.3}
      \begin{pspicture}(12,17.5)
         \psframe[fillstyle=solid,fillcolor=lightgray!50,linewidth=0](12,8)
         \psgrid[subgriddiv=0,gridcolor=lightgray,gridlabels=0](12,18)
         \psline(0,8)(12,8)
         \psset{linewidth=0.3mm}
         \pspolygon(4,0)(10,5)(5,3)(2,4)
         \pspolygon(2,12)(10,11)(5,13)(4,16)
         \psdots(2,12)(10,11)(5,13)(4,16)
      \end{pspicture}
      \begin{pspicture}(12,17)
         \psframe[fillstyle=solid,fillcolor=lightgray!50,linewidth=0](12,8)
         \psgrid[subgriddiv=0,gridcolor=lightgray,gridlabels=0](12,18)
         \psline(0,8)(12,8)
         \psset{linewidth=0.3mm}
         \pspolygon(4,0)(10,5)(5,3)(2,4)
         \pspolygon(2,12)(5,13)(10,11)(4,18)
      \end{pspicture}
      
      \begin{pspicture}(0,0)(18,9)
         \pspolygon[fillstyle=solid,fillcolor=lightgray!50,linewidth=0](14,0)(18,0)(18,8)(6,8)
         \psgrid[subgriddiv=0,gridcolor=lightgray,gridlabels=0](0,0)(18,8)
         \psline(14,0)(6,8)
         \psset{linewidth=0.4mm}
         \psline(13,5)(10,5)(10,7)
         \psline(11,5)(11,6)
         \psline(6,1)(9,4)(7,5)
         \psline(8,3)(7,4)
      \end{pspicture}}
   \end{center}
\end{exercice}

\end{colonne*exercice}


%%%%%%%%%%%%%%%%%%%%%
%%%%%%%%%%%%%%%%%%%%%
\Recreation

\enigme[Multi-symétries]
   Dans les deux quadrillages suivants, construire le symétrique de la figure par rapport à tous les axes tracés.
   \begin{center}
      {\psset{unit=0.5}
      \begin{pspicture}(-11,-8)(11,8)
         \psgrid[subgriddiv=0,gridlabels=0,gridcolor=lightgray](-11,-8)(11,8)
         \psline(-11,0)(11,0)
         \psline(0,-8)(0,8)
         \psset{linewidth=1mm}
         \psarc(3,0){3}{0}{180}
         \psarc(7,0){1}{0}{180}
         \psarc(0,3){3}{-90}{90}
         \psline(8,0)(8,2)(10,4)(10,0)
      \end{pspicture} \\ [10mm]
      \begin{pspicture}(-11,-11)(11,11)
         \psgrid[subgriddiv=0,gridlabels=0,gridcolor=lightgray](-11,-11)(11,11)
         \psline(-11,0)(11,0)
         \psline(0,-11)(0,11)
         \psline(-11,-11)(11,11)
         \psline(-11,11)(11,-11)
         \psset{linewidth=1mm}
         \psline(0,0)(2,0)(2,1)(3,1)(3,2)(4,2)(4,3)(5,3)(5,4)(6,4)(6,5)(7,5)(7,6)(8,6)(10,0)(7,0)(5,2)(3,0)
      \end{pspicture} } 
\end{center}

