\themaM
\graphicspath{{../S03_Aires_et_perimetres/Images/}}

% Piece de curvica
\def\curvica#1{
   \begin{pspicture}(-1,-0.25)(3,2.5)
   #1
   \psset{linewidth=0.4mm,linestyle=dotted} %Curvica vierge
      \psframe(0,0)(2,2)
      \psarc(1,4){2.24}{-116.6}{-63.4}
      \psarc(1,0){2.24}{63.4}{116.6}
      \psarc(1,2){2.24}{-116.6}{-63.4}
      \psarc(1,-2){2.24}{63.4}{116.6}
      \psarc(4,1){2.24}{153.4}{-153.4}
      \psarc(0,1){2.24}{-26.6}{26.6}
      \psarc(2,1){2.24}{153.4}{-153.4}
      \psarc(-2,1){2.24}{-26.6}{26.6}
      \end{pspicture}
      }

\chapter{Périmètres et aires}
\label{S03}


%%%%%%%%%%%%%%%%%%%%%%%%%%%%%%%%%%%%%
%%%%%%%%%%%%%%%%%%%%%%%%%%%%%%%%%%%%%
\begin{prerequis}
   \begin{itemize}
      \item Notion de grandeur produit.
      \item[\com] Vérifier la cohérence des résultats du point de vue des unités.
      \item[\com] Effectuer des conversions d’unités.
   \end{itemize}
\end{prerequis}

\vfill

\begin{debat}[Débat : le SI (Système International)]
   En 1795, il existe en France plus de 700 {\bf unités de mesures différentes} qui varient d'une ville à l'autre. Source d'erreurs et de fraudes lors des transactions commerciales, politiques et scientifiques vont tenter de réformer cet état de fait : leur idée est d'assurer l'invariabilité des mesures en les rapportant à un étalon universel emprunté à un phénomène naturel. Le 26 mars 1791 nait le mètre (du grec {\it metron}, mesure), dont la longueur est établie comme égale à la dix-millionième partie du quart du méridien terrestre. L'unité de mesure de base étant déterminée, il suffit désormais d'établir toutes les autres unités de mesure qui en découlent : le mètre carré et le mètre cube, le litre, le gramme\dots{} Le système international des unités (SI) est nait en 1960. En 2018, les unités de base sont redéfinies à partir de sept constantes physiques. \\
   \begin{center}
      {\psset{unit=0.9}
      \begin{pspicture}(-2.5,-2.5)(2.5,2.25)
         \pscircle*[linecolor=gray](0,0){2.5}
         \pswedge*[linecolor=orange](0,0){2}{0}{51}
         \rput(1.25;25){\large\white m}
         \pswedge*[linecolor=red](0,0){2}{51}{103}
         \rput(1.25;75){\large\white kg}
         \pswedge*[linecolor=magenta](0,0){2}{103}{154}
         \rput(1.25;126){\large\white cd}
         \pswedge*[linecolor=violet](0,0){2}{154}{205}
         \rput(1.25;176){\large\white mol}
         \pswedge*[linecolor=blue](0,0){2}{205}{257}
         \rput(1.25;228){\large\white K}
         \pswedge*[linecolor=cyan](0,0){2}{257}{308}
         \rput(1.25;280){\large\white A}
         \pswedge*[linecolor=green](0,0){2}{308}{360}
         \rput(1.25;334){\large\white s}
      \end{pspicture}}
   \end{center}
   \bigskip
   \begin{cadre}[B2][F4]
      \begin{center}
         Vidéo : \href{https://www.youtube.com/watch?time_continue=2&v=bInHclEN6zQ&feature=emb_logo}{\bf Système International d'unités. L'épopée}, {\it Laboratoire national de métrologie et d'essais}. 
      \end{center}
   \end{cadre}
\end{debat}

\vfill

\textcolor{PartieGeometrie}{\sffamily\bfseries Cahier de compétences} : chapitre 11, exercices 1 à 3 et 14.


%%%%%%%%%%%%%%%%%%%%%%%%%%%%%%%%%%%%
%%%%%%%%%%%%%%%%%%%%%%%%%%%%%%%%%%%%
\activites

\begin{activite}[Comparer sans mesurer]
   {\bf Objectifs :} différencier aire et périmètre ; comparer des périmètres et des aires sans utiliser la mesure.
   \begin{QCM}
      On considère les quatre figures A, B, C et D en bas de page.
      \begin{enumerate}
         \item À l'\oe il nu, classer ces figures dans l'ordre croissant de leur périmètre. \\
            \pf \medskip
         \item Trouver un moyen de vérifier ce classement sans utiliser la règle graduée. \\\pf \medskip
         \item À l'\oe il nu, classer ces figures dans l'ordre croissant de leur aire. \\
            \pf \medskip
         \item Trouver un moyen de vérifier ce classement sans utiliser de formules. \\
         \pf \medskip
         \item Les classements sont-ils les mêmes ? \pf \bigskip
      \end{enumerate}
   \end{QCM}
   \begin{center}
      \begin{pspicture}(0,0)(16,12)
         \psframe(0,0)(16,1)
         \rput(8,0.5){D}
         \psframe(11,4)(16,9)
         \rput(13.5,6.5){C}
         \pspolygon(4,3)(7,3)(7,4)(8,4)(8,7)(7,7)(7,8)(4,8)(4,7)(3,7)(3,4)(4,4)
         \rput(5.5,5.5){B}
         \pspolygon(0,5)(0,11)(16,11)(2,10)
         \rput(1,10){A}
      \end{pspicture}
   \end{center}
\end{activite}


%%%%%%%%%%%%%%%%%%%%%%%%%%%%%%%%%%%%
%%%%%%%%%%%%%%%%%%%%%%%%%%%%%%%%%%%%
\cours 

\section{Longueur et périmètre} %%%%%%%

On peut mesurer une longueur grâce au mètre (\um{}) qui est l'une des sept unités de grandeurs de base du système international, que l'on complète par les unités qui en découlent (multiples et sous-multiples).
   \begin{center}
   \begin{CLtableau}{0.65\linewidth}{8}{p{2.7cm}}
      \hline
      Préfixe & kilo & hecto & déca & & déci & centi & milli \\
      \hline
      Unité de longueur & km & hm & dam & m & dm & cm & mm \\
      \hline
      Exemple & & $9$ & $7$ & $3$ & $2$ & $1$ & \\
      \hline
   \end{CLtableau}
   \end{center}



\begin{exemple*1}
   $\um{973,21} =\udm{9732,1} =\ucm{97321} =\umm{973210} =\udam{97,321} =\uhm{9,7321}\dots$
\end{exemple*1}

\medskip

\begin{propriete}
   Pour calculer le périmètre d'un polygone, on additionne la mesure de chacun des segments qui le compose.
\end{propriete}

\begin{exemple}
   {\psset{unit=0.4}
   \begin{pspicture}(-0.5,0)(13,5.5)
      \psgrid[subgriddiv=0,gridlabels=0pt,gridwidth=0.05,gridcolor=gray](13,5)
      \put(1,1){\pspolygon[fillstyle=solid,fillcolor=B2,linewidth=0.1](0,0)(2,0)(2,3)(0,3)(0,2)(1,2)(1,1)(0,1)(0,0)}
      \put(4,1){\pspolygon[fillstyle=solid,fillcolor=A2,linewidth=0.1](0,0)(2,0)(2,2)(1,2)(1,3)(0,3)(0,0)}
      \put(7,1){\pspolygon[fillstyle=solid,fillcolor=J2,linewidth=0.1](1,0)(2,0)(2,1)(3,1)(3,2)(2,2)(2,3)(1,3)(1,2)(0,2)(0,1)(1,1)(1,0)}
      \psline[linewidth=0.1]{|-|}(11,1)(12,1)
      \rput(11.5,1.7){$u.\ell.$}
      \rput(2.5,2.4){\textbf{A}}
      \rput(5,2.4){\textbf{B}}
      \rput(8.5,2.4){\textbf{C}}
   \end{pspicture}}
\correction
   L'unité de longueur est le côté d'un carreau ($u.\ell.$) :
   \begin{itemize}
      \item le périmètre de la figure A vaut $12\,u.\ell.$   
      \item le périmètre de la figure B vaut $10\,u.\ell.$
      \item le périmètre de la figure C vaut $12\,u.\ell.$
   \end{itemize} 
\end{exemple}


\section{Surface et aire} %%%%%%%

\begin{definition}
   La \textbf{surface} d'une figure est la partie située à l'intérieur de son contour. \\
   Sa mesure s'appelle l'\textbf{aire}, qui est le nombre d'unités d'aire que la figure contient.
\end{definition}

\begin{exemple}
{\psset{unit=0.5}
   \begin{pspicture}(0,-0.2)(10,7.5)
      \psgrid[subgriddiv=0,gridlabels=0pt,gridwidth=0.03,gridcolor=gray](11,7)
         \multido{\i=0+1}{5}{%
	\FPeval{a}{\i+7}
	\psline[linecolor=gray,linewidth=0.02](\i,0)(\a,7)
	\psline[linecolor=gray,linewidth=0.02](\i,7)(\a,0)
	}
      \multido{\i=1+1}{6}{%
	\FPeval{b}{7-\i}
	\FPeval{c}{4+\i}
	\psline[linecolor=gray,linewidth=0.02](0,\i)(\b,7)
	\psline[linecolor=gray,linewidth=0.02](\i,0)(0,\i)
	\psline[linecolor=gray,linewidth=0.02](\c,0)(11,\b)
	\psline[linecolor=gray,linewidth=0.02](\c,7)(11,\i)
	}
      \put(1,1){\pspolygon[fillstyle=solid,fillcolor=B2,linewidth=0.1](0,0)(2,0)(2,3)(0,3)(0,2)(1,2)(1,1)(0,1)(0,0)}
      \put(4,1){\pspolygon[fillstyle=solid,fillcolor=A2,linewidth=0.1](1,0)(2,0)(2,2)(1,2)(1,3)(0,3)(0,1)}
      \put(7,1){\pspolygon[fillstyle=solid,fillcolor=J2,linewidth=0.1](1,0)(1.5,0.5)(2,0)(2,1)(3,1)(2.5,1.5)(3,2)(2,2)(2,3)(1.5,2.5)(1,3)(1,2)(0,2)(0.5,1.5)(0,1)(1,1)(1,0)}
      \rput(2.5,2.4){\textbf{A}}
      \rput(5,2.4){\textbf{B}}
      \rput(8.5,2.4){\textbf{C}}
      \rput(1.5,5.5){{$u_1$}} 
      \psframe[fillstyle=solid,fillcolor=gray,linewidth=0.1](2,5)(3,6)
      \rput(4.5,5.5){{$u_2$}}
      \pspolygon[fillstyle=solid,fillcolor=gray,linewidth=0.1](6,5)(6,6)(5.5,5.5)
      \rput(7.5,5.5){{$u_3$}}\pspolygon[fillstyle=solid,fillcolor=gray,linewidth=0.1](8,5)(9,5)(8,6)
   \end{pspicture}}
   \correction   
   Lorsqu'on n'a pas une unité d'aire entière, on peut \og découper \fg{} une partie de la figure afin de la déplacer ailleurs pour former une unité d'aire. \\ [2mm]
   \begin{cltableau}{0.6\linewidth}{4}
      \hline
      Unité & fig. A & fig. B & fig. C \\
      \hline
      $u_1$ & 5 & 4,5 & 4 \\
      \hline
      $u_2$ & 20 & 18 & 16 \\
      \hline
      $u_3$ & 10 & 9 & 8 \\
      \hline
   \end{cltableau}
\end{exemple}

\bigskip

L'aire est une grandeur composée, correspondant au produit de deux longueurs. Chaque unité d'aire dans le tableau comporte donc deux colonnes. \\
Pour désigner une aire, on utilise le mètre carré (\umq{}) et ses multiples et sous-multiples. Pour les mesures agraires, on utilise l'are (a) qui équivaut à \umq{100} et l'hectare (ha) qui vaut 100 ares, c'est-à-dire \umq{10000}.

\begin{center}
   \begin{ltableau}{0.8\linewidth}{14}
      \hline
      \multicolumn{2}{|c|}{\ukmq{}} & \multicolumn{2}{c|}{\uhmq{}} & \multicolumn{2}{c|}{\udamq{}} & \multicolumn{2}{c||}{\umq{}} & \multicolumn{2}{c|}{\udmq{}} & \multicolumn{2}{c|}{\ucmq{}} & \multicolumn{2}{c|}{\ummq{}} \\
      \hline
      & & & & & $3$ & $7$ & \multicolumn{1}{C{0.5}||}{$0$} & $1$ & $5$ & $0$ & $4$ & & \\
      \hline
   \end{ltableau}
\end{center}

\begin{exemple*1}
   \umq{370,1504} = \udmq{37015,04} = \ummq{370150400} = \udamq{3,701504}\dots
\end{exemple*1}


%%%%%%%%%%%%%%%%%%%%%%%%%%%%%%%%%
%%%%%%%%%%%%%%%%%%%%%%%%%%%%%%%%%
\exercicesbase

\begin{colonne*exercice}

\serie{Différencier aire et des périmètre} %%%

\begin{exercice} %1
   On considère les surfaces suivantes.
   \begin{center}
      {\psset{unit=0.5}
      \begin{pspicture}(0,-3.5)(16,13)
         \psgrid[subgriddiv=0,gridlabels=0,gridcolor=lightgray](0,-4)(16,13)
         \rput[l](1,-0.8){unités de longueur :}
         \rput(8.5,-.8){$u$}
         \psline(9,-1)(9,0)
         \rput(10.5,-0.8){$d$}
         \psline(11,-1)(12,0)
         \rput(13.5,-0.8){$c$}
         \psarc(14,0){1}{270}{0}
         \rput[l](1,-2.8){unité d'aire :}
         \psset{fillstyle=solid,fillcolor=lightgray}
         \rput(6.5,-2.8){$a$}
         \psframe(7,-3)(8,-2)
         \psframe(1,11)(4,12)
         \rput(2.5,11.5){\small A}
         \pscustom{\psarc(6,11){1}{0}{180} \psarcn(5,10){1}{90}{0} \psarcn(7,10){1}{180}{90}}
         \rput(6,11.25){\small B}
         \pspolygon(8,11)(12,11)(11,12)(9,12)
         \rput(10,11.5){\small C}
         \pscustom{\psarc(14,11){1}{0}{180} \psline(13,11)(13,10) \psarc(13,11){1}{270}{0} \psarc(15,11){1}{180}{270} \psline(15,10)(15,11)}
         \rput(14,11.5){\small D}
         \pscustom{\psline(1,9)(3,9) \psarc(2,9){1}{180}{0}}
         \rput(2,8.5){\small E}
         \pscustom{\psarc(5,8){1}{180}{0} \psline(6,8)(7,8) \psarc(6,8){1}{0}{180} \psline(5,8)(4,8)}
         \rput(5.5,8){\small F}
         \pspolygon(8,8)(11,8)(10,9)(9,9)
         \rput(9.5,8.5){\small G}
         \pscustom{\psarc(13,8){1}{90}{180} \psarcn(12,7){1}{90}{0} \psline(13,7)(13,8)(14,8)(14,7) \psarcn(15,7){1}{180}{90} \psarc(14,8){1}{0}{90} \psline(14,9)(13,9)}
         \rput(13.5,8.5){\small H}
         \pscustom{\psarcn(1,4){1}{90}{0} \psarcn(3,4){1}{180}{90} \psarcn(3,6){1}{270}{180} \psarcn(1,6){1}{0}{270}}
         \rput(2,5){\small I}
         \psframe(4,4)(6,6)
         \rput(5,5){\small J}
         \pspolygon(7,5)(7,6)(8,5)(9,6)(9,5)(8,4)
         \rput(8,4.75){\small K}
         \pscircle(11,5){1}
         \rput(11,5){\small L}
         \pspolygon(14,4)(15,5)(14,6)(13,5)
         \rput(14,5){\small M}
         \pscustom{\psline(1,2)(1,3)(3,3) \psarcn(3,2){1}{90}{0} \psline(4,2)(2,2)(2,1) \psarc(1,1){1}{0}{90}}
         \rput(2.5,2.5){\small N}
         \psframe(5,2)(7,3)
         \rput(6,2.5){\small O}
         \pscustom{\psarc(9,2){1}{90}{180} \psarcn(8,1){1}{90}{0} \psarc(9,2){1}{270}{0} \psarcn(10,3){1}{270}{180}}
         \rput(9,2){\small P}
         \pscustom{\psline(12,3)(15,3) \psarcn(14,3){1}{0}{270} \psline(14,2)(11,2) \psarcn(12,2){1}{180}{90}}
         \rput(13,2.5){\small Q}
      \end{pspicture}}
   \end{center}
   \begin{enumerate}
      \item Compléter les tableaux ci-dessous. \\
         Dans la deuxième ligne, indiquer le nombre de côtés d'un carré correspondant à une longueur $u$ ; dans la troisième le nombre de diagonales d'un carré $d$ ; dans la quatrième, le nombre de quarts de cercles $c$ et dans la dernière le nombre d'unités d'aires approximatives en arrondissant à l'unité. \\ [2mm]
         \begin{Ltableau}{0.9\linewidth}{9}{c}
            \hline
            & A & B & C & D & E & F & G & H \\
            \hline
            $u$ & & & & & & & & \\
            \hline
            $d$ & & & & & & & & \\
            \hline
            $c$ & & & & & & & & \\
            \hline
            $a$ & & & & & & & & \\
            \hline
         \end{Ltableau}
         \begin{Ltableau}{\linewidth}{10}{c}
            \hline
            & I & J & K & L & M & N & O & P & Q \\
            \hline
            $u$ & & & & & & & & & \\
            \hline
            $d$ & & & & & & & & & \\
            \hline
            $c$ & & & & & & & & & \\
            \hline
            $a$ & & & & & & & & & \\
            \hline
         \end{Ltableau}
      \item Quelles sont les surfaces ayant une même aire ? Ont-elles le même périmètre ?
      \item Quelles sont les surfaces ayant un même périmètre ? Ont-elle la même aire ?
      \item Classer les surfaces dans l'ordre croissant de leur aire.
   \end{enumerate}
\end{exercice}

\begin{corrige}
   \begin{enumerate}
   \ \\ [-5mm]
      \item On trouve les valeurs suivantes : \\ [1mm]
         \begin{Ltableau}{0.9\linewidth}{9}{c}
            \hline
            & A & B & C & D & E & F & G & H \\
            \hline
            $u$ & \textcolor{blue}{8} & \textcolor{blue}{0} & \textcolor{blue}{6} & \textcolor{blue}{2} & \textcolor{blue}{2} & \textcolor{blue}{2} & \textcolor{blue}{4} & \textcolor{blue}{4} \\
            \hline
            $d$ & \textcolor{blue}{0} & \textcolor{blue}{0} & \textcolor{blue}{2} & \textcolor{blue}{0} & \textcolor{blue}{0} & \textcolor{blue}{0} & \textcolor{blue}{2} & \textcolor{blue}{0} \\
            \hline
            $c$ & \textcolor{blue}{0} & \textcolor{blue}{4} & \textcolor{blue}{0} & \textcolor{blue}{4} & \textcolor{blue}{2} & \textcolor{blue}{4} &\textcolor{blue}{0} & \textcolor{blue}{4} \\
            \hline
            $a$ & \textcolor{blue}{3} & \textcolor{blue}{2} & \textcolor{blue}{3} & \textcolor{blue}{\!$\approx$3} & \textcolor{blue}{\!$\approx$2} & \textcolor{blue}{\!$\approx$3} & \textcolor{blue}{2} & \textcolor{blue}{3} \\
            \hline
         \end{Ltableau}
         \begin{Ltableau}{\linewidth}{10}{c}
            \hline
            & I & J & K & L & M & N & O & P & Q \\
            \hline
            $u$ & \textcolor{blue}{0} & \textcolor{blue}{8} & \textcolor{blue}{2} & \textcolor{blue}{0} & \textcolor{blue}{0} & \textcolor{blue}{6} & \textcolor{blue}{6} & \textcolor{blue}{0} & \textcolor{blue}{6} \\
            \hline
            $d$ & \textcolor{blue}{0} & \textcolor{blue}{0} & \textcolor{blue}{4} & \textcolor{blue}{0} & \textcolor{blue}{4} & \textcolor{blue}{0} & \textcolor{blue}{0} & \textcolor{blue}{0} & \textcolor{blue}{0} \\
            \hline
            $c$ & \textcolor{blue}{4} & \textcolor{blue}{0} & \textcolor{blue}{0} & \textcolor{blue}{4} & \textcolor{blue}{0} & \textcolor{blue}{2} & \textcolor{blue}{0} & \textcolor{blue}{4} & \textcolor{blue}{2} \\
            \hline
            $a$ & \textcolor{blue}{\!$\approx$1} & \textcolor{blue}{4} & \textcolor{blue}{2} & \textcolor{blue}{\!$\approx$3} & \textcolor{blue}{2} & \textcolor{blue}{3} & \textcolor{blue}{2} & \textcolor{blue}{2} & \textcolor{blue}{\!$\approx$4} \\
            \hline
         \end{Ltableau}
      \item {\blue B, G, K, M, O et P} ont la même aire ; \\
        {\blue  A, C, H et N} ont la même aire : $3a$ ; \\
        {\blue D, F, L} ont la même aire qu'un disque de rayon $1u$. \\
        Seuls {\blue D et F} ont le même périmètre.
      \item {\blue A et J} ont le même périmètre : $8u$ ; \\
         {\blue B, I et L} ont le même périmètre : $8c$ ; \\
         {\blue D et F} ont le même périmètre : $2u+4c$. \\
         Seuls {\blue D et F} ont la même aire.
      \item {\blue $I<E<B,G,K,M,O,P<A,C,H,N<D,F,L<Q<J$}.
   \end{enumerate}
\end{corrige}

\medskip

\begin{exercice} %1
   Sachant que l'unité d'aire est le carreau ($u.a.$), déterminer l'aire de chaque surface suivante.
   \begin{center}
      {\psset{unit=0.5}
      \begin{pspicture}(0,-5.25)(15,10)
         \psgrid[subgriddiv=0,gridlabels=0,gridcolor=lightgray](0,-5)(15,10)
         \psset{linewidth=0.3mm}
         \psframe(1,1)(4,3)
         \rput(2.5,2.5){\ding{202}}
         \pspolygon(5,0)(11,0)(10,4)(8,4)(5,2)
         \rput(8.5,1.5){\ding{205}}
         \pspolygon(1,4)(1,9)(5,9)(5,8)(2,8)(2,7)(4,7)(4,6)(2,6)(2,4)
         \rput(1.5,6.5){\ding{203}}
         \pspolygon(6,7)(12,9)(6,9)
         \rput(7.5,8.4){\ding{204}}
         \pspolygon(5,4)(5,6)(12,6)(12,8)(14,8)(14,4)(13,3)(14,2)(14,1)(12,1)(11,3)(12,5)(6,5)
         \rput(12.5,5.5){\ding{206}}
         \psline(3,-2)(1,-2)(1,-4)(2,-4)
         \psarc(3,-4){1}{0}{180}
         \psline(4,-4)(6,-4)(6,-2)(5,-2)
         \psarc(4,-2){1}{0}{180}
         \rput(3.5,-2.5){\ding{207}} 
         \psline(7,-1)(14,-1)
         \psline(7,-4)(14,-4)
         \psline(8,-2)(8,-3)
         \psline(13,-2)(13,-3)
         \psarc(8,-4){1}{90}{180}
         \psarc(8,-1){1}{180}{-90}
         \psarc(13,-4){1}{0}{90}
         \psarc(13,-1){1}{-90}{0}
         \pscircle(10,-2.5){1}
         \rput(11.5,-2.5){\ding{208}}
         \psframe[fillstyle=solid,fillcolor=gray](14,9)(15,10)
         \rput(13.25,9.5){$u.a.$}
      \end{pspicture}}
   \end{center}
\end{exercice}

\begin{corrige}
   On peut par exemple procéder par découpage et recollement pour obtenir des unités entières. \\
   \begin{enumerate}
      \item La surface 1 mesure {\blue 6 $u.a.$}
      \item La surface 2 mesure  {\blue 10 $u.a.$}
      \item La surface 3 mesure {\blue 6 $u.a.$}
      \item La surface 4 mesure {\blue 19 $u.a.$}
      \item La surface 5 mesure {\blue 22,5 $u.a.$}
      \item La surface 6 mesure {\blue 10 $u.a.$}
      \item La surface 7 mesure {\blue 15 $u.a.$}
   \end{enumerate}
\end{corrige}

\medskip

\serie{Conversion de mesures} %%%

\medskip

\begin{exercice}
   Effectuer les conversions de longueurs suivantes :
   \begin{enumerate}
      \item $\um{1275} =\pf \ucm{}$
      \item $\udm{32,5} =\pf\udam{}$
      \item $\ucm{345697,34} =\pf\ukm{}$
      \item $\um{0,003} =\pf\umm{}$
      \item $\ukm{2,5} =\pf \um{}$
   \end{enumerate}
\end{exercice}

\begin{corrige}
   \ \\ [-5mm]
   \begin{enumerate}
      \item $\um{1275} =\blue\ucm{127500}$
      \item $\udm{32,5} =\blue\udam{0,325}$
      \item $\ucm{345697,34} =\blue\ukm{3,4569734}$
      \item $\um{0,003} =\blue\umm{3}$
      \item $\ukm{2,5} =\blue\um{2500}$
   \end{enumerate}
\end{corrige}

\medskip

\begin{exercice}
   Effectuer les conversions d'aires suivantes :
   \begin{enumerate}
      \item $\umq{1275} =\pf \ucmq{}$
      \item $\udmq{32,5} =\pf\udamq{}$
      \item $\ucmq{345697,34} =\pf\umq{}$
      \item $\umq{0,003} =\pf\ummq{}$
      \item $\ukmq{2,5} =\pf \umq{}$
   \end{enumerate}
\end{exercice}

\begin{corrige}
\ \\ [-5mm]
   \begin{enumerate}
      \item $\umq{1275} =\blue\ucmq{12750000}$
      \item $\udmq{32,5} =\blue\udamq{0,00325}$
      \item $\ucmq{345697,34} =\blue\umq{34,569734}$
      \item $\umq{0,003} =\blue\ummq{3000}$
      \item $\ukmq{2,5} =\blue\umq{2500000}$
   \end{enumerate}
\end{corrige}

\medskip

\begin{exercice}
   Donner une unité de mesure usuelle pour chacun des objets suivants :
   \begin{enumerate}
      \item Longueur d'une piste d'athlétisme.
      \item Hauteur d'un livre.
      \item Surface d'un jardin.
      \item Longueur d'une fourmi.
      \item Surface d'une feuille.
      \item Largeur d'une pièce.
      \item Surface d'un champ.
      \item Rayon d'une planète.
   \end{enumerate}
\end{exercice}

\begin{corrige}
\ \\ [-5mm]
   \begin{enumerate}
      \item La longueur d'une piste d'athlétisme se mesure généralement en {\blue mètres}.
      \item La hauteur d'un livre se mesure généralement en {\blue centimètres}.
      \item La surface d'un jardin  se mesure généralement en {\blue mètres carrés} ou en {\blue ares}.
      \item La longueur d'une fourmi se mesure généralement en {\blue millimètres}.
      \item La surface d'une feuille se mesure généralement en {\blue centimètres carrés}.
      \item La largeur d'une pièce se mesure généralement en {\blue mètres}.
      \item La surface d'un champ se mesure généralement en {\blue hectares}.
      \item Le rayon d'une planète se mesure généralement en {\blue kilomètres}.
   \end{enumerate}

\bigskip

\corec{Curvica} \\ [1mm]
     Pour chaque défi, une solution est proposée. \\
     La présence d'un astérisque signifie que la solution n'est pas unique. \\ \medskip
      \begin{tabular}{|p{5cm}|C{1.6}|}
         \hline
         1. Aire max. & \textcolor{blue}{k} \\
         \hline
         2. Périmètre min. & \textcolor{blue}{d} \\
         \hline
         3. Rectangle avec 2 pièces & \textcolor{blue}{a et h} \\
         \hline
         4. Périmètre maxi., aire min. & \textcolor{blue}{s} \\
         \hline
         5. Un axe de symétrie & \textcolor{blue}{a*} \\
         \hline
         6. Deux axes de symétrie & \textcolor{blue}{n*} \\
         \hline
         7. Périmètres égaux, aires diff. & \textcolor{blue}{b et c*} \\
         \hline
         8. Aires égales, périmètres diff. & \textcolor{blue}{j et u*} \\
         \hline
         9. Même aire et périmètre & \textcolor{blue}{b et f*} \\
         \hline
         10. Carré avec 4 pièces & \textcolor{blue}{b, c, x, u*} \\
         \hline
         11. Même aire et périmètre, un axe & \textcolor{blue}{o et j} \\
         \hline
         12. Périmètre grand, aire petite & \textcolor{blue}{s et d*} \\
         \hline
         13. Aires et péri. max., 2 pièces & \textcolor{blue}{k et r} \\
         \hline
         14. Pas d'axe, ni même aire et péri.  & \textcolor{blue}{b et p*} \\
          \hline
         15. Rectangle avec 5 ou 6 pièces & \textcolor{blue}{a, e, f, g, h*} \\
         \hline
      \end{tabular}
\end{corrige}

\end{colonne*exercice}



%%%%%%%%%%%%%%%%%%%%%%%%%%%%%%%%%%%%%
%%%%%%%%%%%%%%%%%%%%%%%%%%%%%%%%%%%%%
\Recreation

\enigme[Le curvica]
Curvica est un jeu puzzle avec 24 pièces inventé par Jean Fromentin en 1982 afin de travailler sur les notions de périmètre, d'aire et de symétrie, entre autres ! À partir d’un carré, on obtient une pièce du puzzle curvica en \og creusant \fg, en \og bombant \fg{} ou en laissant droits les côtés.
      
   \partie[les 24 formes du curvica] \smallskip

   \curvica{\rput(1,1){\textcolor{B1}{a}} \psline(0,0)(0,2) \psline(0,2)(2,2) \psline(2,2)(2,0) \psarc(1,-2){2.24}{63.4}{116.6}}
   \curvica{\rput(1,1){\textcolor{B1}{b}} \psline(0,0)(0,2) \psline(0,2)(2,2) \psarc(4,1){2.24}{153.4}{-153.4} \psarc(1,2){2.24}{-116.6}{-63.4}}
   \curvica{\rput(1,1){\textcolor{B1}{c}} \psarc(2,1){2.24}{153.4}{-153.4} \psline(2,0)(2,2) \psline(0,2)(2,2) \psarc(1,2){2.24}{-116.6}{-63.4}}
   \curvica{\rput(1,1){\textcolor{B1}{d}} \psline(0,0)(0,2) \psline(2,0)(2,2) \psline(0,2)(2,2) \psline(0,0)(2,0)} \\ [1mm]
   
   \curvica{\rput(1,1){\textcolor{B1}{e}} \psline(0,0)(0,2) \psline(2,0)(2,2) \psarc(1,0){2.24}{63.4}{116.6} \psarc(1,2){2.24}{-116.6}{-63.4}}
   \curvica{\rput(1,1){\textcolor{B1}{f}} \psline(0,0)(0,2) \psline(2,0)(2,2) \psarc(1,4){2.24}{-116.6}{-63.4} \psarc(1,2){2.24}{-116.6}{-63.4}}
   \curvica{\rput(1,1){\textcolor{B1}{g}} \psline(0,0)(0,2) \psline(2,0)(2,2) \psarc(1,4){2.24}{-116.6}{-63.4} \psarc(1,-2){2.24}{63.4}{116.6}}
   \curvica{\rput(1,1){\textcolor{B1}{h}} \psline(0,0)(0,2) \psline(2,0)(2,2) \psline(0,2)(2,2) \psarc(1,2){2.24}{-116.6}{-63.4}} \\ [1mm]
   
   \curvica{\rput(1,1){\textcolor{B1}{i}} \psline(0,0)(0,2) \psarc(1,4){2.24}{-116.6}{-63.4} \psarc(4,1){2.24}{153.4}{-153.4} \psarc(1,2){2.24}{-116.6}{-63.4}} 
   \curvica{\rput(1,1){\textcolor{B1}{j}} \psarc(2,1){2.24}{153.4}{-153.4} \psarc(1,4){2.24}{-116.6}{-63.4} \psarc(4,1){2.24}{153.4}{-153.4} \psarc(1,-2){2.24}{63.4}{116.6}}
   \curvica{\rput(1,1){\textcolor{B1}{k}} \psarc(2,1){2.24}{153.4}{-153.4} \psarc(0,1){2.24}{-26.6}{26.6} \psarc(1,0){2.24}{63.4}{116.6} \psarc(1,2){2.24}{-116.6}{-63.4}}
   \curvica{\rput(1,1){\textcolor{B1}{l}} \psarc(-2,1){2.24}{-26.6}{26.6} \psline(2,0)(2,2) \psarc(1,4){2.24}{-116.6}{-63.4} \psarc(1,-2){2.24}{63.4}{116.6}} \\ [1mm]
   
   \curvica{\rput(1,1){\textcolor{B1}{m}} \psline(0,0)(0,2) \psarc(0,1){2.24}{-26.6}{26.6} \psarc(1,4){2.24}{-116.6}{-63.4} \psarc(1,-2){2.24}{63.4}{116.6}} 
   \curvica{\rput(1,1){\textcolor{B1}{n}} \psarc(-2,1){2.24}{-26.6}{26.6} \psarc(4,1){2.24}{153.4}{-153.4} \psarc(1,0){2.24}{63.4}{116.6} \psarc(1,2){2.24}{-116.6}{-63.4}}
   \curvica{\rput(1,1){\textcolor{B1}{o}} \psarc(2,1){2.24}{153.4}{-153.4} \psarc(1,4){2.24}{-116.6}{-63.4} \psarc(4,1){2.24}{153.4}{-153.4} \psarc(1,2){2.24}{-116.6}{-63.4}}
   \curvica{\rput(1,1){\textcolor{B1}{p}} \psarc(2,1){2.24}{153.4}{-153.4} \psline(2,0)(2,2) \psarc(1,0){2.24}{63.4}{116.6} \psarc(1,-2){2.24}{63.4}{116.6}} \\ [1mm]
   
      \curvica{\rput(1,1){\textcolor{B1}{q}} \psline(0,0)(0,2) \psarc(4,1){2.24}{153.4}{-153.4} \psarc(1,0){2.24}{63.4}{116.6} \psarc(1,2){2.24}{-116.6}{-63.4}}
      \curvica{\rput(1,1){\textcolor{B1}{r}} \psarc(2,1){2.24}{153.4}{-153.4} \psarc(0,1){2.24}{-26.6}{26.6} \psarc(1,4){2.24}{-116.6}{-63.4} \psarc(1,2){2.24}{-116.6}{-63.4}}
      \curvica{\rput(1,1){\textcolor{B1}{s}} \psarc(-2,1){2.24}{-26.6}{26.6} \psarc(1,4){2.24}{-116.6}{-63.4} \psarc(4,1){2.24}{153.4}{-153.4} \psarc(1,-2){2.24}{63.4}{116.6}}
      \curvica{\rput(1,1){\textcolor{B1}{t}} \psarc(2,1){2.24}{153.4}{-153.4} \psline(2,0)(2,2) \psarc(1,0){2.24}{63.4}{116.6} \psarc(1,2){2.24}{-116.6}{-63.4}} \\ [1mm]
      
      \curvica{\rput(1,1){\textcolor{B1}{u}} \psline(0,0)(0,2) \psarc(1,4){2.24}{-116.6}{-63.4} \psarc(4,1){2.24}{153.4}{-153.4} \psline(0,0)(2,0)} 
      \curvica{\rput(1,1){\textcolor{B1}{v}} \psarc(2,1){2.24}{153.4}{-153.4} \psarc(1,4){2.24}{-116.6}{-63.4} \psarc(4,1){2.24}{153.4}{-153.4} \psline(0,0)(2,0)}
      \curvica{\rput(1,1){\textcolor{B1}{w}} \psarc(2,1){2.24}{153.4}{-153.4} \psarc(1,0){2.24}{63.4}{116.6} \psarc(4,1){2.24}{153.4}{-153.4} \psline(0,0)(2,0)}
      \curvica{\rput(1,1){\textcolor{B1}{x}} \psarc(2,1){2.24}{153.4}{-153.4} \psarc(1,4){2.24}{-116.6}{-63.4} \psline(2,0)(2,2) \psline(0,0)(2,0)}

\pagebreak

\partie[grille réponse] \bigskip

{\hautab{1.75}
\begin{tabular}{|p{11cm}|C{3}|C{1}|}
   \hline
   \cellcolor{J2}{Défis} & \cellcolor{J2}{Réponses} & \cellcolor{J2}{Points} \\
   \hline
   \multicolumn{2}{|c|}{\cellcolor{FondTableaux}{Niveau facile}} & \cellcolor{FondTableaux}{1 pt} \\
   \hline
   1. Trouver la pièce dont l'aire est la plus grande. & & \\
   \hline
   2. Trouver la pièce dont le périmètre est le plus petit. & & \\
   \hline
   3. Assembler deux pièces pour obtenir un rectangle. & & \\
   \hline
   4. Trouver la pièce de plus grand périmètre et de plus petite aire. & & \\
   \hline
   5. Trouver une pièce ayant un seul axe de symétrie. & & \\
   \hline
   6. Trouver une pièce ayant exactement deux axes de symétrie. & & \\
   \hline
   7. Trouver deux pièces de même périmètre mais d'aires différentes. & & \\
   \hline
   \hline
   \multicolumn{2}{|c|}{\cellcolor{FondTableaux}{Niveau moyen}} & \cellcolor{FondTableaux}{1,5 pt} \\
   \hline
   8. Trouver deux pièces de même aire mais de périmètres différents. & & \\
   \hline
   9. Trouver deux pièces de même aire et de même périmètre. & & \\
   \hline
   10. Assembler quatre pièces pour obtenir un carré. & & \\
   \hline
   11. Trouver deux pièces ayant même aire, même périmètre et au moins un axe de symétrie chacune. & & \\
   \hline
   12. Trouver deux pièces dont l'une a un périmètre plus grand que l'autre mais une aire plus petite. & & \\
   \hline
   13. Assembler deux pièces pour obtenir une figure dont l'aire et le périmètres sont les plus grands possibles. & & \\
   \hline
   \hline
   \multicolumn{2}{|c|}{\cellcolor{FondTableaux}{Niveau difficile}} & \cellcolor{FondTableaux}{2 pts} \\
   \hline
   14. Trouver deux pièces ayant ni axe de symétrie, ni même périmètre, ni même aire. & & \\
   \hline
   15. Assembler cinq ou six pièces pour obtenir un rectangle. & & \\
   \hline
   \hline
   \multicolumn{2}{|r|}{\cellcolor{J2}{Total sur 20 points}} & \\
   \hline
\end{tabular}

\vfill

\hspace*{8cm}{\footnotesize\it Source : Yves Martin. Curvica - activités mathématiques ludiques, 2015, p.75}}

