\themaN
\graphicspath{{../../S23_Nombres_premiers/Images/}}

\chapter{Nombres premiers}
\label{S23}

\setlength{\doublerulesep}{0.3pt}


%%%%%%%%%%%%%%%%%%%%%%%%%%%%%%
%%%%%%%%%%%%%%%%%%%%%%%%%%%%%%
\begin{autoeval}
   \small
   \begin{enumerate}
      \item Il détermine les nombres premiers inférieurs ou égaux à 30.
      \item Il décompose un nombre entier strictement positif en produit de facteurs premiers inférieurs à 30.
      \item Il utilise la décomposition en facteurs premiers inférieurs à 30 pour produire des fractions égales (simplification ou mise au même dénominateur).
   \end{enumerate}
\end{autoeval}

\begin{prerequis}
   \begin{itemize}
      \item Définition d'un nombre premier.
      \item Liste des nombres premiers inférieurs ou égaux à 30.
      \item[\com] Décomposer un nombre entier en produit de facteurs premiers.
      \item[\com] Simplifier une fraction pour la rendre irréductible.
   \end{itemize}
\end{prerequis}

\vfill

\begin{debat}[Débat : la cryptologie]
   La {\bf cryptologie} est un art ancien et une science nouvelle : un art ancien car Jules César l’utilisait déjà ; une science nouvelle parce que ce n’est un thème de recherche scientifique que depuis les années 1970. Ce mot vient du grec {\it krypton} - caché et {\it logos} - science et signifie science du secret. Elle englobe la {\bf cryptographie} (l’écriture secrète) et la {\bf cryptanalyse} (l’analyse de cette dernière). Actuellement, on utilise en cryptographie des méthodes basées sur la difficulté de trouver la décomposition d'un nombre en produit de facteurs premiers.
   \begin{center}
      \begin{pspicture}(0,0.5)(2.5,3)
         \psset{fillstyle=solid}
         \psframe[fillcolor=A1,framearc=0.2,linecolor=A1](0.5,0.5)(2,2)
         \pscircle[fillcolor=white,linecolor=white](1.25,1.5){0.15}
         \psline[linewidth=0.15,linecolor=white]{c-c}(1.25,0.9)(1.25,1.35)
         \psarc[linewidth=0.25](1.25,2){0.5}{0}{180}         
      \end{pspicture}
      \begin{pspicture}(0,0.5)(2.5,3)
         \psset{fillstyle=solid}
          \psarc[linewidth=0.25](1.2,2.1){0.5}{45}{225} 
         \psframe[fillcolor=B1,framearc=0.2,linecolor=B1](0.5,0.5)(2,2)
          \pscircle[fillcolor=white,linecolor=white](1.25,1.5){0.15}
         \psline[linewidth=0.15,linecolor=white]{c-c}(1.25,0.9)(1.25,1.35)       
      \end{pspicture}
      \begin{pspicture}(0,0.5)(2.5,3)
         \psset{fillstyle=solid}
         \psframe[fillcolor=A1,framearc=0.2,linecolor=A1](0.5,0.5)(2,2)
         \pscircle[fillcolor=white,linecolor=white](1.25,1.5){0.15}
         \psline[linewidth=0.15,linecolor=white]{c-c}(1.25,0.9)(1.25,1.35)
         \psarc[linewidth=0.25](1.25,2){0.5}{0}{180}         
      \end{pspicture}
   \end{center}
   \bigskip
   \begin{cadre}[B2][F4]
      \begin{center}
         Vidéo : \href{https://www.yout-ube.com/watch?v=4jPtEsDS-qI}{\bf Les nombres premiers}, site Internet {\it Le blob}, épisode de la série {\it Petits contes mathématiques}.
      \end{center}
   \end{cadre}
\end{debat}


%%%%%%%%%%%%%%%%%%%%%%%%%%%%%%%%%%%%
%%%%%%%%%%%%%%%%%%%%%%%%%%%%%%%%%%%%
\activites

\begin{activite}[Le crible d'Ératosthène]
   {\bf Objectifs} : calculer des multiples ; suivre un algorithme ; déterminer des nombres premiers.
   \begin{QCM}
   On considère le tableau des nombres entiers de 1 à 100 ci-dessous. \\
   \begin{center}
      {\hautab{1.8}
      \begin{tabular}{|*{10}{C{0.5}|}}
         \hline
         1 & 2 & 3 & 4 & 5 & 6 & 7 & 8 & 9 & 10 \\
         \hline
         11 & 12 & 13 & 14 & 15 & 16 & 17 & 18 & 19 & 20 \\
         \hline
         21 & 22 & 23 & 24 & 25 & 26 & 27 & 28 & 29 & 30 \\
         \hline
         31 & 32 & 33 & 34 & 35 & 36 & 37 & 38 & 39 & 40 \\
         \hline
         41 & 42 & 43 & 44 & 45 & 46 & 47 & 48 & 49 & 50 \\
         \hline
         51 & 52 & 53 & 54 & 55 & 56 & 57 & 58 & 59 & 60 \\
         \hline
         61 & 62 & 63 & 64 & 65 & 66 & 67 & 68 & 69 & 70 \\
         \hline
         71 & 72 & 73 & 74 & 75 & 76 & 77 & 78 & 79 & 80 \\
         \hline
         81 & 82 & 83 & 84 & 85 & 86 & 87 & 88 & 89 & 90 \\
         \hline
         91 & 92 & 93 & 94 & 95 & 96 & 97 & 98 & 99 & 100 \\
         \hline
      \end{tabular}}
   \end{center}
   \medskip
   \partie[recherche des nombres premiers jusqu'à 100]
      \begin{enumerate}
         \item Barrer le nombre 1.
         \item Entourer le nombre 2, premier nombre non barré après 1, puis barrer tous les multiples de 2 plus grands que 2.
         \item Entourer le nombre 3, premier nombre non barré après 2, puis barrer tous les multiples de 3 plus grands que 3.
         \item Entourer le nombre 5, premier nombre non barré après 3, puis barrer tous les multiples de 5 plus grands que 5.
         \item Continuer ainsi de suite jusqu'à 10 puis entourer les nombres restants.
      \end{enumerate}
      
   \partie[conclusion]
      {\it Eratosthène} ($-276 ; -194$) était un mathématicien, géographe, philosophe, astronome, poète grec. Cet algorithme qu'il a établi porte son nom et permet de trouver tous les {\bf nombres premiers} (des nombres entiers divisibles uniquement par 1 et eux-même) inférieurs à un certain nombre $n$, ici 100. \\ [3mm]
      Lister tous les nombres entourés dans le tableau : ce sont les nombres premiers inférieurs à 100. \par \medskip
      \pointilles \par \medskip
      \pointilles \medskip
   \end{QCM}
 \end{activite}


%%%%%%%%%%%%%%%%%%%%%%%%%%%%%%%%%%
%%%%%%%%%%%%%%%%%%%%%%%%%%%%%%%%%%
\cours 

%%%%%%%%%%%%%%
\section{Nombres premiers}

\begin{definition}
   Un entier naturel est un \textbf{nombre premier} s'il admet comme seuls diviseurs 1 et lui-même.
\end{definition}

\begin{propriete}
   Les nombres premiers inférieurs à 30 sont 2, 3, 5, 7, 11, 13, 17, 19, 23 et 29.
\end{propriete}

\begin{exemple*1}
   \begin{itemize}
      \item 23 est un nombre premier car il est dans la liste des nombres premiers (divisible uniquement par 1 et 23).
      \item 49 est divisible par 1 et par 49, mais aussi par 7. Donc, 49 n'est pas un nombre premier. \vspace*{-5mm}
   \end{itemize}
\end{exemple*1}


%%%%%%%%%%%%%%%%%%%%%%%%%%%%
\section{Décomposition en produit de facteurs premiers}

\begin{propriete}
   Tout nombre entier admet une décomposition en produit de facteurs premiers, unique à l'ordre des facteurs près. 
\end{propriete}

\medskip

Pour déterminer cette décomposition, on teste si le nombre est divisible par les nombres premiers successifs, éventuellement plusieurs fois. Sur la calculatrice, on peut utiliser la fonction \og {\it Décomp} \fg.

\begin{exemple*1}
   Voilà quelques exemples de représentations de la décomposition de 150 en produit de facteurs premiers. \par
   \quad
   \begin{minipage}{2.5cm}
      \Decomposition[TableauVertical]{150}
   \end{minipage}
   \begin{minipage}{7.5cm}
      \Decomposition[Arbre,Entoure]{150}
      \qquad\;
      \Decomposition[ArbreComplet,Entoure]{150} 
   \end{minipage}
   \begin{minipage}{3cm}
      $150 =\Decomposition[Longue]{150}$, \\
      ou $150 =\Decomposition[Exposant]{150}$.
   \end{minipage}
\end{exemple*1}

%%%%%%%%%%%%%%%%%%%%%%%%%%
\section{Simplifier une fraction}

\begin{methode}[Rendre une fraction irréductible]
   Pour simplifier une fraction, c'est-à-dire écrire une fraction égale mais avec des nombres plus petits au numérateur et au dénominateur, on procède de la façon suivante :
   \begin{itemize}
      \item On décompose le numérateur en produit de facteurs premiers.
      \item On décompose le dénominateur en produit de facteurs premiers.
      \item On simplifie par tout nombre commun au numérateur et au dénominateur.
   \end{itemize}
   \exercice
   Simplifier la fraction $\dfrac{90}{84}$.
   \correction
   $\dfrac{90}{84} =\dfrac{2\times3^2\times5}{2^2\times3\times7} =\dfrac{\cancel{2}\times\cancel{3}\times3\times5}{\cancel{2}\times2\times\cancel{3}\times7} =\dfrac{3\times5}{2\times7} =\dfrac{15}{14}$.
\end{methode}


%%%%%%%%%%%%%%%%%%%%%%%%%%%%%%%%%%%
%%%%%%%%%%%%%%%%%%%%%%%%%%%%%%%%%%%
\exercicesbase

\begin{colonne*exercice}

\begin{exercice} %1
   Les nombres suivants sont-ils des nombres premiers ? Justifier.
   \begin{colenumerate}{3}
      \item 0 \smallskip
      \item 1 \smallskip
      \item 11
      \item 23
      \item 35
      \item 36 
      \item 38 
      \item 51 
      \item 99
   \end{colenumerate}
\end{exercice}
 
\begin{corrige}
   \begin{itemize}
      \item {\blue 0 et 1} ne sont pas premiers car un nombre premier est supérieur ou égal à 2. 
      \item {\blue 11 et 23} sont des nombres premiers.
      \item {\blue 35} n'est pas un nombre premier car il est divisible au moins par 5.
      \item {\blue 36 et 38} ne sont pas des nombres premiers car ils sont divisibles au moins par 2.
      \item {\blue 51 et 99} ne sont pas des nombres premiers car ils sont divisibles au moins par 3.
   \end{itemize}
\end{corrige}

\bigskip


\begin{exercice} %2
   Décomposer les nombres suivants en produits de facteurs premiers.
   \begin{colenumerate}{3}
      \item 12 \smallskip
      \item 18 \smallskip
      \item 28 \smallskip
      \item 48
      \item 210
      \item 442
      \item 2\,048
      \item 30\,375
      \item \, 100\,000
   \end{colenumerate}
\end{exercice}

\begin{corrige}
   \begin{colenumerate}{2}
      \item $12 =\blue 2^2\times3$
      \item $18 =\blue 2\times3^2$
      \item $28 =\blue 2^2\times7$
      \item $48 =\blue 2^4\times3$
      \item $210 =\blue 2\times3\times5\times7$
      \item $442 =\blue 2\times13\times17$
      \item $2\,048 =\blue 2^{11}$
      \item $30\,375 =\blue 3^5\times5^3$
      \item \, $100\,000 =\blue 2^5\times5^5$
   \end{colenumerate}
\end{corrige}

\bigskip


\begin{exercice} %3
   Simplification de fractions.
   \begin{enumerate}
      \item
         \begin{enumerate}
            \item Décomposer les nombres 90 et 75 en produit de facteurs premiers.
            \item Simplifier la fraction $\dfrac{90}{75}$ \medskip
            \item Simplifier la fraction $\dfrac{75}{90}$. Que remarque-t-on ? \smallskip
         \end{enumerate}
      \item
         \begin{enumerate}
            \item Décomposer les nombres 242 et 165 en produit de facteurs premiers.
            \item Simplifier la fraction $\dfrac{242}{165}$ \medskip
            \item Simplifier la fraction $\dfrac{165}{242}$
         \end{enumerate}
   \end{enumerate}
\end{exercice}

\begin{corrige}
\ \\ [-5mm]
   \begin{enumerate}
      \item
         \begin{enumerate}
            \item $90 ={\blue 2\times3^2\times5}$ et $75 ={\blue 3\times5^2}$ \smallskip
            \item $\dfrac{90}{75} =\dfrac{2\times\cancel{3}\times3\times\cancel{5}}{\cancel{3}\times\cancel{5}\times5} =\dfrac{2\times3}{5} ={\blue \dfrac65}$ \smallskip
            \item $\dfrac{75}{90}$ est l'inverse de $\dfrac{90}{75}$ donc, $\dfrac{75}{90}={\blue \dfrac56}$ \smallskip
         \end{enumerate}
      \setcounter{enumi}{1}
      \item          
         \begin{enumerate}
            \item $242 =\blue 2\times11^2$ et $165 =\blue3\times5\times11$ \smallskip
            \item $\dfrac{242}{165} =\dfrac{2\times\cancel{11}\times11}{3\times5\times\cancel{11}} =\dfrac{2\times11}{3\times5} ={\blue \dfrac{22}{15}}$ \smallskip
            \item $\dfrac{165}{242}$ est l'inverse de $\dfrac{242}{165}$ donc, $\dfrac{165}{242} ={\blue \dfrac{15}{22}}$
         \end{enumerate}
   \end{enumerate}
\end{corrige}

\bigskip


\begin{exercice} %4
   Simplifier les fractions suivantes : \smallskip
   \begin{colenumerate}{3}
      \item $\dfrac{5}{15}$ \medskip
      \item $\dfrac{12}{23}$ \medskip
      \item $\dfrac{15}{35}$ 
      \item $\dfrac{20}{90}$
      \item $\dfrac{57}{98}$
      \item $\dfrac{125}{45}$      
   \end{colenumerate}
\end{exercice}

\begin{corrige}
   \ \\ [-5mm]
   \begin{colenumerate}{2}
      \item $\dfrac{5}{15} =\dfrac{1\times\cancel{5}}{3\times\cancel{5}} ={\blue \dfrac13}$ \medskip
      \item $\dfrac{12}{23} =\dfrac{2\times2\times3}{23} ={\blue \dfrac{12}{23}}$ \medskip
      \item $\dfrac{15}{35} =\dfrac{3\times\cancel{5}}{\cancel{5}\times7} ={\blue \dfrac37}$ \medskip
      \item $\dfrac{20}{90} =\dfrac{2\times\cancel{10}}{9\times\cancel{10}} ={\blue \dfrac29}$ \medskip
      \item $\dfrac{125}{45} =\dfrac{5\times5\times\cancel{5}}{3\times3\times\cancel{5}} ={\blue \dfrac{25}{9}}$ \medskip
      \item $\dfrac{57}{98} =\dfrac{3\times19}{2\times7\times7} ={\blue \dfrac{57}{98}}$
   \end{colenumerate}
\end{corrige}

\bigskip


\begin{exercice} %5
   Voici les diviseurs de trois nombres :
   \begin{center}
      {\hautab{1.3}
      \begin{CLtableau}{0.9\linewidth}{2}{c}
         \hline
         & Liste des diviseurs \\
         \hline
         42 & 1 ; 2 ; 3 ; 6 ; 7 ; 14 ; 21 ; 42 \\
         \hline
         56 & 1 ; 2 ; 4 ; 7 ; 8 ; 14 ; 28 ; 56 \\
         \hline
         60 & 1 ; 2 ; 3 ; 4 ; 5 ; 6 ; 10 ; 12 ; 15 ; 20 ; 30 ; 60 \\
         \hline
      \end{CLtableau}}
   \end{center}
   À l'aide de cette liste, simplifier au maximum chaque fraction. \bigskip
   \begin{colenumerate}{3}
      \item $\dfrac{42}{56}$ \bigskip
      \item $\dfrac{56}{60}$ \bigskip
      \item $\dfrac{60}{42}$
      \item $\dfrac{56}{42}$  
      \item $\dfrac{60}{56}$
      \item $\dfrac{42}{60}$
   \end{colenumerate}
\end{exercice}

\begin{corrige}
\ \\ [-5mm]
    \begin{colenumerate}{2}
       \item $\dfrac{42}{56} =\dfrac{42\div14}{56\div14} ={\blue \dfrac{3}{4}}$ \medskip
       \item $\dfrac{56}{60} =\dfrac{56\div4}{60\div4} ={\blue \dfrac{14}{15}}$ \medskip
       \item $\dfrac{60}{42} =\dfrac{60\div6}{42\div6} ={\blue \dfrac{10}{7}}$
       \item $\dfrac{56}{42} =\dfrac{56\div14}{42\div14} ={\blue \dfrac{4}{3}}$
       \item $\dfrac{60}{56} =\dfrac{60\div4}{56\div4} ={\blue \dfrac{15}{14}}$    
       \item $\dfrac{42}{60} =\dfrac{42\div6}{60\div6} ={\blue \dfrac{7}{10}}$
   \end{colenumerate}
\end{corrige}

\bigskip


\begin{exercice} %6
   Une conjecture est un résultat que l'on pense vrai, mais qui n'a pas encore été démontré. La conjecture de Goldbach dit que tout nombre pair strictement supérieur à 2 peut s'écrire comme la somme de deux nombres premiers. Par exemple, $8 =3+5$ ; $40 =23+17$.
   \begin{enumerate}
      \item Trouver deux telles sommes pour 28.
      \item Trouver deux telles sommes pour 42.
      \item Trouver deux telles sommes pour 52.
   \end{enumerate}
\end{exercice}

\begin{corrige}
   \ \\ [-5mm]
   \begin{enumerate}
      \item $28 ={\blue 5+23 =11+17}$ \smallskip
      \item $42 ={\blue 11+31 =13+29 =19+23}$ \smallskip
      \item $52 ={\blue 11+41 = 23+29}$
   \end{enumerate}
\end{corrige}

\medskip


\begin{exercice} %7
   On considère un nombre entier naturel $n$. \\
   On note $S$ la somme de tous ses diviseurs stricts (c'est-à-dire ses diviseurs autres que lui-même). 
   \begin{itemize}
      \item $n$ est dit parfait lorsque $S=n$.
      \item $n$ est dit déficient lorsque $S<n$.
      \item $n$ est dit abondant lorsque $S>n$.
   \end{itemize}
   Par exemple, 8 a comme diviseurs stricts 1; 2 et 4. \\
   $S =1+2+4 =7< 8$. Donc, 8 est déficient.
   \begin{enumerate}
      \item Vérifier que 28 et 496 sont des nombre parfait.
      \item Trouver le plus petit nombre déficient, le plus petit nombre parfait et le plus petit nombre abondant.
      \item Quelle est la nature des nombres 7; 11 et 29 ?
      \item Quelle est la nature d'un nombre premier ?
   \end{enumerate}
\end{exercice}

\begin{corrige}
   \ \\ [-5mm]
   \begin{enumerate}
      \item -- Les diviseurs stricts de 28 sont 1, 2, 4, 7 et 14. \\
         $S =1+2+4+7+14 =28$ donc, {\blue 28 est parfait}. \\
         -- Les diviseurs stricts de 496 sont 1, 2, 4, 8, 16, 31, 62, 124 et 248. Donc, {\blue 496 est un nombre parfait} car
         $S =1+2+4+8+16+31+62+124+248 =496$.
      \item \textcolor{G1}{$\bullet$} Diviseur de 1 : aucun. $S =0$, {\blue 1 est déficient}.
      \begin{itemize}
         \item Diviseur de 2 : 1. $S =1$ donc, 2 est déficient.
         \item Diviseur de 3 : 1. $S =1$ donc, 3 est déficient.
         \item Diviseurs de 4 : 1 ; 2. \\
            $S =1+2 =3$ donc, 4 est déficient.
         \item Diviseur de 5 : 1. $S =1$ donc, 5 est déficient.
         \item Diviseurs de 6 : 1 ; 2 ; 3. \\
            $S =1+2+3 =6$ donc, {\blue 6 est parfait}.
         \item Diviseur de 7 : 1. $S =1$ donc, 7 est déficient.
         \item Diviseurs de 8 : 1 ; 2 ; 4. \\
            $S =1+2+4 =7$ donc, 8 est déficient.
         \item Diviseurs de 9 : 1 ; 3. \\
            $S =1+3 =4$ donc, 9 est déficient.
         \item Diviseurs de 10 : 1 ; 2 ; 5. \\
            $S =1+2+5 =8$ donc, 10 est déficient.
         \item Diviseur de 11 : 1. $S =1$ donc, 11 est déficient.
         \item Diviseurs de 12 : 1 ; 2 ; 3 ; 4 ; 6. \\
            $S =1+2+3+4+6 =16$ donc, {\blue 12 est abondant}.
      \end{itemize}
      \item Le seul diviseur strict de 7, 11 et 29 est 1 donc, {\blue 7, 11 et 29 sont déficients}
      \item Le seul diviseur strict d'un nombre premier est 1 donc, {\blue les nombres premiers sont déficients}.
   \end{enumerate}
\end{corrige}

\medskip


\begin{exercice} %8
   Les chrysodes traduisent des propriétés relatives à la division euclidienne et aux nombres premiers. Ils s'obtiennent à partir d'un cercle gradué.
   \begin{center}
      \begin{pspicture}(-4,-2.7)(4,3.3)
         \pscircle(0,0){3}
         \multido{\n=90.0+-51.4,\i=0+1}{7}{\psdot(3;\n) \rput(3.3;\n){\i}}
      \end{pspicture}
   \end{center}
   \begin{enumerate}
      \item Choisir un nombre de 1 à 6. Le multiplier par 3, puis calculer le reste de la division par 7. Tracer le corde d'extrémités le point correspondant au nombre choisi et le point correspondant au reste obtenu.
      \item À partir de ce reste, recommencer plusieurs fois. Que constate-ton ?
   \end{enumerate}  
\end{exercice}

\begin{corrige}
\ \\
   \begin{multicols}{2}
      On remarque que tous les points sont atteints et que la ligne brisée est fermée. \\
      \begin{pspicture}(-2,-1)(2,1)
         \pscircle(0,0){1.5}
         \multido{\n=90.0+-51.4,\i=0+1}{7}{\psdot(1.5;\n) \rput(1.8;\n){\i}}
         \pspolygon[linecolor=blue](1.5;38.6)(1.5;-61.2)(1.5;-12.8)(1.5;141.4)(1.5;-115.6)(1.5;192.8)
      \end{pspicture}
   \end{multicols}
\end{corrige}

\end{colonne*exercice}


%%%%%%%%%%%%%%%%%%%%%%%%%%%%%%%%%%
%%%%%%%%%%%%%%%%%%%%%%%%%%%%%%%%%%
\Recreation

\begin{enigme}[Le jeu de Juniper-Green]

   Ce jeu mathématique se joue à deux. Il a été créé par {\it Richard Porteous}, enseignant à l'école de Juniper Green. \bigskip
   
   \partie[règles du jeu]
      Deux joueurs jouent sur une grille de $N$ nombres suivant les règles suivantes :
      \begin{itemize}
         \item Règle 1 : le premier joueur choisit un nombre pair entre 1 et $N$ et le barre sur la grille.
         \item Règle 2 : chacun son tour, les deux joueurs choisissent un nombre parmi les multiples ou les diviseurs du nombre choisi précédemment par son adversaire et inférieur à $N$ puis le barre.
         \item Règle 3 : un nombre ne peut être joué qu'une seule fois.
      \end{itemize}
      Le joueur qui ne peut plus jouer a perdu. \\
      
   \partie[avec des grilles de 20 nombres]
      En binôme, jouer quelques parties sur une grille de 20 nombres. Sous la grille, noter la suite de nombres obtenue.\bigskip
         \begin{center}
            \hspace*{-5mm}
            \hautab{1.4}
            \begin{tabular}{|*{5}{C{0.3}|}}
               \hline
               1 & 2 & 3 & 4 & 5 \\
                \hline
               6 & 7 & 8 & 9 & 10 \\
               \hline
               11 & 12 & 13 & 14 & 15 \\
               \hline
               16 & 17 & 18 & 19 & 20 \\
               \hline
            \end{tabular}
            \qquad
            \begin{tabular}{|*{5}{C{0.3}|}}
               \hline
               1 & 2 & 3 & 4 & 5 \\
               \hline
               6 & 7 & 8 & 9 & 10 \\
               \hline
               11 & 12 & 13 & 14 & 15 \\
               \hline
               16 & 17 & 18 & 19 & 20 \\
               \hline
            \end{tabular}     
            \qquad
            \begin{tabular}{|*{5}{C{0.3}|}}
               \hline
               1 & 2 & 3 & 4 & 5 \\
               \hline
               6 & 7 & 8 & 9 & 10 \\
               \hline
               11 & 12 & 13 & 14 & 15 \\
               \hline
               16 & 17 & 18 & 19 & 20 \\
               \hline
            \end{tabular}
            \qquad
            \begin{tabular}{|*{5}{C{0.3}|}}
               \hline
               1 & 2 & 3 & 4 & 5 \\
               \hline
               6 & 7 & 8 & 9 & 10 \\
               \hline
               11 & 12 & 13 & 14 & 15 \\
               \hline
               16 & 17 & 18 & 19 & 20 \\
               \hline
            \end{tabular} \\ [7mm]
            \pointilles \qquad \pointilles  \qquad \pointilles  \qquad \pointilles  \hspace*{3mm} \\ [3mm]
         \end{center}
      Trouver une suite minimale : \pointilles  \\ [3mm]
      Trouver une suite maximale : \pointilles  \\
         
   \partie[avec des grilles de 100 nombres] 
      Jouer avec ces grilles de 100 nombres. \\
         \begin{center}
            {\hautab{1.4}
            \begin{tabular}{|*{10}{C{0.3}|}}
               \hline
               1 & 2 & 3 & 4 & 5 & 6 & 7 & 8 & 9 & 10 \\
               \hline
               11 & 12 & 13 & 14 & 15 & 16 & 17 & 18 & 19 & 20 \\
               \hline
               21 & 22 & 23 & 24 & 25 & 26 & 27 & 28 & 29 & 30 \\
               \hline
               31 & 32 & 33 & 34 & 35 & 36 & 37 & 38 & 39 & 40 \\
               \hline
               41 & 42 & 43 & 44 & 45 & 46 & 47 & 48 & 49 & 50 \\
               \hline
               51 & 52 & 53 & 54 & 55 & 56 & 57 & 58 & 59 & 60 \\
               \hline
               61 & 62 & 63 & 64 & 65 & 66 & 67 & 68 & 69 & 70 \\
               \hline
               71 & 72 & 73 & 74 & 75 & 76 & 77 & 78 & 79 & 80 \\
               \hline
               81 & 82 & 83 & 84 & 85 & 86 & 87 & 88 & 89 & 90 \\
               \hline
               91 & 92 & 93 & 94 & 95 & 96 & 97 & 98 & 99 & 100 \\
               \hline
            \end{tabular}
            \qquad
            \begin{tabular}{|*{10}{C{0.3}|}}
               \hline
               1 & 2 & 3 & 4 & 5 & 6 & 7 & 8 & 9 & 10 \\
               \hline
               11 & 12 & 13 & 14 & 15 & 16 & 17 & 18 & 19 & 20 \\
               \hline
               21 & 22 & 23 & 24 & 25 & 26 & 27 & 28 & 29 & 30 \\
               \hline
               31 & 32 & 33 & 34 & 35 & 36 & 37 & 38 & 39 & 40 \\
               \hline
               41 & 42 & 43 & 44 & 45 & 46 & 47 & 48 & 49 & 50 \\
               \hline
               51 & 52 & 53 & 54 & 55 & 56 & 57 & 58 & 59 & 60 \\
               \hline
               61 & 62 & 63 & 64 & 65 & 66 & 67 & 68 & 69 & 70 \\
               \hline
               71 & 72 & 73 & 74 & 75 & 76 & 77 & 78 & 79 & 80 \\
               \hline
               81 & 82 & 83 & 84 & 85 & 86 & 87 & 88 & 89 & 90 \\
               \hline
               91 & 92 & 93 & 94 & 95 & 96 & 97 & 98 & 99 & 100 \\
               \hline
            \end{tabular}}
         \end{center}
\end{enigme}

\begin{corrige}
   Exemple de suite minimale : {\blue 12 - 1 - 17}. \\
Exemple de suite maximale : {\blue 20 - 10 - 5 - 15 - 3 - 9 - 18 - 6 - 12 - 4 - 16 - 8 - 2 - 14 - 7 - 1 - 19}. 
\end{corrige}
