\annexe{Plans de travail}
\label{PDT}

%macros pour les bulles

\newcommand{\bulle}[3]{
   \psframe[linewidth=1mm,framearc=0.1,linecolor=#1](0,0)(5,3)
   \psline[linecolor=#1](0,2.3)(5,2.3)
   \rput(2.5,2.65){#2}
   \rput(2.5,1.15){\begin{minipage}{4cm}
                           {\small\darkgray #3}
                         \end{minipage}}}
                        
\newcommand{\bullelongue}[3]{
   \psframe[linewidth=1mm,framearc=0.1,linecolor=#1](0,0)(8.5,3)
   \psline[linecolor=#1](0,2.3)(8.5,2.3)
   \rput(4.25,2.65){#2}
   \rput(4.25,1.15){\begin{minipage}{8cm}
                           {\small\darkgray #3}
                         \end{minipage}}}
   
\newcommand{\bullecours}[3]{
   \psframe[linewidth=1mm,framearc=0.1,linecolor=#1](0,0)(12.5,3)
   \psline[linecolor=#1](0,2.3)(12.5,2.3)
   \rput(6.25,2.65){#2}
   \rput(6.25,1.15){\begin{minipage}{12cm}
                        {\small\darkgray #3}
                      \end{minipage}}}
   
\newcommand{\bulleQR}[3]{
   \psframe[linewidth=1mm,framearc=0.1,linecolor=#1](0,0)(4,7)
   \psline[linecolor=#1](0,6.3)(4,6.3)
   \rput(2,6.65){#2}
   \rput(2,3.15){\begin{minipage}{3.5cm}
                           \begin{center}
                              {\small\darkgray #3}
                           \end{center}
                         \end{minipage}}}
 
\begin{center}


%%%%%%%%%% Séquence 1 %%%%%%%%%%
%%%%%%%%%%%%%%%%%%%%%%%%%%%
\begin{pspicture}(0.5,0)(18,11)            
   {\color{red}
      \rput(9,5.75){\parbox{5cm}{\centering\large S1 \par  ENCHAÎNEMENT D'OPĖRATIONS}}} %bulle centrale  
   \rput[l](0,8){%bulle NNO : connaissances et compétences
      \pspolygon[fillstyle=solid,fillcolor=A1,linecolor=A1](6,0)(8,-1.5)(6.4,0)
      \bullecours
         {A1}
         {Je connais mon cours}
         {C1 : J'utilise et je range différentes représentations d'un même nombre décimal \hfill $\square$ \par
          C2 : Je traduis un enchaînement d’opérations à l’aide d’une expression et inversement \hfill $\square$ \par
          C3 : J'effectue un enchaînement d'opérations en respectant les priorités opératoires \hfill $\square$}}         
   \rput[l](14,4){%bulle ENE : Aide vidéo
      \pspolygon[fillstyle=solid,fillcolor=A1,linecolor=A1](0,3.2)(-2.5,2)(0,3.5)
      \bulleQR
         {A1}
         {Aide en vidéo}
         {\qrcode{https://www.youtube.com/watch?v=TJH-fiwAt5s} \par \medskip
          Calculer avec des priorités \par \bigskip
          \qrcode{https://www.youtube.com/watch?v=_yF5ItbcN28&feature=youtu.be} \par \medskip
          Traduire une expression}}    
      \rput[l](0,4){%bulle O : Questions flash
         \pspolygon[fillstyle=solid,fillcolor=Goldenrod,linecolor=Goldenrod](5,1.35)(6.5,1.5)(5,1.65)
         \bulle
            {Goldenrod}
            {Questions flash}
            {\psline[linecolor=darkgray](1.75,-0.5)(2.25,0.5)
             \rput(2.75,0){\darkgray\Huge 5}}}     
      \rput[l](0,0){%bulle SO : Compétence 1
         \pspolygon[fillstyle=solid,fillcolor=B1,linecolor=B1](5,2)(7.2,4.5)(5,2.35)
         \bulle
            {B1}
            {Compétence 1}
            {Activité d'approche \hfill $\star$ \hfill $\square$ \par
             Exercice 1 \hfill $\star$ \hfill $\square$ \par
             Exercice 2 \hfill $\star\star$ \hfill $\square$ \par
             Exercice 3 \hfill $\star\star\star$ \hfill $\square$}}
      \rput[l](6.5,0){%bulle S : compétence 2
         \pspolygon[fillstyle=solid,fillcolor=B1,linecolor=B1](2.35,3)(2.5,4.5)(2.65,3)
         \bulle
            {B1}
            {Compétence 2}
            {Exercice 4 \hfill $\star$ \hfill $\square$ \par
             Exercice 5 \hfill $\star\star$ \hfill $\square$ \par
             Exercice 6 \hfill $\star\star$ \hfill $\square$}}             
      \rput[l](13,0){%bulle SE : compétence 3
          \pspolygon[fillstyle=solid,fillcolor=B1,linecolor=B1](0,2)(-2.3,4.5)(0,2.35)
          \bulle
            {B1}
            {Compétence 3}
            {Exercice 7 \hfill $\star$ \hfill $\square$ \par
             Exercice 8 \hfill $\star$ \hfill $\square$ \par
             Exercice 9 \hfill $\star\star$ \hfill $\square$ \par
             Exercice 10 \hfill $\star\star\star$ \hfill $\square$ \par
             Récréation \hfill $\star\star\star$ \hfill $\square$}}                  
\end{pspicture}
   

%%%%%%%%%% Séquence 2 %%%%%%%%%%
%%%%%%%%%%%%%%%%%%%%%%%%%%%
\begin{pspicture}(0.5,0.5)(18,12.25)            
   {\color{DodgerBlue}
      \rput(9,5.75){\parbox{5cm}{\centering\large S2 \par ANGLES \par PARTICULIERS}}} %bulle centrale  
   \rput[l](0,8){%bulle NNO : connaissances et compétences
      \pspolygon[fillstyle=solid,fillcolor=A1,linecolor=A1](6,0)(8,-1.5)(6.4,0)
      \bullecours
         {A1}
         {Je connais mon cours}
         {C1 : Je reconnais des angles alternes-internes \hfill $\square$ \par
          C2 : Je reconnais des angles correspondants \hfill $\square$ \par
          C3 : J'utilise les propriétés des angles alternes-internes et correspondants pour montrer que des droites sont parallèles ou pour déterminer des angles \hfill $\square$}}         
   \rput[l](14,4){%bulle ENE : Aide vidéo
      \pspolygon[fillstyle=solid,fillcolor=A1,linecolor=A1](0,3.2)(-2.5,2)(0,3.5)
      \bulleQR
         {A1}
         {Aide en vidéo}
         {\qrcode{https://www.youtube.com/watch?v=c8CuPY-KaNM&list=PLVUDmbpupCaoTCiYBCUGfCyenktNbkIdt} \par \medskip
          Angles alternes-internes \par \bigskip
          \qrcode{https://www.youtube.com/watch?v=ErUq2wdA_PE&list=PLVUDmbpupCaoTCiYBCUGfCyenktNbkIdt&index=2} \par \medskip
          Angles correspondants}}    
      \rput[l](0,4){%bulle O : Questions flash
         \pspolygon[fillstyle=solid,fillcolor=Goldenrod,linecolor=Goldenrod](5,1.35)(6.5,1.5)(5,1.65)
         \bulle
            {Goldenrod}
            {Questions flash}
            {\psline[linecolor=darkgray](1.75,-0.5)(2.25,0.5)
             \rput(2.75,0){\darkgray\Huge 5}}}     
      \rput[l](0,0){%bulle SSO : Compétence 1
         \pspolygon[fillstyle=solid,fillcolor=B1,linecolor=B1](5,3)(8,4.5)(5.5,3)
         \bullelongue
            {B1}
            {Compétences 1 et 2}
            {Activité d'approche \hfill $\star$ \hfill $\square$ \par
             Exercice 1 \hfill $\star$ \hfill $\square$ \par
             Exercice 2 \hfill $\star$ \hfill $\square$ \par
             Exercice 3 \hfill $\star\star\star$ \hfill $\square$}}
      \rput[l](10,0){%bulle S : compétence 2
         \pspolygon[fillstyle=solid,fillcolor=B1,linecolor=B1](3,3)(0.5,4.5)(3.5,3)
         \bullelongue
            {B1}
            {Compétence 3}
            {Exercice 4 \hfill $\star$ \hfill $\square$ \par
             Exercice 5 \hfill $\star\star$ \hfill $\square$ \par
             Exercice 6 \hfill $\star\star$ \hfill $\square$ \par
             Exercice 7 \hfill $\star\star\star$ \hfill $\square$}}                             
\end{pspicture}


%%%%%%%%%% Séquence 3 %%%%%%%%%%
%%%%%%%%%%%%%%%%%%%%%%%%%%%
\begin{pspicture}(0.5,0)(18,10)            
   {\color{orange}
      \rput(9,5.75){\parbox{5cm}{\centering\large S3 \par EN ROUTE VERS LA PROGRAMMATION}}} %bulle centrale  
   \rput[l](0,8){%bulle NNO : connaissances et compétences
      \pspolygon[fillstyle=solid,fillcolor=A1,linecolor=A1](6,0)(8,-1.5)(6.4,0)
      \bullecours
         {A1}
         {Je connais mon cours}
         {C1 : Je réalise des activités d’algorithmique débranchée \hfill $\square$ \par
          C2 : Je traduis un script de déplacement ou de construction géométrique \hfill $\square$ \par
          C3 : J'écrit un script de déplacement ou de construction géométrique \hfill $\square$}}         
   \rput[l](14,4){%bulle ENE : Aide vidéo
      \pspolygon[fillstyle=solid,fillcolor=A1,linecolor=A1](0,3.2)(-2.5,2)(0,3.5)
      \bulleQR
         {A1}
         {Aide en vidéo}
         {\qrcode{https://www.youtube.com/watch?v=pdtMUgnmRa4&list=PLVUDmbpupCaqKLNci7_86rbIt61SMhJPd&index=1} \par \medskip
          Prise en main de Scratch \par \bigskip
          \qrcode{https://www.youtube.com/watch?v=8Sfarvw6jgg&list=PLVUDmbpupCaqKLNci7_86rbIt61SMhJPd&index=2} \par \medskip
          Utiliser une boucle}}    
      \rput[l](0,4){%bulle O : Questions flash
         \pspolygon[fillstyle=solid,fillcolor=Goldenrod,linecolor=Goldenrod](5,1.35)(6.5,1.5)(5,1.65)
         \bulle
            {Goldenrod}
            {Questions flash}
            {\psline[linecolor=darkgray](1.75,-0.5)(2.25,0.5)
             \rput(2.75,0){\darkgray\Huge 5}}}    
      \rput[l](0,0){%bulle SO : Compétence 1
         \pspolygon[fillstyle=solid,fillcolor=B1,linecolor=B1](5,2)(7.2,4.5)(5,2.35)
         \bulle
            {B1}
            {Compétence 1}
            {Activité d'approche \hfill $\star\star\star$ \hfill $\square$ \par
             Récréation \hfill $\star\star\star$ \hfill $\square$}}
      \rput[l](6.5,0){%bulle S : compétence 2
         \pspolygon[fillstyle=solid,fillcolor=B1,linecolor=B1](2.35,3)(2.5,4.5)(2.65,3)
         \bulle
            {B1}
            {Compétence 2}
            {Exercice 1 \hfill $\star$ \hfill $\square$ \par
             Exercice 2 \hfill $\star\star$ \hfill $\square$}}           
      \rput[l](13,0){%bulle SE : compétence 3
          \pspolygon[fillstyle=solid,fillcolor=B1,linecolor=B1](0,2)(-2.3,4.5)(0,2.35)
          \bulle
            {B1}
            {Compétence 3}
            {Exercice 1 \hfill $\star$ \hfill $\square$ \par
             Exercice 3 \hfill $\star$ \hfill $\square$ \par
             Exercice 4 \hfill $\star\star$ \hfill $\square$}}               
\end{pspicture}


%%%%%%%%%% Séquence 4 %%%%%%%%%%
%%%%%%%%%%%%%%%%%%%%%%%%%%%
\begin{pspicture}(0.5,0.5)(18,12.5)            
   {\color{red}
      \rput(9,5.75){\parbox{5cm}{\centering\large S4 \par NOMBRES \par RELATIFS}}} %bulle centrale  
   \rput[l](0,8){%bulle NNO : connaissances et compétences
      \pspolygon[fillstyle=solid,fillcolor=A1,linecolor=A1](6,0)(8,-1.5)(6.4,0)
      \bullecours
         {A1}
         {Je connais mon cours}
         {C1 : Je connais la notion de nombre relatif et d'opposé \hfill $\square$ \par
          C2 : Je repère et je place sur un axe gradué des nombres relatifs \hfill $\square$ \par
          C3 : Je compare, range, encadre des nombres relatifs \hfill $\square$}}         
   \rput[l](14,4){%bulle ENE : Aide vidéo
      \pspolygon[fillstyle=solid,fillcolor=A1,linecolor=A1](0,3.2)(-2.5,2)(0,3.5)
      \bulleQR
         {A1}
         {Aide en vidéo}
         {\qrcode{https://www.youtube.com/watch?v=SImiMoRB0vU} \par \medskip
          Droite graduée \par \bigskip
          \qrcode{https://www.youtube.com/watch?v=DYbRr4B42h8} \par \medskip
          Comparer des nombres}}    
      \rput[l](0,4){%bulle O : Questions flash
         \pspolygon[fillstyle=solid,fillcolor=Goldenrod,linecolor=Goldenrod](5,1.35)(6.5,1.5)(5,1.65)
         \bulle
            {Goldenrod}
            {Questions flash}
            {\psline[linecolor=darkgray](1.75,-0.5)(2.25,0.5)
             \rput(2.75,0){\darkgray\Huge 5}}}    
      \rput[l](0,0){%bulle SO : Compétence 1
         \pspolygon[fillstyle=solid,fillcolor=B1,linecolor=B1](5,2)(7.2,4.5)(5,2.35)
         \bulle
            {B1}
            {Compétence 1}
            {Activité d'approche \hfill $\star\star$ \hfill $\square$ \par
             Exercice 3 \hfill $\star$ \hfill $\square$ \par
             Exercice 4 \hfill $\star$ \hfill $\square$ \par
             Récréation \hfill $\star\star\star$ \hfill $\square$}}
      \rput[l](6.5,0){%bulle S : compétence 2
         \pspolygon[fillstyle=solid,fillcolor=B1,linecolor=B1](2.35,3)(2.5,4.5)(2.65,3)
         \bulle
            {B1}
            {Compétence 2}
            {Exercice 1 \hfill $\star\star$ \hfill $\square$ \par
             Exercice 2 \hfill $\star\star$ \hfill $\square$ \par
             Exercice 5 \hfill $\star$ \hfill $\square$ \par
             Exercice 6 \hfill $\star$ \hfill $\square$}}           
      \rput[l](13,0){%bulle SE : compétence 3
          \pspolygon[fillstyle=solid,fillcolor=B1,linecolor=B1](0,2)(-2.3,4.5)(0,2.35)
          \bulle
            {B1}
            {Compétence 3}
            {Exercice 7 \hfill $\star$ \hfill $\square$ \par
             Exercice 8 \hfill $\star\star$ \hfill $\square$ \par
             Exercice 9 \hfill $\star\star\star$ \hfill $\square$ \par
             Exercice 10 \hfill $\star\star$ \hfill $\square$ }}               
\end{pspicture}


%%%%%%%%%% Séquence 5 %%%%%%%%%%
%%%%%%%%%%%%%%%%%%%%%%%%%%%
\begin{pspicture}(0.5,0)(18,10)            
   {\color{DodgerBlue}
      \rput(9,5.75){\parbox{5cm}{\centering\large S5 \par Repérage \par dans le plan}}} %bulle centrale  
   \rput[l](0,8){%bulle NNO : connaissances et compétences
      \pspolygon[fillstyle=solid,fillcolor=A1,linecolor=A1](6,0)(8,-1.5)(6.4,0)
      \bullecours
         {A1}
         {Je connais mon cours}
         {C1 : Je sais lire les coordonnées dans un repère du plan \hfill $\square$ \par
          C2 : Je sais placer des points dans un repère du plan \hfill $\square$}}         
   \rput[l](14,4){%bulle ENE : Aide vidéo
      \pspolygon[fillstyle=solid,fillcolor=A1,linecolor=A1](0,3.2)(-2.5,2)(0,3.5)
      \bulleQR
         {A1}
         {Aide en vidéo}
         {\qrcode{https://www.youtube.com/watch?v=AHNYuKCoCvU&t=280s} \par \medskip
          Repère du plan \par \bigskip
          \qrcode{https://www.youtube.com/watch?v=kAtwKV3DqKI&list=PLVUDmbpupCap2d3CpwVNB_SrkYWEFsmBX&index=7} \par \medskip
          Lire et placer des points}}    
      \rput[l](0,4){%bulle O : Questions flash
         \pspolygon[fillstyle=solid,fillcolor=Goldenrod,linecolor=Goldenrod](5,1.35)(6.5,1.5)(5,1.65)
         \bulle
            {Goldenrod}
            {Questions flash}
            {\psline[linecolor=darkgray](1.75,-0.5)(2.25,0.5)
             \rput(2.75,0){\darkgray\Huge 5}}}    
      \rput[l](0,0){%bulle SSO : Compétence 1
         \pspolygon[fillstyle=solid,fillcolor=B1,linecolor=B1](5,3)(8,4.5)(5.5,3)
         \bullelongue
            {B1}
            {Compétence 1}
            {Exercice 1 \hfill $\star$ \hfill $\square$ \par
             Exercice 4 \hfill $\star\star$ \hfill $\square$}}
      \rput[l](10,0){%bulle S : compétence 2
         \pspolygon[fillstyle=solid,fillcolor=B1,linecolor=B1](3,3)(0.5,4.5)(3.5,3)
         \bullelongue
            {B1}
            {Compétence 2}
            {Activité d'approche \hfill $\star\star\star$ \hfill $\square$ \par
             Exercice 2 \hfill $\star\star$ \hfill $\square$ \par
             Exercice 3 \hfill $\star\star$ \hfill $\square$ \par
             Récréation \hfill $\star\star$ \hfill $\square$}}             
\end{pspicture}


%%%%%%%%%% Séquence 6 %%%%%%%%%%
%%%%%%%%%%%%%%%%%%%%%%%%%%%
\begin{pspicture}(0.5,0.5)(18,12.5)            
   {\color{red}
      \rput(9,5.75){\parbox{5cm}{\centering\large S4 \par NOMBRES \par RELATIFS}}} %bulle centrale  
   \rput[l](0,8){%bulle NNO : connaissances et compétences
      \pspolygon[fillstyle=solid,fillcolor=A1,linecolor=A1](6,0)(8,-1.5)(6.4,0)
      \bullecours
         {A1}
         {Je connais mon cours}
         {C1 : Je connais la notion de nombre relatif et d'opposé \hfill $\square$ \par
          C2 : Je repère et je place sur un axe gradué des nombres relatifs \hfill $\square$ \par
          C3 : Je compare, range, encadre des nombres relatifs \hfill $\square$}}         
   \rput[l](14,4){%bulle ENE : Aide vidéo
      \pspolygon[fillstyle=solid,fillcolor=A1,linecolor=A1](0,3.2)(-2.5,2)(0,3.5)
      \bulleQR
         {A1}
         {Aide en vidéo}
         {\qrcode{https://www.youtube.com/watch?v=SImiMoRB0vU} \par \medskip
          Droite graduée \par \bigskip
          \qrcode{https://www.youtube.com/watch?v=DYbRr4B42h8} \par \medskip
          Comparer des nombres}}    
      \rput[l](0,4){%bulle O : Questions flash
         \pspolygon[fillstyle=solid,fillcolor=Goldenrod,linecolor=Goldenrod](5,1.35)(6.5,1.5)(5,1.65)
         \bulle
            {Goldenrod}
            {Questions flash}
            {\psline[linecolor=darkgray](1.75,-0.5)(2.25,0.5)
             \rput(2.75,0){\darkgray\Huge 5}}}    
      \rput[l](0,0){%bulle SO : Compétence 1
         \pspolygon[fillstyle=solid,fillcolor=B1,linecolor=B1](5,2)(7.2,4.5)(5,2.35)
         \bulle
            {B1}
            {Compétence 1}
            {Activité d'approche \hfill $\star\star$ \hfill $\square$ \par
             Exercice 3 \hfill $\star$ \hfill $\square$ \par
             Exercice 4 \hfill $\star$ \hfill $\square$ \par
             Récréation \hfill $\star\star\star$ \hfill $\square$}}
      \rput[l](6.5,0){%bulle S : compétence 2
         \pspolygon[fillstyle=solid,fillcolor=B1,linecolor=B1](2.35,3)(2.5,4.5)(2.65,3)
         \bulle
            {B1}
            {Compétence 2}
            {Exercice 1 \hfill $\star\star$ \hfill $\square$ \par
             Exercice 2 \hfill $\star\star$ \hfill $\square$ \par
             Exercice 5 \hfill $\star$ \hfill $\square$ \par
             Exercice 6 \hfill $\star$ \hfill $\square$}}           
      \rput[l](13,0){%bulle SE : compétence 3
          \pspolygon[fillstyle=solid,fillcolor=B1,linecolor=B1](0,2)(-2.3,4.5)(0,2.35)
          \bulle
            {B1}
            {Compétence 3}
            {Exercice 7 \hfill $\star$ \hfill $\square$ \par
             Exercice 8 \hfill $\star\star$ \hfill $\square$ \par
             Exercice 9 \hfill $\star\star\star$ \hfill $\square$ \par
             Exercice 10 \hfill $\star\star$ \hfill $\square$ }}               
\end{pspicture}
\end{center}

