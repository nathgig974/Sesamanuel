\usetikzlibrary{calc}
\themaL
\thispagestyle{empty}
  

\tikzset{fondA/.style=green}
\tikzset{fondB/.style=cyan}

%=======================================
\begin{tikzpicture}[remember picture,overlay]
% fond bicolore
\coordinate (cp) at (current page);
\coordinate (cpc) at (current page.center);
\coordinate (cpe) at ($ (current page.east) + (1.7cm,0cm) $);
\coordinate (cpne) at ($ (current page.north east) + (1.7cm,2.9cm) $);
\coordinate (cpn) at ($ (current page.north) + (1.5cm,2.9cm) $);
\coordinate (cpnw) at ($ (current page.north west) + (-0.2cm,2.9cm) $);
\coordinate (cpw) at ($ (current page.west) + (-0.2cm,0cm) $);
\coordinate (cpsw) at ($ (current page.south west) + (-0.2cm,-0.2cm) $);
\coordinate (cps) at ($ (current page.south) + (1.5cm,-0.2cm) $);
\coordinate (cpse) at ($ (current page.south east) + (1.7cm,-0.2cm) $);
\fill[fondA] (cps) .. controls (cpw) and (cpe) .. (cpn) -- (cpnw)  -- (cpsw) -- cycle;
\fill[fondB] (cps) .. controls (cpw) and (cpe) .. (cpn) -- (cpne)  -- (cpse) -- cycle;

\end{tikzpicture}


