\themaE
\graphicspath{{../../S05_Reperage_et_deplacements/Images/}}

\chapter{Repérage\\dans le plan}
\label{S05}


%%%%%%%%%%%%%%%%%%%%%%%%%%%%%%%%%
%%%%%%%%%%%%%%%%%%%%%%%%%%%%%%%%%
\begin{autoeval}
   \small
   \begin{enumerate}
      \item Il se repère dans le plan muni d'un repère orthogonal.
   \end{enumerate}
\end{autoeval}

\begin{prerequis}
   \begin{itemize}
      \item Abscisse, ordonnée.
      \item[\com] (Se) repérer dans le plan muni d'un repère orthogonal.
   \end{itemize}
\end{prerequis}

\vfill

\begin{debat}[Débat : repère, ou repaire ?] 
   Ces deux mots ont la même origine, le nom latin {\it rapatrirare}, \og rentrer chez soi, rentrer dans sa patrie \fg. \\
   {\bf Repaire} a assez vite pris le sens de \og gîte d'animaux sauvages \fg. Au moyen-âge, l'écriture du nom n'était pas encore stabilisée et s'écrivait également {\bf repère}, que l'on rattacha à tort au nom latin {\it reperire}, \og retrouver \fg. Finalement, repère se spécialise pour désigner une marque permettant de retrouver quelque chose. \\ [2mm]
    Revenons aux mathématiques, comment se repérer, selon le type d'objet où l'on se trouve :
   \begin{center}
      {\psset{unit=0.8,linecolor=B1}
      \begin{pspicture}(0,-0.5)(15,3.3)
         \psline(0,0)(3,3)
         \rput(1.5,-0.7){\small Sur une droite ?}
         \psframe(4,0)(7,3)
         \rput(5.5,-0.7){\small Sur un plan ?}
         \psframe(8,0)(10,2) 
         \psline(10,0)(11,1)(11,3)(9,3)(8,2)
         \psline(10,2)(11,3)
         \rput(9.5,-0.7){\small Dans l'espace ?}
         \pscircle(13.5,1.5){1.5}
         \psellipticarc(13.5,1.5)(1.5,0.6){180}{0}
         \rput(13.5,-0.7){\small Sur une sphère ?}
      \end{pspicture}}    
   \end{center}
   \bigskip
   \begin{cadre}[B2][J4]
      \begin{center}
         Vidéo : \href{https://www.youtube.com/watch?v=Tu2kRuZcWRI}{\bf C'est quoi un peu en 4D (et 1D) ?}, chaîne YouTube {\it Trash Bandicoot}.
      \end{center}
   \end{cadre}
\end{debat}


%%%%%%%%%%%%%%%%%%%%%%%%%%%%%%%%
%%%%%%%%%%%%%%%%%%%%%%%%%%%%%%%%
\activites

\begin{activite}[Dessin gradué]
   {\bf Objectif :} créer un dessin en utilisant un repérage particulier. \\ [-5mm]
   \begin{QCM}
      Pour découvrir le dessin codé, il faut placer les points A, B, C\dots{} selon les indications du tableau ci-dessous. Par exemple, le point A est sur la première ligne et son abscisse est 8. \\
      Une fois tous les points placés les relier en suivant les instructions données.
      \begin{center}
         \begin{tabular}{|*{3}{C{0.5}|}}
            \hline
            \cellcolor{lightgray}{\!\!\!\small Ligne} & \cellcolor{lightgray}{\!\!\!\small Point} & \cellcolor{lightgray}{\!\small Abs.} \\
            \hline
            (1) & A & 8 \\
            \hline
            (1) & B & 9 \\
            \hline
            (1) & C & 12 \\
            \hline
            (2) & D & 17 \\
            \hline
            (2) & E & 18 \\
            \hline
            (2) & F& 19 \\
            \hline
            (2) & G & 20 \\
            \hline
            (2) & H & 21 \\
            \hline
            (2) & I & 22 \\
            \hline
            (2) & J & 23 \\
            \hline
            (3) & K & 57 \\
            \hline
         \end{tabular}
         \hfill
         \begin{tabular}{|*{3}{C{0.5}|}}
            \hline
            \cellcolor{lightgray}{\!\!\!\small Ligne} & \cellcolor{lightgray}{\!\!\!\small Point} & \cellcolor{lightgray}{\!\small Abs.} \\
            \hline
            (3) &  L & 58 \\
            \hline
            (3) & M & 59 \\
            \hline
            (3) & N & 63 \\
            \hline
            (3) & O & 64 \\
            \hline
            (4) & P & 23 \\
            \hline
            (4) & Q & 24 \\
            \hline
            (4) & R & 25 \\
            \hline
            (4) & S & 28 \\
            \hline
            (4) & T & 29 \\
            \hline
            (5) & U & 22 \\
            \hline
            (5) & V & 24 \\
            \hline
         \end{tabular}
         \hfill
         \begin{tabular}{|*{3}{C{0.5}|}}
            \hline
            \cellcolor{lightgray}{\!\!\!\small Ligne} & \cellcolor{lightgray}{\!\!\!\small Point} & \cellcolor{lightgray}{\!\small Abs.} \\
            \hline
            (5) & W & 26 \\
            \hline
            (6) & X & 36 \\
            \hline
            (6) & Y & 44 \\
            \hline
            (7) & Z & 6 \\
            \hline
            (7) & A' & 14 \\
            \hline
            (7) & B' & 18 \\
            \hline
            (7) & C' & 22 \\
            \hline
            (8) & D' & 15 \\
            \hline
            (8) & E' & 18 \\
            \hline
            (8) & F' & 27 \\
            \hline
            (9) & G' & 103 \\
            \hline
         \end{tabular}
         \hfill
         \begin{tabular}{|*{3}{C{0.5}|}}
            \hline
            \cellcolor{lightgray}{\!\!\!\small Ligne} & \cellcolor{lightgray}{\!\!\!\small Point} & \cellcolor{lightgray}{\!\small Abs.} \\
            \hline
            (9) & H' & 107 \\
            \hline
            (9) & I' & 108 \\
            \hline
            (10) & J' & 2 \\
            \hline
             (10) & K' & 4 \\
            \hline
            (10) & L' & 16 \\
            \hline
            (11) & M' & 50 \\
            \hline
            (11) & N' & 80 \\
            \hline
            (12) & O' & 32 \\
            \hline
            (12) & P' & 44 \\
            \hline
            (13) & Q' & 0,1 \\
            \hline
            (13) & R' & 0,2 \\
            \hline
         \end{tabular}
         \hfill
         \begin{tabular}{|*{3}{C{0.5}|}}
            \hline
            \cellcolor{lightgray}{\!\!\!\small Ligne} & \cellcolor{lightgray}{\!\!\!\small Point} & \cellcolor{lightgray}{\!\small Abs.} \\
            \hline
            (13) & S' & 0,3 \\
            \hline
            (13) & T' & 0,5 \\
            \hline
            (13) & U' & 0,6 \\
            \hline
            (13) & V' & 0,7 \\
            \hline
            (13) & W' & 0,8 \\
            \hline
            (13) & X' & 0,9 \\
            \hline
            (14) & Y' & $-15$ \\
            \hline
            (14) & Z' & $-14$ \\
            \hline
            (14) & A'' & $-11$ \\
            \hline
            (14) & B'' & $-7$ \\
            \hline
            (14) & C'' & $-6$ \\
            \hline
         \end{tabular}
      \end{center}
      \begin{minipage}{4cm}
         Tracer les lignes brisées \\
         suivantes : \\ [3mm]
         FELMPKDACOTWVY \\
         C'P’C"B"V’O’W’X’B’XA’  \\ [3mm]
         GB \\ [3mm]
         HJNI \\ [3mm]
         ST \\ [3mm]
         QRUC’ \\ [3mm]
         D’E’L’N’U’T’ \\ [3mm]
         F’I’H’ \\ [3mm]
         U’V’ \\ [3mm]
         B"A"S’M’G’K’R’Z’Y’Q’J’ZK \\ [3mm]
         ZG’
      \end{minipage}
      \qquad
      \begin{minipage}{10cm}
            \DessinGradue[LignesIdentiques=false,Echelle=1,EcartVertical=0.8]
            {0/15/15,
            10/25/15,
            50/65/15,
            15/30/15,
            0/30/30, %5
            20/50/15,
            0/30/15,
            0/45/15,
            100/115/15,
            0/30/15, %10
            0/150/15,
            0/60/30,
            0/1.5/15,
            -15/0/15}
            {1/A/8,1/B/9,1/C/12,
            2/D/7,2/E/8,2/F/9,2/G/10,2/H/11,2/I/12,2/J/13,
            3/K/7,3/L/8,3/M/9,3/N/13,3/O/14,
            4/P/8,4/Q/9,4/R/10,4/S /13,4/T/14,
            5/U/22,5/V/24,5/W/26,
            6/X/8,6/ Y/12,
            7/Z/3,7/A'/7,7/B'/9,7/C'/11,
            8/D'/5 ,8/E'/6,8/F'/9,
            9/G'/3,9/H'/7,9/I'/8,
            10/J'/1,10/K'/2,10/L'/8,
            11/M'/5,11/N'/8,
            12/O'/16,12/P'/22,
            13/Q'/1,13/R'/2,13/S'/3,13/ T'/5,13/U'/6,13/V'/7,13/W'/8,13/X'/9,
            14/Y'/0,14/Z'/1,14/A''/4,14/B''/8,14/C''/9}
            {chemin(F,E,L,M,P,K,D,A,C,O,T,W,V,Y,C',P',C '',B'',V',O',W',X',B',X,A'),
            chemin(G,B),
            chemin(H,J,N,I),
            chemin(S,T),
            chemin(Q,R,U ,C'),
            chemin(D',E',L',N',U',T'),
            chemin(F ',I',H'),
            chemin(U',V'),
            chemin(B'',A'',S',M',G',K',R',Z',Y',Q',J',Z,K),
            chemin(Z, G')}
      \end{minipage} \smallskip
   \end{QCM}
   \vfill\hfill{\it\small Activité inspirée de la brochure APMEP n°169 : \og Jeux 7 \fg}
\end{activite}


%%%%%%%%%%%%%%%%%%%%%%%%%%%%%%%%%%%%%%%%
%%%%%%%%%%%%%%%%%%%%%%%%%%%%%%%%%%%%%%%%
\cours 

%%%%%%%%%%%%%%%%%%
\section{La droite graduée (rappels)}

\begin{definition}
   Pour graduer une droite, il faut choisir une {\bf origine} qui correspond au \og 0 \fg{} et une {\bf unité} qui sera reportée de manière régulière. \\
   Sur une droite graduée, un point est repéré par son {\bf abscisse}.
\end{definition}
      
\begin{center}
   \begin{pspicture}(-3,-1.3)(5.1,1)
      \psaxes[yAxis=false]{->}(0,0)(-3,0)(5.1,0)
      \psline[linecolor=gray]{<-}(0,0)(-0.5,0.5)
      \rput(-1,0.7){\textcolor{gray}{origine}}
      \psline[linecolor=gray]{<->}(0,0.3)(1,0.3)
      \rput(0.5,0.6){\textcolor{gray}{unité}}
      \rput(3,0.4){\textcolor{A1}{A}}
      \rput(3,-0.9){\textcolor{A1}{l'abscisse du}}
      \rput(3,-1.3){\textcolor{A1}{point A est 3}}
      \rput(3,-1.7){\textcolor{A1}{on note A(3)}}
   \end{pspicture}
\end{center}


%%%%%%%%%%%%%%%%%%%%%%%%%%%%%%%%%%%%%%%
\section{Repérer un point dans un repère du plan}


\begin{definition}  
   Un \textbf{repère orthogonal} est constitué de deux axes gradués perpendiculaires et sécants en O.
    \begin{itemize}
      \item O est l'\textbf{origine} du repère ;
      \item la droite horizontale est l'\textbf{axe des abscisses} ;
      \item la droite verticale est l'\textbf{axe des ordonnées}.
   \end{itemize} 
\end{definition}

\medskip

\begin{propriete}
   Dans un repère, un point $M$ est repéré par un couple $(x;y)$ appelé coordonnées du point $M$. \\
   $x$ est l'\textbf{abscisse} du point et $y$ est l'\textbf{ordonnée}.
\end{propriete}

\begin{exemple*1}
\ \\
   \begin{pspicture}(-7,-3.5)(7,3.5)
      \psgrid[gridlabels=0,subgriddiv=0,gridcolor=lightgray!70](-4,-3)(4,3)
      \pstGeonode[PosAngle=45](3,2){A}(-2,1.5){B}(-3,-2){C}(2.5,-1.5){D}
      \rput(-0.3,-0.3){\small $O$}
      \rput[l](4.5,0){\textcolor{A1}{axe des abscisses}}
      \rput(0,3.5){\textcolor{B1}{axe des ordonnées}}
      \rput[l](4.5,-2.5){\gray origine du repère}
      \footnotesize
      \psaxes[yAxis=false,linecolor=A1,labels=none]{->}(0,0)(-4,0)(4,0)
      \multido{\n=-4+1}{4}{\rput(\n,-0.4){\textcolor{A1}{\n}}}
      \multido{\n=1+1}{4}{\rput(\n,-0.4){\textcolor{A1}{\n}}}
      \psaxes[xAxis=false,linecolor=B1,labels=none]{->}(0,0)(0,-3)(0,3)
      \multido{\n=-3+1}{3}{\rput(-0.4,\n){\textcolor{B1}{\n}}}
      \multido{\n=1+1}{3}{\rput(-0.4,\n){\textcolor{B1}{\n}}}
      \psline[linestyle=dashed,linecolor=gray]{->}(4,-2.5)(2,-2.5)(0.1,-0.1)
      \psline[linestyle=dashed]{<->}(0,2)(3,2)(3,0)
   \end{pspicture}
   \correction
      Les coordonnées des points $O, A, B, C$ et $D$ sont : \\
      $O(\textcolor{A1}{0}\,;\textcolor{B1}{0})$ \qquad $A(\textcolor{A1}{3}\,;\textcolor{B1}{2})$ \qquad $B(\textcolor{A1}{-2}\,;\textcolor{B1}{1,5})$ \qquad $C(\textcolor{A1}{-3}\,;\textcolor{B1}{-2})$ \qquad $D(\textcolor{A1}{2,5}\,;\textcolor{B1}{-1,5})$
\end{exemple*1}


%%%%%%%%%%%%%%%%%%%%%%%%%%%%%%%%
%%%%%%%%%%%%%%%%%%%%%%%%%%%%%%%%
\exercicesbase

\begin{colonne*exercice}

\begin{exercice} %1
  Lire puis écrire les coordonnées des points A à L.
  {\psset{unit=0.4}
  \begin{center}
  \begin{pspicture}(-9,-10)(9,10.25)
   \psgrid[gridlabels=0,subgriddiv=0,gridcolor=lightgray](-9,-10)(9,10)
      \psaxes[labels=none,ticks=none]{->}(0,0)(-9,-10)(9,10)
      \psline(2,-0.2)(2,0.2)
      \psline(-0.2,2)(0.2,2)
      \small
      \rput(-0.4,-00.4){$O$}
      \rput(2,-0.5){1}
      \rput(-0.5,2){1}
      \pstGeonode[PointSymbol=+,PosAngle=45,linewidth=1mm](3,3){A}(-4,7){B}(-8,-7){C}(7,-3){D}(6,9){E}(-8,2){F}(-2,-4){G}(2,-6){H}(8,0){I}(0,5){J}(-4;0){K}(0,-8){L}
   \end{pspicture}
   \end{center}}
\end{exercice}

\begin{corrige}
   \begin{colitemize}{2}
      \item \blue $A(1,5\,;\,1,5)$
      \item \blue $B(-2\,;\,3,5)$
      \item \blue $C(-4\,;\,-3,5)$
      \item \blue $D(3,5\,;\,-1,5)$
      \item \blue $E(3\,;\,4,5)$
      \item \blue $F(-4\,;\,1)$
      \item \blue $G(-1\,;\,-2)$
      \item \blue $H(1\,;\,-3)$
      \item \blue $I(4\,;\,0)$
      \item \blue $J(0\,;\,2,5)$
      \item \blue $K(-2\,;\,0)$
      \item \blue $L(0\,;\,-4)$
   \end{colitemize}
\end{corrige}

\medskip


\begin{exercice} %2
   On considère les points de coordonnées : \\
   {\hautab{1.2}
   \begin{tabular}{p{1.6cm}p{1.6cm}p{1.6cm}p{1.6cm}}
      $M(-9\,;-5)$ & $N(-4\,;0)$ & $O'(2,5\,;7)$ & $P(5\,;3)$ \\
      $Q(-1\,;-1)$ & $R(2\,;-3)$ & $S(5\,;-2)$ & $T(-6,5\,;-2)$ \\
      $U(-1\,;-4)$ & $V(2\,;0)$ & $W(-6,5\,;4)$ & $X(-9\,,0)$ \\
      $Y(-4\,;-5)$ & $Z(-6,5\,;-1)$ & & \\
   \end{tabular}}
   \begin{enumerate}
      \item Créer un repère orthogonal en prenant un centimètre pour une unité qui puisse contenir tous les points.
      \item Placer les points dans le repère.
      \item Relier dans l'ordre les points suivants :
      \begin{itemize}
         \item $W-X-M-Y-N-W-O'-P-S-Y$.
         \item $U-Q-V-R$.
         \item $X-N-P$.
         \item Tracer le cercle de centre $T$ passant par $Z$.
      \end{itemize}
      Imane reconnait un dessin familier. Quel est-il ?
   \end{enumerate}
\end{exercice}

\begin{corrige}
   Imane reconnait le dessin d'{\blue une maison}. \\
   {\psset{unit=0.777,PointSymbol=none}
   \begin{pspicture}(-10,-5.6)(6,8.3)
   \psgrid[gridlabels=0,subgriddiv=0,gridcolor=lightgray](-10,-6)(6,8)
      \psaxes[labels=none,ticks=none]{->}(0,0)(-10,-6)(6,8)
      \psline(1,-0.2)(1,0.2)
      \psline(-0.2,1)(0.2,1)
      \footnotesize
      \rput(-0.4,-00.4){$O$}
      \rput(1,0.5){1}
      \rput(-0.5,1){1}
      \psset{linecolor=blue}
      \pstGeonode[PosAngle={90,135,-135,-45,50},CurveType=polygon](-6.5,4){W}(-9,0){X}(-9,-5){M}(-4,-5){Y}(-4,0){N}
      \pstGeonode[PosAngle=45,CurveType=polyline](2.5,7){O'}(5,3){P}(5,-2){S}
      \pstLineAB{W}{O'}
      \pstLineAB{S}{Y}
      \pstGeonode[PosAngle={-90,135,45,-45},CurveType=polyline](-1,-4){U}(-1,-1){Q}(2,0){V}(2,-3){R}
      \pstLineAB{X}{N}
      \pstLineAB{N}{P}
      \pstGeonode(-6.5,-2){T}(-6.1,-1){Z}
      \psdot(-6.5,-2)
      \pstCircleOA{T}{Z}
   \end{pspicture}}
\end{corrige}

\medskip


\begin{exercice} %3
   On se place dans un repère orthogonal d'origine $O$ et d'unité \ucm{1}.
   \begin{enumerate}
      \item Placer le point $A$ de coordonnées $(3,5\,;1,5)$.
      \item Placer le point $B$ symétrique du point $A$ par rapport à l'axe des abscisses. Quelles sont ses coordonnées ?
      \item Placer le point $C$ symétrique du point $B$ par rapport à l'axe des ordonnées. Quelles sont ses coordonnées ?
      \item Placer le point $D$ symétrique du point $C$ par rapport à l'axe des abscisses. Quelles sont ses coordonnées ?
      \item Quelle est la nature du quadrilatère $ABCD$ ?
   \end{enumerate}
\end{exercice}

\begin{corrige}
   On obtient un {\blue rectangle}. \\
   \begin{pspicture}(-3.5,-2)(4,2.2)
      \footnotesize
      \psgrid[gridlabels=0,subgriddiv=2,gridcolor=lightgray](-4,-2)(4,2)
      \pstGeonode[PointSymbol=none,PosAngle={45,-45,-135,135},CurveType=polygon,linecolor=blue](3.5,1.5){A}(3.5,-1.5){B}(-3.5,-1.5){C}(-3.5,1.5){D}
      \psaxes[labels=none,ticks=none]{->}(0,0)(-4,-2)(4,2)
      \psline(1,-0.2)(1,0.2)
      \psline(-0.2,1)(0.2,1)
      \rput(-0.4,-00.4){$O$}
      \rput(1,0.5){1}
      \rput(-0.5,1){1}
   \end{pspicture} \\
   {\blue $B(3,5\,;-1.5)\,; C(-3,5\,;-1.5)\,;D(-3,5\,;1.5)$} \\
\end{corrige}

\columnbreak


\begin{exercice} %4
   L'image suivante représente la position obtenue au déclenchement du bloc \og Départ \fg{} d'un programme. \\
   L'arrière-plan est constitué de points espacés de 40 unités. Le chat a pour coordonnées $(-120\,;\,-80)$. \\
   Le but du jeu est de positionner le chat sur la balle représentée par le petit disque.
   \begin{center}
   {\psset{unit=0.4}
   \begin{pspicture}(-6,-3.3)(6,4.3)
      \multido{\n=-6+1}{13}{
      \multido{\na=-4+1}{9}{\psdots[dotscale=0.5](\n,\na)}}
      \psaxes[Dx=10,Dy=10]{->}(0,0)(-6,-4)(6,4)
      \uput[dl](0,0){O}
      \psdots[dotscale=2](4,3)
      \rput(-3,-2){Chat}
   \end{pspicture}}
   \end{center}
   \begin{enumerate}
      \item Quelles sont les coordonnées du centre de la balle représentée dans cette position?
      \item Dans cette question, le chat est dans la position obtenue au déclenchement du bloc départ. \\
      Voici le script du lutin \og chat \fg{} qui se déplace. \\ [2mm]
      \begin{Scratch}[Echelle=0.75]
         Place Drapeau;
         Place Bloc("Départ");
      \end{Scratch}
      \quad
      \begin{Scratch}[Echelle=0.75]
         Place QPresse("n'importe laquelle");
         Place Si(TestCapToucheObjet("Balle"));
            Place DireT("Je t'ai attrapée","2");
            Place Bloc("Départ");
         Place FinBlocSi;
      \end{Scratch} \\ [2mm]
      \hspace*{-5mm}   
      \begin{Scratch}[Echelle=0.65]
         Place QPresse("flèche droite");
         Place AjouterVar("80","x");
      \end{Scratch}
      \begin{Scratch}[Echelle=0.65]
         Place QPresse("flèche haut");
         Place AjouterVar("80","y");
      \end{Scratch} \\ [2mm]
      \hspace*{-5mm}
      \begin{Scratch}[Echelle=0.65]
         Place QPresse("flèche gauche");
         Place AjouterVar("-40","x");
      \end{Scratch}
      \begin{Scratch}[Echelle=0.65]
         Place QPresse("flèche bas");
         Place AjouterVar("-40","y");
      \end{Scratch} 
      \begin{enumerate}
         \item Expliquer pourquoi le chat ne revient pas à sa position de départ si le joueur appuie sur la touche $\to$ puis sur la touche $\gets$.
         \item Le joueur appuie sur la succession de touches suivante : $\to$ $\to$  $\uparrow$ $\gets$ $\downarrow$. Quelles sont les coordonnées $x$ et $y$ du chat après ce déplacement ? Justifier.
         \item Parmi les propositions ci-dessous, laquelle permet au chat d'atteindre la balle ? \medskip
      \end{enumerate}
      \hspace*{-5mm}
      {\small
      \hautab{1.2}
      \begin{tabular}{|c|c|c|}
         \hline
         déplacement 1 & déplacement 2 & déplacement 3\\ \hline
         $\to \to \to \to \to \to \to \uparrow\uparrow\uparrow\uparrow\uparrow$
         &
         $\to\to\to \uparrow\uparrow\uparrow \to \downarrow \gets$
         &
         $\uparrow \to \uparrow \to \uparrow \to \to \downarrow \downarrow$ \\
         \hline
      \end{tabular}}
      \smallskip
      \item Que se passe-t-il quand le chat atteint la balle ?
   \end{enumerate}
\end{exercice}

\begin{corrige}
   \ \\ [-5mm]
   \begin{enumerate}
      \item La balle est située à quatre espaces vers la droite et trois espaces vers le haut, soit $4\times40$ unités = 160 unités et $3\times40$ unités = 120 unités. \\
      Ses coordonnées sont donc {\blue (160\,;\,120)}.
      \item \\
      \begin{enumerate}
         \item La touche $\to$ ajoute 80 à l'abscisse $x$ ; la touche $\gets$ ajoute $-40$ à l'abscisse $x$ ; donc, la succession $\to \, \gets$ ajoute $80+(-40) =40$ à l'abscisse $x$. \\
            Le chat a donc \og avancé \fg{} de 40 unités vers la droite,  {\blue il ne revient pas à sa position de départ}. \vspace*{11.5cm}
         \item On résume dans un tableau les déplacements : \\ \smallskip
            {\hautab{1.3}
            \begin{tabular}{|*{7}{c|}}
               \hline
               & départ & $\to$ & $\to$ & $\uparrow$ & $\gets$ & $\downarrow$ \\
               \hline
               $x$ & $-120$ & $-40$ & 40 & 40 & 0 & 0 \\
               \hline 
               $y$ & $-80$ & $-80$ & $-80$ & 0 & 0 & $-40$ \\
               \hline
            \end{tabular}} \\ \smallskip
         Les coordonnées du chat après ces cinq déplacements sont {\blue $(0\,;\,-40)$}.
         \item Seul le {\blue déplacement 2} permet au chat d'attraper la balle.
      \end{enumerate}
      \setcounter{enumi}{2}
      \item Quand le chat atteint la balle, {\blue il dit \og Je t'ai attrapée \fg{} pendant 2 sec.} puis retourne au départ. 
   \end{enumerate}
\end{corrige}

\end{colonne*exercice}


%%%%%%%%%%%%%%%%%%%%%%%%%%%%%%%
%%%%%%%%%%%%%%%%%%%%%%%%%%%%%%%
\Recreation

\begin{enigme}[Déformations]
   \partie[dessin]
      \begin{minipage}{10cm}
         \begin{enumerate}
            \item Placer ces points dans le repérage ci-contre. \\ [-9mm]
               \begin{multicols}{4}
                 \textls[400]{A($-$3;2) \\ B($-$3;$-5$) \\ C(3;$-$5) \\ D(3;2) \\ E(4;1) \\
            F(4;2) \\ G(0;6) \\ H($-$4;2) \\ I($-$4;1) \\ J(0;5) \\
            K($-$1;$-$5) \\ L($-$1;$-$2) \\ M(1;$-$2) \\ N(1;$-$5) \\
            O($-$2;0) \\ P($-$2;2) \\ Q($-$1;2) \\ R($-$1;0) \\ S(1;$-$1) \\
            T(1;1) \\ U(2;1) \\ V(2;$-$1)}
               \end{multicols}
               \vspace*{-4mm}
         \item Relier les points suivants : \\
            ABCDEFGHIJD \\
            KLMN \\
            OPQRO \\
            STUVS
         \end{enumerate}
      \end{minipage}
      \qquad
      \begin{minipage}{6cm}
         {\psset{unit=0.5}
         \begin{pspicture}(-6,-6)(5,6)
            \psgrid[subgriddiv=0,gridcolor=lightgray,gridlabels=0](0,0)(-5,-6)(5,7)
            \psaxes[labels=none]{->}%
(0,0)(-5,-6)(5,7)
            \rput(-0.4,-0.4){\scriptsize 0}
            \rput(1,-0.5){\scriptsize 1}
            \rput(-0.5,1){\scriptsize 1}
         \end{pspicture}}
      \end{minipage}
      \bigskip
      
   \partie[déformations]
      Tracer le dessin dans les repères suivants : que se passe-t-il ? \\ [2mm]
      \begin{minipage}{6cm}
         {\psset{xunit=0.5}
         \begin{pspicture}(-6,-5)(5,7)
            \psgrid[subgriddiv=0,gridcolor=lightgray,gridlabels=0](0,0)(-5,-6)(5,7)
            \psaxes[labels=none]{->}%
(0,0)(-5,-6)(5,7)
            \rput(-0.4,-0.4){\scriptsize 0}
            \rput(1,-0.5){\scriptsize 1}
            \rput(-0.5,1){\scriptsize 1}
         \end{pspicture}}
      \end{minipage}
      \begin{minipage}{11cm}
         {\psset{yunit=0.5}
         \begin{pspicture}(-6,-6)(5,7)
            \psgrid[subgriddiv=0,gridcolor=lightgray,gridlabels=0](0,0)(-5,-6)(5,7)
            \psaxes[labels=none]{->}%
(0,0)(-5,-6)(5,7)
            \rput(-0.4,-0.4){\scriptsize 0}
            \rput(1,-0.5){\scriptsize 1}
            \rput(-0.5,1){\scriptsize 1}
         \end{pspicture}} \\
         \pstilt{55}{
         {\psset{xunit=0.7,yunit=0.6}
         \begin{pspicture}(-5,-5)(5,8)
            \psgrid[subgriddiv=0,gridcolor=lightgray,gridlabels=0](0,0)(-5,-6)(5,7)
            \psaxes[labels=none]{->}%
(0,0)(-5,-6)(5,7)
            \rput(-0.4,-0.4){\scriptsize 0}
            \rput(1,-0.5){\scriptsize 1}
            \rput(-0.5,1){\scriptsize 1}
         \end{pspicture}}}
      \end{minipage}
\end{enigme}

\begin{corrige}
   \partie[dessin]
      {\psset{unit=0.5}
         \begin{pspicture}(-6,-6)(5,8)
            \psgrid[subgriddiv=0,gridcolor=lightgray,gridlabels=0](0,0)(-5,-6)(5,7)
            \psaxes[labels=none]{->}%
(0,0)(-5,-6)(5,7)
            \rput(-0.4,-0.4){\scriptsize 0}
            \rput(1,-0.5){\scriptsize 1}
            \rput(-0.5,1){\scriptsize 1}
            \psset{linecolor=blue}
            \psline(-3,2)(-3,-5)(3,-5)(3,2)(4,1)(4,2)(0,6)(-4,2)(-4,1)(0,5)(3,2) 
            \psframe(-1,-5)(1,-2)
            \psframe(-2,0)(-1,2)
            \psframe(1,-1)(2,1)
         \end{pspicture}}
         \bigskip
         
   \partie[déformations]
      {\psset{xunit=0.5}
         \begin{pspicture}(-6,-5)(5,7.5)
            \psgrid[subgriddiv=0,gridcolor=lightgray,gridlabels=0](0,0)(-5,-6)(5,7)
            \psaxes[labels=none]{->}%
(0,0)(-5,-6)(5,7)
            \rput(-0.4,-0.4){\scriptsize 0}
            \rput(1,-0.5){\scriptsize 1}
            \rput(-0.5,1){\scriptsize 1}
            \psset{linecolor=blue}
            \psline(-3,2)(-3,-5)(3,-5)(3,2)(4,1)(4,2)(0,6)(-4,2)(-4,1)(0,5)(3,2) 
            \psframe(-1,-5)(1,-2)
            \psframe(-2,0)(-1,2)
            \psframe(1,-1)(2,1)
         \end{pspicture}}

\Coupe

      {\psset{yunit=0.5}
         \begin{pspicture}(-4.05,-6)(5,10.55)
            \psgrid[subgriddiv=0,gridcolor=lightgray,gridlabels=0](0,0)(-5,-6)(5,7)
            \psaxes[labels=none]{->}%
(0,0)(-5,-6)(5,7)
            \rput(-0.4,-0.4){\scriptsize 0}
            \rput(1,-0.5){\scriptsize 1}
            \rput(-0.5,1){\scriptsize 1}
            \psset{linecolor=blue}
            \psline(-3,2)(-3,-5)(3,-5)(3,2)(4,1)(4,2)(0,6)(-4,2)(-4,1)(0,5)(3,2) 
            \psframe(-1,-5)(1,-2)
            \psframe(-2,0)(-1,2)
            \psframe(1,-1)(2,1)
         \end{pspicture}}
         
      \pstilt{55}{
      {\psset{xunit=0.7,yunit=0.6}
         \begin{pspicture}(-2,-5)(5,10.5)
            \psgrid[subgriddiv=0,gridcolor=lightgray,gridlabels=0](0,0)(-5,-6)(5,7)
            \psaxes[labels=none]{->}%
(0,0)(-5,-6)(5,7)
            \rput(-0.4,-0.4){\scriptsize 0}
            \rput(1,-0.5){\scriptsize 1}
            \rput(-0.5,1){\scriptsize 1}
            \psset{linecolor=blue}
            \psline(-3,2)(-3,-5)(3,-5)(3,2)(4,1)(4,2)(0,6)(-4,2)(-4,1)(0,5)(3,2) 
            \psframe(-1,-5)(1,-2)
            \psframe(-2,0)(-1,2)
            \psframe(1,-1)(2,1)
         \end{pspicture}}}
\end{corrige}