\themaM
\graphicspath{{../Ch18_Aires_et_surfaces/Images/}}

\chapter{Aires par pavages}
\label{C10}


%%%%%%%%%%%%%%%%%%%%%%%%%%%%%%%%%%%%%
%%%%%%%%%%%%%%%%%%%%%%%%%%%%%%%%%%%%%
\begin{prerequis}[Connaissances et compétences associées]
   \begin{itemize}
      \item Différencier périmètre et aire d’une figure.
      \item Comparer des surfaces selon leurs aires sans avoir recours à la mesure, par superposition ou par découpage et recollement.
      \item Déterminer la mesure de l’aire d’une surface à partir d’un pavage simple.
      \item Unités usuelles d’aire et leurs relations : multiples et sous-multiples du m$^2$.
      \item Estimer la mesure d’une aire et l’exprimer dans une unité adaptée.
   \end{itemize}
\end{prerequis}

\vfill

\begin{debat}[Débat : le SI (Système International)]
   En 1795, il existe en France plus de 700 {\bf unités de mesures différentes} qui varient d'une ville à l'autre. Source d'erreurs et de fraudes lors des transactions commerciales, politiques et scientifiques vont tenter de réformer cet état de fait : leur idée est d'assurer l'invariabilité des mesures en les rapportant à un étalon universel emprunté à un phénomène naturel. Le 26 mars 1791 nait le mètre (du grec {\it metron}, mesure), dont la longueur est établie comme égale à la dix-millionième partie du quart du méridien terrestre. L'unité de mesure de base étant déterminée, il suffit désormais d'établir toutes les autres unités de mesure qui en découlent : le mètre carré et le mètre cube, le litre, le gramme\dots{} Le système international des unités (SI) est né en 1960. En 2018, les unités de base sont redéfinies à partir de sept constantes physiques. \\
   \begin{center}
      {\psset{unit=0.8}
      \begin{pspicture}(-2.5,-2)(2.5,1.8)
         \pscircle*[linecolor=gray](0,0){2.5}
         \pswedge*[linecolor=orange](0,0){2}{0}{51}
         \rput(1.25;25){\large\white m}
         \pswedge*[linecolor=red](0,0){2}{51}{103}
         \rput(1.25;75){\large\white kg}
         \pswedge*[linecolor=magenta](0,0){2}{103}{154}
         \rput(1.25;126){\large\white cd}
         \pswedge*[linecolor=violet](0,0){2}{154}{205}
         \rput(1.25;176){\large\white mol}
         \pswedge*[linecolor=blue](0,0){2}{205}{257}
         \rput(1.25;228){\large\white K}
         \pswedge*[linecolor=cyan](0,0){2}{257}{308}
         \rput(1.25;280){\large\white A}
         \pswedge*[linecolor=green](0,0){2}{308}{360}
         \rput(1.25;334){\large\white s}
      \end{pspicture}}
   \end{center}
   \bigskip
   \begin{cadre}[B2][F4]
      \begin{center}
         Vidéo : \href{https://www.youtube.com/watch?time_continue=2&v=bInHclEN6zQ&feature=emb_logo}{\bf Système International d'unités. L'épopée}, {\it Laboratoire national de métrologie et d'essais}. 
      \end{center}
   \end{cadre}
\end{debat}

\vfill

\textcolor{PartieGeometrie}{\sffamily\bfseries Cahier de compétences} : chapitre 11, exercices 12 ; 26 à 28 ; 30 à 35 ; 42.


%%%%%%%%%%%%%%%%%%%%%%%%%%%%%%%%%%%%
%%%%%%%%%%%%%%%%%%%%%%%%%%%%%%%%%%%%
\activites

\begin{activite}[Curvica]
   {\bf Objectif :} différencier aire et périmètre ; comparer des périmètres ; comparer des aires.
   \begin{QCM}
      \partie[présentation]
         À partir d’un carré, on obtient une pièce du puzzle Curvica en \og creusant \fg, en \og bombant \fg{} ou en \og laissant droit \fg{} les côtés. Par exemple, voici une pièce de Curvica : \\
         \begin{center}
           \curvica{}
            \begin{pspicture}(0,-0.25)(2,2.25)
               \rput(1,1){$\Longrightarrow$}
            \end{pspicture}
            \curvica{
               \psline(0,0)(2,0)(2,2)
               \psarc(1,4){2.24}{-116.6}{-63.4}
               \psarc(2,1){2.24}{153.4}{-153.4}}
            \begin{pspicture}(-1.5,-0.25)(2.5,2.25)
               \rput(1,1.75){\it\small tous les arcs de cercles}
               \rput(1,1.25){\it\small (creusés et bombés) reliant}
               \rput(1,0.75){\it\small deux sommets du carré}
               \rput(1,0.25){\it\small sont superposables.} 
            \end{pspicture}
         \end{center}
         \bigskip
      \partie[défis]
         Construire deux pièces différentes (non superposables par rotation, déplacement ou retournement) telles que : \\ [8mm]
         \parbox{5cm}{les deux pièces ont la même aire mais des périmètres différents}\parbox{1.5cm}{\phantom{}}\parbox{5cm}{\curvica{}}\parbox{5cm}{\curvica{}} \\ [10mm]
         \parbox{5cm}{les deux pièces ont le même périmètre mais des aires différentes.}\parbox{1.5cm}{\phantom{}}\parbox{5cm}{\curvica{}}\parbox{5cm}{\curvica{}} \\ [10mm]
         \parbox{5cm}{les deux pièces ont le même périmètre et la même aire} \parbox{1.5cm}{\phantom{}}\parbox{5cm}{\curvica{}}\parbox{5cm}{\curvica{}} \\ [5mm] 
   \end{QCM}
   \vfill\hfill {\footnotesize\it Source : Yves Martin. Curvica - activités mathématiques ludiques. 2015, pp.75. hal-01502901}
\end{activite}


%%%%%%%%%%%%%%%%%%%%%%%%%%%%%%%%%%%%
%%%%%%%%%%%%%%%%%%%%%%%%%%%%%%%%%%%%
\cours 

%%%%%%%%%%%%%%%%%%%%%%%%%%%%
\section{Surface et aire}

\begin{definition}
   La \textbf{surface} d'une figure est la partie située à l'intérieur de son contour. \\
   Sa mesure s'appelle l'\textbf{aire}, qui est le nombre d'unités d'aire que la figure contient.
\end{definition}

\begin{remarque}
   attention à ne pas confondre avec le périmètre qui est une mesure de longueur !
\end{remarque}

Pour déterminer l'aire d'une surface, on peut découper la figure en figure simples ou utiliser un pavage simple de la figure.

\begin{exemple*1}
\ \\
   {\psset{unit=0.8}
   \begin{pspicture}(-3,-0.5)(10,6.6)
      \put(1,1){\pspolygon[fillstyle=solid,fillcolor=B2,linewidth=0.1](0,0)(2,0)(2,3)(0,3)(0,2)(1,2)(1,1)(0,1)(0,0)}
      \put(4,1){\pspolygon[fillstyle=solid,fillcolor=A2,linewidth=0.1](1,0)(2,0)(2,2)(1,2)(1,3)(0,3)(0,1)}
      \put(7,1){\pspolygon[fillstyle=solid,fillcolor=J2,linewidth=0.1](1,0)(1.5,0.5)(2,0)(2,1)(3,1)(2.5,1.5)(3,2)(2,2)(2,3)(1.5,2.5)(1,3)(1,2)(0,2)(0.5,1.5)(0,1)(1,1)(1,0)}
      \rput(2.5,2.4){\textbf{A}}
      \rput(5,2.4){\textbf{B}}
      \rput(8.5,2.4){\textbf{C}}
      \rput(1.5,5.5){{$u_1$}} 
      \psframe[fillstyle=solid,fillcolor=darkgray,linewidth=0.1](2,5)(3,6)
      \rput(4.5,5.5){{$u_2$}}
      \pspolygon[fillstyle=solid,fillcolor=darkgray,linewidth=0.1](6,5)(6,6)(5.5,5.5)
      \rput(7.5,5.5){{$u_3$}}\pspolygon[fillstyle=solid,fillcolor=darkgray,linewidth=0.1](8,5)(9,5)(8,6)
      \psgrid[subgriddiv=0,gridlabels=0pt,gridwidth=0.02,gridcolor=darkgray](11,7)
      \multido{\i=0+1}{5}{%
	\FPeval{a}{\i+7}
	\psline[linecolor=darkgray,linewidth=0.02](\i,0)(\a,7)
	\psline[linecolor=darkgray,linewidth=0.02](\i,7)(\a,0)
	}
      \multido{\i=1+1}{6}{%
	\FPeval{b}{7-\i}
	\FPeval{c}{4+\i}
	\psline[linecolor=darkgray,linewidth=0.02](0,\i)(\b,7)
	\psline[linecolor=darkgray,linewidth=0.02](\i,0)(0,\i)
	\psline[linecolor=darkgray,linewidth=0.02](\c,0)(11,\b)
	\psline[linecolor=darkgray,linewidth=0.02](\c,7)(11,\i)
	}
   \end{pspicture}}
   \correction   
   Lorsque l'on n'a pas une unité d'aire entière $u_1$, on prend une partie de l'unité : 
   \begin{itemize}
      \item $u_2$ correspond à la moitié d'un carré $=\dfrac12 =0,5$ ;
      \item $u_3$ correspond au quart d'un carré $=\dfrac14 =0,25$. \smallskip
   \end{itemize}
   On peut aussi \og découper \fg{} une partie de la figure afin de la déplacer ailleurs pour former une unité d'aire.
   \begin{center}
      \begin{cltableau}{0.9\linewidth}{4}
         \hline
         Unité & fig. A & fig. B & fig. C \\
         \hline
            $u_1$ & $5$ & $4,5$ & $4$ \\
         \hline
         $u_2$ & $20$ & $18$ & $16$ \\
         \hline
         $u_3$ & $10$ & $9$ & $8$ \\
         \hline
      \end{cltableau}
   \end{center}
\end{exemple*1}


\section{Unités de mesure d'aires}

Pour désigner une aire, on utilise le mètre carré (\umq{}) et ses multiples et sous-multiples. Pour les mesures agraires, on utilise l'are (a) qui équivaut à \umq{100} et l'hectare (ha) qui vaut 100 ares, c'est-à-dire \umq{10000}.

\begin{center}
   \renewcommand{\arraystretch}{1}
   \begin{ltableau}{0.8\linewidth}{14}
      \hline
      \multicolumn{2}{|c|}{\ukmq{}} & \multicolumn{2}{c|}{\uhmq{}} & \multicolumn{2}{c|}{\udamq{}} & \multicolumn{2}{c||}{\umq{}} & \multicolumn{2}{c|}{\udmq{}} & \multicolumn{2}{c|}{\ucmq{}} & \multicolumn{2}{c|}{\ummq{}} \\
      \hline
      & & & & & $3$ & $7$ & \multicolumn{1}{C{0.5}||}{$0$} & $1$ & $5$ & $0$ & $4$ & & \\
      \hline
   \end{ltableau}
\end{center}

\begin{exemple*1}
   \umq{370,1504} = \udmq{37015,04} = \ummq{370150400} = \udamq{3,701504}.
\end{exemple*1}


%%%%%%%%%%%%%%%%%%%%%%%%%%%%%%%%%%%
%%%%%%%%%%%%%%%%%%%%%%%%%%%%%%%%%%%
\exercicesbase

\begin{colonne*exercice}

\serie{Différencier aire et périmètre}

\begin{exercice} %1
   On choisit comme unité de longueur $u.\ell.$ la longueur du côté d'un carreau du cahier et comme unité d'aire $u.a.$ la surface d'un carreau.
   \begin{enumerate}
      \item Construire cinq polygones de périmètre 10 $u.\ell.$ \\
         Ont-ils tous la même aire ?
      \item Construire huit polygones d'aire 5 $u.a.$ \\
         Ont-ils tous le même périmètre ?
   \end{enumerate}
\end{exercice}

\medskip

\begin{exercice} %2
   On considère les deux figures A et B suivantes :
   \begin{center}
      {\psset{unit=0.5}
      \begin{pspicture}(-1,-1)(8,3)
         \pspolygon[fillstyle=solid,fillcolor=A2,linecolor=gray](0,0)(3,0)(1,2)(0,2)
         \pspolygon[fillstyle=solid,fillcolor=B2,linecolor=gray](4,0)(7,0)(7,1)(6,1)(6,2)(4,2)
         \psgrid[subgriddiv=1,gridlabels=0,gridcolor=gray](-1,-1)(8,3)
         \rput(1,1){A}
         \rput(5,1){B}
      \end{pspicture}}
   \end{center}
   Dans toute la suite, les sommets des polygones doivent être des noeuds du quadrillage.
   \begin{enumerate}
      \item Construire une figure différente de A mais de même périmètre que A.
      \item Construire une figure différente de B mais de même périmètre que B.
      \item Construire une figure de même aire que la figure A.
      \item Construire une figure de même aire que la figure B.
      \item Construire trois figures différentes qui ont à la fois le même périmètre que la figure A et la même aire que la figure B.
   \end{enumerate}
\end{exercice}

\medskip

\begin{exercice} %3
   On considère les figures 1 et 2 :
   \begin{center}
      {\psset{unit=0.5}
      \begin{pspicture}(0,-5.5)(10,7)
         \psgrid[subgriddiv=1,gridlabels=0,gridcolor=gray](0,-6)(10,7)
         \pspolygon[linewidth=0.5mm](1,1)(3,1)(3,2)(4,1)(8,1)(9,2)(9,4)(8,5)(6,5)(6,4)(4,6)(1,6)(1,5)(2,4)(1,3)
         \rput(5,3){figure 1}
         \pspolygon[linewidth=0.5mm](1,0)(9,0)(9,-2)(8,-3)(7,-2)(6,-3)(7,-4)(6,-5)(5,-4)(3,-4)(2,-5)(1,-4)(1,-3)(2,-3)(2,-1)(1,-1)
         \rput(4.5,-2){figure 2}
      \end{pspicture}}
   \end{center}
   \begin{enumerate}
      \item Comparer le périmètre de ces deux figures sans les calculer.
      \item Comparer l'aire de ces deux figures sans les calculer.
    \end{enumerate}
\end{exercice}


%%%%%%%
\serie{Calculer des aires}

\begin{exercice} %4
   Sachant que l'unité d'aire est le carreau, déterminer l'aire de chaque figure suivante.
   \begin{center}
      {\psset{unit=0.5}
      \begin{pspicture}(0,-5)(15,10)
         \psgrid[subgriddiv=0,gridlabels=0pt,gridcolor=gray](0,-5)(15,10)
         \psset{linewidth=0.5mm}
         \psframe(1,1)(4,3)
         \rput(2.5,2){\bf 1}
         \pspolygon(5,0)(11,0)(10,4)(8,4)(5,2)
         \rput(8,2){\bf 4}
         \pspolygon(1,4)(1,9)(5,9)(5,8)(2,8)(2,7)(4,7)(4,6)(2,6)(2,4)
         \rput(1.5,6.5){\bf 2}
         \pspolygon(6,7)(12,9)(6,9)
         \rput(7.5,8.4){\bf 3}
         \pspolygon(5,4)(5,6)(12,6)(12,8)(14,8)(14,4)(13,3)(14,2)(14,1)(12,1)(11,3)(12,5)(6,5)
         \rput(13,5.5){\bf 5}
         \psline(3,-2)(1,-2)(1,-4)(2,-4)
         \psarc(3,-4){1}{0}{180}
         \psline(4,-4)(6,-4)(6,-2)(5,-2)
         \psarc(4,-2){1}{0}{180}
         \rput(4,-2.5){\bf 6} 
         \psline(7,-1)(14,-1)
         \psline(7,-4)(14,-4)
         \psline(8,-2)(8,-3)
         \psline(13,-2)(13,-3)
         \psarc(8,-4){1}{90}{180}
         \psarc(8,-1){1}{180}{-90}
         \psarc(13,-4){1}{0}{90}
         \psarc(13,-1){1}{-90}{0}
         \pscircle(10,-2.5){1}
         \rput(12,-2.5){\bf 7}
      \end{pspicture}}
   \end{center}
\end{exercice}

\begin{exercice} %5
   Quelle est l'aire, en \ucmq{}, de la figure grisée sachant que \begin{pspicture}(3,2)(4,3)
         \psframe[fillstyle=solid,fillcolor=lightgray](3,2)(4,3)
         \psline(3,2)(4,3)
         \psline(3,3)(4,2)
      \end{pspicture} vaut \ucmq{1} ? Expliquer.
   \begin{center}
      \begin{pspicture*}(5,0)(10,5)
         \pspolygon[fillstyle=solid,fillcolor=lightgray](6,0)(9,0)(9,1)(10,2)(8,4)(9,4)(8,5)(7,5)(6,4)(7,4)(5,2)(6,1)
         \pspolygon[fillstyle=solid,fillcolor=white](6.5,0.5)(7.5,1.5)(8.5,0.5)(9,1)(9,2)(8,2)(7.5,2.5)(8,3)(7.5,3.5)(8,4)(7,4)(7.5,3.5)(7,3)(7.5,2.5)(7,2)(6,2)(6,1)
         \psgrid[subgriddiv=1](5,0)(10,5)
         \multido{\i=10+-1}{10}{%
	   \FPeval{a}{\i+5}
	   \psline(\i,0)(\a,5)
	   \psline(\i,5)(\a,0)}
      \end{pspicture*}
   \end{center}
\end{exercice}

\begin{exercice} %6
   Estimer l'aire de la figure suivante :
   \begin{center}
      {\psset{unit=0.6,linewidth=0.5mm}
      \small
         \begin{pspicture}(0,-3)(10,3)
            \psgrid[subgriddiv=0,gridlabels=0pt,gridcolor=gray](0,-3)(10,3)
            \psarc(2,0){2}{180}{0}
            \psarc(3,0){3}{180}{0}
            \psarc(8,0){2}{0}{180}
            \psarc(7,0){3}{0}{180}
            \psline{<->}(8,-1)(9,-1)
            \rput(8.5,-1.5){2 cm}
         \end{pspicture}}
   \end{center}
\end{exercice}

\end{colonne*exercice}


%%%%%%%%%%%%%%%%%%%%%%%%%%%%%%%%%%%%%
%%%%%%%%%%%%%%%%%%%%%%%%%%%%%%%%%%%%%
\Recreation

\vspace*{-5mm}

\enigme[Le retour du Curvica !]

   \partie[les 24 formes du Curvica] \smallskip

   \curvica{\rput(1,1){\textcolor{B1}{a}} \psline(0,0)(0,2) \psline(0,2)(2,2) \psline(2,2)(2,0) \psarc(1,-2){2.24}{63.4}{116.6}}
   \curvica{\rput(1,1){\textcolor{B1}{b}} \psline(0,0)(0,2) \psline(0,2)(2,2) \psarc(4,1){2.24}{153.4}{-153.4} \psarc(1,2){2.24}{-116.6}{-63.4}}
   \curvica{\rput(1,1){\textcolor{B1}{c}} \psarc(2,1){2.24}{153.4}{-153.4} \psline(2,0)(2,2) \psline(0,2)(2,2) \psarc(1,2){2.24}{-116.6}{-63.4}}
   \curvica{\rput(1,1){\textcolor{B1}{d}} \psline(0,0)(0,2) \psline(2,0)(2,2) \psline(0,2)(2,2) \psline(0,0)(2,0)} \\ [1mm]
   
   \curvica{\rput(1,1){\textcolor{B1}{e}} \psline(0,0)(0,2) \psline(2,0)(2,2) \psarc(1,0){2.24}{63.4}{116.6} \psarc(1,2){2.24}{-116.6}{-63.4}}
   \curvica{\rput(1,1){\textcolor{B1}{f}} \psline(0,0)(0,2) \psline(2,0)(2,2) \psarc(1,4){2.24}{-116.6}{-63.4} \psarc(1,2){2.24}{-116.6}{-63.4}}
   \curvica{\rput(1,1){\textcolor{B1}{g}} \psline(0,0)(0,2) \psline(2,0)(2,2) \psarc(1,4){2.24}{-116.6}{-63.4} \psarc(1,-2){2.24}{63.4}{116.6}}
   \curvica{\rput(1,1){\textcolor{B1}{h}} \psline(0,0)(0,2) \psline(2,0)(2,2) \psline(0,2)(2,2) \psarc(1,2){2.24}{-116.6}{-63.4}} \\ [1mm]
   
   \curvica{\rput(1,1){\textcolor{B1}{i}} \psline(0,0)(0,2) \psarc(1,4){2.24}{-116.6}{-63.4} \psarc(4,1){2.24}{153.4}{-153.4} \psarc(1,2){2.24}{-116.6}{-63.4}} 
   \curvica{\rput(1,1){\textcolor{B1}{j}} \psarc(2,1){2.24}{153.4}{-153.4} \psarc(1,4){2.24}{-116.6}{-63.4} \psarc(4,1){2.24}{153.4}{-153.4} \psarc(1,-2){2.24}{63.4}{116.6}}
   \curvica{\rput(1,1){\textcolor{B1}{k}} \psarc(2,1){2.24}{153.4}{-153.4} \psarc(0,1){2.24}{-26.6}{26.6} \psarc(1,0){2.24}{63.4}{116.6} \psarc(1,2){2.24}{-116.6}{-63.4}}
   \curvica{\rput(1,1){\textcolor{B1}{l}} \psarc(-2,1){2.24}{-26.6}{26.6} \psline(2,0)(2,2) \psarc(1,4){2.24}{-116.6}{-63.4} \psarc(1,-2){2.24}{63.4}{116.6}} \\ [1mm]
   
   \curvica{\rput(1,1){\textcolor{B1}{m}} \psline(0,0)(0,2) \psarc(0,1){2.24}{-26.6}{26.6} \psarc(1,4){2.24}{-116.6}{-63.4} \psarc(1,-2){2.24}{63.4}{116.6}} 
   \curvica{\rput(1,1){\textcolor{B1}{n}} \psarc(-2,1){2.24}{-26.6}{26.6} \psarc(4,1){2.24}{153.4}{-153.4} \psarc(1,0){2.24}{63.4}{116.6} \psarc(1,2){2.24}{-116.6}{-63.4}}
   \curvica{\rput(1,1){\textcolor{B1}{o}} \psarc(2,1){2.24}{153.4}{-153.4} \psarc(1,4){2.24}{-116.6}{-63.4} \psarc(4,1){2.24}{153.4}{-153.4} \psarc(1,2){2.24}{-116.6}{-63.4}}
   \curvica{\rput(1,1){\textcolor{B1}{p}} \psarc(2,1){2.24}{153.4}{-153.4} \psline(2,0)(2,2) \psarc(1,0){2.24}{63.4}{116.6} \psarc(1,-2){2.24}{63.4}{116.6}} \\ [1mm]
   
      \curvica{\rput(1,1){\textcolor{B1}{q}} \psline(0,0)(0,2) \psarc(4,1){2.24}{153.4}{-153.4} \psarc(1,0){2.24}{63.4}{116.6} \psarc(1,2){2.24}{-116.6}{-63.4}}
      \curvica{\rput(1,1){\textcolor{B1}{r}} \psarc(2,1){2.24}{153.4}{-153.4} \psarc(0,1){2.24}{-26.6}{26.6} \psarc(1,4){2.24}{-116.6}{-63.4} \psarc(1,2){2.24}{-116.6}{-63.4}}
      \curvica{\rput(1,1){\textcolor{B1}{s}} \psarc(-2,1){2.24}{-26.6}{26.6} \psarc(1,4){2.24}{-116.6}{-63.4} \psarc(4,1){2.24}{153.4}{-153.4} \psarc(1,-2){2.24}{63.4}{116.6}}
      \curvica{\rput(1,1){\textcolor{B1}{t}} \psarc(2,1){2.24}{153.4}{-153.4} \psline(2,0)(2,2) \psarc(1,0){2.24}{63.4}{116.6} \psarc(1,2){2.24}{-116.6}{-63.4}} \\ [1mm]
      
      \curvica{\rput(1,1){\textcolor{B1}{u}} \psline(0,0)(0,2) \psarc(1,4){2.24}{-116.6}{-63.4} \psarc(4,1){2.24}{153.4}{-153.4} \psline(0,0)(2,0)} 
      \curvica{\rput(1,1){\textcolor{B1}{v}} \psarc(2,1){2.24}{153.4}{-153.4} \psarc(1,4){2.24}{-116.6}{-63.4} \psarc(4,1){2.24}{153.4}{-153.4} \psline(0,0)(2,0)}
      \curvica{\rput(1,1){\textcolor{B1}{w}} \psarc(2,1){2.24}{153.4}{-153.4} \psarc(1,0){2.24}{63.4}{116.6} \psarc(4,1){2.24}{153.4}{-153.4} \psline(0,0)(2,0)}
      \curvica{\rput(1,1){\textcolor{B1}{x}} \psarc(2,1){2.24}{153.4}{-153.4} \psarc(1,4){2.24}{-116.6}{-63.4} \psline(2,0)(2,2) \psline(0,0)(2,0)}

\pagebreak

\partie[la grille réponse] \bigskip

{\hautab{1.75}
\begin{tabular}{|p{11cm}|C{3}|C{1}|}
   \hline
   Défis & Réponses & Points \\
   \hline
   \multicolumn{2}{|c|}{Niveau facile} & 1 pt \\
   \hline
   1. Trouver la pièce dont l'aire est la plus grande. & & \\
   \hline
   2. Trouver la pièce dont le périmètre est le plus petit. & & \\
   \hline
   3. Assembler deux pièces pour obtenir un rectangle. & & \\
   \hline
   4. Trouver la pièce de plus grand périmètre et de plus petite aire. & & \\
   \hline
   5. Trouver une pièce ayant un seul axe de symétrie. & & \\
   \hline
   6. Trouver une pièce ayant exactement deux axes de symétrie. & & \\
   \hline
   7. Trouver deux pièces de même périmètre mais d'aires différentes. & & \\
   \hline
   \hline
   \multicolumn{2}{|c|}{Niveau moyen} & 1,5 pt \\
   \hline
   8. Trouver deux pièces de même aire mais de périmètres différents. & & \\
   \hline
   9. Trouver deux pièces de même aire et de même périmètre. & & \\
   \hline
   10. Assembler quatre pièces pour obtenir un carré. & & \\
   \hline
   11. Trouver deux pièces ayant même aire, même périmètre et au moins un axe de symétrie chacune. & & \\
   \hline
   12. Trouver deux pièces dont l'une a un périmètre plus grand que l'autre mais une aire plus petite. & & \\
   \hline
   13. Assembler deux pièces pour obtenir une figure dont l'aire et le périmètres sont les plus grands possibles. & & \\
   \hline
   \hline
   \multicolumn{2}{|c|}{Niveau difficile} & 2 pts \\
   \hline
   14. Trouver deux pièces ayant ni axe de symétrie, ni même périmètre, ni même aire. & & \\
   \hline
   15. Assembler cinq ou six pièces pour obtenir un rectangle. & & \\
   \hline
   \hline
   \multicolumn{2}{|r|}{Total sur 20 points} & \\
   \hline
\end{tabular}}

